\pdfoutput=1
\documentclass[a4paper,pdftex,10pt]{article}
% \usepackage[whole,autotilde]{bxcjkjatype}
\usepackage[T1]{fontenc}
\usepackage{tgtermes}

% ---Display \subsubsection at the Index
\setcounter{tocdepth}{3}

% ---Setting about the geometry of the document----
\usepackage{a4wide}
% \pagestyle{empty}

% ---Physics and Math Packages---
\usepackage{amssymb,amsfonts,amsthm,mathtools}
\usepackage{physics,braket,bm,slashed}

% ---underline---
\usepackage[normalem]{ulem}

% ---cancel---
\usepackage{cancel}

% --- surround the texts or equations
\usepackage{fancybox,ascmac}

% ---settings of theorem environment---
\theoremstyle{definition}
\newtheorem{dfn}{Definition}
\newtheorem{prop}{Proposition}
\newtheorem{thm}{Theorem}

% ---settings of proof environment---
\renewcommand{\proofname}{\textbf{Proof}}
\renewcommand{\qedsymbol}{$\blacksquare$}

% ---Ignore the Warnings---
\usepackage{silence}
\WarningFilter{latexfont}{Some font shapes,Font shape}
\ExplSyntaxOn
\msg_redirect_name:nnn{hooks}{generic-deprecated}{none}
\ExplSyntaxOff

% ---Insert the figure (If insert the `draft' at the option, the process becomes faster.)---
% \usepackage{graphicx}
% \usepackage{subcaption}

% ----Add a link to a text---
\usepackage{url,hyperref}
\usepackage[dvipsnames,svgnames]{xcolor}
\hypersetup{colorlinks=true,citecolor=FireBrick,linkcolor=Navy,urlcolor=purple}

% ---Tikz---
\usepackage{tikz,pgf,pgfplots,circuitikz}
\pgfplotsset{compat=1.15}
\usetikzlibrary{intersections, arrows.meta, angles, calc, 3d, decorations.pathmorphing}
\usepackage[compat=1.1.0]{tikz-feynhand}

% ---tcolorbox---
\usepackage{tcolorbox}
\tcbuselibrary{raster,skins,breakable}
\newtcolorbox{graybox}[1][]{frame empty, colback=black!07!white, sharp corners}

% ---Add the section number to the equation, figure, and table number---
\makeatletter
   \renewcommand{\theequation}{$\thesection.\arabic{equation}$}
   \@addtoreset{equation}{section}
   
   \renewcommand{\thefigure}{\thesection.\arabic{figure}}
   \@addtoreset{figure}{section}
   
   \renewcommand{\thetable}{\thesection.\arabic{table}}
   \@addtoreset{table}{section}
\makeatother

% ---enumerate---
% \renewcommand{\labelenumi}{$\arabic{enumi}.$}
% \renewcommand{\labelenumii}{$(\arabic{enumii})$}

% ---Index---
% \usepackage{makeidx}
% \makeindex 

% ---footnotes---
\renewcommand{\thefootnote}{$\ast$\arabic{footnote}}

% ---Title---
\title{Notes on Anomalies}
\author{Itsuki Miyane}
\date{Last modified:\ \today}

\begin{document}

\maketitle

\begin{enumerate}
  \item
        Let us consider the variation of the action
        \begin{equation}
          S_{m}
          =
          \int\dd^4 \sqrt{-g}K(\phi,X)
        \end{equation}
        where $X$ denotes
        \begin{equation}
          X
          \equiv
          -
          \frac{1}{2}g^{\mu\nu}\partial_[\mu]\phi\partial_{\nu}\phi
          .
        \end{equation}
        It obeys
        \begin{align}
          \delta S_{m}
           & =
          \int\dd^4
          \left[
            (\delta\sqrt{-g})K
            +
            \sqrt{-g}(\delta K)
            \right]
          \nonumber
          \\
           & =
          -\frac{1}{2}
          \int\dd^4 x
          \sqrt{-g}
          \left\{
          g_{\mu\nu}K
          +
          K_{X}\partial_{\mu}\partial_{\nu}\phi
          \right\}
          \delta g^{\mu\nu}
          \label{eqn:variation}
        \end{align}
        where $K_{X}\equiv \partial_{X}K(\phi,X)$. The definition of the energy-momentum tensor so far is
        \begin{equation}
          \delta S_{m}
          =
          -\frac{1}{2}\int\dd^4 x\
          \delta g^{\mu\nu}\sqrt{-g}T_{\mu\nu}
          \label{eqn:dfn_energy-momentum_tensor}
        \end{equation}
        and by comparing this definition with the equation \eqref{eqn:variation}, we find
        \begin{graybox}
          \begin{equation}
            T_{\mu\nu}
            =
            g_{\mu\nu}K
            +
            K_{X}\partial_{\mu}\phi\partial_{\nu}\phi
            \hspace{15pt}
            (K_{X}\equiv \partial_{X}K(\phi,X))
            .
          \end{equation}
        \end{graybox}

  \item
        The perturbed FLRW metric
        \begin{equation}
          g_{\mu\nu}
          =
          a^2(\eta)
          \begin{pmatrix}
            -(1+2A)       & \partial_{i}B                                   \\
            \partial_{i}B & (1+2\psi)\delta_{ij}+2\partial_{i}\partial_{j}E
          \end{pmatrix}
        \end{equation}
        is obtained by metric perturbation. The energy-momentum tensor becomes
        \begin{equation}
          T^{\mu}_{\ \nu}
          =
          g^{\mu\rho}T_{\rho\nu}
          =
          \delta^{\mu}_{\nu}K(\phi,X)
          +
          g^{\mu\rho}
          K_{X}
          \partial_{\rho}\phi\partial_{\nu}\phi
          \label{eqn:upperd_energy-momentum_tensor}
        \end{equation}
        from the previous result. By calculating the inverse matrix of the metric, we obtain\footnote{
        This result is obtained by purterbating $g^{\mu\nu}=g_{0}^{\mu\nu}+\delta g^{\mu\nu}$. This implies $g_{0}^{\mu\nu}=0$ naively and $\delta g^{\mu\nu}$ should contribute to cancel out the perturbative parts in $g_{\mu\nu}$.
        }
        \begin{equation}
          g^{\mu\nu}
          =
          \frac{1}{a^2(\eta)}
          \begin{pmatrix}
            1-2A(\eta,\bm{x}) & \partial_{x}B               & \partial_{y}B               & \partial_{z}B                \\
            \partial_{x}B     & 1-2(\psi+\partial_{x}^{2}E) & -2\partial_{x}\partial_{y}E & -2\partial_{x}\partial_{z}E  \\
            \partial_{y}B     & -2\partial_{x}\partial_{y}E & 1-2(\psi+\partial_{y}^{2}E) & -2\partial_{y}\partial_{z} E \\
            \partial_{z}B     & -2\partial_{x}\partial_{z}E & -2\partial_{y}\partial_{z}E & 1-2(\psi+\partial_{z}^{2}E)
          \end{pmatrix}
          .
        \end{equation}

        To compute $\delta T^{\mu}_{\ \nu}$, let me consider the variation $\phi\rightarrow\phi(\eta)+\delta\phi(\eta,\bm{x})$, i.e., we will calculate
        \begin{align}
          T^{\mu}_{\ \nu}
           & =
          \delta^{\mu}_{\nu}
          K(\phi+\delta\phi, X+\delta X)
          +
          g^{\mu\rho}
          K_{X}(\phi+\delta\phi, X+\delta X)
          \partial_{\mu}(\phi+\delta\phi)
          \partial_{\nu}(\phi+\delta\phi)
          \label{eqn:T_def}
        \end{align}
        and take the first order of the perturbative term and identify it as $\delta T^{\mu}_{\ \nu}$. When we evaluate the above value, we should be careful with the fact that the background field does not depend on the space coordinates, and the derivative to that direction vanishes.

        Thus we obtain
        \begin{gather}
          \delta X
          =
          \frac{\phi^{\prime} (\delta \phi^{\prime}-A \phi^{\prime})}{a^2}
          ,
          \\
          K
          =
          \frac{K_X \phi^{\prime} (\delta \phi^{\prime}-A \phi^{\prime})}{a^2}+\delta \phi  K_{\phi }
          ,
          \\
          K_{X}
          =
          \frac{K_{XX} \phi^{\prime} (\delta \phi^{\prime}-A \phi^{\prime})}{a^2}+\delta \phi  K_{X \phi}
        \end{gather}
        and we can get \eqref{eqn:T_def} as
        \begin{graybox}
          \vspace*{-10pt}
          \begin{gather}
            \delta T^{0}_{\ 0}
            =
            \frac{K_{XX} {\phi^{\prime}}^3 (A {\phi^{\prime}}-{\delta \phi^{\prime}})}{a^4}+\frac{{\phi^{\prime}} (A K_X {\phi^{\prime}}+{\delta \phi^{\prime}} (-K_X)-\delta \phi  K_{X \phi} {\phi^{\prime}})}{a^2}+\delta \phi  K_{\phi }
            \\
            \delta T^{ii}
            =
            \frac{K_{X}\phi^{\prime}(\delta\phi^{\prime}-\phi^{\prime}A)}{a^2}+K_{\phi}\delta\phi
            \\
            \delta T^{i}_{\ 0}
            =
            \frac{1}{a^2}
            \left\{
            \partial_{i}(\delta\phi)+\phi^{\prime}\partial_{i}B
            \right\}
            \\
            \delta T^{0}_{\ i}
            =
            -\frac{1}{a^2}K_{X}\partial_{i}(\delta\phi)\delta\phi^{\prime}
          \end{gather}
          where $i=1,2,3$.
        \end{graybox}

  \item
        The perturbative Einstein equation is given as
        \begin{gather}
          3\mathcal{H}(\psi^{\prime}-\mathcal{H}A)-\nabla^2\psi+\mathcal{H}\nabla^2(E^{\prime}-B)
          =
          4\pi Ga^2\delta\rho
          \label{eqn:hoge_1}
          \\
          \psi^{\prime}-\mathcal{H}A=4\pi Ga\delta q
          \label{eqn:hoge_2}
        \end{gather}
        in this course. In this case, we will use the notation as follows:
        \begin{gather}
          K=-\Lambda
          ,
          \\
          \delta q=a\rho(v+B)
          \label{eqn:deltaq}
          ,
          \\
          \Phi = \psi-\mathcal{H}(E^{\prime}-B)
          \label{eqn:PHI}
          ,
          \\
          \delta\rho_{m}
          =
          \delta\rho
          -
          3\frac{\mathcal{H}}{a}\delta q
          \label{eqn:deltarhom}
          .
        \end{gather}
        The first setup is irrelevant at this moment.

        For the first step, insert \eqref{eqn:hoge_2} into \eqref{eqn:hoge_1} and organize some terms. We immediately find
        \begin{equation}
          \nabla^2(\psi-\mathcal{H}(E^{\prime}-B))
          =
          4\pi Ga
          \left\{
          3\mathcal{H}\delta q
          -
          a\delta\rho
          \right\}
          .
        \end{equation}
        We notice the foremost term corresponds with $\Phi$. In addition, we try to put \eqref{eqn:deltarhom} into the $\delta\rho$ on the right-hand side. Thus we obtain slick equality
        \begin{graybox}
          \vspace*{-5pt}
          \begin{equation}
            \nabla^2\Phi
            +
            4\pi Ga^2\delta\rho_{m}
            =
            0
            .
            \label{eqn:reault_3}
          \end{equation}
        \end{graybox}

  \item        
        Let me confirm the given relation again:
        \begin{equation}
          \delta
          \equiv
          \frac{\delta\rho_{m}}{\rho}
          \ ,\quad
          \Psi=A
          ,\ \quad
          \Phi=\psi
          ,\ \quad
          \Phi+\Phi=0
          .
        \end{equation}
        Naively, we just put that relation to the previous result \eqref{eqn:reault_3} and attain the equality
        \begin{equation}
          -\nabla^2 A+4\pi Ga^2 \rho\delta=0
          .
          \label{eqn:hoge_3}
        \end{equation}
        We will try to evaluate undetermined factors $\rho$ and $A$. From the problem 2, we obtain $T^{0}_{\ 0}=-\rho$ as 
        \begin{equation}
          -\rho
          =
          K-\frac{K_X {\phi^{\prime}}^2}{a^2}+\frac{K_{XX} {\phi^{\prime}}^3 (A {\phi^{\prime}}-{\delta \phi^{\prime}})}{a^4}+\frac{{\phi^{\prime}} (A K_X {\phi^{\prime}}+{\delta \phi^{\prime}} (-K_X)-\delta \phi  K_{X\phi} {\phi^{\prime}})}{a^2}+\delta \phi  K_{\phi }
        \end{equation} 
        up to the first order of the perturbation. In this case, since we set $K=-\Lambda$, the derivation of $K$, such as $K_{X}, K_{X\phi}$, etc., vanishes. Thus we obtain the relation
        \begin{equation}
          \rho=\Lambda
          .
          \label{eqn:rhoequalLambda}
        \end{equation}
        
        Let us figure out $A$ in terms of $\delta$. To accomplish this, we compute \eqref{eqn:hoge_2} and \eqref{eqn:deltarhom}. These relations imply equality
        \begin{equation}
          -(1+\mathcal{H})A
          =
          4\pi Ga^2 \Lambda(v+B)
          .
          \label{eqn:hoge_4}
        \end{equation}
        On the other hand, we evaluate \eqref{eqn:deltarhom} and the relation
        \begin{equation}
          \delta
          =
          \frac{\delta \rho}{\Lambda}-3\mathcal{H}(v+B)
        \end{equation}
        holds. From the result of problem 2, we find $\delta T^{0}_{\ 0}=-\delta \rho=0$. Putting this equation into \eqref{eqn:hoge_4} and vanishing the factor $v+B$, we conclude $A$ as 
        \begin{equation}
          A
          =
          \frac{4\pi a^2 \Lambda}{3\mathcal{H}(1+\mathcal{H})}\delta
        \end{equation}
        and we can organize the equation \eqref{eqn:hoge_4} for
        \begin{graybox}
          \vspace*{-5pt}
          \begin{equation}
            \nabla^2
            \left[ \frac{\delta}{3\mathcal{H}(1+\mathcal{H})} \right]
            =
            \delta
            .
          \end{equation}
        \end{graybox}

\end{enumerate}

\end{document}
