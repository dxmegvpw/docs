\pdfoutput=1
\documentclass[a4paper,pdftex,10pt]{article}
% \usepackage[whole,autotilde]{bxcjkjatype}
\usepackage[T1]{fontenc}
\usepackage{tgtermes}

% ---Display \subsubsection at the Index
\setcounter{tocdepth}{3}

% ---Setting about the geometry of the document----
\usepackage{a4wide}
% \pagestyle{empty}

% ---Physics and Math Packages---
\usepackage{amssymb,amsfonts,amsthm,mathtools}
\usepackage{physics,braket,bm,slashed}

% ---underline---
\usepackage[normalem]{ulem}

% ---cancel---
\usepackage{cancel}

% --- surround the texts or equations
\usepackage{fancybox,ascmac}

% ---settings of theorem environment---
\theoremstyle{definition}
\newtheorem{dfn}{Definition}
\newtheorem{prop}{Proposition}
\newtheorem{thm}{Theorem}

% ---settings of proof environment---
\renewcommand{\proofname}{\textbf{Proof}}
\renewcommand{\qedsymbol}{$\blacksquare$}

% ---Ignore the Warnings---
\usepackage{silence}
\WarningFilter{latexfont}{Some font shapes,Font shape}
\ExplSyntaxOn
\msg_redirect_name:nnn{hooks}{generic-deprecated}{none}
\ExplSyntaxOff

% ---Insert the figure (If insert the `draft' at the option, the process becomes faster.)---
% \usepackage{graphicx}
% \usepackage{subcaption}

% ----Add a link to a text---
\usepackage{url,hyperref}
\usepackage[dvipsnames,svgnames]{xcolor}
\hypersetup{colorlinks=true,citecolor=FireBrick,linkcolor=Navy,urlcolor=purple}

% ---Tikz---
\usepackage{tikz,pgf,pgfplots,circuitikz}
\pgfplotsset{compat=1.15}
\usetikzlibrary{intersections, arrows.meta, angles, calc, 3d, decorations.pathmorphing}
\usepackage[compat=1.1.0]{tikz-feynhand}

% ---tcolorbox---
\usepackage{tcolorbox}
\tcbuselibrary{raster,skins,breakable}
\newtcolorbox{graybox}[1][]{frame empty, colback=black!07!white, sharp corners}

% ---Add the section number to the equation, figure, and table number---
\makeatletter
   \renewcommand{\theequation}{$\thesection.\arabic{equation}$}
   \@addtoreset{equation}{section}
   
   \renewcommand{\thefigure}{\thesection.\arabic{figure}}
   \@addtoreset{figure}{section}
   
   \renewcommand{\thetable}{\thesection.\arabic{table}}
   \@addtoreset{table}{section}
\makeatother

% ---enumerate---
% \renewcommand{\labelenumi}{$\arabic{enumi}.$}
% \renewcommand{\labelenumii}{$(\arabic{enumii})$}

% ---Index---
% \usepackage{makeidx}
% \makeindex 

% ---footnotes---
\renewcommand{\thefootnote}{$\ast$\arabic{footnote}}

% ---Title---
\title{Notes on Anomalies}
\author{Itsuki Miyane}
\date{Last modified:\ \today}

\begin{document}

\maketitle

\begin{enumerate}
  \item
        If we put $\theta=\pi/2$, the quantity $\Sigma$ becomes $r^2$ and the line element is obtained as
        \begin{equation}
          \dd s^2
          =
          -
          c^2
          \left( 1-\frac{2\mu}{r} \right)\dd t^2
          -
          \frac{4\mu ac}{r}\dd t\dd \varphi
          +
          \frac{r^2}{\Delta}\dd r^2
          +
          \left(
          r^2+a^2+\frac{2\mu a^2}{r}
          \right)
          \dd\varphi^2
        \end{equation}
        and the metric also is as
        \begin{equation}
          g_{\mu\nu}
          =
          \begin{pmatrix}
            -c^2(1-2\mu/r) & 0          & -2\mu ac/r           \\
            0              & r^2/\Delta & 0                    \\
            -2\mu ac/r     & 0          & r^2 + a^2 +2\mu a^2/r
          \end{pmatrix}
        \end{equation}
        where $\Delta$ still remain $r^2-2\mu r+a^2$. Therefore the conserved quantities $p_{t}$ and $p_{\varphi}$ are obtained as 
        \begin{align}
          p_{t}
          &=
          -g_{tt}\dot{t}-g_{t\varphi}\dot{\varphi}
          \nonumber
          \\
          &=
          c^2(1-2\mu/r)\dot{t}
          +
          (2\mu ac/r)\dot{\varphi}
          \\
          p_{\varphi}
          &=
          g_{\varphi t}\dot{t}+g_{\varphi\varphi}\dot{\varphi}
          \nonumber
          \\
          &=
          (2\mu ac/r) \dot{t}
          +
          (r^2 + a^2 +2\mu a^2/r)\dot{\varphi}
        \end{align} 
        and we will solve these equations to $\dot{t}$ and $\dot{\varphi}$.

        











\end{enumerate}

% ----------------------------------------------------------------------
\clearpage
\begin{thebibliography}{99}
  \bibitem{url01}
  \href{https://www.roma1.infn.it/teongrav/onde19_20/geodetiche_Kerr.pdf}{Chapter 22 Geodesic motion in Kerr spacetime}. (Last accessed: \today)
  \bibitem{url02}
  \href{https://www.pas.rochester.edu/assets/pdf/undergraduate/kerr_geometry_and_rotating_black_holes.pdf}{Kerr Geometry and Rotating Black Hole}. (Last accessed: \today)
\end{thebibliography}

% ----------------------------------------
% \appendix
% \section{The computation of the radius of the photon sphere}











% ----------------------------------------
% \clearpage
% \bibliography{ref}
% \bibliographystyle{ytamsalpha}

% ----------------------------------------
% \clearpage
% \index{hoge@hoge}
% \printindex

\end{document}
