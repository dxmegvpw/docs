\pdfoutput=1
\documentclass[a4paper,pdftex,10pt]{article}
% \usepackage[whole,autotilde]{bxcjkjatype}
\usepackage[T1]{fontenc}
\usepackage{tgtermes}

% ---Display \subsubsection at the Index
\setcounter{tocdepth}{3}

% ---Setting about the geometry of the document----
\usepackage{a4wide}
% \pagestyle{empty}

% ---Physics and Math Packages---
\usepackage{amssymb,amsfonts,amsthm,mathtools}
\usepackage{physics,braket,bm,slashed}

% ---underline---
\usepackage[normalem]{ulem}

% ---cancel---
\usepackage{cancel}

% --- surround the texts or equations
\usepackage{fancybox,ascmac}

% ---settings of theorem environment---
\theoremstyle{definition}
\newtheorem{dfn}{Definition}
\newtheorem{prop}{Proposition}
\newtheorem{thm}{Theorem}

% ---settings of proof environment---
\renewcommand{\proofname}{\textbf{Proof}}
\renewcommand{\qedsymbol}{$\blacksquare$}

% ---Ignore the Warnings---
\usepackage{silence}
\WarningFilter{latexfont}{Some font shapes,Font shape}
\ExplSyntaxOn
\msg_redirect_name:nnn{hooks}{generic-deprecated}{none}
\ExplSyntaxOff

% ---Insert the figure (If insert the `draft' at the option, the process becomes faster.)---
% \usepackage{graphicx}
% \usepackage{subcaption}

% ----Add a link to a text---
\usepackage{url,hyperref}
\usepackage[dvipsnames,svgnames]{xcolor}
\hypersetup{colorlinks=true,citecolor=FireBrick,linkcolor=Navy,urlcolor=purple}

% ---Tikz---
\usepackage{tikz,pgf,pgfplots,circuitikz}
\pgfplotsset{compat=1.15}
\usetikzlibrary{intersections, arrows.meta, angles, calc, 3d, decorations.pathmorphing}
\usepackage[compat=1.1.0]{tikz-feynhand}

% ---tcolorbox---
\usepackage{tcolorbox}
\tcbuselibrary{raster,skins,breakable}
\newtcolorbox{graybox}[1][]{frame empty, colback=black!07!white, sharp corners}

% ---Add the section number to the equation, figure, and table number---
\makeatletter
   \renewcommand{\theequation}{$\thesection.\arabic{equation}$}
   \@addtoreset{equation}{section}
   
   \renewcommand{\thefigure}{\thesection.\arabic{figure}}
   \@addtoreset{figure}{section}
   
   \renewcommand{\thetable}{\thesection.\arabic{table}}
   \@addtoreset{table}{section}
\makeatother

% ---enumerate---
% \renewcommand{\labelenumi}{$\arabic{enumi}.$}
% \renewcommand{\labelenumii}{$(\arabic{enumii})$}

% ---Index---
% \usepackage{makeidx}
% \makeindex 

% ---footnotes---
\renewcommand{\thefootnote}{$\ast$\arabic{footnote}}

% ---Title---
\title{Notes on Anomalies}
\author{Itsuki Miyane}
\date{Last modified:\ \today}

\begin{document}

\maketitle

\begin{enumerate}
  \item
        Let us consider the variation of the action
        \begin{equation}
          S_{m}
          =
          \int\dd^4 \sqrt{-g}K(\phi,X)
        \end{equation}
        where $X$ denotes
        \begin{equation}
          X
          \equiv
          -
          \frac{1}{2}g^{\mu\nu}\partial_[\mu]\phi\partial_{\nu}\phi
          .
        \end{equation}
        It obeys
        \begin{align}
          \delta S_{m}
           & =
          \int\dd^4
          \left[
            (\delta\sqrt{-g})K
            +
            \sqrt{-g}(\delta K)
            \right]
          \nonumber
          \\
           & =
          -\frac{1}{2}
          \int\dd^4 x
          \sqrt{-g}
          \left\{
          g_{\mu\nu}K
          +
          K_{X}\partial_{\mu}\partial_{\nu}\phi
          \right\}
          \delta g^{\mu\nu}
          \label{eqn:variation}
        \end{align}
        where $K_{X}\equiv \partial_{X}K(\phi,X)$. The definition of the energy-momentum tensor so far is
        \begin{equation}
          \delta S_{m}
          =
          -\frac{1}{2}\int\dd^4 x\
          \delta g^{\mu\nu}\sqrt{-g}T_{\mu\nu}
          \label{eqn:dfn_energy-momentum_tensor}
        \end{equation}
        and by comparing this definition with the equation \eqref{eqn:variation}, we find
        \begin{graybox}
          \begin{equation}
            T_{\mu\nu}
            =
            g_{\mu\nu}K
            +
            K_{X}\partial_{\mu}\partial_{\nu}\phi
            \hspace{15pt}
            (K_{X}\equiv \partial_{X}K(\phi,X))
            .
          \end{equation}
        \end{graybox}

  \item
        The perturbed FLRW metric
        \begin{equation}
          g_{\mu\nu}
          =
          a^2(\eta)
          \begin{pmatrix}
            -(1+2A)       & \partial_{i}B                                   \\
            \partial_{i}B & (1+2\psi)\delta_{ij}+2\partial_{i}\partial_{j}E
          \end{pmatrix}
        \end{equation}
        is obtained by metric perturbation. The energy-momentum tensor becomes
        \begin{equation}
          T^{\mu}_{\ \nu}
          =
          g^{\mu\rho}T_{\rho\nu}
          =
          \delta^{\mu}_{\nu}K(\phi,X)
          +
          g^{\mu\rho}
          K_{X}
          \partial_{\rho}\phi\partial_{\nu}\phi
          \label{eqn:upperd_energy-momentum_tensor}
        \end{equation}
        from the previous result. By calculating the inverse matrix of the metric, we obtain\footnote{
          This result is obtained by purterbating $g^{\mu\nu}=g_{0}^{\mu\nu}+\delta g^{\mu\nu}$. This implies $g_{0}^{\mu\nu}=0$ naively and $\delta g^{\mu\nu}$ should contribute to cancel out the perturbative parts in $g_{\mu\nu}$.
        }
        \begin{equation}
          g^{\mu\nu}
          =
          \frac{1}{a^2(\eta)}
          \begin{pmatrix}
            1-2A(\eta,\bm{x}) & \partial_{x}B & \partial_{y}B & \partial_{z}B \\
            \partial_{x}B & 1-2(\psi+\partial_{x}^{2}E) & -2\partial_{x}\partial_{y}E & -2\partial_{x}\partial_{z}E \\
            \partial_{y}B & -2\partial_{x}\partial_{y}E & 1-2(\psi+\partial_{y}^{2}E) & -2\partial_{y}\partial_{z} E \\
            \partial_{z}B & -2\partial_{x}\partial_{z}E & -2\partial_{y}\partial_{z}E & 1-2(\psi+\partial_{z}^{2}E) 
          \end{pmatrix}
          .
        \end{equation}
        By perturbing $\phi(\eta,\bm{x})$ as $\phi(\eta)+\delta\phi(\eta,\bm{x})$\footnote{
          I followed the instructions in the problem statement and omitted the bar on the background field.
        }, we find $\delta T^{0}_{\ 0}$ as 
        \begin{align}
          \delta T^{0}_{\ 0}
          &=
          K(\phi+\delta\phi,X+X_{\phi}\delta\phi)
          +
          g^{0\rho}
          K_{X}(\phi+\delta\phi,X+X_{\phi}\delta\phi)
          \partial_{\rho}\phi\partial_{0}\phi   
          \nonumber
          \\
          &=      
          K+K_{\phi}\delta\phi+K_{X}X_{\phi}\delta\phi
          \nonumber
          \\
          &
          \hspace*{10pt}
          +
          \frac{1}{a(\eta)}
          \left(  
            K_{X}+K_{\phi X}\delta\phi+K_{XX}X_{\phi}\delta\phi              
          \right)
          \left[  
            \frac{2A}{a(\eta)}\phi^{\prime}
            +
            \partial_{i}B\partial_{i}\phi
          \right]
          \phi^{\prime}
          .
        \end{align}
        Note that $g^{\mu\nu}$ has only the perturbative quantity, thus the contribution of its coefficient should be only the background. In the same way, we can evaluate $\delta T^{0}_{\ i}$ and $\delta T^{i}_{\ j}$. Anyway, we conclude that the result, though it is formal, is 
        \begin{graybox}
          \begin{equation}
            \delta T^{\mu}_{\ \nu}
            =
            \delta^{\mu}_{\nu}\left( K+K_{\phi}\delta\phi+K_{X}X_{\phi}\delta\phi \right)
            +
            g^{\mu\rho}
            \left(  
              K_{X}+K_{\phi X}\delta\phi+K_{XX}X_{\phi}\delta\phi              
            \right)
            \partial_{\rho}\phi\partial_{\nu}\phi
            .
          \end{equation}
        \end{graybox}

  \item
        (incomplete)\\
        The perturbed Einstein equation is given by
        \begin{equation}
          \delta G^{\mu}_{\ \nu}
          =
          8\pi G\delta T^{\mu}_{\ \nu}
        \end{equation}
        in Lecture 14. In this case, by putting $K=-\Lambda\ (=\textrm{constant})$, the energy-momentum tensor becomes
        \begin{equation}
          \delta T^{\mu}_{\ \nu}
          =
          -
          \delta^{\mu}_{\nu}\Lambda
          -
          g^{\mu\rho}
          \Lambda
          \partial_{\rho}\phi\partial_{\nu}\phi          
        \end{equation}
        from the previous result. 








\end{enumerate}









% ----------------------------------------
% \appendix
% \section{The computation of the radius of the photon sphere}


% ----------------------------------------
% \clearpage
% \bibliography{ref}
% \bibliographystyle{ytamsalpha}

% ----------------------------------------
% \clearpage
% \index{hoge@hoge}
% \printindex

\end{document}
