\pdfoutput=1
\documentclass[a4paper,pdftex,10pt]{article}
% \usepackage[whole,autotilde]{bxcjkjatype}
\usepackage[T1]{fontenc}
\usepackage{tgtermes}

% ---Display \subsubsection at the Index
\setcounter{tocdepth}{3}

% ---Setting about the geometry of the document----
\usepackage{a4wide}
% \pagestyle{empty}

% ---Physics and Math Packages---
\usepackage{amssymb,amsfonts,amsthm,mathtools}
\usepackage{physics,braket,bm,slashed}

% ---underline---
\usepackage[normalem]{ulem}

% ---cancel---
\usepackage{cancel}

% --- surround the texts or equations
\usepackage{fancybox,ascmac}

% ---settings of theorem environment---
\theoremstyle{definition}
\newtheorem{dfn}{Definition}
\newtheorem{prop}{Proposition}
\newtheorem{thm}{Theorem}

% ---settings of proof environment---
\renewcommand{\proofname}{\textbf{Proof}}
\renewcommand{\qedsymbol}{$\blacksquare$}

% ---Ignore the Warnings---
\usepackage{silence}
\WarningFilter{latexfont}{Some font shapes,Font shape}
\ExplSyntaxOn
\msg_redirect_name:nnn{hooks}{generic-deprecated}{none}
\ExplSyntaxOff

% ---Insert the figure (If insert the `draft' at the option, the process becomes faster.)---
% \usepackage{graphicx}
% \usepackage{subcaption}

% ----Add a link to a text---
\usepackage{url,hyperref}
\usepackage[dvipsnames,svgnames]{xcolor}
\hypersetup{colorlinks=true,citecolor=FireBrick,linkcolor=Navy,urlcolor=purple}

% ---Tikz---
\usepackage{tikz,pgf,pgfplots,circuitikz}
\pgfplotsset{compat=1.15}
\usetikzlibrary{intersections, arrows.meta, angles, calc, 3d, decorations.pathmorphing}
\usepackage[compat=1.1.0]{tikz-feynhand}

% ---tcolorbox---
\usepackage{tcolorbox}
\tcbuselibrary{raster,skins,breakable}
\newtcolorbox{graybox}[1][]{frame empty, colback=black!07!white, sharp corners}

% ---Add the section number to the equation, figure, and table number---
\makeatletter
   \renewcommand{\theequation}{$\thesection.\arabic{equation}$}
   \@addtoreset{equation}{section}
   
   \renewcommand{\thefigure}{\thesection.\arabic{figure}}
   \@addtoreset{figure}{section}
   
   \renewcommand{\thetable}{\thesection.\arabic{table}}
   \@addtoreset{table}{section}
\makeatother

% ---enumerate---
% \renewcommand{\labelenumi}{$\arabic{enumi}.$}
% \renewcommand{\labelenumii}{$(\arabic{enumii})$}

% ---Index---
% \usepackage{makeidx}
% \makeindex 

% ---footnotes---
\renewcommand{\thefootnote}{$\ast$\arabic{footnote}}

% ---Title---
\title{Notes on Anomalies}
\author{Itsuki Miyane}
\date{Last modified:\ \today}

\begin{document}

\maketitle

\tableofcontents

% ----------------------------------------------------------------------------------------------------
\clearpage
\section{Problem 1}

The Christoffel symbol is defined as
\begin{equation}
  \Gamma_{\mu\alpha\beta}
  =
  \frac{1}{2}
  \left(
  \pdv{g_{\mu\alpha}}{x^{\beta}}
  +
  \pdv{g_{\mu\beta}}{x^{\alpha}}
  -
  \pdv{g_{\alpha\beta}}{x^{\mu}}
  \right)
  \label{eqn:christoffel_symbol}
\end{equation}
in a coordinate system $x^{\lambda}$ and transform into a system $\tilde{x}^{\lambda}$\footnote{
  In the problem statement, we should consider the transformation $x^{\lambda}$ into a system "${x^{\prime}}^{\lambda}$" but I should apologize since we will use a different label "$\tilde{x}^{\lambda}$", though it is trivial.
}. The transformation laws of the coordinate are
\begin{align}
  g^{\mu\nu}
   & =
  \pdv{x^{\mu}}{\tilde{x}^{\rho}}\pdv{x^{\nu}}{\tilde{x}^{\sigma}}\tilde{g}^{\rho\sigma}
  \\
  \pdv{}{x^{\mu}}
   & =
  \pdv{\tilde{x}^{\lambda}}{x^{\mu}}\pdv{}{\tilde{x}^{\lambda}}  .
\end{align}
Putting these relations into the definition \eqref{eqn:christoffel_symbol}, we obtain the transformation law
\begin{align}
  \Gamma_{\mu\alpha\beta}
   & =
  \frac{1}{2}
  \left(
  \pdv{g_{\mu\alpha}}{x^{\beta}}
  +
  \pdv{g_{\mu\beta}}{x^{\alpha}}
  -
  \pdv{g_{\alpha\beta}}{x^{\mu}}
  \right)
  \nonumber
  \\
   & =
  \frac{1}{2}
  \left\{
  \pdv{\tilde{x}^{\delta}}{x^{\beta}}\pdv{}{\tilde{x}^{\delta}}
  \left(
  \pdv{\tilde{x}^{\lambda}}{x^{\mu}}\pdv{\tilde{x}^{\gamma}}{x^{\alpha}}g_{\lambda\gamma}
  \right)
  +
  \pdv{\tilde{x}^{\gamma}}{x^{\alpha}}\pdv{}{\tilde{x}^{\gamma}}
  \left(
  \pdv{\tilde{x}^{\lambda}}{x^{\mu}}\pdv{\tilde{x}^{\delta}}{x^{\beta}}g_{\lambda\delta}
  \right)
  -
  \pdv{\tilde{x}^{\lambda}}{x^{\mu}}\pdv{}{\tilde{x}^{\lambda}}
  \left(
  \pdv{\tilde{x}^{\gamma}}{x^{\alpha}}\pdv{\tilde{x}^{\delta}}{x^{\beta}}g_{\gamma\delta}
  \right)
  \nonumber
  \right\}
  \nonumber
  \\
   & =
  \frac{1}{2}
  \left\{
  \pdv{\tilde{x}^{\delta}}{x^{\beta}}\pdv{}{\tilde{x}^{\delta}}
  \left( \pdv{\tilde{x}^{\lambda}}{x^{\mu}} \right)
  \pdv{\tilde{x}^{\gamma}}{x^{\alpha}}g_{\lambda\gamma}
  +
  \pdv{\tilde{x}^{\delta}}{x^{\beta}}
  \pdv{\tilde{x}^{\lambda}}{x^{\mu}}
  \pdv{}{\tilde{x}^{\delta}}
  \left(  \pdv{\tilde{x}^{\gamma}}{x^{\alpha}} \right)
  g_{\lambda\gamma}
  +
  \uwave{
    \pdv{\tilde{x}^{\delta}}{x^{\beta}}
    \pdv{\tilde{x}^{\lambda}}{x^{\mu}}\pdv{\tilde{x}^{\gamma}}{x^{\alpha}}
    \pdv{g_{\lambda\gamma}}{\tilde{x}^{\delta}}
  }
  \right.
  \nonumber
  \\
   & \qquad
  +
  \pdv{\tilde{x}^{\gamma}}{x^{\alpha}}\pdv{}{\tilde{x}^{\gamma}}
  \left(
  \pdv{\tilde{x}^{\lambda}}{x^{\mu}}
  \right)
  \pdv{\tilde{x}^{\delta}}{x^{\beta}}g_{\lambda\delta}
  +
  \pdv{\tilde{x}^{\gamma}}{x^{\alpha}}
  \pdv{\tilde{x}^{\lambda}}{x^{\mu}}
  \pdv{}{\tilde{x}^{\gamma}}
  \left(  \pdv{\tilde{x}^{\delta}}{x^{\beta}}
  \right)
  g_{\lambda\delta}
  +
  \uwave{
    \pdv{\tilde{x}^{\gamma}}{x^{\alpha}}
    \pdv{\tilde{x}^{\lambda}}{x^{\mu}}\pdv{\tilde{x}^{\delta}}{x^{\beta}}
    \pdv{g_{\lambda\delta}}{\tilde{x}^{\gamma}}
  }
  \nonumber
  \\
   & \qquad
  \left.
  -
  \pdv{\tilde{x}^{\lambda}}{x^{\mu}}\pdv{}{\tilde{x}^{\lambda}}
  \left(
  \pdv{\tilde{x}^{\gamma}}{x^{\alpha}}
  \right)
  \pdv{\tilde{x}^{\delta}}{x^{\beta}}g_{\gamma\delta}
  -
  \pdv{\tilde{x}^{\lambda}}{x^{\mu}}\pdv{\tilde{x}^{\gamma}}{x^{\alpha}}
  \pdv{}{\tilde{x}^{\lambda}}
  \left(
  \pdv{\tilde{x}^{\delta}}{x^{\beta}}
  \right)
  g_{\gamma\delta}
  -
  \uwave{
    \pdv{\tilde{x}^{\lambda}}{x^{\mu}}
    \pdv{\tilde{x}^{\gamma}}{x^{\alpha}}\pdv{\tilde{x}^{\delta}}{x^{\beta}}
    \pdv{g_{\gamma\delta}}{\tilde{x}^{\lambda}}
  }
  \right\}
  .
  \label{eqn:1}
\end{align}
The waved terms $\uwave{\qquad}$ in \eqref{eqn:1} become a twice of the Christoffel symbol:
\begin{equation}
  \pdv{\tilde{x}^{\delta}}{x^{\beta}}
  \pdv{\tilde{x}^{\lambda}}{x^{\mu}}\pdv{\tilde{x}^{\gamma}}{x^{\alpha}}
  \pdv{g_{\lambda\gamma}}{\tilde{x}^{\delta}}
  +
  \pdv{\tilde{x}^{\gamma}}{x^{\alpha}}
  \pdv{\tilde{x}^{\lambda}}{x^{\mu}}\pdv{\tilde{x}^{\delta}}{x^{\beta}}
  \pdv{g_{\lambda\delta}}{\tilde{x}^{\gamma}}
  -
  \pdv{\tilde{x}^{\lambda}}{x^{\mu}}
  \pdv{\tilde{x}^{\gamma}}{x^{\alpha}}\pdv{\tilde{x}^{\delta}}{x^{\beta}}
  \pdv{g_{\gamma\delta}}{\tilde{x}^{\lambda}}
  =
  2
  \pdv{\tilde{x}^{\lambda}}{x^{\mu}}
  \pdv{\tilde{x}^{\gamma}}{x^{\alpha}}\pdv{\tilde{x}^{\delta}}{x^{\beta}}
  \tilde{\Gamma}_{\lambda\gamma\delta}
  .
\end{equation}
But we already find that there are additional terms in \eqref{eqn:1} and they violate the transformation law of the 3-rank covariant tensor. Thus we find the law as
\begin{align}
  \Gamma_{\mu\alpha\beta}
   & =
  \pdv{\tilde{x}^{\lambda}}{x^{\mu}}
  \pdv{\tilde{x}^{\gamma}}{x^{\alpha}}\pdv{\tilde{x}^{\delta}}{x^{\beta}}
  \tilde{\Gamma}_{\lambda\gamma\delta}
  +
  \frac{1}{2}
  \left\{
  \pdv{\tilde{x}^{\lambda}}{x^{\beta}}{x^{\mu}}\pdv{\tilde{x}^{\gamma}}{x^{\alpha}}g_{\lambda\gamma}
  +
  \pdv{\tilde{x}^{\gamma}}{x^{\beta}}{x^{\alpha}}\pdv{\tilde{x}^{\lambda}}{x^{\mu}}g_{\lambda\gamma}
  \right.
  \nonumber
  \\
   & \qquad
  \left.
  +
  \pdv{\tilde{x}^{\lambda}}{x^{\alpha}}{x^{\mu}}\pdv{\tilde{x}^{\delta}}{x^{\beta}}g_{\lambda\delta}
  +
  \pdv{\tilde{x}^{\delta}}{x^{\alpha}}{x^{\beta}}\pdv{\tilde{x}^{\lambda}}{x^{\mu}}g_{\lambda\delta}
  -
  \pdv{\tilde{x}^{\gamma}}{x^{\mu}x^{\alpha}}\pdv{\tilde{x}^{\delta}}{x^{\beta}}g_{\gamma\delta}
  -
  \pdv{\tilde{x}^{\delta}}{x^{\mu}}{x^{\beta}}\pdv{\tilde{x}^{\gamma}}{x^{\alpha}}g_{\delta\gamma}
  \right\}
  \nonumber
  \\
  &=
  \pdv{\tilde{x}^{\lambda}}{x^{\mu}}
  \pdv{\tilde{x}^{\gamma}}{x^{\alpha}}\pdv{\tilde{x}^{\delta}}{x^{\beta}}
  \tilde{\Gamma}_{\lambda\gamma\delta}
  +
  \pdv{\tilde{x}^{\lambda}}{x^{\mu}}\pdv{\tilde{x}^{\gamma}}{x^{\alpha}}{x^{\beta}}g_{\lambda\gamma}
\end{align}
Note we freely change the dummy indices and cancel out equivalent terms.




% ----------------------------------------------------------------------------------------------------
\clearpage
\section{Problem 2}

From the previous problem, we obtain the transformation law
\begin{equation}
  \tilde{\Gamma}^{\lambda}_{\ \mu\nu}
  =
  \pdv{\tilde{x}^{\lambda}}{x^{\gamma}}
  \pdv{x^{\alpha}}{\tilde{x}^{\mu}}
  \pdv{x^{\beta}}{\tilde{x}^{\nu}}
  \Gamma^{\gamma}_{\ \alpha\beta}
  +
  \pdv{\tilde{x}^{\lambda}}{x^{\gamma}}\pdv{x^{\gamma}}{\tilde{x}^{\mu}}{\tilde{x}^{\nu}}
  .
\end{equation}
Since we already know the transformation laws, what we have to do is just to compute them. Thus the transformation should be
\begin{align}
  \tilde{\nabla}_{\mu}\tilde{V}_{\nu}
   & \equiv
  \pdv{\tilde{V}_{\nu}}{\tilde{x}^{\mu}}
  -
  \tilde{\Gamma}^{\lambda}_{\ \mu\nu}\tilde{V}_{\lambda}
  \nonumber
  \\
   & =
  \pdv{x^{\alpha}}{\tilde{x}^{\mu}}\pdv{}{x^{\alpha}}\left( \pdv{x^{\beta}}{\tilde{x}^{\nu}}V_{\beta} \right)
  -
  \left(
  \pdv{\tilde{x}^{\lambda}}{x^{\gamma}}
  \pdv{x^{\alpha}}{\tilde{x}^{\mu}}
  \pdv{x^{\beta}}{\tilde{x}^{\nu}}
  \Gamma^{\gamma}_{\ \alpha\beta}
  +
  \pdv{\tilde{x}^{\lambda}}{x^{\gamma}}\pdv{x^{\gamma}}{\tilde{x}^{\mu}}{\tilde{x}^{\nu}}
  \right)
  \cdot
  \pdv{x^{\delta}}{\tilde{x}^{\lambda}}V_{\delta}
  \nonumber
  \\
   & =
  \cancel{
  \pdv{x^{\beta}}{\tilde{x}^{\mu}}{\tilde{x}^{\nu}}V_{\beta}
  }
  +
  \pdv{x^{\alpha}}{\tilde{x}^{\mu}}\pdv{x^{\beta}}{\tilde{x}^{\nu}}\pdv{V_{\beta}}{x^{\alpha}}
  -
  \pdv{\tilde{x}^{\lambda}}{x^{\gamma}}
  \pdv{x^{\alpha}}{\tilde{x}^{\mu}}
  \pdv{x^{\beta}}{\tilde{x}^{\nu}}
  \Gamma^{\gamma}_{\ \alpha\beta}
  \cdot
  \pdv{x^{\delta}}{\tilde{x}^{\lambda}}V_{\delta}
  -
  \cancel{
  \pdv{\tilde{x}^{\lambda}}{x^{\gamma}}\pdv{x^{\gamma}}{\tilde{x}^{\mu}}{\tilde{x}^{\nu}}
  \cdot
  \pdv{x^{\delta}}{\tilde{x}^{\lambda}}V_{\delta}
  }
  \nonumber
  \\
   & =
  \pdv{x^{\alpha}}{\tilde{x}^{\mu}}\pdv{x^{\beta}}{\tilde{x}^{\nu}}
  \left(
  \pdv{V_{\beta}}{x^{\alpha}}
  -
  \Gamma^{\gamma}_{\ \alpha\beta}V_{\gamma}
  \right)
  =
  \pdv{x^{\alpha}}{\tilde{x}^{\mu}}\pdv{x^{\beta}}{\tilde{x}^{\nu}}\nabla_{\alpha}V_{\beta}
\end{align}
and we finally attain
\begin{equation}
  \tilde{\nabla}_{\mu}\tilde{V}_{\nu}
  =
  \pdv{x^{\alpha}}{\tilde{x}^{\mu}}\pdv{x^{\beta}}{\tilde{x}^{\nu}}\nabla_{\alpha}V_{\beta}
  .
\end{equation}



% ----------------------------------------------------------------------------------------------------
\clearpage
\section{Problem 3}

The definition of the covariant derivative of two-rank tensor and as followings:
\begin{gather}
  \nabla_{\lambda} T_{\mu\nu}
  =
  \pdv{T_{\mu\nu}}{x^{\lambda}}
  -
  \Gamma^{\rho}_{\ \mu\lambda}T_{\rho\nu}
  -
  \Gamma^{\rho}_{\ \nu\lambda}T_{\mu\rho}
  ,
  \\
  R^{\mu}_{\ \nu\lambda\rho}
  =
  \pdv{}{x^{\lambda}}
  \Gamma^{\mu}_{\ \nu\rho}
  -
  \pdv{}{x^{\rho}}
  \Gamma^{\mu}_{\ \nu\lambda}
  +
  \Gamma^{\mu}_{\ \alpha\lambda}\Gamma^{\alpha}_{\ \nu\rho}
  -
  \Gamma^{\mu}_{\ \alpha\rho}\Gamma^{\alpha}_{\ \nu\lambda}
  .
\end{gather}
We already showed that $\nabla_{\lambda}A_{\sigma}$ is the 2-rank covariant tensor in the previous problem. Thus we should find $[\nabla_{\mu},\nabla_{\nu}]T_{\lambda\rho}$ for a general two-rank tensor $T_{\mu\nu}$. Let us compute $\nabla_{\mu}\nabla_{\nu}T_{\lambda\rho}$ first. It is
\begin{align}
  \nabla_{\mu}\nabla_{\nu}T_{\lambda\rho}
  &=
  \pdv{}{x^{\mu}}
  \left(  \nabla_{\nu}T_{\lambda\rho} \right)
  -
  \textcolor{gray}{\Gamma^{\alpha}_{\ \nu\mu}(\nabla_{\delta}T_{\lambda\rho})}
  -
  \Gamma^{\sigma}_{\ \lambda\mu}(\nabla_{\nu}T_{\sigma\rho})
  -
  \Gamma^{\sigma}_{\ \rho\mu}(\nabla_{\nu}T_{\lambda\sigma})
  \nonumber
  \\
  &=
  \textcolor{gray}{\pdv{T_{\lambda\rho}}{x^{\mu}}{x^{\nu}}}
  -
  \pdv{\Gamma^{\sigma}_{\ \lambda\nu}}{x^{\mu}}T_{\sigma\rho}
  -
  \Gamma^{\sigma}_{\ \lambda\nu}\pdv{T_{\sigma\rho}}{x^{\mu}}
  -
  \pdv{\Gamma^{\sigma}_{\ \rho\nu}}{x^{\mu}}T_{\lambda\sigma}
  -
  \Gamma^{\sigma}_{\ \rho\nu}\pdv{T_{\lambda\sigma}}{x^{\mu}}
  -
  \textcolor{gray}{\Gamma^{\alpha}_{\ \nu\mu}(\nabla_{\delta}T_{\lambda\rho})}
  \nonumber
  \\
  &\quad
  -
  \Gamma^{\sigma}_{\ \lambda\mu}\pdv{T_{\sigma\rho}}{x^{\nu}}
  +
  \Gamma^{\sigma}_{\ \lambda\mu}\Gamma^{\alpha}_{\ \sigma\nu}T_{\alpha\rho}
  +
  \Gamma^{\sigma}_{\ \lambda\mu}\Gamma^{\alpha}_{\ \rho\nu}T_{\sigma\alpha}
  \nonumber
  \\
  &\qquad
  -
  \Gamma^{\sigma}_{\ \rho\mu}\pdv{T_{\lambda\sigma}}{x^{\nu}}
  +
  \Gamma^{\sigma}_{\ \rho\mu}\Gamma^{\alpha}_{\ \lambda\nu}T_{\alpha\sigma}
  +
  \Gamma^{\sigma}_{\ \rho\mu}\Gamma^{\alpha}_{\ \sigma\nu}T_{\lambda\alpha}
  .
  \nonumber
\end{align}
Note that grayed terms vanish from contributions of $\nabla_{\nu}\nabla_{\mu}T_{\lambda\rho}$. When we evaluate $\nabla_{\mu}\nabla_{\nu}T_{\lambda\rho}$, we just need to flip the indices as $\mu\leftrightarrow\nu$ and we get 
\begin{align}
  [\nabla_{\mu},\nabla_{\nu}]T_{\lambda\rho}
  &=
  -
  \uline{
  \pdv{\Gamma^{\sigma}_{\ \lambda\nu}}{x^{\mu}}T_{\sigma\rho}
  }
  -
  \cancel{
  \Gamma^{\sigma}_{\ \lambda\nu}\pdv{T_{\sigma\rho}}{x^{\mu}}
  }
  -
  \uwave{
  \pdv{\Gamma^{\sigma}_{\ \rho\nu}}{x^{\mu}}T_{\lambda\sigma}
  }
  -
  \cancel{
  \Gamma^{\sigma}_{\ \rho\nu}\pdv{T_{\lambda\sigma}}{x^{\mu}}
  }
  -
  \cancel{
  \Gamma^{\sigma}_{\ \lambda\mu}\pdv{T_{\sigma\rho}}{x^{\nu}}
  }
  \nonumber
  \\
  &\qquad
  +
  \uline{
  \Gamma^{\sigma}_{\ \lambda\mu}\Gamma^{\alpha}_{\ \sigma\nu}T_{\alpha\rho}
  }
  +
  \textcolor{DarkGreen}{
  \Gamma^{\sigma}_{\ \lambda\mu}\Gamma^{\alpha}_{\ \rho\nu}T_{\sigma\alpha}
  }
  -
  \cancel{
  \Gamma^{\sigma}_{\ \rho\mu}\pdv{T_{\lambda\sigma}}{x^{\nu}}
  }
  +
  \textcolor{DarkRed}{
  \Gamma^{\sigma}_{\ \rho\mu}\Gamma^{\alpha}_{\ \lambda\nu}T_{\alpha\sigma}
  }
  +
  \uwave{
  \Gamma^{\sigma}_{\ \rho\mu}\Gamma^{\alpha}_{\ \sigma\nu}T_{\lambda\alpha}
  }
  \nonumber
  \\
  &\qquad
  +
  \uline{
  \pdv{\Gamma^{\sigma}_{\ \lambda\mu}}{x^{\nu}}T_{\sigma\rho}
  }
  +
  \cancel{
  \Gamma^{\sigma}_{\ \lambda\mu}\pdv{T_{\sigma\rho}}{x^{\nu}}
  }
  +
  \uwave{
  \pdv{\Gamma^{\sigma}_{\ \rho\mu}}{x^{\nu}}T_{\lambda\sigma}
  }
  +
  \cancel{
  \Gamma^{\sigma}_{\ \rho\mu}\pdv{T_{\lambda\sigma}}{x^{\nu}}
  }
  +
  \cancel{
  \Gamma^{\sigma}_{\ \lambda\nu}\pdv{T_{\sigma\rho}}{x^{\mu}}
  }
  \nonumber
  \\
  &\qquad
  -
  \uline{
  \Gamma^{\sigma}_{\ \lambda\nu}\Gamma^{\alpha}_{\ \sigma\mu}T_{\alpha\rho}
  }
  -
  \textcolor{DarkRed}{
  \Gamma^{\sigma}_{\ \lambda\nu}\Gamma^{\alpha}_{\ \rho\mu}T_{\sigma\alpha}
  }
  +
  \cancel{
  \Gamma^{\sigma}_{\ \rho\nu}\pdv{T_{\lambda\sigma}}{x^{\mu}}
  }
  -
  \textcolor{DarkGreen}{
  \Gamma^{\sigma}_{\ \rho\nu}\Gamma^{\alpha}_{\ \lambda\mu}T_{\alpha\sigma}
  }
  -
  \uwave{
  \Gamma^{\sigma}_{\ \rho\nu}\Gamma^{\alpha}_{\ \sigma\mu}T_{\lambda\alpha}
  }
  \nonumber
  \\
  &=
  \uline{
    \left(  
      \pdv{\Gamma^{\sigma}_{\ \lambda\mu}}{x^{\nu}}
      -
      \pdv{\Gamma^{\sigma}_{\ \lambda\nu}}{x^{\mu}}
      +
      \Gamma^{\alpha}_{\ \lambda\mu}\Gamma^{\sigma}_{\ \alpha\nu}
      +
      \Gamma^{\alpha}_{\ \lambda\nu}\Gamma^{\sigma}_{\ \alpha\mu}
    \right)
    T_{\sigma\rho}
  }
  \nonumber
  \\
  &\qquad
  +
  \uwave{
    \left(
      \pdv{\Gamma^{\sigma}_{\ \rho\mu}}{x^{\nu}}
      -
      \pdv{\Gamma^{\sigma}_{\ \rho\nu}}{x^{\mu}}
      +
      \Gamma^{\alpha}_{\ \rho\mu}\Gamma^{\sigma}_{\ \alpha\nu}
      +
      \Gamma^{\alpha}_{\ \rho\nu}\Gamma^{\sigma}_{\ \alpha\mu}
    \right)
    T_{\lambda\sigma}
  }
  \nonumber
  \\
  &=R^{\sigma}_{\ \lambda\nu\mu}T_{\sigma\rho}+R^{\sigma}_{\ \rho\nu\mu}T_{\lambda\sigma}
  .
\end{align}
By substituting $T_{\lambda\rho}$ with $\nabla_{\lambda}A_{\rho}$, the required result realized.


% ----------------------------------------------------------------------------------------------------
\clearpage
\section{Problem 4}

Let us show the following formula\footnote{
  The exponential of matrix $A$ satisfies
  \begin{equation}
    \det e^{A}=e^{\Tr A}
    \nonumber
  \end{equation}
  and taking $A\equiv\ln g$, we find \eqref{eqn:4-1}.
}:
\begin{equation}
  \pdv{}{x^{\mu}}\sqrt{-g}
  =
  \sqrt{-g}\Gamma^{\lambda}_{\ \mu\lambda}
  .
  \label{eqn:4-1}
\end{equation}
Assuming this relation, we immediately reach the answer
\begin{align}
  \frac{1}{\sqrt{-g}}\pdv{}{x^{\lambda}}\left(\sqrt{-g}A^{\lambda}\right)
   & =
  \frac{1}{\sqrt{-g}}\left( \pdv{}{x^{\lambda}}\sqrt{-g} \right)A^{\lambda}
  +
  \pdv{A^{\lambda}}{x^{\lambda}}
  \nonumber
  \\
   & =
  \Gamma^{\sigma}_{\ \lambda\sigma}A^{\kappa}
  +
  \pdv{A^{\lambda}}{x^{\lambda}}
  =
  \nabla_{\lambda}A^{\lambda}
  .
\end{align}
So what is left is to prove the relation \eqref{eqn:4-1}.
\begin{proof}
  We will use the relation
  \begin{equation}
    \ln g
    =
    \Tr \ln g
    \label{eqn:4-2}
  \end{equation}
  and take the derivative to $x^{\lambda}$. Then we get
  \begin{equation}
    \pdv{}{x^{\lambda}}\ln g
    =
    \pdv{}{x^{\lambda}}\Tr\ln g
    \label{eqn:4-3}
  \end{equation}
  and by computing both sides carefully, we can obtain
  \begin{equation}
    \frac{1}{g}\pdv{g}{x^{\lambda}}
    =
    g^{\alpha\beta}\pdv{g_{\alpha\beta}}{x^{\lambda}}
    .
    \label{eqn:4-4}
  \end{equation}
  Note that we use the relation
  \begin{equation}
    \pdv{}{x^{\lambda}}\ln g_{\alpha\beta}
    =
    (g^{-1})_{\alpha\gamma}\pdv{g_{\gamma\beta}}{x^{\lambda}}
  \end{equation}
  when we derivate the RHS of \eqref{eqn:4-3}. Thus we find
  \begin{align}
    \pdv{}{x^{\lambda}}\sqrt{-g}
     & =
    -\pdv{g}{x^{\lambda}}
    \cdot
    \frac{1}{2}\frac{1}{\sqrt{-g}}
    \nonumber
    \\
     & =
    \frac{1}{2}\sqrt{-g}g^{\alpha\beta}\pdv{g_{\alpha\beta}}{x^{\lambda}}
  \end{align}
  by using \eqref{eqn:4-4}. On the other hand, the Christoffel symbol is defined as \eqref{eqn:christoffel_symbol} and contracting the indices, the relation holds
  \begin{equation}
    \Gamma^{\lambda}_{\ \mu\lambda}
    =
    \frac{1}{2}g^{\alpha\beta}\pdv{g_{\alpha\beta}}{x^{\lambda}}
  \end{equation}
  and finally we attain \eqref{eqn:4-1}.
\end{proof}



% ----------------------------------------
% \clearpage
% \appendix
% \section{Notes}

% ----------------------------------------
% \clearpage
% \bibliography{ref}
% \bibliographystyle{ytamsalpha}

% ----------------------------------------
% \clearpage
% \index{hoge@hoge}
% \printindex

\end{document}
