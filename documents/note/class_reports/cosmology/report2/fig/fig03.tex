\documentclass{standalone}

\usepackage{graphics}
\usepackage[dvipsnames,svgnames]{xcolor}

\usepackage{tikz,pgf,pgfplots,circuitikz}
\pgfplotsset{compat=1.15}
\usetikzlibrary{shapes.symbols,intersections,arrows.meta,angles,calc,3d,decorations.pathmorphing}
\usepackage[compat=1.1.0]{tikz-feynhand}

\usepackage{amssymb,amsfonts,amsthm,mathtools}
\usepackage{physics,braket,bm}

\begin{document}

\begin{tikzpicture}
  % Define the function parameters
  \def\Hzero{1/145}
  \def\w{-1.2}
  \def\tzero{145}
  \def\tscale{12}
  \def\func(#1){2 * \Hzero / (3 * (\Hzero * (\w + 1) * (#1 - \tzero) + 1))}

  % Draw axes
  \draw[-latex] (500/\tscale/10,0) -- (1200/\tscale/10,0) node[right] {$t$};
  \draw[-latex] (600/\tscale/10,-1.2) -- (600/\tscale/10,2) node[above] {$H(t)$};
  \draw[dashed] (870/\tscale/10,-1.2) -- (870/\tscale/10,2);
  \draw (850/\tscale/10,-0.30) node {$t'$};
  
  % Plot the function
  \draw[scale=0.1,domain=500:869,smooth,variable=\t] plot ({\t/\tscale},{\func(\t)*20});
  \draw[scale=0.1,domain=879:1000,smooth,variable=\t] plot ({\t/\tscale},{\func(\t)*20});

\end{tikzpicture}

\end{document}
