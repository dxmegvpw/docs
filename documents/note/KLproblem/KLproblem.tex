\documentclass[unicode,a4paper,10pt]{ltjsarticle}
% ---fonts---
\PassOptionsToPackage{quiet}{fontspec}
\usepackage{luatexja-fontspec}
\setmainfont{TeX Gyre Termes}
\setmainjfont[BoldFont = IPAGothic]{IPAMincho}
% \setmainjfont{Noto Sans CJK JP}
\setmathrm{Latin Modern Roman}

% ---Display \subsubsection at the Index
% \setcounter{tocdepth}{3}

% ---Setting about the geometry of the document----
% \usepackage{a4wide}
% \pagestyle{empty}

% ---Physics and Math Packages---
\usepackage{amssymb,amsfonts,amsthm,mathtools}
\usepackage{physics,braket,bm}

% ---underline---
\usepackage[normalem]{ulem}

% ---cancel---
\usepackage{cancel}

% --- surround the texts or equations
% \usepackage{fancybox,ascmac}

% ---settings of theorem environment---
\theoremstyle{definition}
\newtheorem{dfn}{定義}
\newtheorem{prop}{命題}
\newtheorem{thm}{定理}
\newtheorem{exm}{例}
\newtheorem{exc}{演習}

% ---settings of proof environment---
\renewcommand{\proofname}{\textbf{証明}}
\renewcommand{\qedsymbol}{$\blacksquare$}

% ---Ignore the Warnings---
\usepackage{silence}
\WarningFilter{latexfont}{Some font shapes}
\WarningFilter{latexfont}{Font shape}
\WarningFilter{latexfont}{Size substitutions}
\ExplSyntaxOn
\msg_redirect_name:nnn{hooks}{generic-deprecated}{none}
\ExplSyntaxOff

% ---Insert the figure (If insert the `draft' at the option, the process becomes faster.)---
\usepackage{graphicx}
% \usepackage{subcaption}

% ----Add a link to a text---
\usepackage{url,hyperref}
\usepackage[dvipsnames,svgnames]{xcolor}
\hypersetup{colorlinks=true,citecolor=FireBrick,linkcolor=Navy,urlcolor=purple}
% ---refer `texdoc xcolor' at the command line---

% ---Tikz---
% \usepackage{tikz,pgf,pgfplots,circuitikz}
% \pgfplotsset{compat=1.15}
% \usetikzlibrary{intersections,arrows.meta,angles,calc,3d,decorations.pathmorphing}

% ---Add the section number to the equation, figure, and table number---
\makeatletter
   \renewcommand{\theequation}{$\thesection.\arabic{equation}$}
   \@addtoreset{equation}{section}
   
   \renewcommand{\thefigure}{\thesection.\arabic{figure}}
   \@addtoreset{figure}{section}
   
   \renewcommand{\thetable}{\thesection.\arabic{table}}
   \@addtoreset{table}{section}
\makeatother

% ---enumerate---
% \renewcommand{\labelenumi}{$\arabic{enumi}.$}
% \renewcommand{\labelenumii}{$(\arabic{enumii})$}

% ---Index---
% \usepackage{makeidx}
% \makeindex 

% ---Title---
\title{
  title
}
\author{
  author
}
\date{最終更新:\today}

\begin{document}

\maketitle

\section*{KL problemのノート}

KL problemの簡単にまとめたい。

この話題自体がそんなに自分の研究に関係あるわけではないが、どうやら手法が使えそうということなので、軽く読んでみた。そのメモ。

本当はオリジナルを読むべきだが、調べてたら\href{https://doi.org/10.48550/arXiv.1211.1455}{山田さんの文章}があったので、そこのレビューの部分を読んだ。


\subsection*{KL problemとは} 

KKLT模型では
$$
    \begin{cases}
        W
        =
        w_{0}+Ae^{-aT}
        \\
        K
        =
        -3\ln(T+\bar{T})
    \end{cases}
$$
と超ポテンシャルとKählerポテンシャルが与えられ、これらからなるスカラーポテンシャルは
$$
    V
    =
    e^{K}(K^{I\bar{J}}D_{I}WD_{\bar{J}}\bar{W}-3|W|^2)
$$
である。

KKLT模型のVEVは、SUSYを保ってかつ$V_{\mathrm{min}}<0$という性質がある。実際に実験との整合性云々から、SUSYを破る項を入れて$V_{\mathrm{min}}=\mathcal{O}(120)$くらいまでupliftしなければならない。

そう考えて、インフラトン場$\phi$を導入して、それを込みのポテンシャルを考えると$V_{\mathrm{min}}$の値が大きくなりすぎて、障壁をこえて$\sigma\equiv\mathrm{Re}T$が$\mathrm{Re}T\rightarrow\infty$で$V_{\mathrm{min}}\rightarrow0$とランウェイしてしまう。これを回避するためにはハッブルパラメターが$H<m_{3/2}$を満たしていなければならず、これが\textbf{KL problem}らしい\footnote{
  ハッブル定数の問題については、良く分からない。ゲージーノmassよりも小さいのは良くないのだろうか。
}。

\subsection*{KalloshとLindeの仕事}

KL problemに対して、\href{https://arxiv.org/abs/hep-th/0411011}{KalloshとLindeは解決法を見出した}という。それは次のような超ポテンシャル
\begin{gather*}
    W_{\mathrm{KL}}
    =
    w_{0}
    +
    Ae^{-aT}
    -
    Be^{-bT}
    \\
    w_{0}
    \equiv
    B\left(\frac{aA}{bB}\right)^{b/b-a}
    +
    A\left(\frac{aA}{bB}\right)^{a/b-a}
\end{gather*}
を考えることにある。この超ポテンシャルを考えてもやはり最小となるのはKKLTと同じで$D_{T}W=0$でSUSYを破らないが、$w_{0}$の値を調整しているおかげで、$W=0$となり、さらにMinkowski vacuumを実現している。

計算したMathematicaのコードは\href{/docs/documents/note/study_notes/short_memos/notebook/kahler_meson_polonyi.nb}{ここ}。値は\href{https://arxiv.org/abs/hep-th/0411011}{オリジナルの}を参考にしている。










% ----------------------------------------
% \clearpage

% \makeatletter
% \renewcommand{\appendix}{\par
%   \setcounter{section}{0}%
%   \setcounter{subsection}{0}%
%   \gdef\presectionname{\appendixname}%
%   \gdef\postsectionname{}%
%   \gdef\thesection{\presectionname\@Alph\c@section\postsectionname}%
%   \gdef\thesubsection{\@Alph\c@section.\@arabic\c@subsection}%
%   \renewcommand{\theequation}{\@Alph\c@section.\arabic{equation}}%
%   \renewcommand{\thefigure}{\@Alph\c@section.\arabic{figure}}%
%   \renewcommand{\thetable}{\@Alph\c@section.\arabic{table}}%
% }
% \makeatother

% \appendix

% \section{Notes}


% ----------------------------------------
% \clearpage
% \bibliography{ref}
% \bibliographystyle{ytamsalpha}


% ----------------------------------------
% \clearpage
% \index{hoge@hoge}
% \printindex


\end{document}
