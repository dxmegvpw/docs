\documentclass[unicode,a4paper,10pt]{ltjsarticle}
% ---fonts---
\PassOptionsToPackage{quiet}{fontspec}
\usepackage{luatexja-fontspec}
\setmainfont{TeX Gyre Termes}
\setmainjfont[BoldFont = IPAGothic]{IPAMincho}
% \setmainjfont{Noto Sans CJK JP}
\setmathrm{Latin Modern Roman}

% ---Display \subsubsection at the Index
% \setcounter{tocdepth}{3}

% ---Setting about the geometry of the document----
% \usepackage{a4wide}
% \pagestyle{empty}

% ---Physics and Math Packages---
\usepackage{amssymb,amsfonts,amsthm,mathtools}
\usepackage{physics,braket,bm}

% ---underline---
\usepackage[normalem]{ulem}

% ---cancel---
\usepackage{cancel}

% --- surround the texts or equations
% \usepackage{fancybox,ascmac}

% ---settings of theorem environment---
\theoremstyle{definition}
\newtheorem{dfn}{定義}
\newtheorem{prop}{命題}
\newtheorem{thm}{定理}
\newtheorem{exm}{例}
\newtheorem{exc}{演習}

% ---settings of proof environment---
\renewcommand{\proofname}{\textbf{証明}}
\renewcommand{\qedsymbol}{$\blacksquare$}

% ---Ignore the Warnings---
\usepackage{silence}
\WarningFilter{latexfont}{Some font shapes}
\WarningFilter{latexfont}{Font shape}
\WarningFilter{latexfont}{Size substitutions}
\ExplSyntaxOn
\msg_redirect_name:nnn{hooks}{generic-deprecated}{none}
\ExplSyntaxOff

% ---Insert the figure (If insert the `draft' at the option, the process becomes faster.)---
\usepackage{graphicx}
% \usepackage{subcaption}

% ----Add a link to a text---
\usepackage{url,hyperref}
\usepackage[dvipsnames,svgnames]{xcolor}
\hypersetup{colorlinks=true,citecolor=FireBrick,linkcolor=Navy,urlcolor=purple}
% ---refer `texdoc xcolor' at the command line---

% ---Tikz---
% \usepackage{tikz,pgf,pgfplots,circuitikz}
% \pgfplotsset{compat=1.15}
% \usetikzlibrary{intersections,arrows.meta,angles,calc,3d,decorations.pathmorphing}

% ---Add the section number to the equation, figure, and table number---
\makeatletter
   \renewcommand{\theequation}{$\thesection.\arabic{equation}$}
   \@addtoreset{equation}{section}
   
   \renewcommand{\thefigure}{\thesection.\arabic{figure}}
   \@addtoreset{figure}{section}
   
   \renewcommand{\thetable}{\thesection.\arabic{table}}
   \@addtoreset{table}{section}
\makeatother

% ---enumerate---
% \renewcommand{\labelenumi}{$\arabic{enumi}.$}
% \renewcommand{\labelenumii}{$(\arabic{enumii})$}

% ---Index---
% \usepackage{makeidx}
% \makeindex 

% ---Title---
\title{
  title
}
\author{
  author
}
\date{最終更新:\today}

\begin{document}

\maketitle
\tableofcontents

\clearpage
\section{はじめに}



\clearpage
\section{ゲージ理論の古典論}

\subsection{相対論的な場の理論とローレンツ群}

相対論的な場の理論を構成するためには、ローレンツ変換に対して共変的な場を用意しておくと見通しがよい。特に、ローレンツ群の代数の表現を調べておけば、そのような場を構成することができる。この節では、そういった観点から場の理論を構成する。

ローレンツ群とは、次のような内積
\begin{equation}
  A^{\mu}B_{\mu}
  \equiv
  -
  A^{0}B^{0}
  +
  A^{1}B^{1}
  +
  A^{2}B^{2}
  +
  A^{3}B^{3}
\end{equation}
を保存するような変換$A^{\mu}\rightarrow\Lambda^{\mu}_{\nu}A^{\nu}$のなす群のことであり、$SO(3,1)$と表すこととする。

















% ----------------------------------------
% \clearpage

% \makeatletter
% \renewcommand{\appendix}{\par
%   \setcounter{section}{0}%
%   \setcounter{subsection}{0}%
%   \gdef\presectionname{\appendixname}%
%   \gdef\postsectionname{}%
%   \gdef\thesection{\presectionname\@Alph\c@section\postsectionname}%
%   \gdef\thesubsection{\@Alph\c@section.\@arabic\c@subsection}%
%   \renewcommand{\theequation}{\@Alph\c@section.\arabic{equation}}%
%   \renewcommand{\thefigure}{\@Alph\c@section.\arabic{figure}}%
%   \renewcommand{\thetable}{\@Alph\c@section.\arabic{table}}%
% }
% \makeatother

% \appendix

% \section{Notes}


% ----------------------------------------
\clearpage
\bibliography{ref}
\bibliographystyle{ytphys}

\nocite{Peskin:1995}
\nocite{Sato:2016}
\nocite{Mogi:2001}

% ----------------------------------------
% \clearpage
% \index{hoge@hoge}
% \printindex


\end{document}
