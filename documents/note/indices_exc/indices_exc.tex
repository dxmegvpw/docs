\pdfoutput=1
\documentclass[a4paper,pdftex]{article}
% \usepackage[whole]{bxcjkjatype}
\usepackage[T1]{fontenc}
\usepackage{tgtermes}

% ---Display \subsubsection at the Index
% \setcounter{tocdepth}{3}

% ---Setting about the geometry of the document----
\usepackage{a4wide}
% \pagestyle{empty}

% ---Physics and Math Packages---
\usepackage{amssymb,amsfonts,amsthm,mathtools}
\usepackage{physics,braket,bm}

% ---underline---
\usepackage{ulem}

% ---cancel---
\usepackage{cancel}

% --- surround the texts or equations
% \usepackage{fancybox,ascmac}

% ---settings of theorem environment---
% \usepackage{amsthm}
% \theoremstyle{definition}

% ---settings of proof environment---
% \renewcommand{\proofname}{\textbf{証明}}
% \renewcommand{\qedsymbol}{$\blacksquare$}

% ---Ignore the Warnings---
\usepackage{silence}
\WarningFilter{latexfont}{Some font shapes,Font shape}

% ---Insert the figure (If insert the `draft' at the option, the process becomes faster.)---
% \usepackage{graphicx}
% \usepackage{subcaption}

% ----Add a link to a text---
\usepackage{url}
\usepackage{xcolor,hyperref}
\hypersetup{colorlinks=true,citecolor=orange,linkcolor=blue,urlcolor=magenta}

% ---Tikz---
% \usepackage{tikz,pgf,pgfplots,circuitikz}
% \pgfplotsset{compat=1.15}
% \usetikzlibrary{intersections,arrows.meta,angles,calc,3d,decorations.pathmorphing}

% ---Add the section number to the equation, figure, and table number---
\makeatletter
   \renewcommand{\theequation}{\thesection.\arabic{equation}}
   \@addtoreset{equation}{section}
   
   \renewcommand{\thefigure}{\thesection.\arabic{figure}}
   \@addtoreset{figure}{section}
   
   \renewcommand{\thetable}{\thesection.\arabic{table}}
   \@addtoreset{table}{section}
\makeatother

% ---enumerate---
% \renewcommand{\labelenumi}{$\arabic{enumi}.$}
% \renewcommand{\labelenumii}{$(\arabic{enumii})$}

% ---Index---
% \usepackage{makeidx}
% \makeindex 

% ---Fonts---
% \renewcommand{\familydefault}{\sfdefault}

% ---Title---
\title{Exercise for messy indices!}
\author{I. Miyane}
\date{Last modified:\ \today}

\begin{document}

\maketitle

\begin{abstract}
   This is the exercise sheet for the computation with a lot of indices. If you have trouble working out the equations or quantities in QFT, please check this note and try again. I hope it will help you!
\end{abstract}

\tableofcontents

\clearpage
\section{Problem}

\begin{enumerate}

   \item 
   Answer whether the following quantities are Lorentz invariants or not. And if it is \textit{not} invariant, find the transformation law. (e.g. Mass $m^2$ is a Lorentz invariant, while $x^{\mu}$ is not. It is transformed for $x^{\mu}\rightarrow\Lambda^{\mu}_{\ \nu}x^{\nu}$.)

   \begin{enumerate}

      \item 
      $p_{\mu}$
      
      \item 
      $A^{\mu}(x)A_{\nu}(x)$

      \item 
      $\partial_{\mu}(f(x)A^{\mu}(x))=A^{\mu}(x)(\partial_{\mu}f(x))+f(x)(\partial_{\mu}A^{\mu}(x))$ 

      (Parenthesis are often omitted.)
      
      \item 
      $\dot{\phi}$ and $\dot{\phi}^2$

      \item 
      How about $\dot{\phi}^2-(\nabla\phi)^2$ and $\ddot{\phi}-\nabla^2\phi$?
   \end{enumerate}

   \item 
   \begin{enumerate}
      \item 
      Show that $\partial_{\mu}(x^2)=2x_{\mu}$. It is intuitively true but, in our notation, it is not a trivial relation. You should keep it in mind.

      \item 
      Show that $\partial_{\mu}e^{ik\cdot x}=ik_{\mu}e^{ik\cdot x}$. I will tell you again. This relation seems to be trivial but it is not. Work it out precisely.
   \end{enumerate}

   \item 
   \begin{enumerate}
      \item 
      Let $\varepsilon_{\mu\nu\lambda}$ is  antisymmetric tensor, $A^{\mu\nu}$ is  symmetric tensor and $B^{\mu}$ is any vector. Show that the quantity $\varepsilon_{\mu\nu\lambda}A^{\mu\nu}B^{\lambda}$ vanishes. You may notice that the product of the antisymmetric one and symmetric one will be zero. I think this is a very useful tip because we may be able to omit calculations.
      
      \item 
      One of the covariant forms of the Maxwell's equations is
      \begin{equation}
         \varepsilon^{\mu\nu\rho\sigma}\partial_{\nu}F_{\rho\sigma}
         =
         0
         .
      \end{equation}
      Check that the electromagnetic field tensor $F_{\mu\nu}=\partial_{\mu}A_{\nu}-\partial_{\nu}A_{\mu}$ follows this equation. 
   \end{enumerate}
   
   \item 
   \begin{enumerate}
      \item 
      We have defined $\partial_{\mu}$ as
      \begin{equation}
         \partial_{\mu}
         \equiv
         \pdv{}{x^{\mu}}
         .
      \end{equation}
      You may notice the position of the index is opposite, the index of $x^{\mu}$ is upper while the index of $\partial_{\mu}$ is lower. Why\footnote{
         This is an advanced quiz. So you should not be able to answer it for the first time. I hope you will be able to answer it smoothly when you get used to such a computation.
      }?

      \item 
      This relation holds when you carry out the derivative by the fields. When you consider vector fields $A^{\mu}$, for instance, the functional derivative looks like the vector with the lower index:
      \begin{equation}
         \pdv{}{(A^{\mu})}
         =
         (\text{something})_{\mu}
         .
      \end{equation}
      Consider the following quantity:
      \begin{equation}
         \partial_{?}^{?}
         \left(  
            \pdv{\mathcal{L}}{(\partial^{\mu}\varphi)}
         \right)
         .
         \label{problem4b}
      \end{equation}
      The derivative $\partial$ must have indices so that this is Lorentz invariant. Rewrite the quantity \eqref{problem4b} again with the correct indices.
   \end{enumerate}

   \item 
   Consider the Lagrangian
   \begin{equation}
      \mathcal{L}
      =
      \frac{1}{2}
      (\partial_{\nu}\varphi)
      (\partial^{\nu}\varphi)
      -
      \frac{1}{2}m^2\varphi^2
      .
   \end{equation}
   We will work out the least action principle and derive the equation of motion\footnote{
      We usually denote $(\partial_{\mu}\varphi)(\partial^{\mu}\varphi)$ as $(\partial_{\mu}\varphi)^2$. Note that this is just an abbreviation. When you compute it, you should be careful about it.
   }. The Euler-Lagrange equation is as follows:
   \begin{equation}
      \partial_{\rho}
      \left(  
         \pdv{\mathcal{L}}{(\partial_{\rho}\varphi)}
      \right)
      -
      \pdv{\mathcal{L}}{\varphi}
      =
      0
      .
   \end{equation}
   \begin{enumerate}
      \item 
      Find the easier term
      \begin{equation}
         \pdv{\mathcal{L}}{\varphi}
         .
      \end{equation}

      \item 
      Show that 
      \begin{equation}
         \pdv{\mathcal{L}}{(\partial_{\mu}\varphi)}
         =
         \partial^{\mu}\varphi
      \end{equation}
      Note that $\varphi$ and $\partial_{\mu}\varphi$ are independent. Thus, it follows
      \begin{equation}
         \pdv{}{(\partial_{\mu}\varphi)}
         \left(\varphi^2\right)
         =
         0
         \quad
         ,
         \quad
         \pdv{}{(\\partial_{\mu}\varphi)}
         \left(\partial_{\nu}\varphi\right)
         =
         \delta^{\mu}_{\nu}
         .
      \end{equation}

      \item 
      Find an equation of motion. It is the \textit{Klein-Gordon equation}.
   \end{enumerate}   

   \item 
   Consider the Lagrangian
   \begin{equation}
      \mathcal{L}
      =
      -
      \frac{1}{4}
      F^{\mu\nu}F_{\mu\nu}
      -
      eA^{\mu}j_{\mu}
      .
   \end{equation}
   We can derive Maxwell's equation
   \begin{equation}
      \partial_{\mu}F^{\mu\nu}
      =
      ej^{\nu}
   \end{equation}
   from this Lagrangian by applying the least action principle. Let's do it! If you finish this computation by yourself, you will understand the system of such a messy calculation\footnote{
      Unfortunately, there are still more confusing concepts that make the computation difficult in QFT. But, when you meet a difficult one, you will be able to understand one by one slowly and surely.
   }.

\end{enumerate}


\section{Solution}

\begin{enumerate}

   \item 
   \begin{enumerate}

      \item 
      Of course, it is not invariant. It is transformed for $p_{\mu}\rightarrow\Lambda_{\mu}^{\ \nu}p_{\nu}$. 

      \item 
      No. It becomes $A^{\mu}A_{\nu}\rightarrow\Lambda^{\mu}_{\ \rho}\Lambda_{\nu}^{\ \sigma}A^{\rho}A_{\sigma}$. You should be careful about the indices. If you wrote the transformation law like $A^{\mu}A_{\nu}\rightarrow\Lambda^{\mu}_{\ \uline{\nu}}\Lambda_{\uline{\nu}}^{\ \sigma}A^{\uline{\nu}}A_{\sigma}$, you will be confused because we can not distinguish which indices are contracted\footnote{
         Although you may understand how to calculate, no one can work it out correctly. This is a promise so that no physicists will be confused. So please be sure to follow the rules!
      }.

      \item 
      It is invariant. However, the invariance of this quantity will be discussed later, in Chapter 3. Note that the reason for the invariance varies from textbook\footnote{
         There are two definitions of the Lorentz transformation, ``active'' or ``passive''. The textbook takes the ``active'' point of view. I think you did not care about that and I did not too. Pay attention when you read Chapter 3.
      }. So you should be careful when you discuss the invariance of such quantities which include the derivative $\partial_{\mu}$. For now, remember that it is invariant it should be for now until you reach that chapter.

      \item 
      Neither are invariants. Rewriting those quantities in covariant forms, we obtain
      \begin{equation}
         \dot{\phi}
         =
         \partial_{0}\phi
         ,\quad
         \ddot{\phi}
         =
         \partial_{0}^2\phi         
         .
      \end{equation}
      They are transformed since the indices are not contracted. Transformation laws are the following (But those are topics in Chapter 3, so you do not have to understand when you read the earlier chapter):
      \begin{equation}
         \partial_{0}\phi(x)
         \rightarrow
         \Lambda_{0}^{\ \nu}(\partial_{\nu}\phi)(\Lambda^{-1}x)
         ,\quad
         \partial_{0}^2\phi     
         \rightarrow
         \Lambda_{0}^{\ \rho}\Lambda^{\sigma}_{\ 0}
         (\partial_{\rho}\partial^{\sigma}\phi)(\Lambda^{-1}x)
         .
      \end{equation}

      \item 
      Both are invariant. We find those quantities as 
      \begin{equation}
         \dot{\phi}^2-(\nabla\phi)^2
         =
         (\partial_{\mu}\phi)(\partial^{\mu}\phi)
         \equiv
         (\partial_{\mu}\phi)^2
         ,\quad
         \ddot{\phi}-\nabla^2\phi
         =
         \partial_{\mu}\partial^{\mu}\phi
         \equiv
         \partial^2\phi
      \end{equation}
      in covariant form, and we notice that all indices are contracted.
   \end{enumerate}

   \item 
   \begin{enumerate}
      \item 
      $x^2$ is an abbreviation of $x^{\mu}x_{\mu}=g_{\mu\nu}x^{\mu}x^{\nu}$ and it holds
      \begin{equation}
         \pdv{(x^{\mu})}{x^{\nu}}
         =
         \delta^{\mu}_{\nu}
         .
      \end{equation}
      Thus, we will achieve a derivative of $x^2$ by careful computation as
      \begin{align}
         \partial_{\mu}(x^2)
         &=
         \pdv{}{x^{\mu}}
         \left(g_{\rho\nu}x^{\rho}x^{\nu}\right)
         \nonumber
         \\
         &=
         g_{\rho\nu}\delta^{\rho}_{\mu}x^{\nu}
         +
         g_{\rho\nu}x^{\rho}\delta_{\mu}^{\nu}
         \nonumber
         \\
         &=
         2x_{\mu}
         .
      \end{align}

      \item 
      Since $k\cdot x\equiv k_{\mu}x^{\mu}$, we get
      \begin{equation}
         \partial_{\mu}e^{ik\cdot x}
         =
         \pdv{}{x^{\mu}}\left(e^{ik_{\nu}\cdot x^{\nu}}\right)
         =
         ik_{\nu}\delta^{\nu}_{\mu}e^{ik_{\nu}\cdot x^{\nu}}
         =
         ik_{\mu}e^{ik\cdot x}
         .
      \end{equation}
      If you work the above out perfectly, you can omit the process and write it down as ``$\partial_{\mu}e^{ik\cdot x}=ik_{\mu}e^{ik\cdot x}$'' on the blackboard. 
   \end{enumerate}

   \item 
   \begin{enumerate}
      \item 
      If indices are contracted, we can exchange those indices between each other since it is \textit{dummy indices}. So we can achieve the following relation:
      \begin{equation}
         \varepsilon_{\mu\nu\lambda}A^{\mu\nu}B^{\lambda}
         =
         \varepsilon_{\nu\mu\lambda}A^{\nu\mu}B^{\lambda}
         .
         \label{solution3a}
      \end{equation}
      We have just exchanged $\mu\leftrightarrow\nu$. Using the condition $\varepsilon^{\nu\mu\lambda}=-\varepsilon\mu\nu\lambda$ (antisymmetric) and $A^{\nu\mu}=A^{\mu\nu}$ (symmetric) for the right-hand side of \eqref{solution3a}, we find
      \begin{equation}
         \varepsilon_{\mu\nu\lambda}A^{\mu\nu}B^{\lambda}
         =
         -
         \varepsilon_{\mu\nu\lambda}A^{\mu\nu}B^{\lambda}
         .
      \end{equation}
      We notice that both quantities are the same except for signs. Only zero satisfies the relation and finally, we obtain
      \begin{equation}
         \varepsilon_{\mu\nu\lambda}A^{\mu\nu}B^{\lambda}
         =
         0
         .
      \end{equation}
      As we have mentioned in the problem statement, this is a very useful relation, the product of symmetric and antisymmetric vanish.

      \item 
      It is easy to show if you notice that $\partial_{\mu}\partial_{\nu}$ is symmetric since the order of the derivative does not matter in this case.
   \end{enumerate}

   \item 
   \begin{enumerate}
      \item 
      Why a derivative by a contravariant component $x^{\mu}$ is covariant $\partial_{\mu}$ varies from our notation of Lorentz transformation, positive or passive, as we have mentioned. From the passive point of view, which is not in the textbook, the transformation law is as
      \begin{equation}
         \pdv{}{x^{\mu}}
         \rightarrow
         \pdv{}{{x'}^{\mu}}
         =
         \uline{
            \pdv{x^{\nu}}{{x'}^{\mu}}
         }
         \pdv{}{x^{\nu}}
         =
         \uline{
            \Lambda_{\mu}^{\ \nu}
         }
         \pdv{}{x^{\nu}}
         .
      \end{equation}
      Thus, a derivative is transformed as the covariant form $\partial_{\mu}\rightarrow\Lambda_{\mu}^{\ \nu}\partial_{\nu}$.

      From a positive point of view, the derivative is not transformed since the configuration of the space-time does not change. But the variable of the function changes. Then, the transformation law is totally
      \begin{equation}
         \partial_{\mu}f(x)
         \rightarrow
         \partial_{\mu}f(\Lambda^{-1}x)
         =
         (\Lambda^{-1})^{\nu}_{\ \mu}(\partial_{\nu}f)(\Lambda^{-1}x)
         =
         \Lambda_{\mu}^{\ \nu}(\partial_{\nu}f)(\Lambda^{-1}x)
         .
      \end{equation}

      \item 
      The total derivative should have an upper index as $\partial^{\mu}$.
   \end{enumerate}

   \item 
   Just follow the instructions in the problem statement. You can do it if you understand.

   \item 
   You will prove it in front of professors, as I remember. You might as well try it by yourself!
\end{enumerate}

\end{document}
