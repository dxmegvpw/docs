\documentclass[unicode,a4paper,11pt]{ltjsarticle}

\usepackage{luatexja-fontspec}
\setmainfont{TeX Gyre Termes}
\setmainjfont[BoldFont = IPAGothic]{IPAMincho}
\setmathrm{Latin Modern Roman}
% \setmainjfont{Noto Sans JP}

% ---Display \subsubsection at the Index
% \setcounter{tocdepth}{3}

% ---Setting about the geometry of the document----
% \usepackage{a4wide}
% \pagestyle{empty}

% ---Physics and Math Packages---
\usepackage{amssymb,amsfonts,amsthm,mathtools}
\usepackage{physics,braket,bm}

% ---underline---
\usepackage[normalem]{ulem}

% ---cancel---
\usepackage{cancel}

% --- surround the texts or equations
% \usepackage{fancybox,ascmac}

% ---settings of theorem environment---
\theoremstyle{definition}
\newtheorem{dfn}{定義}
\newtheorem{prop}{命題}
\newtheorem{thm}{定理}
\newtheorem{exm}{例}
\newtheorem{exc}{演習}

% ---settings of proof environment---
\renewcommand{\proofname}{\textbf{証明}}
\renewcommand{\qedsymbol}{$\blacksquare$}

% ---Ignore the Warnings---
\usepackage{silence}
\WarningFilter{latexfont}{Some font shapes}
\WarningFilter{latexfont}{Font shape}
\WarningFilter{latexfont}{Size substitutions}
\ExplSyntaxOn
\msg_redirect_name:nnn{hooks}{generic-deprecated}{none}
\ExplSyntaxOff

% ---Insert the figure (If insert the `draft' at the option, the process becomes faster.)---
\usepackage{graphicx}
% \usepackage{subcaption}

% ----Add a link to a text---
\usepackage{url,hyperref}
\usepackage[dvipsnames,svgnames]{xcolor}
\hypersetup{colorlinks=true,citecolor=FireBrick,linkcolor=Navy,urlcolor=purple}
% ---refer `texdoc xcolor' at the command line---

% ---Tikz---
% \usepackage{tikz,pgf,pgfplots,circuitikz}
% \pgfplotsset{compat=1.15}
% \usetikzlibrary{intersections,arrows.meta,angles,calc,3d,decorations.pathmorphing}

% ---Add the section number to the equation, figure, and table number---
\makeatletter
   \renewcommand{\theequation}{$\thesection.\arabic{equation}$}
   \@addtoreset{equation}{section}
   
   \renewcommand{\thefigure}{\thesection.\arabic{figure}}
   \@addtoreset{figure}{section}
   
   \renewcommand{\thetable}{\thesection.\arabic{table}}
   \@addtoreset{table}{section}
\makeatother

% ---enumerate---
% \renewcommand{\labelenumi}{$\arabic{enumi}.$}
% \renewcommand{\labelenumii}{$(\arabic{enumii})$}

% ---Index---
% \usepackage{makeidx}
% \makeindex 

% ---Fonts---
% \renewcommand{\familydefault}{\sfdefault}

% ---Title---
\title{ペスキンゼミ\ 11章から13章}
\author{宮根 一樹}
\date{最終更新日:\today}

\begin{document}

\maketitle

\tableofcontents

\vspace*{10pt}

卒論発表前後のときとは打って変わってとっても元気なので、ゼミ資料とか作ってみました。勢いに任せているので、クオリティはあまり保証しません。それ以前に、ぺスキンのここらへんは評判良くないらしいですし、僕もそう思います。ですので、僕が犠牲になって(この地雷パートの)雰囲気だけは伝えられたらなと思います。

\vspace*{10pt}

\begin{itemize}
   \item
         春の学校の発表の準備や、研究の進捗があまり芳しくありません。全てはこのゼミ資料のせいです。これを身代わりに助かりたいところですが、まあ、そんなわけにはいかないでしょう。頑張ります。
   \item
         Youtubeを見てたら、深夜の首都高一周ドライブをしたくなりました。これを読んだ誰か、一緒に行きましょう。別に危ないことをしたいわけじゃないです。「山手線を徒歩/チャリ/電車で一周しました!」って人たちと同じモチベーションです。
\end{itemize}

\clearpage

\section{くりこみと対称性}

対称性というのがQFT(というか物理全般)で大事なのは周知のこととは思いますが、では、くりこみをした後の物理では、もとの対称性はどうなっているのでしょうか?素朴に考えれば、くりこみでやってることは、ラグランジアンのパラメターをいじることだけですので、ラグランジアン自体の対称性は変わらないような気がするのですが、量子補正は物理量を変えてしまうため、得られる物理量は対称性を反映していない可能性もありそうです。この章では、まずは古典レベルで一部の対称性を破っておいて、くりこんだ理論がどうなるかどうかを見ていきます。

得られる結果としては
\begin{itemize}
   \item
         古典論のレベルでは、対称性を破ると質量が0の粒子が生成される(ゴールドストーンの定理, \S11.1)
   \item
         任意のオーダーでくりこみをした後でも、ゴールドストーンの定理は成立する(\S11.2,\S11.6)
\end{itemize}
ということです\footnote{
   個人的に導入で結論まで言ってしまう構成が好きなので、この資料ではこのスタイルで行こうと思います。
}。特に、2つめの項目を議論するときに、量子補正も取り込んだ有効的な作用を書いておくと見通しがいいので、ここではそれらの道具の整備についても議論していきます。


\subsection{対称性の破れとゴールドストーンの定理}

ここでは、古典レベルで対称性を一部だけ破る例として、$O(N)$対称性のある線形シグマ模型を見ていきます。ラグランジアンは、$\phi^{i}\ (i=1,\cdots,N)$をスカラー場として
\begin{equation}
   \mathcal{L}
   =
   \frac{1}{2}(\partial_{\mu}\phi^{i})^2
   +
   \frac{1}{2}\mu^2(\phi^{i})^2
   -
   \frac{\lambda}{4}[(\phi^{i})^2]^2
   \label{eqn:lagrangian_N_linearsigma}
\end{equation}
です。ただし、$[(\phi^{i})^2]^2\equiv [(\phi^{1})^2+\cdots+(\phi^{N})^2]^2$です。この理論は$\phi^{i}$のノルムでかけているので、$N\times N$の直交行列$R\in O(N)$で回す変換に対して不変です。実際、${\phi^{i}}^{\prime}\equiv R^{ij}\phi^{j}$とすれば
\begin{equation}
   ({\phi^{i}}^{\prime})^2
   =
   \phi^{k}R^{ki}\cdot R^{ij}\phi^{j}
   =
   (\phi^{i})^2
\end{equation}
です。この理論のポテンシャルは
\begin{equation}
   V(\phi)
   =
   -\frac{1}{2}\mu^2 (\phi^{i})^2+\frac{\lambda}{4}[(\phi^{i})^2]^2
\end{equation}
ですが、最小点は
\begin{equation}
   (\phi^{i})^2
   \frac{\mu^2}{\lambda}
\end{equation}
であり、そのような点は複数あります。そのような点の中から、ここでは真空期待値を
\begin{equation}
   \phi_{0}^{i}
   =
   (0,0,\cdots,v)
   ,\ 
   v\equiv\frac{\mu}{\sqrt{\lambda}}
\end{equation}
ととりましょう。この最小点で場$\phi^{i}$を展開すると
\begin{equation}
   \phi^{i}
   =
   (\pi^{1},\pi^{2},\cdots,\pi^{N-1},v+\sigma(x))
\end{equation}
と書けるので、ラグランジアン\eqref{eqn:lagrangian_N_linearsigma}を新しい場$\pi^{k},\sigma$で書き換えると
\begin{align}
   \mathcal{L}
   &=
   \frac{1}{2}(\partial_{\mu}\pi^{k})^2
   +
   \frac{1}{2}(\partial_{\mu}\sigma)^2
   -
   \frac{1}{2}(2\mu^2)\sigma^2
   \nonumber
   \\
   &\qquad
   -\sqrt{\lambda}\mu\sigma^3
   -\sqrt{\lambda}\mu(\pi^{k})^2\sigma
   -\frac{\lambda}{4}\sigma^4 
   -\frac{\lambda}{2}(\pi^{k})^2\sigma^2  
   -\frac{\lambda}{4}[(\pi^{k})^2]^2
   \label{eqn:lagrangian_N_linearsigma_2}
\end{align}
となります。ただし、定数は無視しています。

この理論についての量子論は次節から取り扱うとして、古典論のレベルから分かることは、$(N-1)$個の無質量な場$\pi^{k}$が現れていることです。今、$\pi^{k}$についての対称性が$O(N-1)$として残っているわけですが、このように対称性が敗れると質量がある理論から無質量な粒子が出ることが分かっており、これを\textbf{ゴールドストーンの定理}といいます。また、この自発的対称性の破れによって生じた無質量な粒子を\textbf{ゴールドストーンボソン}といったりします。

ゴールドストーンの定理は教科書に証明があるので、ここでは省略します。読書の手助けのために、その証明のスケッチを述べておくと
\begin{enumerate}
   \item 
   ポテンシャルをVEVの周りで展開して、その2次の係数が質量行列なのだからそこをみる。
   \item 
   もし、そのVEVでは元の対称性が保っていないなら、その分だけ質量行列のランクが削れて固有値に0が含まれることになる。
\end{enumerate}
といった感じです。


\subsection{くりこみと対称性の具体例}

線形シグマ模型\eqref{eqn:lagrangian_N_linearsigma_2}を量子化しましょう。すると、場$\pi^{k}(x),\sigma(x)$のファインマンダイアグラムはすぐに求まります、が、どうせすぐにくりこみを議論するためにラグランジアンを書き換えるのでここでは書きません。ラグランジアンの書き換えは、元々のラグランジアン\eqref{eqn:lagrangian_N_linearsigma}には3つの発散するパラメターがあったことから、$\phi^{i}\rightarrow Z\phi^{i}$とリスケーリングしたあとに
\begin{equation}
   \delta_{Z}
   \equiv
   Z-1
   ,\ 
   \delta_{m}
   \equiv
   m_{0}^2Z-m^2
   ,\ 
   \delta_{\lambda}
   \equiv
   \lambda_{0}Z^2
   -
   \lambda
\end{equation}
と裸の定数を書き換えてやれば、
\begin{equation}
   \mathcal{L}
   \equiv
   \mathcal{L}_{\mathrm{phys.}}+\mathcal{L}_{\mathrm{c.t.}}
\end{equation}
と物理的な項$\mathcal{L}_{\mathrm{phys.}}$と相殺項$\mathcal{L}_{\mathrm{c.t.}}$に分離できます。それぞれ
\begin{align}
   \mathcal{L}_{\mathrm{phys}}
   &=
\end{align}
























\end{document}
