\pdfoutput=1
\documentclass[pdflatex,unicode,ja=standard,12pt]{beamer}

% ---Setting about the geometry of the document----
% \usepackage{a4wide}
% \pagestyle{empty}

% ---Physics and Math Packages---
\usepackage{amssymb,amsfonts,amsthm,mathtools}
\usepackage{physics,braket,bm}

% ---underline---
\usepackage{ulem}

% ---cancel---
\usepackage{cancel}

% --- sorround the texts or equations
% \usepackage{fancybox,ascmac}

% ---settings of theorem environment---
% \usepackage{amsthm}
% \theoremstyle{definition}

% ---settings of proof environment---
% \renewcommand{\proofname}{\textbf{証明}}
% \renewcommand{\qedsymbol}{$\blacksquare$}

% ---Ignore the Warnings---
\usepackage{silence}
\WarningFilter{latexfont}{Some font shapes,Font shape}

% ---Insert the figure (If insert the `draft' at the option, the process becomes faster.)---
% \usepackage{graphicx}
% \usepackage{subcaption}

% ----Add a link to a text---
\usepackage{url}
\usepackage{xcolor,hyperref}
\hypersetup{colorlinks=true,citecolor=orange,linkcolor=blue,urlcolor=magenta}
\usepackage[whole,autotilde]{bxcjkjatype}

% ---Tikz---
% \usepackage{tikz,pgf,pgfplots,circuitikz}
% \pgfplotsset{compat=1.15}
% \usetikzlibrary{intersections,arrows.meta,angles,calc,3d,decorations.pathmorphing}

% ---Add the section number to the equation, figure, and table number---
\makeatletter
   \renewcommand{\theequation}{\thesection.\arabic{equation}}
   \@addtoreset{equation}{section}
   
   \renewcommand{\thefigure}{\thesection.\arabic{figure}}
   \@addtoreset{figure}{section}
   
   \renewcommand{\thetable}{\thesection.\arabic{table}}
   \@addtoreset{table}{section}
\makeatother

% ---beamer settings---
\usefonttheme{professionalfonts}
\usecolortheme{seahorse}
\setbeamercolor{structure}{fg=green}
\setbeamercolor{itemize item}{fg=black}
\setbeamercolor{bibliography entry author}{fg=black}
\setbeamercolor{bibliography item}{fg=black}
\setbeamertemplate{footline}[frame number]
\setbeamertemplate{navigation symbols}{} 

% ---Title---
\title{B4\ ぺスキンゼミ\ 演習問題3.4}
\author{宮根 一樹}
\date{2023年\ 8月\ 11日}

\begin{document}

\begin{frame}
  \titlepage
\end{frame}

\begin{frame}%[allowframebreaks]
  \frametitle{目次}
  \tableofcontents
\end{frame}

\setcounter{section}{0}

\section{問題3.4\ マヨラナフェルミオン}

\subsection{今日の流れと予備知識}

\begin{frame}%[noframenumbering,allowframebreaks]

  \frametitle{\subsecname}
  %\thispagestyle{empty}









\end{frame}


\begin{frame}%[noframenumbering,allowframebreaks]

  \frametitle{\subsecname}
  %\thispagestyle{empty}

  spinor fieldには,\uline{右巻き(right-handed)}なものと\uline{左巻き(left-handed)}なものが存在する.(これらはヘリシティで定義されるが,今回はあまり関係ない.)

  \vspace{10pt}

  これらはローレンツ変換の仕方が異なる:
  \begin{equation}
    \left\{
      \begin{alignedat}{1}
        \text{左巻き:}&\ 
        \psi_{L}(x)
        \rightarrow
        a_{L}\psi(\Lambda^{-1}x)
        ,
        \\
        \text{右巻き:}&\ 
        \psi_{R}(x)
        \rightarrow
        a_{R}\psi(\Lambda^{-1}x)
        .
      \end{alignedat}
    \right.
  \end{equation}
  左巻きのspinor$\chi$について,$i\sigma^2\chi^{*}$は右巻きで変換される.

  \vspace{10pt}

  カイラル表示なら,Dirac fieldは
  \begin{equation}
    \psi(x)
    =
    \begin{pmatrix}
      \psi_{L}(x) \\
      \psi_{R}(x)
    \end{pmatrix}
  \end{equation}
  という形で左巻きと右巻きが関わる.

\end{frame}



\subsection{(1)\ 運動方程式}

\begin{frame}

  \frametitle{\subsecname}

  $\psi=(\psi_L,\psi_R)$とおけば,masslessのディラック方程式は
  \begin{equation}
    i(\partial_0-\bm{\sigma}\cdot\bm{\nabla})\psi_L=0
  \end{equation}
  となる.このとき,$\psi_L(x)$の成分を$\chi_a(x)$とする.

  \vspace{10pt}

  \begin{itemize}

    \item [(1)]
    
    $m\neq 0$のときの$\chi(x)$についての方程式が
    \begin{equation}
      i\bar{\sigma}\cdot\partial\chi
      -
      im\sigma^2\chi^{*}
      =
      0
      \label{eom}
    \end{equation}
    と書けることを示せ.そのために,方程式がrelativistic invariantであり,Klein-Gordon方程式を導くことを確かめよ.

    \vspace{10pt}

    (このように記述されるフェルミオンの質量を\textbf{マヨラナ質量項}という.)

  \end{itemize}

\end{frame}

\begin{frame}
  
  \frametitle{\subsecname}

  方程式\eqref{eom}が運動方程式なら,
  \begin{itemize}
    \item [1.]
    
    Lorentz invariance

    \item [2.]

    Klein-Gordon方程式を含んでいる(解になっている)

  \end{itemize}
  を少なくとも満たしていてほしい.それらをチェックしよう.

  \vspace{10pt}
  
  (この方程式に対応するLagrangianがあるかどうか云々はあとで議論するので,ここではひとまず\eqref{eom}が運動方程式として要求される条件を満たしているかどうかを議論する.)

\end{frame}

\begin{frame}
  
  \frametitle{\subsecname}

  \uline{1. Lorentz invariance}
  
  \vspace{5pt}

  $\chi(x)$はleft-handed spinorなので,Lorentz変換は
  \begin{equation}
    \chi(x)
    \rightarrow
    a_L\chi(\Lambda^{-1}x)
    ,\ 
    a_L
    =
    \exp\left[  
      -
      i\bm{\theta}\cdot\frac{\bm{\sigma}}{2}
      -
      \bm{\beta}\cdot\frac{\bm{\sigma}}{2}
    \right]
  \end{equation}
  で与えられる.

\end{frame}

\begin{frame}
  
  \frametitle{\subsecname}

    このとき,それぞれの項は
    \begin{equation}
      \left\{
        \begin{alignedat}{1}
          \bar{\sigma}\cdot\partial\chi(x)
          &\rightarrow
          \bar{\sigma}^{\mu}
          \partial_{\mu}
          \left(  
            a_L\chi(\Lambda^{-1}x)
          \right)
          =
          a_R\bar{\sigma}\cdot(\partial\chi)(\Lambda^{-1}x)
          \\
          \sigma^2\chi^{*}(x)
          &\rightarrow
          \sigma^2
          a_L^{*}\chi^{*}(\Lambda^{-1}x)
          =
          a_R\sigma^2\chi^{*}(\Lambda^{-1}x)
        \end{alignedat}
      \right.
    \end{equation}
    となる.ここで次の公式
    \begin{equation}
      \bar{\sigma}^{\mu}a_L
      =
      \Lambda^{\mu}_{\ \nu}a_R\bar{\sigma}^{\nu}
      ,\ 
      \sigma^{2} a_L^{*}=a_R \sigma^2
      \label{form1}
    \end{equation}
    を用いた.ただし
    \begin{equation}
      a_R
      \equiv
      \exp
      \left[  
        -
        i\bm{\theta}\cdot\frac{\bm{\sigma}}{2}
        +
        \bm{\beta}\cdot\frac{\bm{\sigma}}{2}
      \right]
      ,\ 
      \Lambda_{\frac{1}{2}}
      =
      \begin{pmatrix}
        a_L & 0 \\
        0 & a_R
      \end{pmatrix}
    \end{equation}
    である.これらの公式\eqref{form1}の計算は補足\ref{append_1}で.    

\end{frame}

\begin{frame}
  
  \frametitle{\subsecname}

    したがって,方程式全体は
    \begin{equation}
      (i\bar{\sigma}\cdot\partial\chi-im\sigma^{2}\chi^{*})(x)
      \rightarrow
      a_R(i\bar{\sigma}\cdot\partial\chi-im\sigma^{2}\chi^{*})(\Lambda^{-1}x)=0
    \end{equation} 
    となるので,運動方程式はLorentz invariant.

    \vspace{10pt}

    \uline{2. K.G.方程式を含んでいること}

    \vspace{5pt}

    運動方程式\eqref{eom}の複素共役は
    \begin{equation}
      -i(\bar{\sigma}^{*})^{\mu}\partial_{\mu}\chi^{*}
      +
      im(\sigma^2)^{*}\chi
      =
      0
      .
    \end{equation}
    \eqref{eom}を$\chi^{*}$について解くと$\chi^{*}=\sigma^2 \bar{\sigma}^{\mu}\partial_\mu \chi /m$.

\end{frame}


\begin{frame}
  
  \frametitle{\subsecname}
    
    これを代入すれば
    \begin{gather}
      -(\bar{\sigma}^{*})^{\mu}\partial_{\mu}(\sigma^2 \bar{\sigma}^{\nu}\partial_\nu)\chi
      +
      m^2(\sigma^2)^{*}\chi
      =
      0
      \nonumber
      \\
      \rightarrow\quad
      \underbrace{
        (\bar{\sigma}^{*})^{\mu}\sigma^2\bar{\sigma}^{\nu}
        \partial_\mu\partial_\nu
      }_{
        \hspace*{10pt}
        \underset{\text{(補足\ref{append_1}で)}}{=\sigma^{2}\partial^2}
      }
      \chi
      +
      \sigma^2m^2
      \chi
      =
      0
      \quad
      (\because\ (\sigma^2)^{*}=-\sigma^2)
      \nonumber
      \\
      \rightarrow\quad
      (\partial^2+m^2)\chi=0
    \end{gather}
    となり,確かにKlein-Gordon方程式の解にもなっている.

\end{frame}

\subsection{(2)\ 作用・ラグランジアン}

\begin{frame}

  \frametitle{\subsecname}

  \begin{itemize}

    \item [(2)]

    マヨラナ方程式を導くラグランジアンはあるだろうか?質量項をみると$(\sigma^2)_{ab}\chi^{*}_{a}\chi^{*}_{b}$を$\chi^{*}$で変分をとればよさそう.しかし,$\sigma^2$が反対称なので,普通に$\chi$がc-nuberだと考えると消えてしまう.(後述)

    \vspace{5pt}

    そこで,$\chi(x)$がグラスマン数だとみなそう.(定義とかは後で.)すると,対応するラグランジアンは
    \begin{equation}
      \mathcal{L}
      =
      \chi^{\dag}i\bar{\sigma}\cdot\partial\chi
      +
      \frac{im}{2}
      (  
        \chi^{T}\sigma^2\chi
        -
        \chi^{\dag}\sigma^2\chi^{*}
      )
      \label{majorana_lagrangian}
    \end{equation}
    である.このラグランジアンの作用$S$が実(i.e. $S=S^{*}$)であることを確認し,$\chi,\chi^{*}$でそれぞれ変分をとることで,マヨラナ方程式を導出せよ.
 
  \end{itemize}

\end{frame}

\begin{frame}
  
  \frametitle{\subsecname}

  ラグランジアンに$(\sigma^2)_{ab}\chi^{*}_a\chi^{*}_b=\chi^{\dag}\sigma^{2}\chi^{*}$という項があれば,これを$\chi^{*}_a$で変分することで$\sigma^{2}\chi^{\dag}$となり,質量項が取り出せそうに思える.しかし,$\sigma^{2}$が反対称$(\sigma^2)^{T}=-\sigma^2$であることによって,そもそも
  \begin{equation}
    \chi^{\dag}\sigma^{2}\chi^{*}
    =
    (\chi^{\dag}\sigma^{2}\chi^{*})^{T}
    =
    -\chi^{\dag}\sigma^{2}\chi^{*}
    \quad\rightarrow\quad
    \chi^{\dag}\sigma^{2}\chi^{*}=0
  \end{equation}
  となってしまう.そこで次のような性質を満たすグラスマン数を導入する:
  \begin{gather}
    \alpha \beta
    =
    -\beta \alpha
    ,
    \\
    (\alpha\beta)^{*}
    \equiv
    \beta^{*}\alpha^{*}
    =
    -\alpha^{*}\beta^{*}
    .
  \end{gather}
  これを用いれば$\chi^{\dag}\sigma^{2}\chi^{*}$が対称になるので,non-zero.

\end{frame}

\begin{frame}
  
  \frametitle{\subsecname}

  作用は
  \begin{equation}
    S
    =
    \int\dd^4 x
    \left[  
      \chi^{\dag}i\bar{\sigma}\cdot\partial\chi
      +
      \frac{im}{2}
      (  
        \chi^{T}\sigma^2\chi
        -
        \chi^{\dag}\sigma^2\chi^{*}
      )
    \right]
    .
    \label{action}
  \end{equation}
  ただ,これが実なのかをみたいだけなら$\mathcal{L}^{*}=\mathcal{L}$のみを調べればよい:
  \begin{align}
    \mathcal{L}^{*}
    &
    \uncover<1->{
      =
      -
      \chi^{T}i\bar{\sigma}^{*}\cdot\overset{\rightarrow}{\partial}\chi^{*}
      -
      \frac{im}{2}
      \left(  
        -\chi^{\dag}\sigma^2\chi^{*}
        +\chi^{T}\sigma^2\chi
      \right)
    }
    \nonumber
    \\
    &
    \uncover<2->{
      =
      \chi^{T}i\bar{\sigma}^{*}\cdot\overset{\leftarrow}{\partial}\chi^{*}
      +
      \frac{im}{2}
      (  
        \chi^{T}\sigma^2\chi
        -
        \chi^{\dag}\sigma^2\chi^{*}
    )
    }
    \nonumber
    \\
    &\uncover<3->{
      =
      \chi^{\dag}i\bar{\sigma}\cdot\overset{\rightarrow}{\partial}\chi
      +
      \frac{im}{2}
      (  
        \chi^{T}\sigma^2\chi
        -
        \chi^{\dag}\sigma^2\chi^{*}
      )
      .
    }
  \end{align}
  ただし,${(\bar{\sigma}^*)^{\mu}}^{T}=(\bar{\sigma}^{\mu})^{\dag}=\bar{\sigma}^{\mu}$を用いた.

\end{frame}

\begin{frame}
  
  \frametitle{\subsecname}

  作用をもう一度:
  \begin{equation}
    S
    =
    \int\dd^4 x
    \left[  
      \chi^{\dag}i\bar{\sigma}\cdot\partial\chi
      +
      \frac{im}{2}
      (  
        \chi^{T}\sigma^2\chi
        -
        \chi^{\dag}\sigma^2\chi^{*}
      )
    \right]
    .
    \tag{
      \ref{action}
    }
  \end{equation}
  $\chi$と$\chi^{*}$がindependentだと思って,$\chi$の変分をとってみると
  \begin{align}
    \frac{\delta}{\delta \chi_a}
    (
      \chi^{\dag}i\bar{\sigma}^{\mu}\partial_{\mu}\chi
     )
    &=
    \frac{\delta}{\delta \chi_a}
    (
      -\chi^{\dag}_{b}i(\bar{\sigma}^{\mu})_{bc}\overset{\leftarrow}{\partial}_{\mu}\chi_{c}
    )
    \nonumber
    \\
    &
    =
    -i(\bar{\sigma})_{ab}\cdot\partial\chi^{*}_{b}
    \\
    \frac{\delta}{\delta \chi_a}
    (
      \chi^{T}\sigma^2\chi
      -
      \chi^{\dag}\sigma^2\chi^{*}
    )
    &=
    \frac{\delta}{\delta \chi_a}
    (\chi_{b}(\sigma^2)_{bc}\chi_{c})
    \nonumber
    \\
    &
    =
    \frac{\delta}{\delta\chi_a}
    (
      -2i\chi_1\chi_2
    )
    =
    2(\sigma^2)_{ab}\chi_b
    .
  \end{align}

\end{frame}


\begin{frame}
  
  \frametitle{\subsecname}

  \begin{align}
    &\frac{\delta S}{\delta \chi}
    =
    \int\dd^4 x
    \left[  
      -
      i\bar{\sigma}\cdot\partial\chi^{*}
      +
      im\sigma^2\chi
      \vphantom{\frac{1}{2}}
    \right]
    =
    0
    \nonumber
    \\
    &
    \hspace{-1cm}
    \rightarrow
    \quad
    i\bar{\sigma}\cdot\partial\chi^{*}
    -
    im\sigma^2\chi
    =
    0
    .
    \label{eom_conj}
  \end{align}
  $\chi^{\dag}$での変分も同様.こっちはちゃんと運動方程式がでてくる.(\eqref{eom_conj}はその複素共役.)

\end{frame}


\subsection{(3)\ ディラックスピノール}

\begin{frame}

  \frametitle{\subsecname}

  \begin{itemize}

    \item [(3)]
    
    $\psi=(\psi_{L},\psi_{R})$と書いたとき,$\psi_{R}$の変換は$\psi_{L}$の複素共役の変換とユニタリー同値.(補足\ref{unitary}で)よって,left-handed spinors$\chi_1,\chi_2$で
    \begin{equation}
      \psi_L=\chi_1
      ,\ 
      \psi_R=i\sigma^2\chi_2^{*}
    \end{equation}
    と書き換えるとすれば,Dirac fieldのラグランジアンとその質量項はどうなるか?

  \end{itemize}

  \vspace{10pt}

  (
    left-handed spinor $\psi_L$について,$\sigma^2\psi_{L}^{*}$はright-handedの変換をしていた.よって,問題文の$\psi_{R}$はちゃんとright-handed.よって,$\psi=(\psi_L,\psi_R)$はディラック場(の必要条件を満たしている)といえる.
  )
  
\end{frame}



\begin{frame}
  
  \frametitle{\subsecname}

  ラグランジアンを計算するだけ:
  \begin{align}
    \mathcal{L}
    &\uncover<1->{
      =
      \bar{\psi}(i\gamma^{\mu}\partial_{\mu}-m)\psi
    }
    \nonumber
    \\
    &
    \uncover<2->{
      =
      \begin{pmatrix}
        \psi_R^{\dag} & \psi_L^{\dag}
      \end{pmatrix}
      \begin{pmatrix}
        -m & i\sigma^{\mu}\partial_{\mu} \\
        i\bar{\sigma}^{\mu}\partial_{\mu} & -m
      \end{pmatrix}
      \begin{pmatrix}
        \psi_L \\
        \psi_R
      \end{pmatrix}
    }
    \nonumber
    \\
    &
    \uncover<3->{
      =
      -m(-i\chi_2^{T}\sigma^2)\chi_1
      +
      i(-i\chi_2^{T}\sigma^2)\sigma^{\mu}\partial_{\mu}(i\sigma^2\chi^{*}_2)
    }
    \nonumber
    \\
    &\qquad
    \uncover<3->{
      +
      i\chi^{\dag}_1\bar{\sigma}^{\mu}\partial_{\mu}\chi_1
      -
      m\chi^{\dag}_1(i\sigma^2\chi_2^{*})
    }
    \nonumber
    \\
    &\qquad\uncover<3->{\vdots}
    \nonumber
    \\
    &
    \uncover<4->{
      =
      i\chi^{\dag}_1\bar{\sigma}^{\mu}\partial_{\mu}\chi_1
      +
      i\chi^{\dag}_2\bar{\sigma}^{\mu}\partial_{\mu}\chi_2
      +
      im
      (
        \chi_2^{T}\sigma^2\chi_1
        -
        \chi_1^{\dag}\sigma^2\chi_2^{*}
      )
      .
    }
    \label{lagrangian}
  \end{align}
  \uncover<4->{最後の項が質量項.(途中計算は補足\ref{q3_appendix}で.)}

\end{frame}


\subsection{(4)\ 対称性}

\begin{frame}
  
  \frametitle{\subsecname}

  \begin{itemize}
    
    \item [(4)]

    前問の作用がglobal symmetryをもつことを示せ.また,(2),(3)のラグランジアンについて
    \begin{equation}
      J^{\mu}
      =
      \chi^{\dag}\bar{\sigma}^{\mu}\chi
      ,\ 
      J^{\mu}
      =
      \chi_1^{\dag}\bar{\sigma}^{\mu}\chi_1
      -
      \chi_2^{\dag}\bar{\sigma}^{\mu}\chi_2
      \label{current}
    \end{equation}
    がそれぞれ保存カレントであることを確認し,それぞれの関係を確認せよ.そして,$O(N)$の対称性をもつ$N$個のmassiveな2成分自由場の理論を構築せよ.

  \end{itemize}

\end{frame}


\begin{frame}
  
  \frametitle{\subsecname}

  (3)でのラグランジアンは
  \begin{equation}
    \mathcal{L}
    =
    i\chi^{\dag}_1\bar{\sigma}^{\mu}\partial_{\mu}\chi_1
    +
    i\chi^{\dag}_2\bar{\sigma}^{\mu}\partial_{\mu}\chi_2
    +
    im
    (
      \chi_2^{T}\sigma^2\chi_1
      -
      \chi_1^{\dag}\sigma^2\chi_2^{*}
    )
    .
    \nonumber
  \end{equation}
  ディラック場は$U(1)$対称性$\psi\rightarrow e^{i\alpha}\psi$をもっていたので,このラグランジアンも
  \begin{equation}
    \psi
    =
    \begin{pmatrix}
      \chi_1 \\
      i\sigma^2\chi_2
    \end{pmatrix}
    \rightarrow
    \begin{pmatrix}
      e^{i\alpha}\chi_1 \\
      i\sigma^2(e^{-i\alpha}\chi_2)^{*}
    \end{pmatrix}
    =
    e^{i\alpha}\psi
  \end{equation}
  の対称性をもっているはずであり,実際にそうなっている.($U(1)_{\mathrm{v}}$不変性;$U(1)_L\times U(1)_R$はmassiveでは破れている.)

\end{frame}



\begin{frame}
  
  \frametitle{\subsecname}

  (2)のラグランジアン
  \begin{equation}
    \mathcal{L}
    =
    \chi^{\dag}i\bar{\sigma}\cdot\partial\chi
    +
    \frac{im}{2}(\chi^{T}\sigma^{2}\chi-\chi^{\dag}\sigma^2\chi^{*})
    \nonumber
  \end{equation}
  は,$m=0$で$U(1)$対称性をもつ.(こっちは質量項があるときは,その対称性が破れている.これはleft-handed spinorのみを考えているため,masslessのときのみ保存されるから.)
  
  \vspace{10pt}

  $m=0$のときのネーターカレントをもとめ,それが\eqref{current}の第1式に対応していることを示そう.
  
\end{frame}


\begin{frame}
  
  \frametitle{\subsecname}

  ラグランジアンは$U(1)$変換で不変なので,ネーターカレントをもとめるためには$\partial_{\mu}\chi$での微分のみを考えればよい.よって,
  \begin{equation}
    J^{\mu}
    =
    \pdv{\mathcal{L}}{(\partial_{\mu}\chi)}
    \cdot
    (-i\chi)
    =
    \chi^{\dag}\bar{\sigma}^{\mu}\chi
    .
  \end{equation}
  (本文でもそうだったが,位相の変換に関しては符号は気にしない.)

  masslessの運動方程式は$\bar{\sigma}^{\mu}\partial_{\mu}\chi=0,\ (\bar{\sigma}^{\mu})^{*}\partial_{\mu}\chi^{*}=0$なので,代入すれば確かに保存カレント.

\end{frame}

\begin{frame}
  
  \frametitle{\subsecname}

  (3)のラグランジアンはそもそもディラック場から導出されていたため,ネーターカレントも同じ.よって,
  \begin{align}
    J^{\mu}
    &=
    \bar{\psi}\gamma^{\mu}\psi
    \nonumber
    \\
    &=
    \begin{pmatrix}
      -i\chi_2^{T}\sigma_2 & \chi_1^{\dag}
    \end{pmatrix}
    \begin{pmatrix}
      0 & \sigma^{\mu} \\
      \bar{\sigma}^{\mu} & 0
    \end{pmatrix}
    \begin{pmatrix}
      \chi_1 \\
      i\sigma^2\chi_2^{*}
    \end{pmatrix}
    \nonumber
    \\
    &=
    \chi_1^{\dag}\bar{\sigma}^{\mu}\chi_1
    -
    \chi_2^{\dag}\bar{\sigma}^{\mu}\chi_2
    .
  \end{align}
  ただし,
  \begin{equation}
    \chi_2^{T}\sigma^2\sigma^{\mu}\sigma^2\chi_2^{*}
    =
    (\chi_2^{T}(\bar{\sigma}^{\mu})^*\chi_2^{*})^{T}
    =
    -\chi_2^{\dag}\bar{\sigma}^{\mu}\chi_2
  \end{equation}
  を用いた.($\sigma^2\sigma^{\mu}\sigma^2=(\bar{\sigma}^{\mu})^*$とGrassmann numberの交換.)

  \vspace{10pt}

  こっちは,運動方程式から\uline{massiveでも}ちゃんと保存することがわかる.
  
\end{frame}


\begin{frame}
  
  \frametitle{\subsecname}

  spnor fieldsが$(\chi_1,\cdots,\chi_N)$のとき,これらが$O(N)$対称ならば,
  \begin{equation}
    \chi_1^{T}\chi_1
    +
    \cdots
    +
    \chi_N^{T}\chi_N
    =
    \mathrm{inv.}
  \end{equation}
  である.他にも$\sum\chi_i^{T}\sigma\cdot\partial\chi_i$などの形も不変.

\end{frame}

\begin{frame}
  
  \frametitle{\subsecname}

  one spinor fieldのラグランジアン
  \begin{equation}    
    \mathcal{L}
    =
    \chi^{\dag}i\bar{\sigma}\cdot\partial\chi
    +
    \frac{im}{2}(\chi^{T}\sigma^{2}\chi-\chi^{\dag}\sigma^2\chi^{*})
    \nonumber
  \end{equation}
  をそれぞれの場について足し上げれば
  \begin{equation}
    \mathcal{L}_N
    =
    \sum_{i=1}^{N}
    \left[  
      \chi_i^{\dag}i\bar{\sigma}\cdot\partial\chi_i
      +
      \frac{im}{2}(\chi_i^{T}\sigma^{2}\chi_i+(\chi_i^{T}\sigma^2\chi_i)^{*})
    \right]
    \label{L_ON}
  \end{equation}
  であり,これは確かに$O(N)$変換で不変である.

\end{frame}



\subsection{(5)\ マヨラナ理論の量子化}

\begin{frame}
  
  \frametitle{\subsecname}
  
  \begin{itemize}    
    \item [(5)]    
    (1),(2)のマヨラナ理論(one spinor field)
    \begin{equation}
      \mathcal{L}
      =
      \chi^{\dag}i\bar{\sigma}\cdot\partial\chi
      +
      \frac{im}{2}(\chi^{T}\sigma^{2}\chi-\chi^{\dag}\sigma^2\chi^{*})
      \nonumber
    \end{equation}
    を量子化しよう.つまり,$\chi(x)$は次の正準交換関係
    \begin{equation}
      \{  
        \chi_a(\mathbf{x})
        ,
        \chi_b^{\dag}(\mathbf{y})
      \}
      =
      \delta^{(3)}(\mathbf{x}-\mathbf{y})\delta_{ab}
    \end{equation}
    を満たすとして,
    \begin{itemize}
      \item 
      ハミルトニアンを構成し,

      \item
      ハミルトニアンを対角化する生成消滅演算子の交換関係をもとめよ.
    \end{itemize}
    (ヒント:$\chi(x)$とディラック場の$\psi_L$を比較してみよ.)
  \end{itemize}

\end{frame}


\begin{frame}  
  \frametitle{\subsecname}

  ハミルトニアンに関しては量子論以前の話.エネルギー・運動量テンソルは
  \begin{equation}
    T^{\mu}_{\ \nu}
    =
    \pdv{\mathcal{L}}{(\partial_\mu \chi)}\partial_\nu \chi
    -
    \mathcal{L}\delta^{\mu}_{\ \nu}
  \end{equation}
  である.この(0,0)成分のchargeがハミルトニアンだった.

\end{frame}


\begin{frame}%[noframenumbering,allowframebreaks]

  \frametitle{\subsecname}
  %\thispagestyle{empty}

  よって,$T^{00}$を積分して
  \begin{align}
    H
    &
    \uncover<1->{
      =
      \int\dd^3\mathbf{x}\left(  
        \pdv{\mathcal{L}}{(\partial_0 \chi)}\partial^0 \chi
        -
        \mathcal{L}
      \right)
    }
    \nonumber
    \\
    &
    \uncover<2->{
      =
      \int\dd^3 \mathbf{x}
      \left(  
        (i\chi^{\dag}\sigma^0)\partial_0\chi
        -
        \left[  
          \chi^{\dag}i\sigma^0\partial_0\chi
          -
          \chi^{\dag}i\bm{\sigma}\cdot\bm{\nabla}\chi
      \right.\right.
    }
    \nonumber
    \\
    &\hspace{5cm}
    \uncover<2->{
      \left.\left.
          +
          \frac{im}{2}(\chi^{T}\sigma^{2}\chi-\chi^{\dag}\sigma^2\chi^{*})
        \right]
      \right)
    }
    \nonumber
    \\
    &
    \uncover<3->{
      =
      \int\dd^3\mathbf{x}
      \left(  
        \chi^{\dag}i\bm{\sigma}\cdot\bm{\nabla}\chi
        -
        \frac{im}{2}(\chi^{T}\sigma^{2}\chi-\chi^{\dag}\sigma^2\chi^{*})
      \right)
      .
    }
    \label{hamiltonian}
  \end{align}
  \uncover<3->{
  (後から知ったのですが,運動項を半分に分けて,片方だけ部分積分するとかなり楽になるそうです.が,今はこのままやります.)
  }

\end{frame}


\begin{frame}
  
  \frametitle{\subsecname}

  素直にヒントに従ってみる.ディラック場の展開は
  \begin{gather}
    \psi(x)
    =
    \int\frac{\dd^3 \mathbf{p}}{(2\pi)^3}
    \frac{1}{\sqrt{2E_{\mathbf{p}}}}
    \sum_{s}
    \left[  
      a^s_{\mathbf{p}}u^{s}(p)e^{-ip\cdot x}
      +
      {b_{\mathbf{p}}^s}^{\dag}v^s(p)e^{+ip\cdot x}
    \right]
    \\
    u^s(p)
    =
    \begin{pmatrix}
      \sqrt{p\cdot \sigma}\xi^s \\
      \sqrt{p\cdot \bar{\sigma}}\xi^s
    \end{pmatrix}
    ,\ 
    v^s(p)
    =
    \begin{pmatrix}
      \sqrt{p\cdot \sigma}(-i\sigma^2(\xi^s)^{*}) \\
      -\sqrt{p\cdot \bar{\sigma}}(-i\sigma^2(\xi^s)^{*}) 
    \end{pmatrix}
  \end{gather}
  だった.

\end{frame}


\begin{frame}%[noframenumbering,allowframebreaks]

  \frametitle{\subsecname}
  %\thispagestyle{empty}

  $\chi(x)$はleft-handedだったので,$i\sigma^2\chi^*$はright-handed.これらを組み合わせたDirac fieldを
  \begin{equation}
    \psi(x)
    =
    \begin{pmatrix}
      \chi(x) \\
      i\sigma^2 \chi^{*}(x)
    \end{pmatrix}
  \end{equation}
  とすると,$\psi(x)$の第2量子化と比べて
  \begin{equation}
    \left\{
      \begin{alignedat}{1}        
        \chi(x)
        &=    
        \int\frac{\dd^3 \mathbf{p}}{(2\pi)^3}
        \frac{1}{\sqrt{2E_{\mathbf{p}}}}
        \sum_s
        \left[  
          a_{\mathbf{p}}^s \cdot \sqrt{p\cdot \sigma}\xi^s \cdot e^{-ip\cdot x}
        \right.
        \\
        &\hspace{2.3cm}
        \left.
          +
          {b_{\mathbf{p}}^s}^{\dag} \cdot \sqrt{p\cdot \sigma}(-i\sigma^2(\xi^s)^{*}) \cdot e^{+ip\cdot x}
        \right]
        \\
        i\sigma^2\chi^{*}(x)
        &=
        \int\frac{\dd^3 \mathbf{p}}{(2\pi)^3}
        \frac{1}{\sqrt{2E_{\mathbf{p}}}}
        \sum_s
        \left[  
          a_{\mathbf{p}}^s \cdot \sqrt{p\cdot \bar{\sigma}}\xi^s \cdot e^{-ip\cdot x}
        \right.
        \\
        &\hspace{2.3cm}
        \left.
          -
          {b_{\mathbf{p}}^s}^{\dag} \cdot \sqrt{p\cdot \bar{\sigma}}(-i\sigma^2(\xi^s)^{*}) \cdot e^{+ip\cdot x}
        \right]
      \end{alignedat}
    \right.
    \label{spinor_quantized}
  \end{equation}
  と量子化できるだろう.

\end{frame}


\begin{frame}%[noframenumbering,allowframebreaks]

  \frametitle{\subsecname}
  %\thispagestyle{empty}

  \eqref{spinor_quantized}の第2式を$\chi(x)$について解いてみる.すなわち,$-i\sigma^2$を左から作用させて,全体のcomplex conjugateをとれば
  \begin{align}
    \chi(x)
    &=    
    \int\frac{\dd^3 \mathbf{p}}{(2\pi)^3}
    \frac{1}{\sqrt{2E_{\mathbf{p}}}}
    \sum_s
    \left[  
      b_{\mathbf{p}}^s \cdot \sqrt{p\cdot \sigma}\xi^s \cdot e^{-ip\cdot x}
    \right.
    \nonumber
    \\
    &\hspace{2.3cm}
    \left.
      +
      {a_{\mathbf{p}}^s}^{\dag} \cdot \sqrt{p\cdot \sigma}(-i\sigma^2(\xi^s)^{*}) \cdot e^{+ip\cdot x}
    \right]
  \end{align}
  となるので,$a_{\mathbf{p}}=b_{\mathbf{p}}$.($a_{\mathbf{p}}^s$のcomplex conjugateの定義は補足\ref{append_1}の\eqref{formula05}で.要するに,operatorに関してはhermitian conjugateだと思えばよい.)

\end{frame}

\begin{frame}%[noframenumbering,allowframebreaks]

  \frametitle{\subsecname}
  %\thispagestyle{empty} 

  したがって,
  \begin{align}
    \chi(x)
    &=    
    \int\frac{\dd^3 \mathbf{p}}{(2\pi)^3}
    \frac{1}{\sqrt{2E_{\mathbf{p}}}}
    \sum_s
    \left[  
      a_{\mathbf{p}}^s \cdot \sqrt{p\cdot \sigma}\xi^s \cdot e^{-ip\cdot x}
    \right.
    \nonumber
    \\
    &\hspace{2.3cm}
    \left.
      +
      {a_{\mathbf{p}}^s}^{\dag} \cdot \sqrt{p\cdot \sigma}(-i\sigma^2(\xi^s)^{*}) \cdot e^{+ip\cdot x}
    \right]
    \label{spinor_quantized2}
  \end{align}
  である.

  生成消滅演算子の正準交換関係はDirac場と同じで
  \begin{equation}
    \{
      a_{\mathbf{p}}^s
      ,
      {a_{\mathbf{q}}^r}^{\dag}
    \}
    =
    (2\pi)^{3}
    \delta^{(3)}(\mathbf{p}-\mathbf{q})\delta^{sr}
    \label{anti_com}
  \end{equation}
  でよい.

\end{frame}

\begin{frame}%[noframenumbering,allowframebreaks]

  \frametitle{\subsecname}
  %\thispagestyle{empty}

  あとは,$\chi(x)$の表示\eqref{spinor_quantized2}をハミルトニアン\eqref{hamiltonian}に代入するだけ.ただし,今は$H$はtime invariantなので,Schrödinger pictureで計算する.つまり,  
  \begin{align}
    \chi(\mathbf{x})
    &=    
    \int\frac{\dd^3 \mathbf{p}}{(2\pi)^3}
    \frac{1}{\sqrt{2E_{\mathbf{p}}}}
    \sum_s
    \left[  
      a_{\mathbf{p}}^s \sqrt{p\cdot \sigma}\xi^s e^{+i\mathbf{p}\cdot\mathbf{x}}
    \right.
    \nonumber
    \\
    &\hspace{2.3cm}
    \left.
      +
      {a_{\mathbf{p}}^s}^{\dag} \sqrt{p\cdot \sigma}(-i\sigma^2(\xi^s)^{*}) e^{-i\mathbf{p}\cdot\mathbf{x}}
    \right]
    \nonumber
    \\
    &=
    \int\frac{\dd^3 \mathbf{p}}{(2\pi)^3}
    \frac{1}{\sqrt{2E_{\mathbf{p}}}}
    e^{+i\mathbf{p}\cdot\mathbf{x}}
    \nonumber
    \\
    &\hspace{1cm}
    \times
    \sum_s    
    \left[  
      a_{\mathbf{p}}^s\sqrt{p\cdot \sigma}\xi^s
      +
      {a_{-\mathbf{p}}^{s\dag}} \sqrt{p\cdot \bar{\sigma}}(-i\sigma^2(\xi^s)^{*}) 
    \right]
  \end{align}
  を使う.これで少し計算がラクになる.

\end{frame}

\begin{frame}%[noframenumbering,allowframebreaks]

  \frametitle{\subsecname}
  %\thispagestyle{empty}

  \begin{align}
    H
    &=
    \int\dd^3\mathbf{x}
    \chi^{\dag}i\bm{\sigma}\cdot\bm{\nabla}\chi
    -    
    \frac{im}{2}
    \int\dd^3\mathbf{x}
    (\chi^{T}\sigma^{2}\chi-\chi^{\dag}\sigma^2\chi^{*})
    \nonumber
    \\
    &=\cdots
    \nonumber
    \\
    &=
    \uwave{
      \int\frac{\dd^3\mathbf{p}}{(2\pi)^3}
      \frac{1}{2E_{\mathbf{p}}}\sum_{s,r}
      (a_{\mathbf{p}}^{r\dag}\xi^{r\dag}\sqrt{p\cdot\sigma}+a_{-\mathbf{p}}^{r}(i\xi^{rT}\sigma^2)\sqrt{p\cdot\bar{\sigma}})
    }
    \nonumber
    \\
    &\hspace{2cm}
    \underset{=A_1}{
      \uwave{
      \times
      \bm{\sigma}\cdot(-\mathbf{p})
      (a_{\mathbf{p}}^s\sqrt{p\cdot\sigma}\xi^s+a_{-\mathbf{p}}^{s\dag}\sqrt{p\cdot\bar{\sigma}}(-i\sigma^2\xi^{s*}))
      }
    }
    \nonumber
    \\
    &\quad
    -
    \frac{im}{2}
    \underset{=A_2}{
      \uline{
        \int\dd^3\mathbf{x}
        \left(  
          \chi^{T}\sigma^{2}\chi
          -
          \chi^{\dag}\sigma^2\chi^{*}
        \right)
      }
    }
    .
    \nonumber
  \end{align}

\end{frame}

\begin{frame}%[noframenumbering,allowframebreaks]

  \frametitle{\subsecname}
  %\thispagestyle{empty}

  $A_1,A_2$をそれぞれ計算すると
  \begin{align}
    A_1
    &=
    -
    \int\frac{\dd^3\mathbf{p}}{(2\pi)^3}
    \frac{|\mathbf{p}|^2}{2E_{\mathbf{p}}}
    \sum_{s}
    (
      a_{\mathbf{p}}^{s\dag}a_{\mathbf{p}}^{s}
      -
      a_{\mathbf{p}}^{s}a_{\mathbf{p}}^{s\dag}
    )
    +
    \alpha
    \label{a_1}
    \\
    A_2
    &=
    -i
    \int\frac{\dd^3\mathbf{p}}{(2\pi)^3}
    \frac{m}{E_{\mathbf{p}}}
    \sum_{s}
    (      
      a_{\mathbf{p}}^{s}a_{\mathbf{p}}^{s\dag}
      -
      a_{\mathbf{p}}^{s\dag}a_{\mathbf{p}}^{s}
    )
    +
    \frac{2}{im}\alpha
    \label{a_23}
  \end{align}
  となる.$\xi^1=(1,0),\xi^2=(0,1)$として計算した.細かい計算は補足\ref{append_5}でやっている.なお,$\alpha$は
  \begin{align}
    \alpha
    &=
    \int
    \frac{\dd^3\mathbf{p}}{(2\pi)^3}
    \frac{im}{2E_{\mathbf{p}}}
    \sum_{s,r}
    \left[  
      a_{-\mathbf{p}}^ra_{\mathbf{p}}^s
      \xi^{rT}\sigma^2\mathbf{p}\cdot\bm{\sigma}\xi^s
    \right.
    \nonumber
    \\
    &\hspace{5cm}
    \left.
      -
      a_{\mathbf{p}}^{r\dag}a_{-\mathbf{p}}^{s\dag}
      \xi^{rT}\mathbf{p}\cdot\bm{\sigma}\sigma^2\xi^s      
    \right]
  \end{align}
  であり,ハミルトニアンを計算するときは相殺される.

\end{frame}

\begin{frame}%[noframenumbering,allowframebreaks]

  \frametitle{\subsecname}
  %\thispagestyle{empty}

  \begin{align}
    H
    &=
    \int\frac{\dd^3\mathbf{p}}{(2\pi)^3}
    \frac{-|\mathbf{p}|^3+m^2}{2E_{\mathbf{p}}}
    \sum_{s}
    (
      a_{\mathbf{p}}^{s\dag}a_{\mathbf{p}}^{s}
      -
      a_{\mathbf{p}}^{s}a_{\mathbf{p}}^{s\dag}
    )
    \nonumber
    \\
    &=
    \int\frac{\dd^3\mathbf{p}}{(2\pi)^3}E_{\mathbf{p}}
    \sum_{s}a_{\mathbf{p}}^{s\dag}a_{\mathbf{p}}^{s}
    -
    \delta^{(3)}(\mathbf{0})\int\dd^3\mathbf{p}\frac{E_{\mathbf{p}}}{2}\sum_{s}
    .
  \end{align}
  
  無限大の項は(エネルギーの基準を変えると思えば)無視していいので
  \begin{equation}
    H
    =    
    \int\frac{\dd^3\mathbf{p}}{(2\pi)^3}E_{\mathbf{p}}
    \sum_{s}a_{\mathbf{p}}^{s\dag}a_{\mathbf{p}}^{s}
  \end{equation}
  となる.

\end{frame}


\subsection{まとめ}

\begin{frame}
  
  \frametitle{\subsecname}

  left-handed spnor$\chi(x,t)$のマヨラナ理論
  \begin{equation}
    \mathcal{L}_M
    =
    \chi^{\dag}i\bar{\sigma}\cdot\partial\chi
    +
    \frac{im}{2}
    (  
      \chi^{T}\sigma^2\chi
      -
      \chi^{\dag}\sigma^2\chi^{*}
    )
    \tag{\ref{majorana_lagrangian}}
  \end{equation}
  を第2量子化すると
  \begin{align}
    \chi(x)
    &=    
    \int\frac{\dd^3 \mathbf{p}}{(2\pi)^3}
    \frac{1}{\sqrt{2E_{\mathbf{p}}}}
    \sum_s
    \left[  
      a_{\mathbf{p}}^s \cdot \sqrt{p\cdot \sigma}\xi^s \cdot e^{-ip\cdot x}
    \right.
    \nonumber
    \\
    &\hspace{2.3cm}
    \left.
      +
      {a_{\mathbf{p}}^s}^{\dag} \cdot \sqrt{p\cdot \sigma}(-i\sigma^2(\xi^s)^{*}) \cdot e^{+ip\cdot x}
    \right]
    \tag{\ref{spinor_quantized2}}
    \\
    &
    \{
      a_{\mathbf{p}}^s
      ,
      {a_{\mathbf{q}}^r}^{\dag}
    \}
    =
    (2\pi)^{3}
    \delta^{(3)}(\mathbf{p}-\mathbf{q})\delta^{sr}
    \tag{\ref{anti_com}}
  \end{align}
  となる.
  
\end{frame}


\begin{frame}

  \frametitle{\subsecname}

  Dirac場$\psi(x)$と同様に,マヨラナ場も4成分表示ができ
  \begin{eqnarray}
    \psi_{M}(x)
    =
    \begin{pmatrix}
      \chi(x) \\
      i\sigma^2\chi^{\ast}(x)
    \end{pmatrix}
  \end{eqnarray}
  という4成分spinor場を構成することができるはず.

  この4成分場のcharge conjugateを考えるが,そもそも$Ca_pC=a_p$なので,
  \begin{equation}
    C\psi_MC
    =
    \psi_M
    .
  \end{equation}
  $\rightarrow$中性粒子.
  \\
  
  スカラー場と同じで,粒子自身が反粒子であることに起因.


\end{frame}

\section{参考文献}

\begin{frame}[noframenumbering]

  \frametitle{\secname}
  \thispagestyle{empty}

  \beamertemplatetextbibitems

  \begin{thebibliography}{99}

    \bibitem{peskin}    
    M. Peskin, D. Schroeder, \textit{An Introduction to Quantum Field Theory,} Addison-Wesley Pub. Co, Reading, Mass, 1995.

    \bibitem{kugo1}
    九後汰一郎, 『ゲージ場の量子論 I』, 新物理学シリーズ, 培風館, 1989年.

    \bibitem{solution}
    Z. Xianyu, \textit{A Complete Solution to Problems in “An Introduction to Quantum Field Theory” by Peskin and Schroeder,} 2016, \url{https://zzxianyu.files.wordpress.com/2017/01/peskin_problems.pdf}.

  \end{thebibliography}

\end{frame}


\section{Appendix}

\begin{frame}[noframenumbering]

  \frametitle{\secname}
  \thispagestyle{empty}

  補足

\end{frame}

\subsection{公式集}
\label{append_1}

\begin{frame}[noframenumbering]

  \frametitle{補足\ \subsecname}
  \thispagestyle{empty}

  ここでは,いくつかの公式を示しておく:
  \begin{gather}
    \bar{\sigma}^{\mu}a_L
    =
    \Lambda^{\mu}_{\ \nu}a_R\bar{\sigma}^{\nu}
    ,
    \label{formula01}
    \\
    \sigma^2 a_L^{*}
    =
    a_R \sigma^2
    ,
    \label{formula02}
    \\    
    (\bar{\sigma}^{*})^{\mu}\sigma^2\bar{\sigma}^{\nu}
    \partial_\mu\partial_\nu
    =
    \sigma^2\partial^2
    ,
    \label{formula03}
    \\
    \sigma^2(\sigma^{\mu})^{T}\sigma^2
    =
    -\bar{\sigma}^{\mu}
    ,
    \label{formula04}
    \\
    (a_{\mathbf{p}}^s)^{*}
    =    
    (a_{\mathbf{p}}^s)^{\dag}
    .
    \label{formula05}
  \end{gather}
  ただし,最後の\eqref{formula05}は「operatorのcomplex conjugateとは?」という定義の問題である.

\end{frame}

\begin{frame}[noframenumbering]

  \frametitle{補足 \subsecname}
  \thispagestyle{empty}  

  \begin{equation}
    \left\{
      \begin{alignedat}{1}
        (p\cdot\sigma)^2
        &=
        E_{\mathbf{p}}^2
        +
        |\mathbf{p}|^2
        -
        2E_{\mathbf{p}}\mathbf{p}\cdot\bm{\sigma}
        \\
        (p\cdot\bar{\sigma})^2
        &=
        E_{\mathbf{p}}^2
        +
        |\mathbf{p}|^2
        +
        2E_{\mathbf{p}}\mathbf{p}\cdot\bm{\sigma}
        ,
        \\
        (p\cdot\sigma)(p\cdot\bar\sigma)
        &=
        (p\cdot\bar\sigma)(p\cdot\sigma)
        =
        m^2
        ,
        \\
        \mathbf{p}\cdot\bm{\sigma}
        &=
        \frac{p\cdot\bar{\sigma}-p\cdot\sigma}{2}
        .
      \end{alignedat}
    \right.
    \label{formula06}
  \end{equation}

  \begin{equation}
    \left\{
      \begin{alignedat}{1}
        \sqrt{p\cdot\sigma}
        \mathbf{p}\cdot\bm{\sigma}
        \sqrt{p\cdot\sigma}
        &=
        -
        |\mathbf{p}|^2
        +
        E_{\mathbf{p}}\mathbf{p}\cdot\sigma
        ,
        \\
        \sqrt{p\cdot\sigma}
        \mathbf{p}\cdot\bm{\sigma}
        \sqrt{p\cdot\bar{\sigma}}
        &=
        \sqrt{p\cdot\bar{\sigma}}
        \mathbf{p}\cdot\bm{\sigma}
        \sqrt{p\cdot\sigma}
        \\
        &=   
        m\mathbf{p}\cdot\bm{\sigma}
        ,
        \\
        \sqrt{p\cdot\bar{\sigma}}
        \mathbf{p}\cdot\bm{\sigma}
        \sqrt{p\cdot\bar{\sigma}}
        &=
        |\mathbf{p}|^2
        +
        E_{\mathbf{p}}\mathbf{p}\cdot\sigma
        .
      \end{alignedat}
    \right.
    \label{formula07}
  \end{equation}

\end{frame}

\begin{frame}[noframenumbering]

  \frametitle{補足 \subsecname}
  \thispagestyle{empty}  

  \begin{equation}
    \left\{
      \begin{alignedat}{1}
        \sqrt{p\cdot\sigma^{T}}
        \sigma^2
        \sqrt{p\cdot\bar{\sigma}}
        &=
        \sigma^2
        (p\cdot\bar{\sigma})
        ,
        \\
        \sqrt{p\cdot\bar{\sigma}^{T}}
        \sigma^2
        \sqrt{p\cdot\sigma}
        &=
        \sigma^2(p\cdot\sigma)
        ,
        \\
        \sqrt{p\cdot\bar{\sigma}^{T}}
        \sigma^2
        \sqrt{p\cdot\bar{\sigma}}
        &=
        \sqrt{p\cdot\sigma^{T}}
        \sigma^2
        \sqrt{p\cdot\sigma}
        =
        m\sigma^2
        .
      \end{alignedat}
    \right.
    \label{formula08}
  \end{equation}

  \vspace{10pt}

  ちなみに,出回ってる解答集\cite{solution}はここらへんの公式が間違っている.結果は合っていたけど.

\end{frame}

\begin{frame}[noframenumbering]

  \frametitle{補足 \subsecname}
  \thispagestyle{empty}  

  Dirac theoryとMajorana theoryの比較.

  \begin{itemize}

    \item 

    \uline{Dirac theory}

    \begin{equation}
      \mathcal{L}_{D}
      =
      i\chi_1^{\dag}\bar{\sigma}^{\mu}\partial_{\mu}\chi_1
      +
      i\chi_2^{\dag}\bar{\sigma}^{\mu}\partial_{\mu}\chi_2
      +
      im(\chi_{2}^{T}\sigma^2\chi_1-\chi_1^{\dag}\sigma^2\chi_2^{\ast})
      .
    \end{equation}

    \item 

    \uline{Majorana theory}

    \begin{equation}    
      \mathcal{L}_{M}
      =
      \chi^{\dag}i\bar{\sigma}\cdot\partial\chi
      +
      \frac{im}{2}(\chi^{T}\sigma^{2}\chi-\chi^{\dag}\sigma^2\chi^{*})
    \end{equation}

  \end{itemize}

  \vspace{10pt}

  Dirac thoryで$\chi_1=\chi_2=\chi$とすれば,
  \begin{equation}
    \mathcal{L}_{D}
    =
    2\mathcal{L}_{M}
    .
  \end{equation}


\end{frame}

\begin{frame}[noframenumbering]

  \frametitle{補足 \subsecname}
  \thispagestyle{empty}

  \uline{\eqref{formula01}の証明}:

  \vspace{10pt}  

  次の関係
  \begin{equation}
    \Lambda_{\frac{1}{2}}^{-1}
    \gamma_{\mu}
    \Lambda_{\frac{1}{2}}
    =
    \Lambda^{\mu}_{\ \nu}\gamma^{\nu}
  \end{equation}
  が成立しているので(本文p.42の式(3.29)),次の表現
  \begin{equation}
    \Lambda_{\frac{1}{2}}
    =
    \begin{pmatrix}
      a_L & 0 \\
      0 & a_R
    \end{pmatrix}
    ,\ 
    \gamma^{\mu}
    =
    \begin{pmatrix}
      0 & \sigma^{\mu} \\
      \bar{\sigma}^{\mu} & 0
    \end{pmatrix}
  \end{equation}
  をとれば
  \begin{gather}
    a_{R}^{-1}\bar{\sigma}^{\mu}a_{L}
    =
    \Lambda^{\mu}_{\ \nu}\sigma^{\nu}
    \quad
    (
      \&
      \quad
      a_{L}^{-1}\sigma^{\mu}a_R
      =
      \Lambda^{\mu}_{\ \nu}\sigma^{\nu}
    )
    \nonumber
    \\
    \rightarrow\quad
    \bar{\sigma}^{\mu}a_{L}
    =
    \Lambda^{\mu}_{\ \nu}a_{R}\bar{\sigma}^{\nu}
    .
  \end{gather}

\end{frame}

\begin{frame}[noframenumbering]

  \frametitle{補足 \subsecname}
  \thispagestyle{empty}

  \uline{\eqref{formula02}の証明}:

  \vspace{10pt}

  $\sigma^2\bm{\sigma}^{*}=-\bm{\sigma}\sigma^2$を使えば
  \begin{align}
    \sigma^2 a_{L}^{*}
    &=
    \sigma^2\exp\left[  
      i\bm{\theta}\cdot\frac{\bm{\sigma}^{*}}{2}
      -
      \bm{\beta}\cdot\frac{\bm{\sigma}^{*}}{2}
    \right]
    \nonumber
    \\
    &=
    \exp\left[  
      -
      i\bm{\theta}\cdot\frac{\bm{\sigma}}{2}
      +
      \bm{\beta}\cdot\frac{\bm{\sigma}}{2}
    \right]
    \sigma^2
    =
    a_R \sigma^2
    .
  \end{align}

\end{frame}

\begin{frame}[noframenumbering]

  \frametitle{補足 \subsecname}
  \thispagestyle{empty}

  \uline{\eqref{formula03}の証明}:

  \vspace{10pt}

  $\bm{\sigma}^{*}\sigma^2=-\sigma^2\bm{\sigma}$より
  \begin{equation}
    (\bar{\sigma}^{\mu})^{*}\sigma^2\bar{\sigma}^{\nu}\partial_{\mu}\partial_{\nu}
    =
    \sigma^2\sigma^{\mu}\bar{\sigma}^{\nu}\partial_{\mu}\partial_{\nu}
    .
  \end{equation}
  $\mu\neq\nu$なら,$\sigma^{\mu}$と$\bar{\sigma}^{\nu}$はanti-commute.一方,$\partial_{\mu}$と$\partial_{\nu}$はもちろんcommuteなので,対角成分$\mu=\nu$のみが残る.よって,
  \begin{equation}
    \sigma^2\sigma^{\mu}\bar{\sigma}^{\nu}\partial_{\mu}\partial_{\nu}
    =
    \sigma^2 \partial^2
    .
  \end{equation}

\end{frame}

\begin{frame}[noframenumbering]

  \frametitle{補足 \subsecname}
  \thispagestyle{empty}

  \uline{\eqref{formula04}の証明}:

  \vspace{10pt}

  $\mu=0$なら
  \begin{equation}
    \sigma^2(\sigma^0)^{T}\sigma^2
    =
    -\sigma^0(\sigma^2)^2
    =
    -\sigma^0
    .
  \end{equation}
  $\mu=i$なら
  \begin{equation}
    \sigma^2\bm{\sigma}^{T}\sigma^2
    =
    -(\bm{\sigma}\sigma^2)^{T}\sigma^2
    =
    \bm{\sigma}^{\dag}(\sigma^2)^2
    =
    \bm{\sigma}
    .
  \end{equation} 
  ただし,$\bm{\sigma}\sigma^2=-\sigma^2\bm{\sigma}^{*}$を用いた. 

\end{frame}


\begin{frame}[noframenumbering]

  \frametitle{補足 \subsecname}
  \thispagestyle{empty}

  \uline{\eqref{formula05}の証明}:

  \vspace{10pt}

  まず,場$\chi^{\dag}$の複素共役の定義を
  \begin{equation}
    \chi^{*}(x)
    \equiv
    (\chi^{T})^{\dag}(x)
  \end{equation}
  とする.両辺の展開を考えてみると
  \begin{align}    
    {\chi^{T}(x)}^{\dag}
    &=
    \int\frac{\dd^3\mathbf{p}}{(2\pi)^3}
    \frac{1}{\sqrt{E}_{\mathbf{p}}}
    \sum_{s}
    \left[  
      a_{\mathbf{p}}^{s\dag}
      \sqrt{p\cdot\sigma^{*}}\xi^{s*}e^{+ip\cdot x}
      +
      \cdots
    \right]
    \\    
    \chi^{*}(x)
    &=
    \int\frac{\dd^3\mathbf{p}}{(2\pi)^3}
    \frac{1}{\sqrt{E}_{\mathbf{p}}}
    \sum_{s}
    \left[  
      a_{\mathbf{p}}^{s*}
      \sqrt{p\cdot\sigma^{*}}\xi^{s*}e^{+ip\cdot x}
      +
      \cdots
    \right]
  \end{align}
  となるので,比較すると\eqref{formula05}が出てくる.(転置$T$が状態空間に作用しないことが原因.)

\end{frame}


\begin{frame}[noframenumbering]

  \frametitle{補足 \subsecname}
  \thispagestyle{empty}

  \uline{\eqref{formula06}の証明}:

  \vspace{10pt}

  $\sqrt{p\cdot\sigma},\sqrt{p\cdot\bar{\sigma}}$は
  \begin{equation}
    \left.
      \begin{alignedat}{1}
        &\sqrt{p\cdot\sigma}
        \\
        &\sqrt{p\cdot\bar{\sigma}}
      \end{alignedat}
    \right\}
    =
    \sqrt{E_{\mathbf{p}}+|\mathbf{p}|}
    \frac{1\mp\hat{\mathbf{p}}\cdot\sigma}{2}
    +
    \sqrt{E_{\mathbf{p}}-|\mathbf{p}|}
    \frac{1\pm\hat{\mathbf{p}}\cdot\sigma}{2}
  \end{equation}
  である.(上が$\sqrt{p\cdot\sigma}$.)ただし,$\hat{\mathbf{p}}\equiv\mathbf{p}/|\mathbf{p}|$である.ここで,
  \begin{equation}
    \left(  
      \frac{1\pm\hat{\mathbf{p}}\cdot\sigma}{2}
    \right)^2
    =    
    \frac{1\pm\hat{\mathbf{p}}\cdot\sigma}{2}
    ,\ 
    \frac{1\pm\hat{\mathbf{p}}\cdot\sigma}{2}
    \cdot
    \frac{1\mp\hat{\mathbf{p}}\cdot\sigma}{2}
    =
    0
  \end{equation}
  の関係(射影)を用いればよい.

\end{frame}

\begin{frame}[noframenumbering]

  \frametitle{補足 \subsecname}
  \thispagestyle{empty}

  \uline{\eqref{formula06}の証明}:

  \vspace{10pt}

  例えば
  \begin{align}
    \left.
      \begin{alignedat}{1}
        &(p\cdot\sigma)^2
        \\
        &(p\cdot\bar{\sigma})^2
      \end{alignedat}
    \right\}
    &=
    \left(  
      \sqrt{E_{\mathbf{p}}+|\mathbf{p}|}
      \frac{1\mp\hat{\mathbf{p}}\cdot\sigma}{2}
      +
      \sqrt{E_{\mathbf{p}}-|\mathbf{p}|}
      \frac{1\pm\hat{\mathbf{p}}\cdot\sigma}{2}
    \right)^4
    \nonumber
    \\
    &=
    (E_{\mathbf{p}}+|\mathbf{p}|)^2
    \frac{1\mp\hat{\mathbf{p}}\cdot\sigma}{2}
    +
    (E_{\mathbf{p}}-|\mathbf{p}|)^2
    \frac{1\pm\hat{\mathbf{p}}\cdot\sigma}{2}
    \nonumber
    \\
    &=
    E_{\mathbf{p}}^2+|\mathbf{p}|^2\mp 2E_{\mathbf{p}}\mathbf{p}\cdot\bm{\sigma}
  \end{align}
  である.(上が$(p\cdot\sigma)^2$.)

\end{frame}

\begin{frame}[noframenumbering]

  \frametitle{補足 \subsecname}
  \thispagestyle{empty}

  \uline{\eqref{formula06}の証明}:

  \vspace{10pt}

  同様にして
  \begin{align}
    &\hspace{-0.5cm}
    (p\cdot\sigma)
    (p\cdot\bar{\sigma})  
    =    
    (p\cdot\bar{\sigma}) 
    (p\cdot\sigma)
    \nonumber
    \\
    &=
    (E_{\mathbf{p}}^2-|\mathbf{p}|^2)
    \frac{1+\hat{\mathbf{p}}\cdot\sigma}{2}
    +    
    (E_{\mathbf{p}}^2-|\mathbf{p}|^2)
    \frac{1-\hat{\mathbf{p}}\cdot\sigma}{2}
    \nonumber
    \\
    &=
    E_{\mathbf{p}}^2-|\mathbf{p}|^2
    =
    m^2
    .
  \end{align}

  また,$p\cdot\sigma=p^{0}\sigma^{0}-\mathbf{p}\cdot\bm{\sigma},p\cdot\bar{\sigma}=p^{0}\sigma^{0}+\mathbf{p}\cdot\bm{\sigma}$より
  \begin{equation}    
    \mathbf{p}\cdot\bm{\sigma}
    =
    \frac{p\cdot\bar{\sigma}-p\cdot\sigma}{2}
    .
  \end{equation}

\end{frame}

\begin{frame}[noframenumbering]

  \frametitle{補足 \subsecname}
  \thispagestyle{empty}

  \uline{\eqref{formula07}の証明}:

  \vspace{5pt}

  \eqref{formula06}を使えばすぐに示すことができる:
  \begin{align}
    \sqrt{p\cdot\sigma}
    \mathbf{p}\cdot\bm{\sigma}
    \sqrt{p\cdot\sigma}
    &=
    \frac{1}{2}
    \left(  
      (p\cdot\sigma)(p\cdot\bar{\sigma})
      -
      (p\cdot\sigma)^2
    \right)
    \nonumber
    \\
    &=
    -|\mathbf{p}|^2
    +
    E_{\mathbf{p}}\mathbf{p}\cdot\sigma
    ,
    \\
    \sqrt{p\cdot\bar{\sigma}}
    \mathbf{p}\cdot\bm{\sigma}
    \sqrt{p\cdot\bar{\sigma}}
    &=
    \frac{1}{2}
    \left(  
      (p\cdot\bar{\sigma})^2
      -
      (p\cdot\sigma)(p\cdot\bar{\sigma})
    \right)
    \nonumber
    \\
    &=
    |\mathbf{p}|^2
    +
    E_{\mathbf{p}}\mathbf{p}\cdot\sigma
    ,
    \\
    \sqrt{p\cdot\sigma}
    \mathbf{p}\cdot\bm{\sigma}
    \sqrt{p\cdot\bar{\sigma}}
    &=
    \frac{1}{2}
    \left.(
      \sqrt{(p\cdot\sigma)(p\cdot\bar{\sigma})}(p\cdot\bar{\sigma})
    \right.
    \nonumber
    \\
    &\hspace{2cm}
    \left.
      -
      \sqrt{(p\cdot\sigma)(p\cdot\bar{\sigma})}(p\cdot\sigma)
    \right)
    \nonumber
    \\
    &=
    m\mathbf{p}\cdot\bm{\sigma}
    =
    \sqrt{p\cdot\bar{\sigma}}
    \mathbf{p}\cdot\bm{\sigma}
    \sqrt{p\cdot\sigma}
    .
  \end{align}

\end{frame}


\begin{frame}[noframenumbering]

  \frametitle{補足 \subsecname}
  \thispagestyle{empty}

  \uline{\eqref{formula08}の証明}:

  \vspace{5pt}

  $\bm{\sigma}^{T}\sigma^2=-\sigma^2\bm{\sigma}^{\dag}=-\sigma^2\bm{\sigma}$なので,例えば
  \begin{equation}
    \sqrt{p\cdot\sigma^{T}}\sigma^2
    =
    \sigma^2\sqrt{p\cdot\bar{\sigma}}
  \end{equation}
  のようになる.よって,
  \begin{align}
    \sqrt{p\cdot\sigma^{T}}
    \sigma^2
    \sqrt{p\cdot\bar{\sigma}}
    &=
    \sigma^2
    (p\cdot\bar{\sigma})
    \nonumber
    ,
    \\
    \sqrt{p\cdot\bar{\sigma}^{T}}
    \sigma^2
    \sqrt{p\cdot\sigma}
    &=
    \sigma^2
    (p\cdot\sigma)
    \nonumber
    ,
    \\
    \sqrt{p\cdot\bar{\sigma}^{T}}
    \sigma^2
    \sqrt{p\cdot\bar{\sigma}}
    &=
    \sqrt{p\cdot\sigma^{T}}
    \sigma^2
    \sqrt{p\cdot\sigma}
    \nonumber
    \\
    &=
    \sigma^2\sqrt{(p\cdot\sigma)(p\cdot\bar{\sigma})}
    =
    m\sigma^2
    .
  \end{align}

\end{frame}


\subsection{右巻きと左巻きのspinorの変換の関係}
\label{unitary}

\begin{frame}[noframenumbering]
  
  \frametitle{補足 \subsecname}
  \thispagestyle{empty}

  (3)の問題文「 $\psi_{R}$の変換は$\psi_{L}$の複素共役の変換とユニタリー同値」の意味について.

  \vspace{10pt}

  $\psi_{R}$の変換は$a_R$,$\psi_L^*$の変換は$a_L$であり,\eqref{formula02}より
  \begin{equation}
    a_R
    =
    \sigma^2 a_L \sigma^2
  \end{equation}
  と結びつけられる.$\sigma^2$はユニタリーなので,$a_R$と$a_L^*$はユニタリー同値である.

  \vspace{10pt}

  したがって,後々量子化して状態空間を考えることになるので,right-handed spinorは$\psi_L^*$という形で導入すればよい.

\end{frame}


\subsection{\texorpdfstring{%
  \eqref{lagrangian}%
  }{
  ラグランジアン
}の途中計算}
\label{q3_appendix}

\begin{frame}[noframenumbering]

  \frametitle{補足\ \subsecname}
  \thispagestyle{empty}

  \eqref{lagrangian}の計算を一部省略していたので,ここで補足を.ラグランジアンを途中まで計算すると
  \begin{align}
    \mathcal{L}
    &=
    \bar{\psi}(im\gamma^{\mu}\partial_{\mu}-m)\psi
    \nonumber
    \\
    &=
    \begin{pmatrix}
      \psi_{R}^{\dag} & \psi_{L}^{\dag}
    \end{pmatrix}
    \begin{pmatrix}
      -m & i\sigma^{\mu}\partial_{\mu} \\
      i\bar{\sigma}^{\mu}\partial_{\mu} & -m
    \end{pmatrix}
    \begin{pmatrix}
      \psi_{L} \\
      \psi_{R}
    \end{pmatrix}
    \nonumber
    \\
    &=
    -m\psi_{R}^{\dag}\psi_{L}
    +
    i\psi_{R}^{\dag}\sigma^{\mu}\partial_{\mu}\psi_{R}
    \nonumber
    \\
    &\qquad
    +
    i\psi_{L}^{\dag}\bar{\sigma}^{\mu}\partial_{\mu}\psi_{L}
    -
    m\psi_{L}^{\dag}\psi_{R}
    \label{append3_1}
  \end{align}

\end{frame}

\begin{frame}[noframenumbering]

  \frametitle{補足\ \subsecname}
  \thispagestyle{empty}

  $\psi_{L}=\chi_1,\psi_{R}=i\sigma^2\chi_2^*$を\eqref{append3_1}に代入すると
  \begin{align}
    &\quad    
    -m\psi_{R}^{\dag}\psi_{L}
    +
    i\psi_{R}^{\dag}\sigma^{\mu}\partial_{\mu}\psi_{R}
    \nonumber
    \\
    &\qquad
    +
    i\psi_{L}^{\dag}\bar{\sigma}^{\mu}\partial_{\mu}\psi_{L}
    -
    m\psi_{L}^{\dag}\psi_{R}
    \nonumber
    \\
    &=
    -m(-i\chi_2^T\sigma^2)\chi_1+i(-i\chi_2^T\sigma^2)\sigma^{\mu}\partial_{\mu}(-i\sigma^2\chi_2^*)
    \nonumber
    \\
    &\qquad
    +
    i\chi_1^{\dag}
    \bar{\sigma}^{\mu}\partial_{\mu}\chi_1
    -
    m\chi_1^{\dag}(i\sigma^2)\chi_2^*
    \nonumber
    \\
    &=
    i\chi_1^{\dag}\bar{\sigma}^{\mu}\partial_{\mu}\chi_1
    +
    i\chi_2^T\uwave{\sigma^2\sigma^{\mu}\sigma^2\partial_{\mu}}\chi_2^*
    \nonumber
    \\
    &\qquad
    +
    im\chi_2^T\sigma^2\chi_1
    -
    im\chi_1^{\dag}\sigma^2\chi_2^*
    .
    \label{append3_2}
  \end{align}
  あとは$\uwave{\qquad}$の部分を整理すればよい.

\end{frame}

\begin{frame}[noframenumbering]

  \frametitle{補足\ \subsecname}
  \thispagestyle{empty}

  ここは
  \begin{equation}
    \sigma^2\sigma^{\mu}\sigma^2\partial_{\mu}
    =
    \sigma^2(\sigma^{0}\partial_{0}+\bm{\sigma}\cdot\bm{\nabla})\sigma^2
    =
    (\bar{\sigma}^{\mu})^*\partial_{\mu}
  \end{equation}
  となるので,
  \begin{align}
    \chi_2^T
    \uwave{
      \sigma^2\sigma^{\mu}\sigma^2\partial_{\mu}
    }
    \chi_2^*
    &=
    \chi_2^T
    (\bar{\sigma}^{\mu})^*\partial_{\mu}
    \chi_2^*
    \nonumber
    \\
    &=
    (
      \chi_2^T
      (\bar{\sigma}^{\mu})^*\partial_{\mu}
      \chi_2^*
    )^{T}
    \nonumber
    \\
    &=
    -
    \chi_2^{\dag}(\bar{\sigma}^{\mu})^{\dag}\overset{\leftarrow}{\partial}_{\mu}\chi_2
  \end{align}
  である.ただし,$\chi_2$がGrassmann numberであることから,転置をとったときにマイナスがでてくる.これを部分積分すれば,\eqref{lagrangian}を得る.

\end{frame}



\subsection{(5)の途中計算}
\label{append_5}

\begin{frame}[noframenumbering]
  
  \frametitle{補足\ \subsecname}
  \thispagestyle{empty}
  
  公式を駆使して\eqref{a_1},\eqref{a_23}を導出する.なお,今回行う計算はほとんど$\mathbf{p}$の積分とスピンの総和$\sum$の中で行われるので,$\mathbf{p}$の符号の反転やスピン添え字の入れ替えを行ってもよい.これらの事実を今後はゴリゴリ使っていく.

  \vspace{10pt}

  まずは\eqref{a_1}から.$\sum$以下を素直に書き下してみると
  \begin{align}
    \sum_{s,r}
    &
    \left[\ 
      a_{\mathbf{p}}^{r\dag}\xi^{r\dag}
      \sqrt{p\cdot\sigma}\mathbf{p}\cdot\bm{\sigma}\sqrt{p\cdot\sigma}
      a_{\mathbf{p}}^s\xi^s
    \right.
    \nonumber
    \\
    &
    +
    a_{\mathbf{p}}^{r\dag}\xi^{r\dag}
    \sqrt{p\cdot\sigma}\mathbf{p}\cdot\bm{\sigma}\sqrt{p\cdot\bar{\sigma}}
    \sigma^2 a_{-\mathbf{p}}^{s\dag}(-i\xi^{s*})
    \nonumber
    \\
    &
    +
    a_{-\mathbf{p}}^{r}(i\xi^{rT})\sigma^2
    \sqrt{p\cdot\bar{\sigma}}\mathbf{p}\cdot\bm{\sigma}\sqrt{p\cdot\sigma}
    a_{\mathbf{p}}^s\xi^s
    \nonumber
    \\
    &
    +
    \left.
      a_{-\mathbf{p}}^{r}(i\xi^{rT})\sigma^2
      \sqrt{p\cdot\bar{\sigma}}\mathbf{p}\cdot\bm{\sigma}\sqrt{p\cdot\bar{\sigma}}
      \sigma^2 a_{-\mathbf{p}}^{s\dag}(-i\xi^{s*})
    \right]
  \end{align}
  である.

\end{frame}

\begin{frame}[noframenumbering]
  
  \frametitle{補足\ \subsecname}
  \thispagestyle{empty}

  第1項と第4項を混ぜて計算すると
  \begin{align}
    &\quad
    a_{\mathbf{p}}^{r\dag}\xi^{r\dag}
    \sqrt{p\cdot\sigma}\mathbf{p}\cdot\bm{\sigma}\sqrt{p\cdot\sigma}
    a_{\mathbf{p}}^s\xi^s
    \nonumber
    \\
    &\qquad
    +
    a_{-\mathbf{p}}^{r}(i\xi^{rT})\sigma^2
    \sqrt{p\cdot\bar{\sigma}}\mathbf{p}\cdot\bm{\sigma}\sqrt{p\cdot\bar{\sigma}}
    \sigma^2 a_{-\mathbf{p}}^{s\dag}(-i\xi^{s*})
    \nonumber
    \\
    &=
    a_\mathbf{p}^{r\dag}a_{\mathbf{p}}^s
    \xi^{r\dag}
    (-|\mathbf{p}|^2+E_{\mathbf{p}\mathbf{p}\cdot\bm{\sigma}})
    \xi^s
    \nonumber
    \\
    &\qquad
    +a_{\mathbf{p}}^ra_{\mathbf{p}}^{s\dag}
    \xi^{r\dag}\sigma^2
    (|\mathbf{p}|^2-E_{\mathbf{p}\mathbf{p}\cdot\bm{\sigma}})
    \sigma^2\xi^s
    \nonumber
    \\
    &=
    -|\mathbf{p}|^2\delta^{rs}(a_\mathbf{p}^{r\dag}a_{\mathbf{p}}^s-a_{\mathbf{p}}^ra_{\mathbf{p}}^{s\dag})
    \nonumber
    \\
    &\quad
    +
    E_{\mathbf{p}}
    ( 
      \cancel{ 
        a_\mathbf{p}^{r\dag}a_{\mathbf{p}}^s
        \xi^{r\dag}\mathbf{p}\cdot\bm{\sigma}\xi^s
      }
      +
      \underbrace{
        \cancel{
          a_{\mathbf{p}}^ra_{\mathbf{p}}^{s\dag}
          \xi^{r\dag}\mathbf{p}\cdot\bm{\sigma}^*\xi^s
        }
      }_{
        \text{$a\leftrightarrow a^{\dag},r\leftrightarrow s$でキャンセル}
      }
    )
    \nonumber
    \\
    &=    
    -|\mathbf{p}|^2\delta^{rs}(a_\mathbf{p}^{r\dag}a_{\mathbf{p}}^s-a_{\mathbf{p}}^ra_{\mathbf{p}}^{s\dag})
  \end{align}
  となる.ただし,$\xi^1=(1,0),\xi^2=(0,1)$で計算しているので,転置もエルミート共役も同じ.また,$\delta(0)\sim\infty$は無視している.

\end{frame}

\begin{frame}[noframenumbering]
  
  \frametitle{補足\ \subsecname}
  \thispagestyle{empty}

  第2項と第3項は,$A_2$や$A_3$の計算で出てきたものと相殺される:
  \begin{align}
    &
    a_{\mathbf{p}}^{r\dag}\xi^{r\dag}
    \sqrt{p\cdot\sigma}\mathbf{p}\cdot\bm{\sigma}\sqrt{p\cdot\bar{\sigma}}
    \sigma^2 a_{-\mathbf{p}}^{s\dag}(-i\xi^{s*})
    \nonumber
    \\
    &\quad
    +
    a_{-\mathbf{p}}^{r}(i\xi^{rT})\sigma^2
    \sqrt{p\cdot\bar{\sigma}}\mathbf{p}\cdot\bm{\sigma}\sqrt{p\cdot\sigma}
    a_{\mathbf{p}}^s\xi^s
    \nonumber
    \\
    &=
    im\left[  
      a_{-\mathbf{p}}^ra_{\mathbf{p}}^s
      \xi^{rT}\sigma^2\mathbf{p}\cdot\bm{\sigma}\xi^s
      -
      a_{\mathbf{p}}^{r\dag}a_{-\mathbf{p}}^{s\dag}
      \xi^{rT}\mathbf{p}\cdot\bm{\sigma}\sigma^2\xi^s      
    \right]
    .
  \end{align}

\end{frame}

\begin{frame}[noframenumbering]
  
  \frametitle{補足\ \subsecname}
  \thispagestyle{empty}

  次は\eqref{a_23}を計算する.$A_2$は
  \begin{align}
    A_2
    &=
    \int\dd^3\mathbf{x}
    \left(  
      \chi^{T}\sigma^{2}\chi
      -
      \chi^{\dag}\sigma^2\chi^{*}
    \right)
    \nonumber
    \\
    &=
    \int\dd^3\mathbf{x}\ 
    \chi^{T}\sigma^{2}\chi
    +
    \left(        
      \int\dd^3\mathbf{x}\ 
      \chi^{T}\sigma^{2}\chi
    \right)^*
  \end{align}
  であった.$\chi^{T}\sigma^{2}\chi$の項を調べる.$\chi$の展開を代入すると
  \begin{align}
    \int\dd^3\mathbf{x}\ 
    \chi^{T}\sigma^{2}\chi
    &=
    \int\dd^3\mathbf{x}\int\frac{\dd^3\mathbf{p}\dd^3\mathbf{q}}{(2\pi)^3}\frac{1}{\sqrt{2E_{\mathbf{p}}}\sqrt{2E_{\mathbf{q}}}}e^{i(\mathbf{p}+\mathbf{q})\cdot\mathbf{x}}
    \nonumber
    \\
    &
    \sum_{s,r}
    (
      a_{\mathbf{q}}^{r}\xi^{rT}\sqrt{q\cdot\sigma^{T}}
      +
      a_{-\mathbf{q}}^{r\dag}(+i\xi^{r\dag}\sigma^2)\sqrt{q\cdot\bar{\sigma}^{T}}
    )
    \nonumber
    \\
    &
    \times\sigma^2
    (
      a_{\mathbf{p}}^s\sqrt{p\cdot\sigma}\xi^s
      +
      a_{-\mathbf{p}}^{s\dag}\sqrt{p\cdot\bar{\sigma}}(-i\sigma^2\xi^{s*})
    )
    .
    \label{a2_1}
  \end{align}

\end{frame}


\begin{frame}[noframenumbering]
  
  \frametitle{補足\ \subsecname}
  \thispagestyle{empty}

  $\mathbf{x}$で積分をすれば$\delta^{(3)}(\mathbf{p}+\mathbf{q})$がでてくるので,$\mathbf{q}$を$-\mathbf{p}$とする.つまり,
  \begin{align}
    \eqref{a2_1}
    &=
    \int\frac{\dd^3\mathbf{p}}{(2\pi)^3}\frac{1}{2E_{\mathbf{p}}}
    \nonumber
    \\
    &
    \sum_{s,r}
    (
      a_{-\mathbf{p}}^{r}\xi^{rT}\sqrt{p\cdot\bar{\sigma}^{T}}
      +
      a_{\mathbf{p}}^{r\dag}(+i\xi^{r\dag}\sigma^2)\sqrt{p\cdot\sigma^{T}}
    )
    \nonumber
    \\
    &\qquad
    \times\sigma^2
    (
      a_{\mathbf{p}}^s\sqrt{p\cdot\sigma}\xi^s
      +
      a_{-\mathbf{p}}^{s\dag}\sqrt{p\cdot\bar{\sigma}}(-i\sigma^2\xi^{s*})
    )
  \end{align}
  である.  

\end{frame}


\begin{frame}[noframenumbering]
  
  \frametitle{補足\ \subsecname}
  \thispagestyle{empty}

  $A_1$の計算のときと同様,$\sum$の中身を書き出してみるすると,
  \begin{align}
    \sum_{s,r}
    &
    \left[\ 
      a_{-\mathbf{p}}^ra_{\mathbf{p}}^s
      \xi^{rT}\sqrt{p\cdot\bar{\sigma}^{T}}
      \sigma^2
      \sqrt{p\cdot\sigma}\xi^s
    \right.
    \nonumber
    \\
    &
    +
    a_{-\mathbf{p}}^ra_{-\mathbf{p}}^{s\dag}
    \xi^{rT}\sqrt{p\cdot\bar{\sigma}^{T}}
    \sigma^2
    \sqrt{p\cdot\bar{\sigma}}(-i\sigma^2\xi^{s*})
    \nonumber
    \\
    &
    +
    a_{\mathbf{p}}^{r\dag}a_{\mathbf{p}}^s
    (+i\xi^{r\dag}\sigma^2)\sqrt{p\cdot\sigma^{T}}
    \sigma^2
    \sqrt{p\cdot\sigma}\xi^s
    \nonumber
    \\
    &
    +
    a_{\mathbf{p}}^{r\dag}a_{-\mathbf{p}}^{s\dag}
    (+i\xi^{r\dag}\sigma^2)\sqrt{p\cdot\sigma^{T}}
    \sigma^2
    \sqrt{p\cdot\bar{\sigma}}(-i\sigma^2\xi^{s*})
  \end{align}
  となる.

\end{frame}


\begin{frame}[noframenumbering]
  
  \frametitle{補足\ \subsecname}
  \thispagestyle{empty}

  第2項と第3項を同時に計算すると
  \begin{align}
    &
    a_{-\mathbf{p}}^ra_{-\mathbf{p}}^{s\dag}
    \xi^{rT}
    \underbrace{
      \sqrt{p\cdot\bar{\sigma}^{T}}
      \sigma^2
      \sqrt{p\cdot\bar{\sigma}}
    }_{=m\sigma^2}
    (-i\sigma^2\xi^{s*})
    \nonumber
    \\
    &
    +
    a_{\mathbf{p}}^{r\dag}a_{\mathbf{p}}^s
    (+i\xi^{r\dag}\sigma^2)
    \underbrace{
      \sqrt{p\cdot\sigma^{T}}
      \sigma^2
      \sqrt{p\cdot\sigma}
    }_{=m\sigma^2}
    \xi^s
    \nonumber
    \\
    &=
    -im
    a_{-\mathbf{p}}^ra_{-\mathbf{p}}^{s\dag}
    \xi^{rT}\xi^s
    +
    im
    a_{\mathbf{p}}^{r\dag}a_{\mathbf{p}}^s
    \xi^{rT}\xi^s
    \nonumber
    \\
    &=
    -im\delta^{rs}
    (
      a_{\mathbf{p}}^ra_{\mathbf{p}}^{s\dag}
      -
      a_{\mathbf{p}}^{r\dag}a_{\mathbf{p}}^s
    )
  \end{align}
  となる.ただし,最後の行では変数変換$-\mathbf{p}\rightarrow\mathbf{p}$を用いた.

\end{frame}



\begin{frame}[noframenumbering]
  
  \frametitle{補足\ \subsecname}
  \thispagestyle{empty}

  第1項と第4項から,$A_1$のキャンセルしてほしい項がでてくる.第1項から計算すると
  \begin{align}
    &\quad
    a_{-\mathbf{p}}^ra_{\mathbf{p}}^s
    \xi^{rT}
    \underbrace{
      \sqrt{p\cdot\bar{\sigma}^T}\sigma^2\sqrt{p\cdot\sigma}
    }_{=\sigma^2(p\cdot\sigma)}
    \xi^s
    \nonumber
    \\
    &=
    a_{-\mathbf{p}}^ra_{\mathbf{p}}^s
    \xi^{rT}
    \sigma^2
    (
      E_{\mathbf{p}}
      -
      \mathbf{p}\cdot\bm{\sigma}
    )
    \xi^s
    \nonumber
    \\
    &=
    E_{\mathbf{p}}a_{-\mathbf{p}}^ra_{\mathbf{p}}^s
    \xi^{rT}
    \sigma^2
    \xi^s
    -
    a_{-\mathbf{p}}^ra_{\mathbf{p}}^s
    \xi^{rT}\sigma^2
    \mathbf{p}\cdot\bm{\sigma}
    \xi^s
  \end{align}
  となる.同様にして,第4項も
  \begin{align}
    &\quad
    a_{\mathbf{p}}^{r\dag}a_{-\mathbf{p}}^{s\dag}
    (+i\xi^{r\dag}\sigma^2)
    \sqrt{p\cdot\sigma^{T}}\sigma^2\sqrt{p\cdot\sigma}
    (-i\sigma^2\xi^{s*})
    \nonumber
    \\
    &=    
    a_{\mathbf{p}}^{r\dag}a_{-\mathbf{p}}^{s\dag}
    \xi^{rT}\sigma^2\xi^s
    +
    a_{\mathbf{p}}^{r\dag}a_{-\mathbf{p}}^{s\dag}
    \xi^{rT}\mathbf{p}\cdot\bm{\sigma}\sigma^2\xi^s
    .
  \end{align}

\end{frame}



\begin{frame}[noframenumbering]
  
  \frametitle{補足\ \subsecname}
  \thispagestyle{empty}

  よって,
  \begin{align}
    &\quad
    \int\dd^3\mathbf{x}\ 
    \chi^{T}\sigma^{2}\chi
    \nonumber
    \\
    &=
    \int\frac{\dd^3\mathbf{p}}{(2\pi)^3}\frac{1}{2E_{\mathbf{p}}}\sum_{s,r}
    \left[  
      -im\delta^{rs}
      (
        a_{\mathbf{p}}^ra_{\mathbf{p}}^{s\dag}
        -
        a_{\mathbf{p}}^{r\dag}a_{\mathbf{p}}^s
      )
    \right.
    \nonumber
    \\
    &\hspace{2cm}
    +
    E_{\mathbf{p}}a_{-\mathbf{p}}^ra_{\mathbf{p}}^s
    \xi^{rT}
    \sigma^2
    \xi^s
    -
    a_{-\mathbf{p}}^ra_{\mathbf{p}}^s
    \xi^{rT}\sigma^2
    \mathbf{p}\cdot\bm{\sigma}
    \xi^s
    \nonumber
    \\
    &\hspace{2cm}
    \left.
      +
      E_{\mathbf{p}}a_{\mathbf{p}}^{r\dag}a_{-\mathbf{p}}^{s\dag}
      \xi^{rT}\sigma^2\xi^s
      +
      a_{\mathbf{p}}^{r\dag}a_{-\mathbf{p}}^{s\dag}
      \xi^{rT}\mathbf{p}\cdot\bm{\sigma}\sigma^2\xi^s
    \right]
  \end{align}
  となる.$A_2$はこれを自身の複素共役とを足すことによって得られるので,純虚数に項は消える.(「実数」や「純虚数」というのは,ここでは複素共役をとったときのふるまいで定義する.)

\end{frame}



\begin{frame}[noframenumbering]
  
  \frametitle{補足\ \subsecname}
  \thispagestyle{empty}

  第1項は実:
  \begin{align}
    -im\delta^{rs}
    (
      a_{\mathbf{p}}^ra_{\mathbf{p}}^{s\dag}
      -
      a_{\mathbf{p}}^{r\dag}a_{\mathbf{p}}^s
    )
    &\rightarrow
    \left(  
      -im\delta^{rs}
      (
        a_{\mathbf{p}}^ra_{\mathbf{p}}^{s\dag}
        -
        a_{\mathbf{p}}^{r\dag}a_{\mathbf{p}}^s
      )
    \right)^*
    \nonumber
    \\
    &=
    +im\delta^{rs}
    (
      a_{\mathbf{p}}^{r\dag}a_{\mathbf{p}}^{s}
      -
      a_{\mathbf{p}}^{r}a_{\mathbf{p}}^{s\dag}
    )
    \nonumber
    \\
    &=
    -im\delta^{rs}
    (
      a_{\mathbf{p}}^ra_{\mathbf{p}}^{s\dag}
      -
      a_{\mathbf{p}}^{r\dag}a_{\mathbf{p}}^s
    )
    .
  \end{align}
  第3項と第5項をまとめたものも実:
  \begin{align}
    &\quad
    -a_{-\mathbf{p}}^ra_{\mathbf{p}}^s
    \xi^{rT}\sigma^2\mathbf{p}\cdot\bm{\sigma}\xi^s
    +
    a_{\mathbf{r}}^{r\dag}a_{-\mathbf{p}}^{s\dag}
    \xi^{rT}
    \mathbf{p}\cdot\bm{\sigma}\sigma^2
    \xi^s
    \nonumber
    \\
    &\rightarrow
    (
      -a_{-\mathbf{p}}^ra_{\mathbf{p}}^s
      \xi^{rT}\sigma^2\mathbf{p}\cdot\bm{\sigma}\xi^s
      +
      a_{\mathbf{r}}^{r\dag}a_{-\mathbf{p}}^{s\dag}
      \xi^{rT}
      \mathbf{p}\cdot\bm{\sigma}(-\sigma^2)
      \xi^s
    )^*
    \nonumber
    \\
    &=
    -a_{-\mathbf{p}}^{r\dag}a_{\mathbf{p}}^{s\dag}
    \xi^{rT}(-\sigma^2)\mathbf{p}\cdot\bm{\sigma}^*\xi^s
    +
    a_{\mathbf{r}}^{r}a_{-\mathbf{p}}^{s}
    \xi^{rT}
    \mathbf{p}\cdot\bm{\sigma}^*(-\sigma^2)
    \xi^s
    \nonumber
    \\
    &=
    -a_{-\mathbf{p}}^ra_{\mathbf{p}}^s
    \xi^{rT}\sigma^2\mathbf{p}\cdot\bm{\sigma}\xi^s
    +
    a_{\mathbf{r}}^{r\dag}a_{-\mathbf{p}}^{s\dag}
    \xi^{rT}
    \mathbf{p}\cdot\bm{\sigma}\sigma^2
    \xi^s
    .
  \end{align}
  ただし,最後の行では$\mathbf{p}\rightarrow-\mathbf{p}$とした.

\end{frame}


\begin{frame}[noframenumbering]
  
  \frametitle{補足\ \subsecname}
  \thispagestyle{empty}

  第2項と第4項をまとめたものは純虚数:
  \begin{align}
    &\quad
    a_{-\mathbf{p}}^ra_{\mathbf{p}}^s
    \xi^{rT}
    \sigma^2
    \xi^s
    +
    a_{\mathbf{p}}^{r\dag}a_{-\mathbf{p}}^{s\dag}
    \xi^{rT}\sigma^2\xi^s
    \nonumber
    \\
    &\rightarrow
    (
      a_{-\mathbf{p}}^ra_{\mathbf{p}}^s
      \xi^{rT}
      \sigma^2
      \xi^s
      +
      a_{\mathbf{p}}^{r\dag}a_{-\mathbf{p}}^{s\dag}
      \xi^{rT}\sigma^2\xi^s
    )^*
    \nonumber
    \\
    &=
    a_{-\mathbf{p}}^{r\dag}a_{\mathbf{p}}^{s\dag}
    \xi^{rT}
    (-\sigma^2)
    \xi^s
    +
    a_{\mathbf{p}}^{r}a_{-\mathbf{p}}^{s}
    \xi^{rT}
    (-\sigma^2)
    \xi^s
    \nonumber
    \\
    &=
    -(
      a_{-\mathbf{p}}^ra_{\mathbf{p}}^s
      \xi^{rT}
      \sigma^2
      \xi^s
      +
      a_{\mathbf{p}}^{r\dag}a_{-\mathbf{p}}^{s\dag}
      \xi^{rT}\sigma^2\xi^s
    )
    .
  \end{align}
  ただし,案の定,最後の行では$\mathbf{p}\rightarrow-\mathbf{p}$としている.

  \vspace{10pt}

  よって,この項は足しあげたときに消える.

\end{frame}



\begin{frame}[noframenumbering]
  
  \frametitle{補足\ \subsecname}
  \thispagestyle{empty}

  以上のことから
  \begin{align}
    A_2
    &=
    \int\frac{\dd^3\mathbf{p}}{(2\pi)^3}\frac{1}{E_{\mathbf{p}}}\sum_{s,r}
    \left[  
      -im\delta^{rs}
      (
        a_{\mathbf{p}}^ra_{\mathbf{p}}^{s\dag}
        -
        a_{\mathbf{p}}^{r\dag}a_{\mathbf{p}}^s
      )
    \right.
    \nonumber
    \\
    &\hspace{3cm}
    -
    a_{-\mathbf{p}}^ra_{\mathbf{p}}^s
    \xi^{rT}\sigma^2
    \mathbf{p}\cdot\bm{\sigma}
    \xi^s
    \nonumber
    \\
    &\hspace{3cm}
    \left.
      +
      a_{\mathbf{p}}^{r\dag}a_{-\mathbf{p}}^{s\dag}
      \xi^{rT}\mathbf{p}\cdot\bm{\sigma}\sigma^2\xi^s
    \right]
    \nonumber
    \\
    &=
    -i
    \int\frac{\dd^3\mathbf{p}}{(2\pi)^3}
    \frac{m}{E_{\mathbf{p}}}
    \sum_{s}
    (
      a_{\mathbf{p}}^ra_{\mathbf{p}}^{s\dag}
      -
      a_{\mathbf{p}}^{r\dag}a_{\mathbf{p}}^s
    )
    \nonumber
    \\
    &\hspace{1.2cm}
    +
    \frac{2}{im}
    \int\frac{\dd^3\mathbf{p}}{(2\pi)^3}
    \frac{im}{2E_{\mathbf{p}}}
    \sum_{s,r}
    \left[  
      a_{\mathbf{p}}^{r\dag}a_{-\mathbf{p}}^{s\dag}
      \xi^{rT}\mathbf{p}\cdot\bm{\sigma}\sigma^2\xi^s
    \right.
    \nonumber
    \\
    &\hspace{4.5cm}
    \left.
      -
      a_{-\mathbf{p}}^ra_{\mathbf{p}}^s
      \xi^{rT}\sigma^2
      \mathbf{p}\cdot\bm{\sigma}
    \xi^s
    \right]
  \end{align}
  となり,これは\eqref{a_23}である.

\end{frame}


\end{document}
