\pdfoutput=1
\documentclass[a4paper,pdftex,10pt]{report}
% \usepackage[whole,autotilde]{bxcjkjatype}
\usepackage[T1]{fontenc}
\usepackage{tgtermes}

% ---Display \subsubsection at the Index
\setcounter{tocdepth}{3}

% ---Setting about the geometry of the document----
\usepackage{a4wide}
% \pagestyle{empty}

% ---Physics and Math Packages---
\usepackage{amssymb,amsfonts,amsthm,mathtools}
\usepackage{physics,braket,bm,slashed}

% ---underline---
\usepackage[normalem]{ulem}

% ---cancel---
\usepackage{cancel}

% --- surround the texts or equations
\usepackage{fancybox,ascmac}

% ---settings of theorem environment---
\theoremstyle{definition}
\newtheorem{dfn}{Definition}
\newtheorem{prop}{Proposition}
\newtheorem{thm}{Theorem}

% ---settings of proof environment---
\renewcommand{\proofname}{\textbf{Proof}}
\renewcommand{\qedsymbol}{$\blacksquare$}

% ---Ignore the Warnings---
\usepackage{silence}
\WarningFilter{latexfont}{Some font shapes,Font shape}
\ExplSyntaxOn
\msg_redirect_name:nnn{hooks}{generic-deprecated}{none}
\ExplSyntaxOff

% ---Insert the figure (If insert the `draft' at the option, the process becomes faster.)---
% \usepackage{graphicx}
% \usepackage{subcaption}

% ----Add a link to a text---
\usepackage{url,hyperref}
\usepackage[dvipsnames,svgnames]{xcolor}
\hypersetup{colorlinks=true,citecolor=FireBrick,linkcolor=Navy,urlcolor=purple}

% ---Tikz---
\usepackage{tikz,pgf,pgfplots,circuitikz}
\pgfplotsset{compat=1.15}
\usetikzlibrary{intersections, arrows.meta, angles, calc, 3d, decorations.pathmorphing}
\usepackage[compat=1.1.0]{tikz-feynhand}

% ---tcolorbox---
\usepackage{tcolorbox}
\tcbuselibrary{raster,skins,breakable}
\newtcolorbox{graybox}[1][]{frame empty, colback=black!07!white, sharp corners}

% ---Add the section number to the equation, figure, and table number---
\makeatletter
   \renewcommand{\theequation}{$\thesection.\arabic{equation}$}
   \@addtoreset{equation}{section}
   
   \renewcommand{\thefigure}{\thesection.\arabic{figure}}
   \@addtoreset{figure}{section}
   
   \renewcommand{\thetable}{\thesection.\arabic{table}}
   \@addtoreset{table}{section}
\makeatother

% ---enumerate---
% \renewcommand{\labelenumi}{$\arabic{enumi}.$}
% \renewcommand{\labelenumii}{$(\arabic{enumii})$}

% ---Index---
% \usepackage{makeidx}
% \makeindex 

% ---footnotes---
\renewcommand{\thefootnote}{$\ast$\arabic{footnote}}

% ---Title---
\title{Notes on Anomalies}
\author{Itsuki Miyane}
\date{Last modified:\ \today}

\begin{document}

\maketitle

\tableofcontents

\clearpage
\chapter{Group and its representation theory}

When we construct the field theory, we impose the theory of Lorentz invariance. One of the easiest ways is to use the field which transforms in a covariant manner with the spacetime Lorentz transformation\footnote{
  I guess it is not impossible to follow that method, though it is tough work. In that case, the Lorentz invariance is a miracle.
}. To construct field theories, the representation of the Lorentz group plays a crucial role. In this chapter, we will discuss the property of the Lorentz group and its representation theory. 

\section{Lorentz group}

Here, I will give the formal definition of the Lorentz group:
\begin{graybox}
  \textit{Lorentz group} is the set of linear transformations
  \begin{equation}
    x^{\mu}
    \rightarrow
    \Lambda^{\mu}_{\ \nu}x^{\nu}
  \end{equation}
  which preserves the following inner product  
  \begin{equation}
    x^{\mu}y_{\mu}
    \equiv
    -x^{0}y_{0}
    +x^{1}y_{1}
    +x^{2}y_{2}
    +x^{3}y_{3}
    .
  \end{equation}
  Such a transformation is similar to the orthogonal transformation $O(4)$ but one sign is flipped. In this sense, we will denote the Lorentz group as $O(3,1)$. 
\end{graybox}
In this text, I will use the metric convention as $\eta^{\mu\nu}=\textrm{diag}(-,+,+,+)$. From this definition, we can obtain the explicit conditions for the elements of the Lorentz group. Let $x^{\mu}, y^{\nu}$ are transformed as 
\begin{equation}
  x^{\mu\prime}
  =
  \Lambda^{\mu}_{\ \sigma}x^{\sigma}
  ,\ 
  y^{\mu\prime}
  =
  \Lambda^{\mu}_{\ \rho}y^{\rho}
  .
\end{equation}
Then the inner product should be
\begin{equation}
  y^{\mu\prime}x_{\mu\prime}
  =
  y^{\mu\prime}\eta_{\mu\nu}x^{\nu\prime}
  =
  (\Lambda^{\mu}_{\ \rho}y^{\rho})\eta_{\mu\nu}(\Lambda^{\nu}_{\ \sigma}x^{\sigma})
  =
  y^{\rho}(\Lambda^{\mu}_{\ \rho}\eta_{\mu\nu}\Lambda^{\nu}_{\ \sigma})x^{\sigma}
  .
\end{equation}
By changing the dummy indices, we conclude that the $\Lambda\equiv(\Lambda^{\mu}_{\ \nu})$ should satisfy 
\begin{equation}
  \eta_{\mu\nu}
  =
  \Lambda^{\rho}_{\ \mu}\eta_{\rho\sigma}\Lambda^{\sigma}_{ \nu}
  \hspace*{5pt}\textrm{or}\hspace*{5pt}
  \eta
  =
  \Lambda^{T}\eta\Lambda
\end{equation} 
and, in this manner, we can write 
\begin{equation}
  SO(3,1)
  =
\end{equation}






























% ----------------------------------------
\clearpage
\appendix
\chapter{Seiberg-Witten Theory}




























\clearpage
\chapter{Relationship between Supersymmetry and Morse theory}



\section{Introduction to supersymmetric quantum mechanics}
























% ----------------------------------------
\clearpage
\chapter{Brief notes}

\section{Dirac operator and Index theorem}

This is the summary of the presentation at QFT seminar in Wathematica. In this introductory talk, I will give the relationship between Atiyah-Singer index theorem and anomalies\footnote{
  I am not sure why I gave this notes in English. Anyway, I hope it motivates you to study quantum field theory.
}. 

\subsection{Dirac operator and Spin complex}

Let $\Delta(M)$ the section of the spin bundle $S(M)$, i.e. $\Delta(M)=\Gamma(M,S(M))$. We assume that the dimension of the manifold $M$ is even integer $m=2l$. The spin group is generated by $2l$ Dirac matrices $\{\gamma^{k}\}_{k=1,\cdots,2l}$  which satisfy
\begin{equation}
  {\gamma^{\alpha}}^{\dag}
  =
  \gamma^{\alpha}
  ,\ 
  \{\gamma^{\alpha},\gamma^{\beta}\}
  =
  2\delta^{\alpha\beta}
\end{equation} 
and we can define 
\begin{equation}
  \gamma^{m+1}
  \equiv
  i\gamma^{1}\gamma^{2}\cdots\gamma^{2l}
  .
\end{equation}
Note that we assume that the metric has the Euclidean signature. When we consider the representation of the gamma matrices, we take so that $\gamma^{m+1}$ is diagonal
\begin{equation}
  \gamma^{m+1}
  =
  \begin{pmatrix}
    1 & 0 \\
    0 & -1
  \end{pmatrix}
  .
  \label{eqn:chirality_operator}
\end{equation}

A Dirac operator $\psi\in\Delta(M)$ is not the irreducible representation of the $Spin(m)$ and we can obtain them separating $\Delta(M)$ according to the eigenvalue of $\gamma^{m+1}$, called \textit{chirality}. Since $\gamma^{m+1}$ has eigenvalues $\pm 1$, we divide 
\begin{equation}
  \Delta(M)
  =
  \Delta^{+}(M)
  \oplus
  \Delta^{-}(M)
  .
\end{equation}
The projection operator$P_{\pm}$ is given by 
\begin{equation}
  P_{\pm}
  \equiv
  \frac{1\pm \gamma^{5}}{2}
  =
  \begin{pmatrix}
    1 & 0 \\
    0 & 0
  \end{pmatrix}
  \hspace*{5pt}\textrm{or}\hspace*{5pt}
  \begin{pmatrix}
    0 & 0 \\
    0 & 1
  \end{pmatrix}
\end{equation}
and we write 
\begin{equation}
  \psi^{+}
  =
  \begin{pmatrix}
    \psi^{+} \\
    0
  \end{pmatrix}
  \in 
  \Delta^{+}(M)
  \ ,\ 
  \psi^{-}
  =
  \begin{pmatrix}
    0 \\
    \psi^{-}
  \end{pmatrix}
  \in 
  \Delta^{-}(M)
  .
\end{equation}

The \textit{Dirac operator} is given by
\begin{equation}
  i\slashed{\nabla}\psi
  \equiv
  i\gamma^{\mu}\partial_{\mu}\psi
  .
\end{equation}
In this case, we assume that the base manifold $M$ is flat, so that the spin connection does not appear in this definition. We admit this operator is \textit{elliptic}. 

Now, let us consider the case $m=4$. We will take the representation of the gamma matrices as 
\begin{eqnarray}
  \gamma^{\mu}
  =
  \begin{pmatrix}
    0 & \bar{\sigma}^{\mu} \\
    \sigma^\mu & 0
  \end{pmatrix}
  ,\quad
  \sigma^{\mu}
  =
  (1,\sigma^{i})
  ,\quad 
  \bar{\sigma}^{\mu}
  =
  (1,-\sigma^{i})
\end{eqnarray}
to obtain the chirality operator $\gamma^{5}$ in \eqref{eqn:chirality_operator}.



















% ----------------------------------------
\clearpage
\bibliography{ref}
\bibliographystyle{ytamsalpha}

\nocite{Peskin:1995}
\nocite{Nair:2005}
\nocite{Weinberg:1996kr}
\nocite{Weinberg:1995mt}
\nocite{Weinberg:2000}

\nocite{Witten:1982df}
\nocite{Witten:1982im}

\nocite{Seiberg:1994aj}
\nocite{Seiberg:1994rs}
\nocite{Alvarez-Gaume:1996ohl}
\nocite{Tachikawa:2013kta}

\nocite{Nakahara:2003}

% ----------------------------------------
% \clearpage
% \index{hoge@hoge}
% \printindex

\end{document}
