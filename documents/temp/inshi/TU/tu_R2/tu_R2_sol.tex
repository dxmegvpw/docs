\pdfoutput=1
\documentclass[a4paper,pdflatex,ja=standard]{bxjsarticle}

% ---Setting about the geometry of the document----
% \usepackage{a4wide}
% \pagestyle{empty}

% ---Physics and Math Packages---
\usepackage{amssymb,amsfonts,amsthm,mathtools}
\usepackage{physics,braket,bm}

% ---underline---
\usepackage{ulem}

% --- sorround the texts or equations
% \usepackage{fancybox,ascmac}

% ---settings of theorem environment---
% \usepackage{amsthm}
% \theoremstyle{definition}

% ---settings of proof environment---
% \renewcommand{\proofname}{\textbf{証明}}
% \renewcommand{\qedsymbol}{$\blacksquare$}

% ---Ignore the Warnings---
\usepackage{silence}
\WarningFilter{latexfont}{Some font shapes,Font shape}

% ---Insert the figure (If insert the `draft' at the option, the process becomes faster)---
\usepackage{graphicx}
% \usepackage{subcaption}

% ----Add a link to a text---
\usepackage{url}
\usepackage{xcolor,hyperref}
\hypersetup{colorlinks=true,citecolor=orange,linkcolor=blue,urlcolor=magenta}
\usepackage{bxcjkjatype}

% ---Tikz---
\usepackage{tikz,pgf,pgfplots,circuitikz}
\pgfplotsset{compat=1.15}
\usetikzlibrary{intersections,arrows.meta,angles,calc,3d,decorations.pathmorphing}

% ---Add the section number to the equation, figure, and table number---
\makeatletter
   \renewcommand{\theequation}{\thesubsection.\arabic{equation}}
   \@addtoreset{equation}{section}
   
   \renewcommand{\thefigure}{\thesection.\arabic{figure}}
   \@addtoreset{figure}{section}
   
   \renewcommand{\thetable}{\thesection.\arabic{table}}
   \@addtoreset{table}{section}
\makeatother

% ---enumerate---
\renewcommand{\labelenumi}{\arabic{enumi}.}
\renewcommand{\labelenumii}{(\roman{enumii})}

% ---Index---
% \usepackage{makeidx}
% \makeindex 

% ---Fonts---
\renewcommand{\familydefault}{\sfdefault}

% ---Title---
\title{東京大学\ 令和2年\ 物理学専攻\ 院試\ 解答例}
\author{ミヤネ}
\date{最終更新:\today}

\newcommand{\prb}[2]{
  \phantomsection
  \addcontentsline{toc}{subsection}{問題 #1: #2}
  \subsection*{第#1問}
  \setcounter{subsection}{#1}
  \setcounter{equation}{0}
}

\begin{document}

\maketitle

\tableofcontents
\clearpage

\section{数学パート}

\prb{1}{偏微分}

\setcounter{subsection}{1}

\begin{enumerate}
  \item 
  \begin{enumerate}
    \item
    被積分関数をべき級数展開すれば
    \begin{equation}
      e^{-\xi^2}
      =
      \sum_{n=0}^{\infty}\frac{(-1)^n}{n!}\xi^{2n}
    \end{equation}
    となり,これを積分すれば
    \begin{align}
      \text{erf}(x)
      &=
      \frac{2}{\sqrt{\pi}}\int_{0}^{x}\dd \xi\ e^{-\xi^2}
      \nonumber
      \\
      &=
      \frac{2}{\sqrt{\pi}}\sum_{n=0}^{\infty}\frac{(-1)^n}{n!}\int_{0}^{x}\dd \xi\ 
      \xi^{2n}
      \nonumber
      \\
      &=
      \frac{2}{\sqrt{\pi}}
      \sum_{n=0}^{\infty}\frac{(-1)^n}{n! (2n+1)}x^{2n+1}
    \end{align}
    となる.

    \item 
    $\xi^2=\eta$と積分変数を変換すれば
    \begin{align}
      \text{erf}(x)
      &=
      \frac{1}{\sqrt{\pi}}\int_{0}^{x^2}\dd \eta\ \eta^{-1/2} e^{-\eta}
      \nonumber
      \\
      \nonumber
      &=
      -\frac{1}{\sqrt{\pi}}\left\{ 
      \left[ \eta^{-1/2}e^{-\eta} \right]_{0}^{x^2}
      +
      \frac{1}{2}\int_{0}^{x^2}\dd \eta\ \eta^{-3/2}e^{-\eta}
      \right\}
      \\
      &=
      -\frac{1}{\sqrt{\pi}x}e^{-x^2}
      +
      \lim_{\eta\rightarrow 0}\frac{e^{-\eta}}{\sqrt{\eta}}+\frac{1}{2}\int_{0}^{x^2}\dd \eta\ \eta^{-3/2}e^{-\eta}
    \end{align}
    となる.ここで
    \begin{equation}
      \text{erf}(x)
      \rightarrow
      \frac{2}{\sqrt{\pi}}\int_{0}^{\infty}\dd \xi\ e^{-\xi^2}
      =
      1
    \end{equation}
    なので,$x\sim\infty$で第2,3項が$1$に近いと近似できて
    \begin{equation}
      \text{erf}(x)
      \sim
      1-\frac{1}{\sqrt{\pi}x}e^{-x^2}
    \end{equation}
    である.
  \end{enumerate}

  \item 
  \begin{enumerate}
    \item 
    解が
    \begin{equation}
      u(x,t)
      =
      X(x)T(t)
      .
    \end{equation}
    と書けるとする.これを方程式に代入して,整理すれば
    \begin{equation}
      \frac{1}{X}\dv{X}{x}
      =
      -\frac{1}{cT}\dv{T}{t}
      \coloneqq
      \Lambda
    \end{equation}
    となる.ここで,$\Lambda$は定数である.よって,微分方程式を解けば,それぞれ
    \begin{equation}
      X(x)=Ae^{\Lambda x}\ ,\ T(t)=Be^{-c\Lambda t}
    \end{equation}
    と定数$A,B$を用いてあらわすことができる.この積が解$u(x,t)$だったので
    \begin{equation}
      u(x,t)
      =
      A^{*}e^{\Lambda (x-ct)}
      \label{sol1}
    \end{equation}
    である.ただし,$A^{*}=AB$とおいた.$t=0$とすれば
    \begin{equation}
      U(x)
      =
      A^{*}e^{\Lambda }
    \end{equation}
    と初期条件から求まる.よって\eqref{sol1}より
    \begin{equation}
      u(x,t)=U(x-ct)
    \end{equation}
    となる.

    \item 
    フーリエ変換の定義から,逆フーリエ変換は
    \begin{equation}
      u(x,t)
      =
      \int_{-\infty}^{\infty}\dd k\ \tilde{u}(k,t)e^{ikx}
    \end{equation}
    となる.これを方程式に代入すれば
    \begin{equation}
      \pdv{\tilde{u}}{t}
      =
      -\left( ick+\frac{\lambda}{2}k^2 \right)\tilde{u}
      \label{PDEink}
    \end{equation}
    となり,初期条件は
    \begin{equation}
      \tilde{U}(k)
      =
      \frac{1}{\sqrt{2\pi}}\int_{-\infty}^{\infty}\dd x\ U(x)e^{-ikx}
    \end{equation}
    である.

    \item
    $U(x)=\delta (x)$のとき,初期条件は
    \begin{equation}
      \tilde{U}(k)
      =
      \frac{1}{\sqrt{2\pi}}
    \end{equation}
    である.\eqref{PDEink}を解くと
    \begin{equation}
      \tilde{G}(k,t)
      =
      A(k)\exp\left[ -\left( ick+\frac{\lambda}{2}k^2 \right)t \right]
      \label{solink}
    \end{equation}
    である.ただし,$A$は$k$のみに依存する関数である.初期条件から
    \begin{equation}
      A
      =
      \frac{1}{\sqrt{2\pi}}
    \end{equation}
    となるので
    \begin{equation}
      \tilde{G}(k,t)
      =
      \frac{1}{\sqrt{2\pi}}\exp\left[ -\left( ick+\frac{\lambda}{2}k^2 \right)t \right]
    \end{equation}
    となり,
    \begin{align}
      G(x,t)
      &=
      \int_{-\infty}^{\infty}\dd k\ \tilde{G}(k,t)e^{ikx}
      \nonumber
      \\
      &=
      \frac{1}{\sqrt{2\pi}}\int_{-\infty}^{\infty}\dd k\ e^{-(\lambda t/2)k^2+i(x-ct)k}
      \nonumber
      \\
      &=
      \frac{1}{\sqrt{\lambda t}}e^{-\tfrac{(x-ct)^2}{2\lambda t}}
    \end{align}
    である.

    \item 
    初期条件は
    \begin{equation}
      \tilde{U}(k)
      =
      \frac{1}{\sqrt{2\pi}}\int_{-\infty}^{0}\dd x\ e^{-ikx}
      =
      \frac{1}{\sqrt{2\pi}}\int_{0}^{\infty}\dd x\ e^{ikx}
    \end{equation}
    となる.よって,\eqref{solink}の解は
    \begin{equation}
      A(k)
      =
      \tilde{U}(k)
    \end{equation}
    であり
    \begin{equation}
      \tilde{u}(k,t)
      =
      \tilde{U}(k)
      \exp\left[ -\left( ick+\frac{\lambda}{2}k^2 \right)t \right]
    \end{equation}
    となる.よって,$\tilde{u}(k,t)$を逆変換して$u(x,t)$を求めれば
    \begin{align}
      u(x,t)
      &=
      \int_{-\infty}^{\infty}\dd k\ e^{ikx}
      \frac{1}{\sqrt{2\pi}}\int_{0}^{\infty}\dd x'\ e^{ikx'}
      e^{-\left( ick+(\lambda/2)k^2 \right)t}
      \nonumber
      \\
      &=
      \frac{1}{\sqrt{2\pi}}
      \int_{0}^{\infty}\dd x'\ \int_{-\infty}^{\infty}\dd k\ 
      e^{-(\lambda t/2)k^2+i(x+x'-ct)k}
      \nonumber
      \\
      &=
      \frac{1}{\sqrt{\lambda t}}\int_{0}^{\infty}\dd x'\ e^{-\tfrac{(x+x'-ct)^2}{2\lambda t}}
    \end{align}
    となる.変数を
    \begin{equation}
      \frac{(x+x'-ct)^2}{2\lambda t}
      =
      z^2
    \end{equation}
    と変換すれば
    \begin{align}
      \int_{0}^{\infty}\dd x'\ e^{-\tfrac{(x+x'-ct)^2}{2\lambda t}}
      &=
      \int_{\tfrac{|x-ct|}{\sqrt{2\lambda t}}}^{\infty}\sqrt{2\lambda t}\dd z\ e^{-z^2}
      \nonumber
      \\
      &=
      \sqrt{2\lambda t}\left\{ 
      \int_{0}^{\infty}e^{-z^2}-\int_{0}^{\tfrac{|x-ct|}{\sqrt{2\lambda t}}}\dd z\ e^{-z^2}
      \right\}
      \nonumber
      \\
      &=
      \sqrt{\frac{\lambda\pi t}{2}}\left\{ 1-\text{erf}(|x-ct|/\sqrt{2\lambda t}) \right\}
    \end{align}
    となって
    \begin{equation}
      u(x,t)
      =
      \sqrt{\frac{\pi}{2}}\left\{ 1-\text{erf}\left( \frac{|x-ct|}{\sqrt{2\lambda t}} \right) \right\}
    \end{equation}
    となる.このとき,$u(x,t)$は時間がたつ($t\rightarrow\infty$)と
    \begin{equation}
      u(x,t)
      \rightarrow
      0
    \end{equation}
    となることがわかる.
  \end{enumerate}
\end{enumerate}

\clearpage
\prb{2}{線形代数}
\begin{enumerate}
  \item 
  $w$は
  \begin{equation}
    w^{d}-1=0
    .
  \end{equation}
  の解である.左辺を変形すれば
  \begin{equation}
    (w-1)\tr Z =0
  \end{equation}
  となり,$w\neq 1$であることから
  \begin{equation}
    \tr Z = 0
  \end{equation}
  がわかる.

  \item 
  $XZ$と$ZX$をそれぞれもとめてみると
  \begin{gather}
    XZ
    =
    \begin{pmatrix}
      0 & 0 & \cdots & 0 & 1 \\
      1 & 0 & \cdots & 0 & 0 \\
      \vdots & \vdots & \ddots & \vdots & \vdots \\
      0 & 0 & \cdots & 1 & 0
    \end{pmatrix}
    \begin{pmatrix}
      1 & 0 & \cdots & 0 & 0 \\
      0 & w & \cdots & 0 & 0 \\
      \vdots & \vdots & \ddots & \vdots & \vdots \\
      0 & 0 & \cdots & 0 & w^{d-1}
    \end{pmatrix}
    =
    \begin{pmatrix}
      0 & 0 & \cdots & 0 & w^{d-1} \\
      1 & 0 & \cdots & 0 & 0 \\
      \vdots & \vdots & \ddots & \vdots & \vdots \\
      0 & 0 & \cdots & w^{d-2} & 0
    \end{pmatrix}
    \nonumber
    \\
    ZX
    =
    \begin{pmatrix}
      1 & 0 & \cdots & 0 & 0 \\
      0 & w & \cdots & 0 & 0 \\
      \vdots & \vdots & \ddots & \vdots & \vdots \\
      0 & 0 & \cdots & 0 & w^{d-1}
    \end{pmatrix}
    \begin{pmatrix}
      0 & 0 & \cdots & 0 & 1 \\
      1 & 0 & \cdots & 0 & 0 \\
      \vdots & \vdots & \ddots & \vdots & \vdots \\
      0 & 0 & \cdots & 1 & 0
    \end{pmatrix}
    =    
    \begin{pmatrix}
      0 & 0 & \cdots & 0 & 1 \\
      w & 0 & \cdots & 0 & 0 \\
      \vdots & \vdots & \ddots & \vdots & \vdots \\
      0 & 0 & \cdots & w^{d-1} & 0
    \end{pmatrix}
    .
    \nonumber
  \end{gather}
  となる.このとき,$w^{d}=1$であることに気をつければ
  \begin{equation}
    ZX
    =
    wXZ
  \end{equation}
  という関係がある.

  \item 
  2通りで計算してみると
  \begin{align}
    U^{(1,m)}U^{(n',m')}
    &=
    X\overbrace{Z\cdots Z}^{m}\overbrace{X\cdots X}^{n'}\overbrace{Z\cdots Z}^{m'}
    \nonumber
    \\
    &=
    w^{mn'}\overbrace{X\cdots X}^{n'+1}\overbrace{Z\cdots Z}^{m+m'}
    \\
    U^{(n',m')}U^{(1,m)}
    &=
    \underbrace{X\cdots X}_{n'}\underbrace{Z\cdots Z}_{m'}X\underbrace{Z\cdots Z}_{m}
    \nonumber
    \\
    \nonumber
    &=
    w^{m'}\overbrace{X\cdots X}^{n'+1}\overbrace{Z\cdots Z}^{m+m'}
  \end{align}
  となる.ここで$U^{(1,m)}$と$U^{(n',m')}$が同時対角化可能であるためには,$U^{(1,m)}U^{(n',m')}$と$U^{(n',m')}U^{(1,m)}$が等しければよい.したがって,
  \begin{equation}
    m'-mn'
    =
    nd
    \hspace*{1cm}
    (n\in\mathbb{Z})
  \end{equation}
  が求める条件である.
  
  \item 
  $U^{(1,m)}$は
  \begin{equation}
    U^{(1,m)}
    =
    \begin{pmatrix}
      0 & 0 & 0 & \cdots & 0 & w^{m(d-1)} \\
      1 & 0 & 0 & \cdots & 0 & 0 \\
      0 & w^{m} & 0 & \cdots & 0 & 0 \\
      \vdots & \vdots & \vdots & \ddots & \vdots & \vdots \\
      0 & 0 & 0 & \cdots & 0 & 0 \\
      0 & 0 & 0 & \cdots & w^{m(d-2)} & 0 
    \end{pmatrix}
  \end{equation}
  となっている.よって,この固有多項式は
  \begin{equation}
    \begin{vmatrix}
      \lambda & 0 & 0 & \cdots & 0 & -w^{m(d-1)} \\
      -1 & \lambda & 0 & \cdots & 0 & 0 \\
      0 & -w^{m} & \lambda & \cdots & 0 & 0 \\
      \vdots & \vdots & \vdots & \ddots & \vdots & \vdots \\
      0 & 0 & 0 & \cdots & \lambda & 0 \\
      0 & 0 & 0 & \cdots & -w^{m(d-2)} & \lambda 
    \end{vmatrix}
  \end{equation}
  である.これを計算するために次の関係式
  \begin{equation}
    \begin{vmatrix}
      \lambda & 0 & \cdots & 0&x_{n} \\
      x_1 & \lambda & \cdots & 0&0 \\
      \vdots & \vdots & \ddots & \vdots & \vdots \\
      0 & 0 & \cdots & x_{n-1} & \lambda
    \end{vmatrix}
    =
    \lambda^{n}+(-1)^{n-1}x_{1}x_{2}\cdots x_{n}
    \label{lem}
  \end{equation}
  を示そう.

  \begin{proof}
    余因子展開をすれば
    \begin{align}
      \begin{vmatrix}
        \lambda & 0 & \cdots & 0&x_{n} \\
        x_1 & \lambda & \cdots & 0&0 \\
        \vdots & \vdots & \ddots & \vdots & \vdots \\
        0 & 0 & \cdots & x_{n-1} & \lambda
      \end{vmatrix}
      &=
      \lambda
      \begin{vmatrix}
        \lambda & 0 & \cdots & 0&0 \\
        x_2 & \lambda & \cdots & 0&0 \\
        \vdots & \vdots & \ddots & \vdots & \vdots \\
        0 & 0 & \cdots & x_{n-1} & \lambda
      \end{vmatrix}
      \nonumber
      \\
      &\hspace*{3cm}
      +
      (-1)^{n-1}x_{n}
      \begin{vmatrix}
        x_{1} & \lambda & \cdots & 0&0 \\
        0 & x_{2} & \cdots & 0&0 \\
        \vdots & \vdots & \ddots & \vdots & \vdots \\
        0 & 0 & \cdots & 0 & x_{n-1}
      \end{vmatrix}
      \nonumber
      \\
      &=
      \lambda^{n}+(-1)^{n-1}x_{1}x_{2}\cdots x_{n}
    \end{align}
    とただちにもとめられる.このとき,
    \begin{gather}
      \begin{vmatrix}
        \lambda & 0 & \cdots & 0&0 \\
        x_2 & \lambda & \cdots & 0&0 \\
        \vdots & \vdots & \ddots & \vdots & \vdots \\
        0 & 0 & \cdots & x_{n-1} & \lambda
      \end{vmatrix}
      =
      \lambda^{n-1}
      \\
      \begin{vmatrix}
        x_{1} & \lambda & \cdots & 0&0 \\
        0 & x_{2} & \cdots & 0&0 \\
        \vdots & \vdots & \ddots & \vdots & \vdots \\
        0 & 0 & \cdots & 0 & x_{n-1}
      \end{vmatrix}
      =
      x_{1}x_{2}\cdots x_{n-1}
    \end{gather}
    という関係を用いた.これらも,余因子展開からもとめることができる.
  \end{proof}

  \eqref{lem}をもちいれば
  \begin{equation}
    \begin{vmatrix}
      \lambda & 0 & 0 & \cdots & 0 & -w^{m(d-1)} \\
      -1 & \lambda & 0 & \cdots & 0 & 0 \\
      0 & -w^{m} & \lambda & \cdots & 0 & 0 \\
      \vdots & \vdots & \vdots & \ddots & \vdots & \vdots \\
      0 & 0 & 0 & \cdots & \lambda & 0 \\
      0 & 0 & 0 & \cdots & -w^{m(d-2)} & \lambda 
    \end{vmatrix}
    =
    \lambda^{d}
    -w^{m+\cdots(d-1)m}
    =
    \lambda^{d}
    -
    w^{m(d-1)d/2}
  \end{equation}
  が成り立つ.したがって,これを$0$とした方程式を解けば,
  \begin{equation}
    \lambda^{d}
    =
    w^{m(d-1)d/2}
  \end{equation}
  となり,
  \begin{equation}
    \lambda
    =
    e^{2\pi i\tfrac{n}{d}}w^{\tfrac{m(d-1)}{2}}
    \hspace*{1cm} (n=0,1,\cdots d-1)
  \end{equation}
  がもとめる固有値である.

  \item 
  $d=3$のとき,固有値は
  \begin{equation}
    w^{m}
    \ ,\ 
    e^{2\pi i/3}w^{m}
    \ ,\ 
    e^{4\pi i/3}w^{m}
  \end{equation}
  である.よって,対応する固有ベクトルは
  \begin{align}
    e^{(m)}_{1}
    =
    \frac{1}{\sqrt{3}}
    \begin{pmatrix}
      w^{m} \\
      1 \\
      1
    \end{pmatrix}
    \ &:\ w^{m}
    \\
    e^{(m)}_{2}
    =
    \frac{1}{\sqrt{3}}
    \begin{pmatrix}
      e^{4\pi i/3}w^{m} \\
      e^{2\pi i/3} \\
      1
    \end{pmatrix}
    \ &:\ e^{2\pi i/3}w^{m}
    \\
    e^{(m)}_{3}
    =
    \frac{1}{\sqrt{3}}
    \begin{pmatrix}
      e^{2\pi i/3}w^{m} \\
      e^{4\pi i/3} \\
      1
    \end{pmatrix}
    \ &:\ e^{4\pi i/3}w^{m}
  \end{align}
  である.

  \item 
  固有ベクトルは
  \begin{equation}
    e_{j}^{(m)}
    =
    \frac{1}{\sqrt{3}}
    \begin{pmatrix}
      e^{-2\pi i(j-1)/3}w^{m} \\
      e^{2\pi i(j-1)/3} \\
      1
    \end{pmatrix}
  \end{equation}
  と書き直すことができることに注意する.これを用いて直接計算してみると
  \begin{align}
    |\ev*{e_{j}^{m},e_{j'}^{m'}}|^2
    &=
    \frac{1}{9}
    | e^{-2\pi i(j'-j)/3}w^{m'-m}+e^{2\pi i(j'-j)/3}+ 1 |^2
    \nonumber
    \\
    &=
    \frac{1}{9}
    ( e^{2\pi i(j'-j)/3}w^{-m'+m}+e^{-2\pi i(j'-j)/3}+ 1 )
    \nonumber
    \\
    &\hspace*{2cm}
    \times( e^{-2\pi i(j'-j)/3}w^{m'-m}+e^{2\pi i(j'-j)/3}+ 1 )
    \nonumber
    \\
    &=
    \frac{2}{9}
    \left( \Re\left[ e^{-2\pi i(j'-j)/3}w^{-m'+m} \right]
    +
    \Re\left[ e^{2\pi i(j'-j)/3}w^{-m'+m} \right]
    \right.
    \nonumber
    \\
    &\hspace*{6cm}
    \left.
    +
    \Re\left[ e^{2\pi i(j'-j)/3} \right]
    \right)
    +
    \frac{1}{3}
    \label{ans1}
  \end{align}
  となる.ここで,$J\coloneqq j'-j, M\coloneqq m'-m$遠くことにする.すると,実部の項は
  \begin{align}
    &\ \ \ 
    \Re\left[ e^{-2\pi i(j'-j)/3}w^{-m'+m} \right]
    +
    \Re\left[ e^{2\pi i(j'-j)/3}w^{-m'+m} \right]
    +
    \Re\left[ e^{2\pi i(j'-j)/3} \right]
    \nonumber
    \\
    &=
    \Re\left[ 2\Re\left[ e^{2\pi iJ/3} \right]w^{-M} \right]
    +
    \Re\left[ e^{2\pi iJ/3} \right]
    \nonumber
    \\
    &=
    2\Re\left[ e^{2\pi iJ/3} \right]\Re\left[ w^{-M} \right]
    +
    \Re\left[ e^{2\pi iJ/3} \right]
    \nonumber
    \\
    &=
    \Re\left[ e^{2\pi iJ/3} \right]
    \left( 2\Re\left[ w^{M} \right] + 1\right)
    \label{1.22}
  \end{align}
  となり,すべての$M\ (=-2,-1,1,2)$について$\Re\left[ w^{M} \right]=-1/2$が成立するので,この\eqref{1.22}の値は0である.よって,\eqref{ans1}より
  \begin{equation}
    |\ev*{e_{j}^{m},e_{j'}^{m'}}|^2
    =
    \frac{1}{3}
  \end{equation}
  が成り立つ\footnote{
    すごいごちゃごちゃしちゃいましたが,この解答はどうなんでしょう.
  }.

\end{enumerate}

\clearpage
\section{物理パート}

\prb{1}{量子力学}
\begin{enumerate}
  \item 
  次の関係式
  \begin{equation}
    \sigma_{z}\ket{\uparrow}=\ket{\uparrow}
    \ ,\ 
    \bra{\uparrow}\sigma_{z}\ket{\uparrow}
    =
    1
  \end{equation}
  が成立する.よって,$s_z=1$であり,期待値も$1$である.

  \item 
  $\sigma_{x}$の固有ベクトルは$(1,1)$と$(1,-1)$なので,
  \begin{equation}
    \begin{pmatrix}
      1 \\
      0
    \end{pmatrix}
    =
    \frac{1}{2}
    \begin{pmatrix}
      1 \\
      1
    \end{pmatrix}
    +
    \frac{1}{2}
    \begin{pmatrix}
      1 \\
      -1
    \end{pmatrix}
  \end{equation}
  である.よって,$s_{x}=\pm 1$であり,期待値は$\bra{\uparrow}\sigma_{z}\ket{\uparrow}=0$である.

  ここで,後のために,規格化した固有ベクトルを
  \begin{equation}
    \begin{dcases}
      \ket{\uparrow_{x}}
      \coloneqq
      \frac{1}{\sqrt{2}}
      \begin{pmatrix}
        1 \\
        1
      \end{pmatrix}
      \\
      \ket{\downarrow_{x}}
      \coloneqq
      \frac{1}{\sqrt{2}}
      \begin{pmatrix}
        1 \\
        -1
      \end{pmatrix}
    \end{dcases}
    \label{x_basis}
  \end{equation}
  と書くことにする.

  \item 
  固有多項式は
  \begin{equation}
    \begin{vmatrix}
      \lambda -\cos\theta & -\sin\theta \\
      -\sin\theta & \lambda + \cos\theta
    \end{vmatrix}
    =
    \lambda^2 -1=0
  \end{equation}
  であり,これを解くと$\lambda=\pm 1$である.よって,測定値は$\sigma(\theta)=\pm 1$であり,期待値は
  \begin{equation}
    \bra{\uparrow}\sigma(\theta)\ket{\uparrow}
    =
    \begin{pmatrix}
      1 & 0
    \end{pmatrix}
    \begin{pmatrix}
      \cos\theta & \sin\theta \\
      \sin\theta & -\cos\theta
    \end{pmatrix}
    \begin{pmatrix}
      1 \\
      0
    \end{pmatrix}
    =
    \cos\theta
  \end{equation}
  である.

  前問と同様に
  \begin{equation}
    \begin{dcases}
      \ket{\uparrow_{\theta}}
      \coloneqq
      \begin{pmatrix}
        \cos\tfrac{\theta}{2} \\
        \sin\tfrac{\theta}{2}
      \end{pmatrix}
      \\
      \ket{\downarrow_{\theta}}
      \coloneqq
      \begin{pmatrix}
        -\sin\tfrac{\theta}{2} \\
        \cos\tfrac{\theta}{2}
      \end{pmatrix}
    \end{dcases}
    \label{theta_basis}
  \end{equation}
  と書くことにする.

  \item 
  測定値の組は$(1,-1)$と$(-1,1)$である.このときの期待値は
  \begin{align}
    \bra{\Psi}\sigma_{z}^{A}\sigma_{z}^{B}\ket{\Psi}
    &=
    \frac{1}{2}\left( \bra{\uparrow}_{A}\bra{\downarrow}_{B}-\bra{\downarrow}_{A}\bra{\uparrow}_{B}  \right)
    \left( -\ket{\uparrow}_{A}\ket{\downarrow}_{B}+\ket{\downarrow}_{A}\ket{\uparrow}_{B}  \right)
    \nonumber
    \\
    &=
    -1
  \end{align}
  である.

  \item 
  $\ket{\Psi}$を基底\eqref{x_basis}で展開すれば
  \begin{align}
    \ket{\Psi}
    &=
    \tfrac{1}{\sqrt{2}}
    \left( \ket{\uparrow}_{A}\ket{\downarrow}_{B}
    -
    \ket{\downarrow}_{A}\ket{\uparrow}_{B} \right)
    \nonumber
    \\
    &=
    \tfrac{1}{2\sqrt{2}}
    \left( (\ket{\uparrow_{x}}_{A}+\ket{\downarrow_{x}}_{A})\otimes(\ket{\uparrow_{x}}_{B}-\ket{\downarrow_{x}}_{B})
    -
    (\ket{\uparrow_{x}}_{A}-\ket{\downarrow_{x}}_{A})\otimes(\ket{\uparrow_{x}}_{B}+\ket{\downarrow_{x}}_{B}) \right)
    \nonumber
    \\
    &=
    -\tfrac{1}{\sqrt{2}}
    \left( \ket{\uparrow_{x}}_{A}\ket{\downarrow_{x}}_{B}
    -
    \ket{\downarrow_{x}}_{A}\ket{\uparrow_{x}}_{B} \right)
  \end{align}
  となる.したがって,前問と得られる結果は同じで
  \begin{equation}
    (s_{x}^{A},s_{x}^{B})
    =
    (1,-1),(-1,1)
    \ ,\ \ 
    \bra{\Psi}\sigma_{x}^{A}\sigma_{x}^{B}\ket{\Psi}
    =
    -1
  \end{equation}
  である.

  \item 
  $\ket{\Psi}$を基底\eqref{theta_basis}で展開する.ここで,$\ket{\uparrow}$と$\ket{\downarrow}$を展開すると
  \begin{equation}
    \begin{dcases}
      \ket{\uparrow} 
      =
      \cos\frac{\theta}{2}
      \ket{\uparrow_{\theta}}
      -
      \sin\frac{\theta}{2}
      \ket{\downarrow_{\theta}}
      \\
      \ket{\downarrow}
      =
      \sin\frac{\theta}{2}
      \ket{\uparrow_{\theta}}
      +
      \cos\frac{\theta}{2}
      \ket{\downarrow_{\theta}}
    \end{dcases}
  \end{equation}
  となるので,$\ket{\Psi}$の展開は
  \begin{align}
    \ket{\Psi}
    &=
    \frac{1}{\sqrt{2}}
    \left\{ 
    \left( \cos\frac{\theta}{2}
    \ket{\uparrow_{\theta}}_{A}
    -
    \sin\frac{\theta}{2}
    \ket{\downarrow_{\theta}}_{A}\right)\otimes
    \left( \sin\frac{\theta}{2}
    \ket{\uparrow_{\theta}}_{B}
    +
    \cos\frac{\theta}{2}
    \ket{\downarrow_{\theta}}_{B} \right)
    \right.
    \nonumber
    \\
    &\hspace*{2cm}\left.
    -
    \left( \sin\frac{\theta}{2}
    \ket{\uparrow_{\theta}}_{A}
    +
    \cos\frac{\theta}{2}
    \ket{\downarrow_{\theta}}_{A} \right)\otimes
    \left( \cos\frac{\theta}{2}
    \ket{\uparrow_{\theta}}_{B}
    -
    \sin\frac{\theta}{2}
    \ket{\downarrow_{\theta}}_{B} \right)
    \right\}
    \nonumber
    \\
    &=
    \frac{1}{\sqrt{2}}(\ket{\uparrow_{\theta}}_{A}\ket{\downarrow_{\theta}}_{B}-\ket{\downarrow_{\theta}}_{A}\ket{\uparrow_{\theta}}_{B})
  \end{align}
  である.よって,測定値の組が
  \begin{equation}
    (s_{\theta}^{A},s_{\theta}^{B})
    =
    (1,-1),(-1,1)
  \end{equation}
  となることが確かめられた.

  \item 
  $\ket{\Psi}$を同様に展開すれば
  \begin{align}
    \ket{\Psi}
    &=
    \frac{1}{\sqrt{2}}
    \left\{ 
    \left( \cos\frac{\theta}{2}
    \ket{\uparrow_{\theta}}_{A}
    -
    \sin\frac{\theta}{2}
    \ket{\downarrow_{\theta}}_{A}\right)\otimes
    \left( \sin\frac{\varphi}{2}
    \ket{\uparrow_{\varphi}}_{B}
    +
    \cos\frac{\varphi}{2}
    \ket{\downarrow_{\varphi}}_{B} \right)
    \right.
    \nonumber
    \\
    &\hspace*{2cm}\left.
    -
    \left( \sin\frac{\theta}{2}
    \ket{\uparrow_{\theta}}_{A}
    +
    \cos\frac{\theta}{2}
    \ket{\downarrow_{\theta}}_{A} \right)\otimes
    \left( \cos\frac{\varphi}{2}
    \ket{\uparrow_{\varphi}}_{B}
    -
    \sin\frac{\varphi}{2}
    \ket{\downarrow_{\varphi}}_{B} \right)
    \right\}
    \nonumber
    \\
    &=
    \frac{1}{\sqrt{2}}
    \left\{ 
    \sin\frac{\varphi-\theta}{2}\ket{\uparrow_{\theta}}_{A}\ket{\uparrow_{\varphi}}_{B}
    +
    \cos\frac{\varphi-\theta}{2}\ket{\uparrow_{\theta}}_{A}\ket{\downarrow_{\varphi}}_{B}
    \right.
    \nonumber
    \\
    &\hspace*{3cm}\left.
    -
    \cos\frac{\varphi-\theta}{2}\ket{\downarrow_{\theta}}_{A}\ket{\uparrow_{\varphi}}_{B}
    -
    \sin\frac{\varphi-\theta}{2}
    \ket{\downarrow_{\theta}}_{A}\ket{\downarrow_{\varphi}}_{B}
    \right\}
  \end{align}
  となる.測定値は
  \begin{equation}
    (s_{\theta}^{A},s_{\varphi}^{B})
    =
    (1,1),(1,-1),(-1,1),(-1,-1)
  \end{equation}
  とすべての組み合わせが得られ,期待値は
  \begin{align}
    \bra{\Psi}\sigma^{A}(\theta)\sigma^{B}(\varphi)\ket{\Psi}
    &=
    \frac{1}{2}\left\{ 
    \sin^2\frac{\varphi-\theta}{2}\bra{\uparrow_{\theta}}_{A}\bra{\uparrow_{\varphi}}_{B}\sigma^{A}(\theta)\sigma^{B}(\varphi)\ket{\uparrow_{\theta}}_{A}\ket{\uparrow_{\varphi}}_{B}
    \right.
    \nonumber
    \\
    &\hspace*{1cm}
    +
    \cos^2\frac{\varphi-\theta}{2}\bra{\uparrow_{\theta}}_{A}\bra{\downarrow_{\varphi}}_{B}\sigma^{A}(\theta)\sigma^{B}(\varphi)\ket{\uparrow_{\theta}}_{A}\ket{\downarrow_{\varphi}}_{B}
    \nonumber
    \\
    &\hspace*{1cm}
    +
    \cos^2\frac{\varphi-\theta}{2}\bra{\downarrow_{\theta}}_{A}\bra{\uparrow_{\varphi}}_{B}\sigma^{A}(\theta)\sigma^{B}(\varphi)\ket{\downarrow_{\theta}}_{A}\ket{\uparrow_{\varphi}}_{B}
    \nonumber
    \\
    &\hspace*{1cm}+\left.
    \sin^2\frac{\varphi-\theta}{2}\bra{\downarrow_{\theta}}_{A}\bra{\downarrow_{\varphi}}_{B}\sigma^{A}(\theta)\sigma^{B}(\varphi)\ket{\downarrow_{\theta}}_{A}\ket{\downarrow_{\varphi}}_{B}
    \right\}
    \nonumber
    \\
    &=
    -\cos^2\frac{\varphi-\theta}{2}
  \end{align}
  となる.

  \item 
  すべての$(\sigma^{A},\sigma^{B})$の組について,$\ket{\Psi}$をもとめてみる:

  \begin{itemize}
    \item 
    $(\sigma^{A},\sigma^{B})=(0^{\circ},0^{\circ}),(120^{\circ},120^{\circ}),(240^{\circ},240^{\circ})$

    このときは
    \begin{equation}
      \ket{\Psi}
      =
      \frac{1}{\sqrt{2}}(\ket{\uparrow}_{A}\ket{\downarrow}_{B}-\ket{\downarrow}_{A}\ket{\uparrow}_{B})
    \end{equation}
    である.よって,$s^{A}s^{B}=-1$なので,この場合の期待値は$1$である.

    \item 
    $(\sigma^{A},\sigma^{B})=(0^{\circ},120^{\circ}),(0^{\circ},240^{\circ}),(120^{\circ},0^{\circ}),(120^{\circ},240^{\circ}),(240^{\circ},0^{\circ}),(240^{\circ},120^{\circ})$

    このときは
    \begin{align}
      \ket{\Psi}
      &=
      \frac{1}{\sqrt{2}}
      \left\{ 
      \pm\frac{\sqrt{3}}{2}\ket{\uparrow}_{A}\ket{\uparrow}_{B}
      +
      \frac{1}{2}\ket{\uparrow}_{A}\ket{\downarrow}_{B}
      \right.
      \nonumber
      \\
      &\hspace*{3cm}\left.
      -
      \frac{1}{2}\ket{\downarrow}_{A}\ket{\uparrow}_{B}
      \mp
      \frac{\sqrt{3}}{2}
      \ket{\downarrow}_{A}\ket{\downarrow}_{B}
      \right\}
    \end{align}
    なので,$s^{A}s^{B}=+1$となる確率は$3/4$であり,$s^{A}s^{B}=-1$である確率は$1/4$である.よって,この場合の期待値は
    \begin{equation}
      (+1)\times\frac{3}{4}+(-1)\times\frac{1}{4}
      =
      +\frac{1}{2}
    \end{equation}
    である.
  \end{itemize}

  よって,$s^{A}s^{B}$の期待値は
  \begin{align}
    s^{A}s^{B}
    &=
    \frac{1}{9}\cdot 3\cdot(-1)
    +
    \frac{1}{9}\cdot 6\cdot\left( +\frac{1}{2} \right)
    \nonumber
    \\
    &=
    0
  \end{align}
  であることが示された.

  \item 
  \begin{enumerate}
    \item 
    測定値$(s^{B}_{0^{\circ}},s^{B}_{120^{\circ}},s^{B}_{240^{\circ}})$は
    \begin{equation}
      (s^{B}_{0^{\circ}},s^{B}_{120^{\circ}},s^{B}_{240^{\circ}})
      =
      (-1,-1,-1)
    \end{equation}
    である.このとき,$s^{A}s^{B}=-1$しか得られないので,期待値は$-1$である.

    \item 
    測定値$(s^{B}_{0^{\circ}},s^{B}_{120^{\circ}},s^{B}_{240^{\circ}})$は
    \begin{equation}
      (s^{B}_{0^{\circ}},s^{B}_{120^{\circ}},s^{B}_{240^{\circ}})
      =
      (-1,-1,+1)
      .
    \end{equation}
    である.このとき,5通り$s^{A}s^{B}=-1$となる場合があり,4通り$s^{A}s^{B}=+1$となる場合があるので,期待値は$-1/9$である.

    \item 
    測定値$(s^{A}_{0^{\circ}},s^{A}_{120^{\circ}},s^{A}_{240^{\circ}})$のうち,$-1$が2つある場合も期待値は$-1/9$である.よって,全体の期待値は
    \begin{equation}
      \frac{2}{8}\cdot (-1)
      +
      \frac{6}{8}\cdot \left( -\frac{1}{9} \right)
      =
      -\frac{1}{3}
      <
      0
    \end{equation}
    である.
  \end{enumerate}
\end{enumerate}

\clearpage
\prb{2}{統計力学}
\begin{enumerate}
  \item 
  ハミルトニアンは空間成分に依存しないので
  \begin{equation}
    \int\dd^3 \bm{x}_1\cdots\dd^3 \bm{x}_N\ 
    =
    V^{N}
  \end{equation}
  と書ける.運動量の積分については
  \begin{equation}
    \int
    \dd^3 \bm{p}_1\cdots\dd^3 \bm{p}_N\ 
    e^{-\beta H}
    =
    \left( \int_{-\infty}^{\infty}\dd p\ e^{-(\beta/2m)p^2} \right)^{3N}
    =
    \left( \frac{2\pi m}{\beta} \right)^{3N/2}
  \end{equation}
  となるので,分配関数は
  \begin{equation}
    Z
    =
    \frac{V^{N}}{N!}\left( \frac{2\pi m}{\beta h^2} \right)^{3N/2}
    .
    \label{par}
  \end{equation}
  である.ここで,分配関数と自由エネルギーの関係は
  \begin{equation}
    F
    =
    -\frac{1}{\beta}\log Z
    \label{free}
  \end{equation}
  であたえられるので,$V$で微分すれば
  \begin{align}
    P
    &=
    -\pdv{F}{V}
    \nonumber
    \\
    &=
    \frac{1}{\beta}\pdv{}{V}\left( N\log V+\cdots \right)
    \nonumber
    \\
    &=
    \frac{N}{\beta V}
  \end{align}
  である.

  \item 
  ヘルムホルツの自由エネルギーは,次の示量性の性質
  \begin{equation}
    F(T,\lambda N,\lambda V)
    =
    \lambda F(T,N,V)
    \label{ext}
  \end{equation}
  を満たさなくてはならない.この性質を示すために,\eqref{par}と\eqref{free}を\eqref{ext}の左辺に代入すると
  \begin{align}
    F(T,\lambda N,\lambda V)
    &=
    -\frac{1}{\beta}
    \log\left[ \frac{(\lambda V)^{\lambda N}}{h^{3\lambda N} (\lambda N)!}\left( \frac{2\pi m}{\beta} \right)^{3\lambda N/2} \right]
    \nonumber
    \\
    &=
    -\frac{1}{\beta}\left[ 
      \lambda N\log (\lambda V)
      -
      3\lambda N\log h
      -
      \log(\lambda N)!
      +
      \frac{3\lambda N}{2}\log \frac{2\pi m}{\beta}
    \right]
    \nonumber
    \\
    &\sim
    \frac{1}{\beta}\left\{ 
      -
      \lambda N\log (\lambda V)
      +
      3\lambda N\log h
      +
      \lambda N\log (\lambda N)
      -
      \lambda N
      -
      \frac{3\lambda N}{2}\log \frac{2\pi m}{\beta}
    \right\}
    \nonumber
    \\
    &=
    \lambda\cdot
    \frac{1}{\beta}\left\{ 
      3 N\log h
      +
      N\log (N/V)
      -
      N
      -
      \frac{3N}{2}\log \frac{2\pi m}{\beta}
    \right\}
    \nonumber
    \\
    &=
    \lambda F(T,N,V)
  \end{align}
  となり,示量性を満たすことが示された.

  \item 
  エントロピーは
  \begin{align}
    S
    &=
    -\pdv{F}{T}
    \nonumber
    \\
    &=
    \frac{1}{k_B T^2}\pdv{}{\beta}\left[ \frac{1}{\beta}\left\{ 3 N\log h
    +
    N\log (N/V)
    -
    N
    -
    \frac{3N}{2}\log \frac{2\pi m}{\beta} \right\} \right]
    \nonumber
    \\
    &=
    \frac{5N}{2}
    -3N\log h
    -N\log \frac{N}{V}
    +\frac{3N}{2}\log(2\pi mk_{B}T)
  \end{align}
  である.$T\rightarrow 0$とすると,エントロピー$S$は$-\infty$に発散することがわかる.

  \item 
  熱容量は
  \begin{equation}
    C_{V}
    =
    \pdv{U}{T}
    =
    \pdv{F}{T}
    +
    S
    +
    T\pdv{S}{T}
    =
    T\pdv{S}{T}
    \label{def_cap}
  \end{equation}
  でもとめることができる.なお,エントロピーの定義
  \begin{equation}
    \pdv{F}{T}
    +
    S
    =
    0
  \end{equation}
  を用いた.\eqref{def_cap}を用いれば
  \begin{equation}
    C_{V}
    =
    \frac{3Nk}{2}
  \end{equation}
  となる.

  \item 
  波数の取りうる値をもとめるために,シュレーディンガー方程式
  \begin{equation}
    -\frac{\hbar^2}{2m}\psi(x)
    =
    E_{x}\psi(x)
    \label{sch}
  \end{equation}
  を考えよう.この解が周期境界条件$\psi(x)=\psi(x+L)$を満たしているとする.この方程式\eqref{sch}の一般解は
  \begin{equation}
    \psi(x)
    =
    A\sin(kx)+B\cos(kx)
  \end{equation}
  である.ただし,
  \begin{equation}
    k_{x}
    \coloneqq
    \sqrt{\frac{2mE_{x}}{\hbar^2}}
  \end{equation}
  とおいた.ここで,条件$\psi(x)=\psi(x+L)$を考えれば,波数の取りうる値は
  \begin{equation}
    k_{x}L=2n_{x}\pi
    \ \ \ 
    (n_{x}\in\mathbb{N})
  \end{equation}
  であることがわかる.この議論は$k_{x},k_{y},k_{z}$それぞれに適用できるので,もとめる条件は
  \begin{equation}
    \bm{k}
    =
    \frac{2\pi}{L}(n_{x},n_{y},n_{z})
  \end{equation}
  である.ただし$n_{x},n_{y},n_{z}\in\mathbb{N}$である.

  \item 
  粒子数について
  \begin{equation}
    \overline{N}
    =
    \frac{1}{\beta}\pdv{}{\mu}\log\Xi
  \end{equation}
  が成立していた.この右辺を計算すると
  \begin{align}
    \overline{N}
    &=
    \frac{1}{\beta}\pdv{}{\mu}\sum_{\bm{k}}\log(1+e^{-\beta(\varepsilon_{\bm{k}-\mu})})
    \nonumber
    \\
    \nonumber
    &=
    \frac{1}{\beta}\sum_{\bm{k}}\frac{\beta e^{-\beta(\varepsilon_{\bm{k}}-\mu)}}{1+e^{-\beta(\varepsilon_{\bm{k}}-\mu)}}
    \\
    &=
    \sum_{\bm{k}}\frac{1}{e^{\beta(\varepsilon_{\bm{k}}-\mu)}+1}
  \end{align}
  となるので
  \begin{equation}
    f(\varepsilon_{\bm{k}})
    \coloneqq
    \frac{1}{e^{\beta(\varepsilon_{\bm{k}}-\mu)}+1}
  \end{equation}
  とおけることが示された.

  \item 
  近似して計算してみると
  \begin{align}
    \overline{N}
    &\sim
    \sum_{\bm{k}}
    \exp\left[ -\beta\left( \frac{\hbar^2}{2m}(k_{x}^2+k_{y}^{2}+k_{z}^2)-\mu \right) \right]
    \nonumber
    \\
    &=
    e^{\beta\mu}\sum_{n_{x},n_{y},n_{z}}\exp\left[ -\frac{2\pi^2\hbar^2\beta^2}{mL^2}\left( n_{x}^2+n_{y}^2+n_{z}^2 \right) \right]
    \nonumber
    \\
    &\sim
    e^{\beta\mu}\cdot\sqrt{\frac{mL^2}{2\pi^2\hbar^2\beta}}\left\{ \int_{0}^{\infty}\dd x\ \exp\left[ -\frac{2\pi^2\hbar^2\beta^2}{mL^2}x^2 \right] \right\}^3
    \nonumber
    \\
    &=
    \frac{e^{\beta\mu}m^2 L^4}{32\pi^{5/2}\hbar^4 \beta^2}
  \end{align}
  となる.ここで,和の近似をするときに
  \begin{equation}
    \sum_{n=0}^{\infty}f(an)
    \sim
    \frac{1}{a}\int_{0}^{\infty}\\dx\ f(x)
  \end{equation}
  とした.なお,$a$は定数である.よって,これを整理すれば
  \begin{equation}
    \mu
    =
    \frac{1}{\beta}\log\left[ \frac{32\pi^{5/2}\hbar^4\beta^2 N}{m^2 L^4} \right]
  \end{equation}
  となる.

  \item 
  エントロピーは
  \begin{equation}
    S
    =
    -\pdv{J}{T}
    \ ,\ \ 
    J=-kT\log\Xi
    \label{def_entropy}
  \end{equation}
  でもとめることができる.これを計算していくと
  \begin{align}
    S
    &=
    k\log\Xi
    +
    kT\pdv{}{T} \log\Xi
    \nonumber
    \\
    &=
    k\sum_{\bm{k}}\log(1+e^{-\beta(\varepsilon_{\bm{k}}-\mu)})
    -
    \frac{1}{T}\pdv{}{\beta}\sum_{\bm{k}}\log(1+e^{-\beta(\varepsilon_{\bm{k}}-\mu)})
  \end{align}
  となる.ここで,$|\beta\mu|\gg 1$より,$\beta\mu\ll 1$なので第1項は無視してよい.第2項の微分は
  \begin{equation}
    \pdv{}{\beta}\sum_{\bm{k}}\log(1+e^{-\beta(\varepsilon_{\bm{k}}-\mu)})
    =
    \sum_{\bm{k}}\frac{(\mu-\varepsilon_{\bm{k}})e^{-\beta(\varepsilon_{\bm{k}}-\mu)}}{1+e^{-\beta(\varepsilon_{\bm{k}}-\mu)}}
    \sim
    \mu\sum_{\bm{k}}f(\varepsilon_{\bm{k}})
    =
    \mu N
  \end{equation}
  となるので,
  \begin{equation}
    S
    \sim
    -\frac{\mu}{T}N
  \end{equation}
  が示された.
  
  \clearpage
  \item 
  エントロピー$S$は極限では
  \begin{equation}
    S
    \rightarrow
    \gamma T
    \ \ \ (T\rightarrow0)
  \end{equation}
  となっている.$\mu<0$であることに気をつければ,図\ref{fig1}の実線のようなグラフになる.
  \begin{figure}[ht]
  \centering
  \begin{tikzpicture}    
    \draw[->,>=stealth,semithick](-0.3,0)--(7.0,0)node[below]{$T$};
    \draw[->,>=stealth,semithick](0,-0.3)--(0,3)node[left]{$S$};
    \draw[thick,samples=100,domain=0.01:6.9,name path=A]plot(\x,{2.8*ln(\x+1)/ln(6.9+1)});
    \draw[thick,samples=100,domain=0.8:6.9,name path=B,dashed]plot(\x,{ln(\x)/ln(6.9)*2.7});
  \end{tikzpicture}
  \caption{エントロピー$S$}
  \label{fig1}
  \end{figure}
  
\end{enumerate}

\clearpage
\prb{3}{電磁気学}
\begin{enumerate}
  \item 
  電場と磁場の境界条件は
  \begin{align}
    &\bm{E}(a,z,t)\cdot\bm{t}=0
    \\
    &\bm{B}(a,z,t)\cdot\bm{n}=0
  \end{align}
  である.ここで,$\bm{t}$は接線ベクトルで$\bm{n}$は法線ベクトルである.ここで,円柱芯線の内部では電場と磁場がゼロになっていることに注意する.この境界条件を用いれば
  \begin{equation}
    E_{\theta}=0
    \ ,\ \ 
    B_{r}=0
  \end{equation}
  ということがわかる.

  \item 
  式(1)の回転をとると
  \begin{equation}
    \nabla\times\left( \nabla\times\bm{E} \right)
    =
    -\pdv{}{t}\left( \nabla\times\bm{B} \right)
  \end{equation}
  となるので,式(2)を代入して,次の公式
  \begin{equation}
    \nabla\times\left( \nabla\times\bm{E} \right)
    =
    \nabla(\nabla\cdot\bm{E})-\nabla^2\bm{E}
  \end{equation}
  を用いれば
  \begin{equation}
    \nabla^2\bm{E}=\mu\varepsilon\pdv[2]{\bm{E}}{t}
    \label{wave_eq}
  \end{equation}
  となる.

  \item 
  マクスウェル方程式$(1)\sim(4)$は
  \begin{align}
    ik\mathcal{E}(r)e^{i(kz-\omega t)}\bm{e}_{\theta}
    &=
    i\omega\mathcal{B}(r)e^{i(kz-\omega t)}\bm{e}_{\theta}
    \label{eq1}
    \\
    -ik\mathcal{B}(r)e^{i(kz-\omega t)}\bm{e}_{r}
    +
    \left( \frac{1}{r}\mathcal{B}(r)+\pdv{\mathcal{B}}{r} \right)e^{i(kz-\omega t)}\bm{e}_{z}
    &=
    -i\mu\varepsilon\omega\mathcal{E}(r)e^{i(kz-\omega t)}\bm{e}_{r}
    \label{eq2}
    \\    
    \mathcal{B}(r)+r\pdv{\mathcal{B}}{r}
    &
    =0
    \\
    \mathcal{E}(r)+r\pdv{\mathcal{E}}{r}
    &=0
    \label{eq3}
  \end{align}
  と書ける.\eqref{eq3}を解くと
  \begin{equation}
    \mathcal{E}(r)
    =
    \frac{A}{r} 
  \end{equation}
  である.ここで,$A$は定数である.初期条件$\mathcal{E}(a)=E_{0}$を用いれば
  \begin{equation}
    \mathcal{E}(r)
    =
    E_{0} \frac{a}{r}
    \label{el_field}
  \end{equation}
  となる.

  \item 
  \eqref{eq1}と\eqref{eq2}より,
  \begin{equation}
    \frac{\omega}{k}
    =
    \frac{1}{\sqrt{\mu\varepsilon}}
  \end{equation}
  ということがわかり,これが位相速度である.

  \item 
  ポインティングベクトル$\bm{S}$は,
  \begin{equation}
    \bm{S}
    =
    \frac{1}{\mu}\bm{E}\times\bm{B}
    =
    \frac{1}{\mu}\mathcal{E}\mathcal{B}e^{2i(kz-\omega t)}(\bm{e}_{r}\times\bm{e}_{\theta})
    =
    \frac{1}{\mu}\mathcal{E}\mathcal{B}e^{2i(kz-\omega t)}\bm{e}_{z}
  \end{equation}
  である.ここで,\eqref{eq1}より
  \begin{equation}
    \mathcal{B}
    =
    \sqrt{\varepsilon\mu}\mathcal{E}
  \end{equation}
  なので,これを代入すれば
  \begin{equation}
    \bm{S}
    =
    \sqrt{\frac{\varepsilon}{\mu}}\frac{E_{0}^2 a^2}{r^2}e^{2i(kz-\omega t)}\bm{e}_{z}
  \end{equation}
  となる.これは単位面積あたり,単位時間あたりのエネルギー流量なので,単位時間当たりのエネルギーは,面積で積分して
  \begin{equation}
    \int S(r)\dd S
    =
    \int_{0}^{2\pi}\dd \theta\ \int_{a}^{b}r \dd r\ S(r)
    =
    2\pi
    \sqrt{\frac{\varepsilon}{\mu}}E_{0}^2 a^2 \log\frac{b}{a} e^{2i(kz-\omega t)}
  \end{equation}
  となる.

  \item 
  電場$\bm{E}$を積分すれば$V(z,t)$を得ることができる.よって,電場は\eqref{el_field}だったので
  \begin{align}
    V(z,t)
    &=
    -\int_{b}^{a} E_{r} dr
    \nonumber
    \\
    &=
    a E_{0}\log\frac{b}{a}e^{i(kz-\omega t)}
  \end{align}
  である.

  \item 
  アンペールの法則より
  \begin{equation}
    I(z,t)
    =
    \frac{1}{\mu}\oint_{C}\bm{B}(\bm{r})\cdot\dd \bm{r}
  \end{equation}
  が成立する.今回は,半径$r$の経路をとってみると,
  \begin{equation}
    I(z,t)
    =
    \frac{2\pi r}{\mu}\cdot \sqrt{\varepsilon\mu}E_{\theta}(r,z,t)
    =
    2\pi a\sqrt{\frac{\varepsilon}{\mu}}E_{0}e^{i(kz-\omega t)}
  \end{equation}
  となる.よって,特性インピーダンスは
  \begin{equation}
    Z_{0}
    =
    \frac{V(z,t)}{I(z,t)}
    =
    \frac{1}{2\pi}\sqrt{\frac{\mu}{\varepsilon}}\log\frac{b}{a}
  \end{equation}
  となる.

  \item 
  $|V(z,t)|$を計算してみると
  \begin{align}
    \nonumber
    \left|
      V(z,t)
    \right|
    &=
    \frac{1}{\sqrt{2\pi}}\left|
      \int_{-\infty}^{\infty}
      v(k)
      e^{ikz-i\omega(k_0)t-iAkt+iAk_0 t}
      \dd k
    \right|
    \\
    \nonumber
    &=
    \frac{1}{\sqrt{2\pi}}\left|
      \int_{-\infty}^{\infty}
      v(k)
      e^{i(z-At)k}
      \dd k
    \right|
    \\
    &=
    \left|
      \tilde{v}(z-At)
    \right|
  \end{align}
  となる.ここで
  \begin{equation}
    \tilde{v}(z)
    \coloneqq
    \frac{1}{\sqrt{2\pi}}
    \int_{-\infty}^{\infty}
      v(k)
      e^{ikz}
      \dd k
  \end{equation}
  は,$v(k)$のフーリエ変換である.よって,波形が変化せずに伝搬されることが示された.
\end{enumerate}


\end{document}
