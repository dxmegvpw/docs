\pdfoutput=1
\documentclass[a4paper,pdflatex,ja=standard]{bxjsarticle}

% ---Setting about the geometry of the document----
% \usepackage{a4wide}
% \pagestyle{empty}

% ---Physics and Math Packages---
\usepackage{amssymb,amsfonts,amsthm,mathtools}
\usepackage{physics,braket,bm}

% ---underline---
\usepackage{ulem}

% --- sorround the texts or equations
% \usepackage{fancybox,ascmac}

% ---settings of theorem environment---
% \usepackage{amsthm}
% \theoremstyle{definition}

% ---settings of proof environment---
% \renewcommand{\proofname}{\uline{\textbf{証明}}}
% \renewcommand{\qedsymbol}{$\blacksquare$}

% ---Ignore the Warnings---
\usepackage{silence}
\WarningFilter{latexfont}{Some font shapes,Font shape}

% ---Insert the figure (If insert the `draft' at the option, the process becomes faster)---
\usepackage{graphicx}
% \usepackage{subcaption}

% ----Add a link to a text---
\usepackage{url}
\usepackage{xcolor,hyperref}
\hypersetup{colorlinks=true,citecolor=orange,linkcolor=blue,urlcolor=magenta}
\usepackage{bxcjkjatype}

% ---Tikz---
% \usepackage{tikz,pgf,pgfplots,circuitikz}
% \pgfplotsset{compat=1.15}
% \usetikzlibrary{intersections,arrows.meta,angles,calc,3d,decorations.pathmorphing}

% ---Add the section number to the equation, figure, and table number---
\makeatletter
   \renewcommand{\theequation}{\thesubsection.\arabic{equation}}
   \@addtoreset{equation}{subsection}
   
   \renewcommand{\thefigure}{\thesection.\arabic{figure}}
   \@addtoreset{figure}{section}
   
   \renewcommand{\thetable}{\thesection.\arabic{table}}
   \@addtoreset{table}{section}
\makeatother

% ---enumerate---
\renewcommand{\labelenumi}{\arabic{enumi}.}
\renewcommand{\labelenumii}{(\roman{enumii})}
\renewcommand{\labelenumiii}{(\alph{enumiii})}

% ---Index---
% \usepackage{makeidx}
% \makeindex

% ---Fonts---
\renewcommand{\familydefault}{\sfdefault}

% ---Title---
\title{東京大学\ 平成27年\ 物理学専攻\ 院試\ 解答例}
\author{ミヤネ}
\date{最終更新:\today}

\newcommand{\prb}[2]{
  \phantomsection
  \addcontentsline{toc}{subsection}{問題 #1: #2}
  \subsection*{第#1問\phantom{#2}}
  \setcounter{subsection}{#1}
  \setcounter{equation}{0}
}

\begin{document}

\maketitle

\tableofcontents
\clearpage

\section{数学パート}

\prb{1}{線形代数}

\begin{enumerate}

  \item 

  \begin{enumerate}


    \item 
    
    具体的に
    \begin{equation}
      A
      =
      \begin{pmatrix}
        a & b \\
        c & d
      \end{pmatrix}
    \end{equation}
    とおけば
    \begin{equation}
      \det(xE-A)
      =
      (x-a)(x-d)-bc
      =
      x^2-(\tr A)x+\det A
      =
      0
      .
    \end{equation}
  

    \item 

    $x=A$を代入すれば
    \begin{equation}
      A^2
      =
      (\tr A)A
      -
      E
      =
      2\xi A-E
    \end{equation}
    です.あとは帰納法でいけます.問題文の式(3)が成立していると仮定して,$A^{N+1}$を評価してみましょう.すると
    \begin{align}
      A^{N+1}
      &=
      (U_{N-1}(\xi)A-U_{N-2}(\xi))A
      \nonumber
      \\
      &=
      U_{N-1}(\xi)(2\xi A-E)-(-U_{N}(\xi)+2\xi U_{N-1}(\xi))A
      \nonumber
      \\
      &=
      U_{N}(\xi)A-U_{N-1}(\xi)E
    \end{align}
    となって,$N+1$の場合でも成立することが示されます.

  \end{enumerate}


  \item 

  部分積分をゴリ押しましょう:
  \begin{align}
    \int_{a}^{b}
    v^{*}(x)\mathcal{L}u(x)    
    \dd x
    &=
    \int_{a}^{b}
    \dd x\ 
    v^{*}(x)\dv{}{x}\left[ p(x)\dv{u(x)}{x} \right] 
    \nonumber
    \\
    &=
    \left[  
      v^{*}(x)p(x)\dv{u(x)}{x}
    \right]_{a}^{b}
    -
    \int_{a}^{b}
    \dd x\ 
    \dv{v^{*}(x)}{x}p(x)\dv{u(x)}{x}
    \nonumber
    \\
    &=
    -
    \left[  
      p(x)\dv{v^{*}(x)}{x}u(x)
    \right]_{a}^{b}
    +
    \int_{a}^{b}
    \dd x\ 
    \dv{}{x}\left[  
      p(x)\dv{v^{*}(x)}{x}
    \right]u(x)
    \nonumber
    \\
    &=
    \int_{a}^{b}
    u(x)(\mathcal{L}v(x))^{*}
    \dd x
    .
  \end{align}
  ここで,$p(a)=p(b)=0$の条件を用いて表面項を消しました.また,今回の結果は複素共役がなくても成立することとに注意しましょう.つまり
  \begin{equation}
    \int_{a}^{b}
    v(x)\mathcal{L}u(x)
    \dd x
    =
    \int_{a}^{b}
    u(x)(\mathcal{L}v(x))
    \dd x
    \label{uv_rel}
  \end{equation}
  です.後で使います.


  \item 

  前問の結果で$v=u$とします.すると,
  \begin{equation}
    \int_{a}^{b}
    u^{*}(x)\mathcal{L}u(x)\dd x
    =
    \int_{a}^{b}
    u(x)(\mathcal{L}u(x))^{*}\dd x
  \end{equation}
  となります.両辺に式(6)を代入すると,$w(x)$が実関数であることから
  \begin{equation}
    \lambda
    \int_{a}^{b}w(x)u(x)u^{*}(x)\dd x
    =
    \lambda^{*}
    \int_{a}^{b}w(x)u(x)u^{*}(x)\dd x
  \end{equation}
  となり,$\lambda=\lambda^{*}$.よって,$\lambda$は実数です.

  式(7)については,式(6)を逆に使いましょう.すると
  \begin{align}
    \int_{a}^{b}
    u_1u_2^{*}w
    \dd x
    &=
    \frac{1}{\lambda_1}
    \int_a^b
    u_2^* \mathcal{L}u_1
    \dd x
    \nonumber
    \\
    &=
    \frac{1}{\lambda_1}
    \int_a^b
    u_1\mathcal{L}u_2^*
    \dd x
    \nonumber
    \\
    &=
    \frac{\lambda_2^*}{\lambda_1}
    \int_a^b
    u_1u_2^*w
    \dd x
  \end{align}
  となるので,
  \begin{equation}
    \int_a^b
    u_1u_2^*w
    \dd x
    =    
    \frac{\lambda_2^*}{\lambda_1}
    \int_a^b
    u_1u_2^*w
    \dd x
  \end{equation}
  となりますが,$\lambda_2^*/\lambda_1\neq 1$なので
  \begin{equation}    
    \int_a^b
    u_1(x)u_2^*(x)w(x)
    \dd x
    =
    0
  \end{equation}
  です.


  \item 

  ちょっと考えてみると$p(x)=(1-x^2)^{3/2}$とおけば,うまくいくことが分かります\footnote{
    探し方はいろいろあると思いますが,次の設問から,なんとなく$w(x)=(1-x^2)^{1/2}$になりそうだと分かります.(これは$w(x)>0$を満たしています.)あとは,漸化式(8)を使うことから,$p(x)$を微分したら$(1-x^2)^{1/2}$がでてきそうだとあたりをつけることができ,$U_N'(x)$の係数が$3x$なので,$(1-x^2)^{3/2}$を微分したときにでてきたものだと,予想することができます.
  }.$\mathcal{L}U_N(x)$を計算してみると
  \begin{equation}
    \mathcal{L}U_N(x)
    =
    (1-x^2)^{1/2}\left[  
      (1-x^2)\dv[2]{U_N}{x}-3x\dv{U_N}{x}
    \right]
    =
    -N(N+2)(1-x^2)^{1/2}U_N(x)
  \end{equation}
  となり,$\lambda=-N(N+2), w(x)=(1-x^2)^{1/2}$であることが分かります.さて,等式(9)ですが,設問2の最後に言及した式を用いれば直ちに終わります.設問3と同様にして
  \begin{align}
    \int_{-1}^{1}
    U_M(x)U_N(x)(1-x^2)^{1/2}
    \dd x
    &=
    -\frac{1}{N(N+2)}    
    \int_{-1}^{1}
    U_M(x)\mathcal{L}U_N(x)
    \dd x
    \nonumber
    \\
    &=
    -\frac{1}{N(N+2)}    
    \int_{-1}^{1}
    U_N(x)\mathcal{L}U_M(x)
    \dd x
    \nonumber
    \\
    &=
    \frac{M(M+2)}{N(N+2)}
    \int_{-1}^{1}
    U_M(x)U_N(x)(1-x^2)^{1/2}
    \dd x
  \end{align}
  となります.したがって,$N\neq M$より,
  \begin{equation}
    \int_{-1}^{1}
    U_M(x)U_N(x)(1-x^2)^{1/2}
    \dd x
    =
    0
  \end{equation}
  です.

\end{enumerate}


\clearpage
\prb{2}{微積分}
\begin{enumerate}

  \item 

  フーリエ変換を
  \begin{equation}
    f(x)
    =
    \int\frac{\dd k}{2\pi}
    \tilde{f}'(k,t)e^{-ikx}
  \end{equation}
  とすれば,
  \begin{equation}
    \int
    \frac{\dd k}{2\pi}
    \pdv{\tilde{f}'(k,t)}{t}e^{-ikx}
    =
    \int\frac{\dd k}{2\pi}
    \left( -\lambda k^2\tilde{f}'(k,t) \right)e^{-ikx}
  \end{equation}
  なので,$\tilde{f}'$が満たす式は
  \begin{equation}
    \pdv{\tilde{f}'(k,t)}{t}
    =
    -\lambda k^2\tilde{f}'(k,t)
  \end{equation}
  です.これを解けば
  \begin{equation}
    \tilde{f}'(k,t)
    =
    \tilde{f}(k)e^{-\lambda k^2 t}
  \end{equation}
  となるので,元のフーリエ変換の式に戻せば式(2)が示されます.


  \item 

  \begin{align}
    &\quad
    \pdv{f}{t}-\lambda\pdv[2]{f}{x}-S
    \nonumber
    \\
    &=
    \int\dd t'\int\dd x'\ 
    \left[  
      \pdv{G(t,x,t',x')}{t}-\lambda\pdv[2]{G(t,x,t',x')}{x}
    \right]
    S(t',x')
    -
    S(t,x)
    \nonumber
    \\
    &=
    \int\dd t'\int\dd x'\ 
    S(t',x')\delta(t-t')\delta(x-x')
    -
    S(t,x)    
    =
    0
    .
  \end{align}


  \item 

  式(5)を式(3)に代入すれば
  \begin{equation}
    \int\frac{\dd \omega}{2\pi}\int\frac{\dd k}{2\pi}
    \left[  
      \frac{C(i\omega+\lambda k^2)}{\omega-i\alpha\lambda k^2}
      -
      1
    \right]
    e^{i\omega(t-t')-ik(x-x')}
    =    
    0
    \label{eqn01}
  \end{equation}
  です.ただし,$\delta$-関数は
  \begin{equation}
    \delta(x)
    =
    \int\frac{\dd p}{2\pi}e^{\pm ipx}
  \end{equation}
  で展開しました.\eqref{eqn01}が成立するためには
  \begin{equation}
    C=-i
    ,\ 
    \alpha=1
  \end{equation}
  であることが必要です.


  \item 

  $C,\alpha$が決まったので
  \begin{equation}
    G(t,x,t',x')
    =
    -i
    \int\frac{\dd \omega}{2\pi}\int\frac{\dd k}{2\pi}
    \frac{1}{\omega-i\lambda k^2}
    e^{i\omega(t-t')-ik(x-x')}
  \end{equation}
  を計算しましょう.極が$\omega=+i\lambda k^2$なので,この積分は経路が上半平面を通ったときに値をもちます.$t-t'>0$のときに,上半平面の積分で収束するので
  \begin{equation}
    G(t,x,t',x')
    =
    \left\{
      \begin{alignedat}{1}
        \int\frac{\dd k}{2\pi}e^{-\lambda k^2(t-t')-ik(x-x')}
        &
        \qquad
        (t>t')
        \\
        0
        &
        \qquad
        (t<t')
      \end{alignedat}
    \right.    
  \end{equation}
  です.


  \item 

  指数の肩を平方完成すれば
  \begin{equation}
    G(t,x,t',x')
    =
    \frac{1}{2}
    \exp\left[  
      -\frac{(x-x')^2}{4\lambda(t-t')}
    \right]
    \sqrt{\frac{1}{\pi\lambda(t-t')}}
  \end{equation}
  です.


  \item 

  式(4)を計算します.前問の$G$を代入すれば
  \begin{align}
    f(x,t)
    &=
    \frac{1}{2}
    \int\dd t'\int\dd x'\ 
    \exp\left[  
      -\frac{(x-x')^2}{4\lambda(t-t')}
    \right]
    \sqrt{\frac{1}{\pi\lambda(t-t')}}
    \delta(t')\cos(px')
    \nonumber
    \\
    &=
    \frac{1}{2\sqrt{\pi\lambda t}}
    \int\dd x'
    \exp\left[  
      -\frac{(x-x')^2}{4\lambda t}
    \right]
    \cos(px')
  \end{align}
  となります.最後の積分ですが,こういったのは$\cos$を$\exp$で書きなおしてガウス積分してやりましょう.すると,指数の肩は
  \begin{equation}
    -\frac{(x-x')^2}{4\lambda t}\pm ipx'
    =
    -\frac{1}{4\lambda t}
    \left(  
      x'-(x\pm 2i\lambda pt)
    \right)^2
    \pm
    ipx
    -
    \lambda p^2t
  \end{equation}
  と平方完成できるので,
  \begin{align}    
    &\quad
    \frac{1}{2\sqrt{\pi\lambda t}}
    \int\dd x'
    \exp\left[  
      -\frac{(x-x')^2}{4\lambda t}
    \right]
    \cos(px')
    \nonumber
    \\
    &=
    \frac{e^{-\lambda p^2 t}}{4\sqrt{\pi\lambda t}}
    \left[  
      e^{+ipx}
      \int \dd z_{+}\ e^{-z_{+}^2/4\lambda t}
      +
      e^{-ipx}
      \int \dd z_{-}\ e^{-z_{-}^2/4\lambda t}
    \right]
    \nonumber
    \\
    &=
    e^{-\lambda p^2t}\cos(px)
  \end{align}
  となります.ただし,$z_{\pm}=x\pm 2i\lambda pt$と変数変換しました.したがって,
  \begin{equation}
    f(x,t)
    =
    e^{-\lambda p^2t}\cos(px)
  \end{equation}
  であり,$t$を固定したときの$f(x,t)$の最大値は
  \begin{equation}
    x
    =
    \frac{2\pi n}{p}
    \quad
    (\text{ただし,$n$は整数})
    \text{のとき,}
    \qquad
    f_{\max}
    =
    e^{-\lambda p^2 t}
  \end{equation}
  です.  

\end{enumerate}


\clearpage
\section{物理パート}

\prb{1}{量子力学}

\begin{enumerate}

  \item 

  Schrödinger方程式は
  \begin{equation}
    -\frac{\hbar^2}{2m}\dv[2]{\psi(x)}{x}
    =
    E\psi(x)
  \end{equation}
  なので,$k\equiv\sqrt{2mE/\hbar^2}$とおけば
  \begin{equation}
    \psi(x)
    =
    Ce^{+ikx}
  \end{equation}
  が解です.境界条件$\psi(x)=\psi(x+2\pi L)$が任意の点について成立しているから
  \begin{equation}
    e^{2\pi ikL}=1
  \end{equation}
  です.したがって,
  \begin{equation}
    2\pi ik_{n}L=2\pi n
    \quad
    (n\in\mathbb{Z})
  \end{equation}
  となるので
  \begin{equation}
    E_{n}
    =
    \frac{\hbar^2}{2mL^2}n^2
  \end{equation}
  と量子化されます.
  
  また,縮退については,波動関数が
  \begin{equation}
    \psi_{n}(x)
    =
    C\exp\left[ +i\frac{n}{L}x \right]
  \end{equation}
  となることから
  \begin{equation}
    \left\{
      \begin{alignedat}{1}
        n=0\text{\ のとき,}
        \ 
        &
        \ 
        \text{縮退なし.}
        \\
        n\neq0\text{\ のとき,}
        \ 
        &
        \ 
        \text{
          $\psi_{-n}(x)$と$\psi_{n}(x)$が独立なので,2重に縮退している.($E_{n}=E_{-n}$)
        }
      \end{alignedat}
    \right.
  \end{equation}
  です.


  \item 

  Schrödinger方程式
  \begin{equation}
    -\frac{\hbar^2}{2m}
    \dv[2]{\psi(x)}{x}
    +
    \frac{\hbar^2 v}{2m}\delta(x)
    =
    E\psi(x)
  \end{equation}
  の両辺を原点付近$x\in(-\varepsilon,+\varepsilon)$で積分すれば
  \begin{equation}
    -
    \frac{\hbar^2}{2m}
    \left(  
      \dv{\psi(+\varepsilon)}{x}
      -
      \dv{\psi(-\varepsilon)}{x}
    \right)
    =
    \int_{-\varepsilon}^{+\varepsilon}
    \left(  
      E-\frac{\hbar^2 v}{2m}\delta(x)
    \right)
    \psi(x)
    \dd x
  \end{equation}
  です.$\varepsilon\rightarrow0$の極限を考えれば,$E$の積分は消えて
  \begin{equation}
    \psi'(+0)
    -
    \psi'(-0)
    =
    v\psi(0)
  \end{equation}
  となります.

  
  \item 

  自由場のSchrödinger方程式を満たすことは計算すればすぐに分るでしょう.

  $x=0$における接続条件は$\psi(0)=\psi(2\pi L)$です.よって,$1+A=e^{2\pi\kappa L}+Ae^{-2\pi\kappa L}$であり,
  \begin{equation}
    A
    =
    e^{2\pi\kappa L}
  \end{equation}
  です.したがって,波動関数は
  \begin{equation}
    \psi(x)
    =
    e^{\kappa x}
    +
    e^{2\pi\kappa L}e^{-\kappa x}
  \end{equation}
  であり,境界条件は前問の結果を用いて
  \begin{equation}
    \kappa
    (1-e^{2\pi\kappa L})
    -
    \kappa
    (e^{2\pi\kappa L})
    =
    v(1+e^{2\pi\kappa L})
  \end{equation}
  です.よって
  \begin{equation}
    v
    =
    -2\kappa\tanh(\pi\kappa L)
  \end{equation}
  がもとめる関係です.グラフは略.


  \item 

  今回は$e^{ikx}$で展開できるので,前問と同様にして
  \begin{equation}
    \psi(x)
    =
    e^{+ikx}+Be^{-ikx}
    \label{wave_fun}
  \end{equation}
  としましょう.境界条件$\psi(0)=\psi(2\pi L)$を課すと,
  \begin{equation}
    B
    =
    e^{2\pi ikL}
  \end{equation}
  ともとまります.したがって,同様の計算をすれば
  \begin{equation}
    v
    =
    2k\tan(\pi kL)
  \end{equation}
  です\footnote{
    問題文の「ポテンシャルエネルギーの影響を受けない解もあることに留意せよ」というのは,波動関数が$e^{+ikx}$と$e^{-ikx}$で展開できるということを意味しているように思います.基本的に自由場の解は$\psi(x)=e^{\pm ikx}$と書けますが,それは波が1方向にしか進まないからです.しかしながら,ポテンシャルの存在によって,進行波と後退波が入り混じることになり,波動関数が\eqref{wave_fun}の形で書けるようになります.おそらく,このことを言いたかったのだと思っています.
  }.


  \item 

  波動関数\eqref{wave_fun}
  \begin{equation}
    \psi(x)
    =
    \sqrt{\frac{k}{4\pi kL+\sin(2\pi kL)}}
    (e^{+ikx}+e^{2\pi ikL}e^{-ikx})
  \end{equation}
  と規格化できるので,摂動の1次は
  \begin{align}
    E^{(1)}
    &=
    \frac{\hbar^2 v}{2m}
    \cdot
    \frac{k}{4\pi kL+\sin(2\pi kL)}
    \int_{0}^{2\pi L}
    (1+e^{-2\pi kL})\delta(x)(1+e^{2\pi kL})
    \dd x
    \nonumber
    \\
    &=
    \frac{\hbar^2 kv}{m}
    \cdot
    \frac{1+\cos(2\pi kL)}{4\pi kL+\sin(2\pi kL)}
  \end{align}
  です\footnote{
    $v<0$のときは
  }.

\end{enumerate}


\clearpage

\prb{2}{統計力学}

\begin{enumerate}

  \item 

  $L$についての帰納法で示しましょう.$L+x+1$が奇数なら,$L+x-1$も奇数なので$C(x-1,L),C(x+1,L)$も0.$L+x+1$が偶数なら
  \begin{align}
    C(x,L+1)
    &=
    \dfrac{L!}{\underbrace{\left( \dfrac{L-x+1}{2} \right)!}_{\dfrac{L-x+1}{2}\left( \dfrac{L-x-1}{2} \right)!}\left( \dfrac{L+x-1}{2} \right)!}
    +
    \dfrac{L!}{\left( \dfrac{L-x-1}{2} \right)!\underbrace{\left( \dfrac{L+x+1}{2} \right)!}_{\dfrac{L+x+1}{2}\left( \dfrac{L+x-1}{2} \right)!}}
    \nonumber
    \\
    &=
    \dfrac{\dfrac{L+x+1}{2}+\dfrac{L-x+1}{2}}{\left( \dfrac{L+-x+1}{2} \right)!\left( \dfrac{L+x+11}{2} \right)!}L!
    =
    \dfrac{(L+1)!}{\left( \dfrac{L+-x+1}{2} \right)!\left( \dfrac{L+x+11}{2} \right)!}
  \end{align}
  で,確かに成立しています.


  \item 

  式(1)の両辺に代入すれば
  \begin{equation}
    \frac{1}{2\pi}
    \int_{-\pi}^{\pi}
    e^{ikx}\tilde{C}(k,L+1)
    \dd k
    =
    \frac{1}{2\pi}
    \int_{-\pi}^{\pi}
    e^{ikx}
    \left[  
      e^{-ik}\tilde{C}(k,L)
      +
      e^{+ik}\tilde{C}(k,L)
    \right]
    \dd k
  \end{equation}
  となるので
  \begin{equation}
    \tilde{C}(k,L+1)
    =
    2\cos k
    \tilde{C}(k,L)
  \end{equation}
  です.


  \item 

  $L=0$のときは,$x=0$のときのみなので
  \begin{equation}
    C(x,0)
    =
    \delta(x)
  \end{equation}
  です.ただし,$\delta(x)$は$x=0$のときが1で,その他の$x$では0.したがって,
  \begin{equation}
    \tilde{C}(k,0)
    =
    \sum_{x=-\infty}^{\infty}
    e^{-ikx}C(x,0)
    =
    1
  \end{equation}
  が初期条件です.よって,前問より
  \begin{equation}
    \tilde{C}(k,L)
    =
    (2\cos k)^L\tilde{C}(k,0)
    =
    (2\cos k)^L
  \end{equation}
  です.


  \item 

  $\log \tilde{C}$を計算してみると
  \begin{equation}
    \log \tilde{C}(k,L)
    =
    L\log 2
    +
    L\log \cos k
  \end{equation}
  です.$f(k)\equiv\log\cos k$とおいて,展開すると
  \begin{equation}
    f(k)
    \sim
    -\frac{1}{2}k^2
  \end{equation}
  となるので,
  \begin{equation}
    \tilde{C}(k,L)
    \sim
    2^L e^{-k^2/2}
  \end{equation}
  で計算します.今,$k$が小さいので積分範囲は$\pi\sim\infty$で近似してよいでしょう.よって
  \begin{equation}
    C(x)
    \sim
    \frac{2^L}{2\pi}
    \int_{-\infty}^{\infty}
    e^{ikx-k^2/2}
    \dd k
    =
    \frac{2^L}{\sqrt{2\pi L}}e^{-x^2/2L}
  \end{equation}
  となります.


  \item 

  系のエネルギーは$U=-qEx$です.分配関数は取りうる場合の数を足し上げることに気をつければ
  \begin{equation}
    Z
    =
    \sum_{x=-L}^{L}
    C(x,L)
    e^{\beta qEx}
  \end{equation}
  です.これをさらに計算しましょう.式(2)を用いれば
  \begin{equation}
    Z
    =
    \sum_{x=-L}^{L}
    \dfrac{L!}{
      \left( \dfrac{L-x}{2} \right)!
      \left( \dfrac{L+x}{2} \right)!
    }
    e^{\beta qEx}
  \end{equation}
  と書けますが,$L+x$が偶数だということが分かれば$n\equiv (L+x)/2$とおいて$n=0,\cdots, L$の総和に変換することができます.すると,$x=2n-L$なので
  \begin{equation}
    Z
    =
    e^{-\beta qEL}
    \sum_{n=0}^{L}
    \dfrac{L!}{(L-n)!n!}
    e^{2\beta qEn}
  \end{equation}
  です.ここで,2項定理が使えることに気がつければ
  \begin{equation}
    Z
    =
    e^{-\beta qEL}
    (1+e^{2\beta qE})^L
    =
    (2\cosh(\beta qE))^L
  \end{equation}
  となります.


  \item 

  $x$の期待値は
  \begin{equation}
    \ev*{x}
    =
    \frac{1}{Z}
    \sum_{x=-L}^{L}
    x
    C(x,L)
    e^{\beta qEx}
    =
    \frac{1}{Z}\cdot\frac{1}{\beta q}
    \pdv{}{E}
    \left(  
      \sum_{x=-L}^{L}
      C(x,L)
      e^{\beta qEx}
    \right)
    =
    \frac{1}{\beta q}
    \pdv{}{E}\log Z
  \end{equation}
  と書けるので,
  \begin{equation}
    \ev*{x}
    =
    \frac{Lk_BT}{q}
    \pdv{}{E}
    \log\cosh(\beta qE)
    =
    L\tanh\left( \frac{qE}{k_BT} \right)
  \end{equation}
  です.


  \item 

  $\ev*{x^2}$は
  \begin{equation}
    \ev*{x^2}
    =
    \frac{1}{Z}
    \sum_{x=-L}^{L}
    x^2
    C(x,L)
    e^{\beta qEx}
  \end{equation}
  ですが,一方で$\ev*{x}$を$E$で微分してみると
  \begin{align}
    \pdv{\ev*{x}}{E}
    &=
    \pdv{}{E}
    \left[  
      \frac{1}{Z[E]}
      \sum_{x=-L}^{L}
      x
      C(x,L)
      e^{\beta qEx}
    \right]
    \nonumber
    \\
    &=
    -\frac{Z'[E]}{Z^2[E]}
    \sum_{x=-L}^{L}
    x
    C(x,L)
    e^{\beta qEx}
    +
    \frac{q}{k_BT}
    \cdot
    \frac{1}{Z[E]}
    \sum_{x=-L}^{L}
    x^2
    C(x,L)
    e^{\beta qEx}
    \nonumber
    \\
    &=
    -\frac{Z'[E]}{Z[E]}
    \ev*{x}
    +
    \frac{q}{k_BT}\ev*{x^2}
  \end{align}
  となります.$E=0$とすると,$\ev*{x}=0$となるので,
  \begin{equation}
    \ev*{x^2}_{E=0}
    =
    \frac{k_BT}{q}
    \left.\pdv{\ev*{x}}{E}\right|_{E=0}
  \end{equation}
  となります.よって,比例定数は$k_BT/q$です.

\end{enumerate}


\clearpage

\prb{3}{電磁気学}

\begin{enumerate}

  \item 
  
  電場$E$については荷電粒子群は単位長さあたり$q/a$の電荷をもつ導体だとみなせます.したがって,単位長さの円筒を考えて,ガウスの法則を適用すれば
  \begin{equation}
    2\pi x\cdot E_x
    =
    \frac{q}{\varepsilon_0 a}
  \end{equation}
  であり,電場は
  \begin{equation}
    \bm{E}
    =
    \left(  
      \frac{q}{2\pi\varepsilon_0 xa}
      ,
      0
      ,
      0
    \right)
  \end{equation}
  です.磁場$B$については,電流密度$\bm{j}=qv_z/a$の電流とみなせるので,直線電流が距離$x$の位置に作る電流を考えて
  \begin{equation}
    \bm{B}
    =
    \left(  
      0
      ,
      \frac{\mu_o qv_z}{2\pi xa}
      ,
      0
    \right)
  \end{equation}
  となります.


  \item 

  $t=0$で$z=0,a$にある荷電粒子をローレンツ変換してみると$z'=0,\gamma a$に移るのでOK\footnote{
    もっと一般に示すことはできると思います.
  }.

  
  \item 

  電荷は$z$軸上で静止しているので,電場だけが存在して
  \begin{equation}
    \bm{E}'
    =
    \left(  
      \frac{q}{2\pi\varepsilon_0 \gamma a}
      \cdot
      \frac{x'}{({x'}^2+{y'}^2)}
      ,
      \frac{q}{2\pi\varepsilon_0 \gamma a}
      \cdot
      \frac{y'}{{x'}^2+{y'}^2}
      ,
      0
    \right)
    ,\ 
    \bm{B}'
    =
    \left(  
      0
      ,
      0
      ,
      0
    \right)
  \end{equation}
  です.


  \item 

  前問の結果を代入すれば
  \begin{align}
    (E_x,E_y,E_z)
    &=
    \left(  
      \frac{q}{2\pi\varepsilon_0 a}
      \cdot
      \frac{x}{x^2+y^2}
      ,
      \frac{q}{2\pi\varepsilon_0 a}
      \cdot
      \frac{y}{x^2+y^2}
      ,
      0
    \right)
    \nonumber
    \\
    (B_x,B_y,B_z)
    &=
    \left(  
      -
      \frac{\mu_0qv_z}{2\pi a}
      \cdot
      \frac{y}{x^2+y^2}
      ,
      \frac{\mu_0qv_z}{2\pi a}
      \cdot
      \frac{x}{x^2+y^2}
      ,
      0
    \right)
  \end{align}
  となります.$y=0$とすれば設問1と一致.


  \item 

  原点で静止しているので
  \begin{equation}
    \bm{E}'
    =
    \frac{q}{4\pi \varepsilon_0 {r'}^3}(x',y',z')
    ,\ 
    \bm{B}'
    =
    0
  \end{equation}
  です.

  
  \item 

  設問4と同様の計算をすれば
  \begin{align}
    (E_x,E_y,E_z)
    &=
    \frac{\gamma q}{2\pi\varepsilon_0 (x^2+y^2+\gamma^2(z-v_zt)^2)^{3/2}}
    \left(  
      x
      ,
      y
      ,
      z-v_zt
    \right)
    \nonumber
    \\
    (B_x,B_y,B_z)
    &=
    \frac{\gamma\mu_0qv_z}{4\pi(x^2+y^2+\gamma^2(z-v_z t)^2)^{3/2}}
    \left(   
      -y
      ,
      x
      ,
      0
    \right)
  \end{align}
  となります.よって,$y,z=0$とすれば
  \begin{align}
    (E_x,E_y,E_z)
    &=
    \frac{\gamma q}{2\pi\varepsilon_0 (x^2+(\gamma v_zt)^2)^{3/2}}
    \left(  
      x
      ,
      0
      ,
      -v_zt
    \right)
    \nonumber
    \\
    (B_x,B_y,B_z)
    &=
    \frac{\gamma\mu_0qv_z}{4\pi(x^2+(\gamma v_z t)^2)^{3/2}}
    \left(   
      0
      ,
      x
      ,
      0
    \right)
  \end{align}
  です.

\end{enumerate}


\end{document}
