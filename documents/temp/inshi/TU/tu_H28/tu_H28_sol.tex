\pdfoutput=1
\documentclass[a4paper,pdflatex,ja=standard]{bxjsarticle}

% ---Setting about the geometry of the document----
% \usepackage{a4wide}
% \pagestyle{empty}

% ---Physics and Math Packages---
\usepackage{amssymb,amsfonts,amsthm,mathtools}
\usepackage{physics,braket,bm}

% ---underline---
\usepackage{ulem}

% --- sorround the texts or equations
% \usepackage{fancybox,ascmac}

% ---settings of theorem environment---
% \usepackage{amsthm}
% \theoremstyle{definition}

% ---settings of proof environment---
% \renewcommand{\proofname}{\uline{\textbf{証明}}}
% \renewcommand{\qedsymbol}{$\blacksquare$}

% ---Ignore the Warnings---
\usepackage{silence}
\WarningFilter{latexfont}{Some font shapes,Font shape}

% ---Insert the figure (If insert the `draft' at the option, the process becomes faster)---
\usepackage{graphicx}
% \usepackage{subcaption}

% ----Add a link to a text---
\usepackage{url}
\usepackage{xcolor,hyperref}
\hypersetup{colorlinks=true,citecolor=orange,linkcolor=blue,urlcolor=magenta}
\usepackage{bxcjkjatype}

% ---Tikz---
% \usepackage{tikz,pgf,pgfplots,circuitikz}
% \pgfplotsset{compat=1.15}
% \usetikzlibrary{intersections,arrows.meta,angles,calc,3d,decorations.pathmorphing}

% ---Add the section number to the equation, figure, and table number---
\makeatletter
   \renewcommand{\theequation}{\thesubsection.\arabic{equation}}
   \@addtoreset{equation}{subsection}
   
   \renewcommand{\thefigure}{\thesection.\arabic{figure}}
   \@addtoreset{figure}{section}
   
   \renewcommand{\thetable}{\thesection.\arabic{table}}
   \@addtoreset{table}{section}
\makeatother

% ---enumerate---
\renewcommand{\labelenumi}{\arabic{enumi}.}
\renewcommand{\labelenumii}{(\roman{enumii})}
\renewcommand{\labelenumiii}{(\alph{enumiii})}

% ---Index---
% \usepackage{makeidx}
% \makeindex

% ---Fonts---
\renewcommand{\familydefault}{\sfdefault}

% ---Title---
\title{東京大学\ 平成28年\ 物理学専攻\ 院試\ 解答例}
\author{ミヤネ}
\date{最終更新:\today}

\newcommand{\prb}[2]{
  \phantomsection
  \addcontentsline{toc}{subsection}{問題 #1: #2}
  \subsection*{第#1問\phantom{#2}}
  \setcounter{subsection}{#1}
  \setcounter{equation}{0}
}

\begin{document}

\maketitle

\tableofcontents
\clearpage

\section{数学パート}

\prb{1}{線形代数}

\begin{enumerate}
  \item 
  \begin{enumerate}
    \item 
    \begin{enumerate}
      \item 
      $\det A=\det B=1$で,$\det AB=(\det A)(\det B)$なので
      \begin{equation}
        \det M_n
        =
        \det A\cdots\det A \det B\cdots\det B
        =
        1
      \end{equation}
      です.

      \item 
      $A,B$はorthogonalなので,$M_n$も.$\tr$は転置をとっても変わらないので.

      \item 
      例えば
      \begin{equation}
        M_n
        =
        \begin{pmatrix}
          a & b \\
          c & d
        \end{pmatrix}
      \end{equation}
      とおいてみましょう.$\det M_n=1$より逆行列が決まり,それらを足せばOK.

    \end{enumerate}

    \item 
    $M_n^T=M_{n-2}^TM_{n-1}^{T}$に右から$M_{n-1}$をかければ
    \begin{equation}
      M_{n-2}^{T}=(M_n)^{-1}M_{n-1}
    \end{equation}
    です.また,$M_{n+1}=M_nM_{n-1}$より
    \begin{equation}
      M_{n+1}
      +
      (M_{n-2})^{-1}
      =
      M_nM_{n-1}
      +
      (M_n)^{-1}M_{n-1}
      =
      (\tr M_n)M_{n-1}
    \end{equation}
    となります.また,両辺の$\tr$をとれば
    \begin{equation}
      x_{n+1}
      =
      x_{n-2}
      =
      x_nx_{n-1}
    \end{equation}
    となり,$x_{n+1}=x_nx_{n-1}-x_{n-2}$が示されます.

    \item 
    $x_{n+2}=x_{n+1}x_n-x_{n-1}$を代入すると
    \begin{align}
      I_n
      &=
      (x_{n+1}x_n-x_{n-1})^2
      +
      x_{n+1}^2
      +
      x_n^2
      -
      (x_{n+1}x_n-x_{n-1})x_{n+1}x_n
      \nonumber
      \\
      &=
      x_{n+1}^2+x_n^2+x_{n-1}^2
      -
      x_{n+1}x_nx_{n-1}
      =
      I_{n-1}
    \end{align}
    となることがわかります.また,$\tr M_0=\sqrt{2}, \tr M_1=\sqrt{3}, \tr M_2=(\sqrt{6}-\sqrt{2})/2$なので,計算すれば
    \begin{equation}
      I_n
      =
      2+3\sqrt{2}-2\sqrt{3}-3\sqrt{6}
    \end{equation}
    となります.

  \end{enumerate}

  \item 
  \begin{enumerate}
    \item 
    $0,\pm\sqrt{2b_1b_2}$.

    \item 
    仮に$v_1=0$だとしましょう.すると,固有ベクトル$\bm{v}$の固有値を$\lambda$とすれば,$C\bm{v}=\lambda\bm{v}$より連立方程式
    \begin{equation}
      \left\{
        \begin{alignedat}{1}
          b_1 v_2 &= 0 \\
          b_2 v_3 &= \lambda v_2 \\
          b_2 v_2 + b_3 v_4 &= \lambda v_3 \\
          &\vdots \\
          b_i v_i + b_{i+2} v_{i+3} &= \lambda v_{i+2} \\
          &\vdots \\
          b_{N-3} v_{N-3} + b_{N-1} v_{N} &= \lambda v_{N-1} \\
          b_{N-1} v_{N-1} &= \lambda v_{N}      
        \end{alignedat}
      \right.
    \end{equation}
    が従います.第1,2式から$v_2,v_3=0$となりますが,漸化式から$v_4,\cdots,v_{N-1}=0$となることが決まってしまい,最終式より$v_N=0$となって自明な解となります.よって,$v_1\neq 0$です.

    \item 
    逆に$v_1\neq 0$なら,
    \begin{equation}
      \left\{
        \begin{alignedat}{1}
          b_1 v_2 &= \lambda v_1 \\
          b_1 v_1 + b_2 v_3 &= \lambda v_2 \\
          &\vdots \\
          b_i v_i + b_{i+2} v_{i+3} &= \lambda v_{i+2} \\
          &\vdots \\
          b_{N-3} v_{N-3} + b_{N-1} v_{N} &= \lambda v_{N-1} \\
          b_{N-1} v_{N-1} &= \lambda v_{N}      
        \end{alignedat}
      \right.
    \end{equation}
    という連立方程式を考えることになりますが,$v_1$が決まれば$\bm{v}$の残りの成分が決まってくることがわかります.このことから,ある固有値に対して対応する固有ベクトルが1対1に決まってくることがわかります\footnote{
      固有ベクトルを固有値から決定するときは,スカラー倍の不定性があったことを思い出すと,その不定性を殺す操作と$v_1$の値を決める操作が対応します.そして,今回は$v_1$を決めると固有ベクトルが1つに定まります.
    }.したがって,縮退がないので固有値は全て異なります.

  \end{enumerate}

\end{enumerate}

\clearpage
\prb{2}{微分方程式}
\begin{enumerate}
  \item 
  \begin{enumerate}
    \item 
    変数分離すれば
    \begin{equation}
      \frac{1}{f}\dv[2]{f}{t}
      =
      \frac{1}{g}\dv[2]{g}{x}
      -
      \lambda^2
      \eqqcolon
      \Lambda
    \end{equation}
    となります.$g(x)=\cos kx$より,$\Lambda=-(k^2+\lambda^2)$となっているので
    \begin{equation}
      \dv[2]{f}{t}
      =
      -
      (k^2+\lambda^2)
      f
    \end{equation}
    を解けばよいことになります.初期条件は$f(0)=1,\dot{f}(0)=0$なので$f(t)=\cos\sqrt{k^2+\lambda^2}t$が解です.よって,
    \begin{equation}
      y(x,t)
      =
      \cos\sqrt{k^2+\lambda^2}t\cos kx
    \end{equation}
    がもとめる解です.

    \item 
    前問の解は
    \begin{equation}
      y(x,t)
      =
      \frac{1}{2}
      \cos\left[ kx+\sqrt{k^2+\lambda^2}t \right]
      +
      \frac{1}{2}
      \cos\left[ kx-\sqrt{k^2+\lambda^2}t \right]
    \end{equation}
    となりますが,これで分解できてます.

  \end{enumerate}

  \item 
  \begin{enumerate}
    \item 
    $\dv*{u}{x}$をかけると
    \begin{equation}
      \dv{}{x}
      \left[  
        -\frac{1}{2}\left( \dv{u}{x} \right)^2
        +
        \lambda^2
        \left(  
          \frac{1}{2}u^4
          -
          u^2
        \right)
      \right]
      =
      0
    \end{equation}
    となるので,これを積分して$(\dv*{u}{x})^2$についてもとめると
    \begin{equation}
      \left( \dv{u}{x} \right)^2
      =
      \lambda^2
      (
        u^4-u^2-2A'/\lambda^2
      )
    \end{equation}
    となります.なお$A'$は積分定数.$-2A'/\lambda^2$をまとめて$A$とおけば
    \begin{equation}
      \dv{u}{x}
      =
      \pm\lambda
      \sqrt{
        u^4-2u^2+A
      }
    \end{equation}
    です.

    \item 
    $x\rightarrow\infty$で$u^2=1,u'=0$とすれば
    \begin{equation}
      A
      =
      1
    \end{equation}
    なので,積分すれば\footnote{
      境界条件$|u(x)|\leq 1$はこのとき効いてきます.
    }
    \begin{equation}
      \log
      \frac{1+u}{1-u}
      =
      \pm
      2\lambda x
      +
      C
    \end{equation}
    となります.境界条件$u(0)=0$より$C=0$となるので
    \begin{equation}
      u(x)
      =
      \dfrac{e^{\pm 2\lambda x}-1}{e^{\pm 2\lambda x}+1}
    \end{equation}
    が解です.

    \item 
    $(u+z)^3\sim u^3+3u^2z$で十分なら
    \begin{equation}
      \left[  
        -\dv[2]{u}{x}
        +
        2\lambda^2(u^3-u)
      \right]
      +
      \left[  
        \pdv[2]{z}{t}
        -
        \pdv[2]{z}{x}
        +
        2\lambda^2
        (
          3u^2z
          -
          z
        )
      \right]
      =
      0
    \end{equation}
    なので,第1項は消えて
    \begin{equation}
      \pdv[2]{z}{t}
        -
        \pdv[2]{z}{x}
        +
        2\lambda^2
        (
          3u^2z
          -
          z
        )
      =
      0
    \end{equation}
    が$z$が満たすべき方程式です.

    \item 
    素直に代入してみると
    \begin{equation}
      -
      \omega^2
      \dv{u}{x}
      -
      \dv[3]{u}{x}
      +
      2\lambda^2
      \left(  
        3u^2\dv{u}{x}
        -
        \dv{u}{x}
      \right)
      =
      0
      \label{pdv01}
    \end{equation}
    となります.問題文の式(3)を微分してみると
    \begin{equation}
      \dv[3]{u}{x}
      =
      2\lambda^2
      \left(  
        3u^2\dv{u}{x}
        -
        \dv{u}{x}
      \right)
    \end{equation}
    となっているので,\eqref{pdv01}に代入すると,なんと第3項がちょうど消えて
    \begin{equation}
      -\omega
      \dv{u}{x}
      =
      0
    \end{equation}
    となります.よって,$\omega=0$とすれば,$z_0$は解になっています\footnote{
      $t$についての依存性も完全に消えてしまいますが.
    }.

  \end{enumerate}

\end{enumerate}


\clearpage
\section{物理パート}
\prb{1}{量子力学}
\begin{enumerate}
  \item 
  次の交換関係と反交換関係は既知とします:
  \begin{equation}
    [\sigma_i,\sigma_j]
    =
    2i\varepsilon_{ijk}\sigma
    ,\ 
    \{\sigma_i,\sigma_j\}
    =
    2\delta_{ij}
    .
  \end{equation}
  計算すれば
  \begin{equation}
    [S_x,S_y]
    =
    i\hbar S_z
    ,\ 
    [H,S_z]=0
  \end{equation}
  です.

  \item 
  $D^{\dagger}D=2mH$なので.

  \item 
  $D$と$\sigma_z$の反交換関係は
  \begin{equation}
    \{
      D,\sigma_z
    \}
    =
    p_x
    \{
      \sigma_x,\sigma_z
    \}
    +
    p_y
    \{
      \sigma_y,\sigma_z
    \}
    =
    0
  \end{equation}
  なので,第1式はOK.第2式は
  \begin{equation}
    D^2
    =
    p_x^2+p_y^2
    +
    p_xp_y\sigma_x\sigma_y
    +
    p_yp_x\sigma_y\sigma_x
  \end{equation}
  ですが,後ろの項は
  \begin{equation}
    \left\{
      \begin{alignedat}{1}
        p_xp_y
        &=
        -
        \hbar^2\pdv[2]{}{x}{y}
        +
        e^2A_xA_y
        -
        i\hbar e\pdv{A_y}{x}
        -
        i\hbar eA_y\pdv{}{x}
        -
        i\hbar eA_x\pdv{}{y}
        \\
        p_yp_x
        &=
        -
        \hbar^2\pdv[2]{}{x}{y}
        +
        e^2A_xA_y
        -
        i\hbar e\pdv{A_x}{y}
        -
        i\hbar eA_x\pdv{}{y}
        -
        i\hbar eA_y\pdv{}{x}
      \end{alignedat}
    \right.
  \end{equation}
  より\footnote{
    $p_xp_y$を計算するときには少し注意が必要で
    \begin{align*}
      p_xp_y
      &=
      \left(  
        -
        i\hbar\pdv{}{x}
        +
        eA_x
      \right)
      \left( 
        -
        i\hbar\pdv{}{y}
        +
        eA_y
      \right)
      \\
      &=
      -
      \hbar^2\pdv[2]{}{x}{y}
      +
      e^2A_xA_y
      -
      \uwave{
        i\hbar\pdv{}{x}
        A_y
      }
      -
      i\hbar eA_x\pdv{}{y}      
      \\
      &=
      -
      \hbar^2\pdv[2]{}{x}{y}
      +
      e^2A_xA_y
      -
      \uwave{
        i\hbar e\pdv{A_y}{x}
        -
        i\hbar eA_y\pdv{}{x}
      }
      -
      i\hbar eA_x\pdv{}{y}
    \end{align*}
    のようになります.
  }
  \begin{equation}
    [p_x,p_y]
    =
    -i\hbar e\left( \pdv{A_y}{x}-\pdv{A_x}{y} \right)
    =
    -i\hbar e B_z(x,y)
  \end{equation}
  となるので,
  \begin{equation}
    p_xp_y\sigma_x\sigma_y
    +
    p_yp_x\sigma_y\sigma_x
    =
    [p_x,p_y]
    \sigma_x\sigma_y
    =
    \hbar eB_z \sigma
  \end{equation}
  です\footnote{
    $\sigma_i\sigma_j=i\varepsilon_{ijk}\sigma_k+\delta_{ij}$より,$\sigma_x\sigma_y=i\sigma_z$です.
  }.したがって,
  \begin{equation}
    \frac{1}{2m}D^2
    =
    \frac{p_x^2+p_y^2}{2m}
    +
    \frac{\hbar e B_z(x,y)}{2m}\sigma_z
    =
    H
  \end{equation}
  となります.

  \item 
  $\ev*{\Phi_n|\Phi_n},\ev*{\Psi_n|\Psi_n}>0$なので
  \begin{equation}
    E_n
    =
    \frac{\ev*{\Phi_n|\Phi_n}}{2m\ev*{\Psi_n|\Psi_n}}
    >
    0
  \end{equation}
  です.固有値については,$H$と$D$は可換なので,$H\ket{\Phi_n}=DH\ket{\Psi_n}=E_n\ket{\Phi_n}$です.また,
  \begin{equation}
    \sigma_z \ket{\Phi_n}
    =
    -D\sigma_z\ket{\Psi_n}
    =
    (-1)\ket{\Phi_n}
  \end{equation}
  なので,固有値は$-1$です.

  \item 
  $E=0$なら$\ev*{\Phi|\Phi}=0$です.よって,
  \begin{equation}
    D\ket{\Psi}
    =
    \ket{\Phi}
    =
    0
  \end{equation}
  です.また,
  \begin{equation}
    D
    =
    p_x\sigma_x+p_y\sigma_y
    =
    -
    \begin{pmatrix}
     0 & \hbar\left( i\pdv{}{x}+\pdv{}{y} \right)+e\left( \pdv{\rho}{y}+i\pdv{\rho}{x} \right) \\
     \hbar\left( i\pdv{}{x}-\pdv{}{y} \right)+e\left( \pdv{\rho}{y}-i\pdv{\rho}{x} \right) & 0 \\
    \end{pmatrix}
  \end{equation}
  なので,$D\ket{\Psi}=0$を計算してみると
  \begin{equation}
    \pdv{f_{\uparrow}}{x}
    +
    i\pdv{f_{\uparrow}}{y}
    =
    0
    ,\ 
    \pdv{f_{\downarrow}}{x}
    -
    i\pdv{f_{\downarrow}}{y}
    =
    0
  \end{equation}
  となっています.これは$w=x+iy,\bar{w}=x-iy$に対して,$f_{\uparrow}=f_{\uparrow}(w),f_{\downarrow}=f_{\downarrow}(\bar{w})$であれば,成立する式になっています.例えば
  \begin{equation}
    \pdv{f_{\uparrow}}{x}
    +
    i\pdv{f_{\uparrow}}{y}
    =
    f'+i\cdot(if')
    =
    0
  \end{equation}
  でしょう.

  \item 
  前問より,$E-0$なら
  \begin{equation}
    \psi_{\uparrow}
    =
    f_{\uparrow}(x+iy)
    \exp\left[ \frac{e}{\hbar}\rho(x,y) \right]
    ,\ 
    \psi_{\downarrow}
    =
    f_{\downarrow}(x-iy)
    \exp\left[ -\frac{e}{\hbar}\rho(x,y) \right]
  \end{equation}
  です.束縛状態では$r\rightarrow\infty$のときに$\psi\rightarrow 0$でないといけませんが,スピン上向きの波動関数$\psi_{\uparrow}$は0になりません.よって,束縛状態ならスピン上向きの成分は0です.また,スピン下向きの波動関数は
  \begin{align}
    \psi_{\downarrow}
    &=
    f_{\downarrow}(\bar{w})
    \exp\left[ -\frac{ebR^2}{4\hbar}-\frac{ebR^2}{2\hbar}\log\frac{r}{R} \right]
    \nonumber
    \\
    &=
    f_{\downarrow}(\bar{w})
    \exp\left[ -\frac{ebR^2}{4\hbar} \right]
    \left( \frac{|\bar{w}|}{R} \right)^{-ebR^2/2\hbar}
  \end{align}
  となります.ここで$n\coloneqq ebR^2/2\hbar$とおくと,
  \begin{equation}
    f_{\downarrow}(\bar{w})
    =
    a_0+a_1\bar{w}+a_2\bar{w}^2+\cdots a_{n-1}\bar{w}^{n-1}+\cdots
  \end{equation}
  と展開できますが,$f_{\downarrow}$の最高次が$\bar{w}^n$よりも大きいと$f_{\downarrow}|\bar{w}|^{-n}$が$r\rightarrow$で0に収束してくれません.したがって,$f_{\downarrow}$の最高次は$n-1$で,一次独立なものも$n-1$つあります.つまり,一次独立なものは
  \begin{equation}
    \frac{ebR^2}{2\hbar}
    -
    1
  \end{equation}
  つです.

\end{enumerate}


\clearpage
\prb{2}{力学,統計力学}

\begin{enumerate}
  \item 
  古典論の分配関数は
  \begin{equation}
    Z[\beta]
    =
    \frac{1}{h}
    \int_{-\infty}^{\infty}\dd p_1
    \int_{-\infty}^{\infty}\dd x_1\ 
    \exp\left[ -\beta 
    \left(  
      \frac{1}{2m}p_1^2
      +
      \frac{m\omega^2}{2}(x_1-a)^2
    \right)\right]
    =
    \frac{1}{h}
    \cdot
    \frac{2\pi}{\beta\omega}
  \end{equation}
  です.よって,
  \begin{equation}
    U
    =
    -\pdv{}{\beta}\log Z[\beta]
    =
    k_B T
    ,\ 
    C
    =
    \pdv{U}{T}
    =
    k_B
  \end{equation}
  となります.また,$x_1$の平均ですが
  \begin{equation}
    \ev*{x_1}
    =
    \dfrac{\displaystyle
      \int_{-\infty}^{\infty}\dd x_1\ 
      x_1
      \exp
      \left[  
        -\frac{\beta m\omega^2}{2}(x_1-a)^2
      \right]
    }{\displaystyle
      \int_{-\infty}^{\infty}\dd x_1\ 
      \exp
      \left[  
        -\frac{\beta m\omega^2}{2}(x_1-a)^2
      \right]
    }
    =
    a
  \end{equation}
  となります\footnote{
    平行移動して積分すればよいです.
  }.同様にして,分散を計算すれば
  \begin{align}
    \sigma_1^2
    &=
    \ev*{x_1^2}-\ev*{x_1}^2
    \nonumber
    \\
    &=
    \frac{\displaystyle
      \int_{-\infty}^{\infty}\dd x_1\ 
      x_1^2
      \exp
      \left[  
        -\frac{\beta m\omega^2}{2}(x_1-a)^2
      \right]
    }{\displaystyle
      \int_{-\infty}^{\infty}\dd x_1\ 
      \exp
      \left[  
        -\frac{\beta m\omega^2}{2}(x_1-a)^2
      \right]
    }
    -
    a^2
    =
    \frac{k_BT}{m\omega^2}
  \end{align}
  となります.

  \item 
  分配関数は
  \begin{equation}
    Z_{Q}[\beta]
    =
    \sum_{n=0}^{\infty}e^{-\beta\hbar\omega(n+1/2)}
    =
    \frac{1}{2\sinh(\beta\hbar\omega/2)}
  \end{equation}
  なので,
  \begin{equation}
    U_Q
    =
    -\pdv{}{\beta}\log Z
    =
    \frac{\hbar\omega}{2}
    \coth(\beta\hbar\omega/2)
    ,\ 
    C_Q
    =
    \pdv{U_Q}{T}
    =
    \frac{\hbar^2\omega^2}{4k_BT^2}
    \cdot
    \frac{1}{\sinh^2(\hbar\omega/2k_BT)}
  \end{equation}
  となります.また,$X_1\coloneqq x_1-a$に対して
  \begin{equation}
    \ev*{n|X_1|n}
    =
    0
  \end{equation}
  なので,
  \begin{equation}
    \ev*{X_1}
    =
    \frac{1}{Z_{Q}[\beta]}
    \sum_{n=0}^{\infty}
    \ev*{n|X_1|n}
    e^{-\beta\hbar\omega(n+1/2)}
    =
    0
  \end{equation}
  です.よって,$\ev*{x_1}=a$です.同様に考えれば
  \begin{align}
    \ev*{n|X_1^2|n}
    &=
    \frac{\hbar}{2m\omega}
    (\sqrt{n}\bra{n-1}+\sqrt{n+1}\bra{n+1})
    (\sqrt{n}\ket{n-1}+\sqrt{n+1}\ket{n+1})
    \nonumber
    \\
    &=
    \frac{1}{m\omega^2}
    \cdot
    \hbar\omega^2
    \left(  
      n+\frac{1}{2}
    \right)
  \end{align}
  なので
  \begin{align}
    \ev*{X_1^2}
    &=
    \frac{1}{Z_Q[\beta]}
    \sum_{n=0}^{\infty}
    \frac{1}{m\omega^2}
    \cdot
    \hbar\omega^2
    \left(  
      n+\frac{1}{2}
    \right)
    e^{-\beta\hbar\omega(n+1/2)}
    \nonumber
    \\
    &=
    \frac{\hbar}{2m\omega}\cosh\left( \frac{\hbar\omega}{2k_BT} \right)
  \end{align}
  より,$\ev*{X_1^2}=\ev*{x_1^2}-a^2=\ev*{x_1^2}-\ev*{x_1}^2$なので
  \begin{equation}
    \sigma^2
    =
    \frac{\hbar}{2m\omega}\cosh\left( \frac{\hbar\omega}{2k_BT} \right)
  \end{equation}
  です.内部エネルギーは,高温$k_BT\gg\hbar\omega$では
  \begin{equation}
    U_Q
    \sim
    \frac{\hbar^2\omega^2}{4k_BT^2}
    \cdot
    \frac{4k_B^2T^2}{\hbar^2\omega^2}
    =
    k_BT
    =
    C
  \end{equation}
  で古典論に一致し,低温$k_BT\ll\hbar\omega$では
  \begin{equation}
    U_Q
    \sim
    \frac{\hbar^2\omega^2}{4k_BT^2}
    \cdot
    0
    =
    0
  \end{equation}
  となります.

  \item   
  この系の運動方程式は
  \begin{equation}
    \dv[2]{}{t}
    \begin{pmatrix}
      x_1 \\
      x_2 \\
      \vdots \\
      x_{N-1} \\
      x_N
    \end{pmatrix}
    =
    -\omega^2
    \begin{pmatrix}
      2 & -1 &      &   &   \\
     -1 &  2 & -1   &   &   \\
        &    &\ddots&   &   \\
        &    &   -1 & 2 & -1\\
        &    &      & -1& 2  
   \end{pmatrix}
    \begin{pmatrix}
      x_1 \\
      x_2 \\
      \vdots \\
      x_{N-1} \\
      x_N
    \end{pmatrix}
  \end{equation}
  なので,その係数行列
  \begin{equation}
    A
    \coloneqq
    \begin{pmatrix}
      2 & -1 &      &   &   \\
     -1 &  2 & -1   &   &   \\
        &    &\ddots&   &   \\
        &    &   -1 & 2 & -1\\
        &    &      & -1& 2  
   \end{pmatrix}
  \end{equation}
  の固有値を考えてみることにしましょう.まずは,対角成分を除いた行列
  \begin{equation}
    B
    \coloneqq
    \begin{pmatrix}
      0 & -1 &      &   &   \\
     -1 &  0 & -1   &   &   \\
        &    &\ddots&   &   \\
        &    &   -1 & 0 & -1\\
        &    &      & -1& 0  
   \end{pmatrix}
  \end{equation}
  の固有値を考えます.すると,この行列の固有値と固有ベクトルは
  \begin{equation}
    \lambda_i
    =
    -
    2
    \cos\left( \frac{i\pi}{N+1} \right)
    ,\ 
    (u_i)_j
    =
    \sin\left( \frac{ij\pi}{N+1} \right)
  \end{equation}
  であることがわかります\footnote{
    固有ベクトルのほうの添え字がやかましいかもしれませんが,$i$が固有値のindexで,$j$はcomponentsのindexです.
  }.実際に,例えば$B\bm{u}_i$の第2成分を見てみると
  \begin{equation}
    -\sin\left( \frac{i\pi}{N+1} \right)
    -\sin\left( \frac{3i\pi}{N+1} \right)
    =
    -2\cos\left( \frac{i\pi}{N+1} \right)\sin\left( \frac{2i\pi}{N+1} \right)
  \end{equation}
  となっています\footnote{
    もちろん,もっと一般にできます.
  }.したがって,$\bm{u}_i$は$A$の固有ベクトルでもあります.なぜなら,
  \begin{equation}
    A\bm{u}_i
    =
    (2+B)\bm{u}_i
    =
    \left( 2-\lambda_i \right)\bm{u}_i
  \end{equation}
  だからです.したがって,$\omega^2A$の固有値は
  \begin{equation}
    \omega^2
    \left(  
      2-2\cos\left( \frac{i\pi}{N+1} \right)
    \right)
  \end{equation}
  であり,これが$\omega_l^2$です.

  \item 
  変数を
  \begin{equation}
    \left\{
      \begin{alignedat}{1}
        X_1&= x_1 - a \\
        &\vdots \\
        X_i &= x_i - x_{i-1} -a \\
        &\vdots \\
        X_N &= a(N+1) - X_N
      \end{alignedat}
    \right.
  \end{equation}
  と変換すれば,ハミルトニアンは
  \begin{equation}
    H
    =
    \sum_{i=1}^{N}
    \left[  
      \frac{1}{2m}p_i^2
      +
      \frac{m\omega^2}{2}X_i^2
    \right]
  \end{equation}
  と書き換えることができます.このとき,系が分離できているので,設問1.と同様の議論をすればよいでしょう.$\omega$の依存性がなかったので,
  \begin{equation}
    C_1
    =
    k_B
    .
  \end{equation}

  \item 
  粒子$l$の分配関数は
  \begin{equation}
    Z^{(l)}_Q[\beta]
    =
    \sum_{n=0}^{\infty}e^{-\beta\hbar\omega_l(n+1/2)}
  \end{equation}
  と書けるので,設問2.と同様にして比熱は
  \begin{equation}
    C_{1Q}^{l}
    =
    \frac{\hbar^2\omega_l^2}{4k_BT^2}
    \cdot
    \frac{1}{\sinh^2(\hbar\omega_l/2k_BT)}
  \end{equation}
  ともとまります.よって,平均をとれば
  \begin{equation}
    C_{1Q}
    =
    \frac{1}{N}
    \sum_{l=1}^{N}
    \frac{\hbar^2\omega_l^2}{4k_BT^2}
    \cdot
    \frac{1}{\sinh^2(\hbar\omega_l/2k_BT)}
  \end{equation}
  ですが,$x_l\coloneqq \hbar\omega/k_BT$とすれば
  \begin{equation}
    C_{1Q}
    =
    k_B
    \cdot
    \frac{1}{N}
    \sum_{l=1}^{\infty}
    \frac{x_l^2e^x}{(e^x-1)^2}
  \end{equation}
  となります.今は高温極限$|x_{l+1}-x_l|\ll 1$と$N\gg 1$の状況を考えているので,
  \begin{equation}
    C_{1Q}
    =
    \frac{k_B}{\Delta x}
    \cdot
    \frac{\Delta x}{N}
    \sum_{l=1}^{\infty}
    \frac{x_l^2e^{x_l}}{(e^{x_l}-1)^2}
    \sim
    \frac{2k_B}{\Delta x}
    \int_0^\infty
    \dd x
    \frac{x^2e^x}{(e^x-1)^2}
  \end{equation}
  です.ここで,
  \begin{equation}
    \Delta x
    =
    \frac{\hbar}{k_BT}\Delta\omega
    \sim
    \frac{\hbar\omega}{k_BT}
  \end{equation}
  なので
  \begin{equation}
    C_{1Q}
    =
    \frac{\pi^2k_B^2}{3\hbar\omega}T
  \end{equation}
  となり,$b=1,A=\pi^2k_B^2/3\hbar\omega$です.


\end{enumerate}

\subsection*{補足}
\begin{itemize}
  \item 
  あまり運動方程式とかにまじめに言及してなかったので.まず,この系のラグランジアンは
  \begin{equation}
    L
    =
    \frac{1}{2}m\dot{x}_1^2
    -
    \frac{m\omega^2}{2}
    (x_1-a)^2
    +
    \frac{1}{2}m\dot{x}_2^2
    -
    \frac{m\omega^2}{2}
    (x_2-x_1-a)^2
    +
    \cdots
  \end{equation}
  で与えられます.したがって,$x_i$に共役運動量は
  \begin{equation}
    p_i
    =
    \pdv{L}{\dot{x}_i}
    =
    m\dot{x}_i
  \end{equation}
  です.このことから,運動方程式はEuler-Lagramgeから
  \begin{equation}
    m\ddot{x}_i
    +
    m\omega^2
    (2x_i-x_{i+1}-x_{i-1})
    =
    0
  \end{equation}
  であることがわかります.また,ハミルトニアンは
  \begin{equation}
    H
    =
    \sum_{i}p_i\dot{q}_i
    -
    L
  \end{equation}
  で与えられますが,これを計算すればちゃんと今回のハミルトニアンになっています.

\end{itemize}

\end{document}
