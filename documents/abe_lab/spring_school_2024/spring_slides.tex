\documentclass[
  unicode,a4paper,11pt,aspectratio=169,
  xcolor = {dvipsnames,svgnames},
  hyperref ={colorlinks=true,citecolor=Navy,linkcolor=NavyBlue,urlcolor=purple},
  ja=standard,lualatex
]{beamer}
\renewcommand{\baselinestretch}{1.4}

% ---fonts---
\usefonttheme{serif}
\mathversion{bold}
\usepackage{luatexja-fontspec}
\setmainfont{TeX Gyre Termes}
\setmainjfont{Noto Sans JP}
\setmathrm{Latin Modern Roman}
% \setmainjfont{IPAexGothic}

% ---refer `texdoc xcolor' at the command line---

% ---Display \subsubsection at the Index
% \setcounter{tocdepth}{3}

% ---Setting about the geometry of the document----
% \usepackage{a4wide}
% \pagestyle{empty}

% ---Physics and Math Packages---
\usepackage{amssymb,amsfonts,amsthm,mathtools}
\usepackage{physics,braket,bm}

% ---underline---
\usepackage[normalem]{ulem}

% ---cancel---
\usepackage{cancel}

% --- surround the texts or equations
\usepackage{fancybox,ascmac}

% ---settings of theorem environment---
% \usepackage{amsthm}
% \theoremstyle{definition}

% ---settings of proof environment---
% \renewcommand{\proofname}{\textbf{証明}}
% \renewcommand{\qedsymbol}{$\blacksquare$}

% ---Insert the figure (If insert the `draft' at the option, the process becomes faster.)---
\usepackage{graphicx}
% \usepackage{subcaption}

% ----Add a link to a text---
\usepackage{url,hyperref}
\usepackage{xcolor}

% ---Tikz---
\usepackage{tikz,pgf,pgfplots,circuitikz}
\pgfplotsset{compat=1.15}
\usetikzlibrary{intersections,arrows.meta,angles,calc,3d,decorations.pathmorphing,positioning}

% ---Add the section number to the equation, figure, and table number---
\makeatletter
   \renewcommand{\theequation}{\thesection.\arabic{equation}}
   \@addtoreset{equation}{section}
   
   \renewcommand{\thefigure}{\thesection.\arabic{figure}}
   \@addtoreset{figure}{section}
   
   \renewcommand{\thetable}{\thesection.\arabic{table}}
   \@addtoreset{table}{section}
\makeatother

% ---enumerate---
% \renewcommand{\labelenumi}{$\arabic{enumi}.$}
% \renewcommand{\labelenumii}{$(\arabic{enumii})$}

% ---beamer settings---
\usefonttheme{professionalfonts}
\usecolortheme{seahorse}
\setbeamercolor{structure}{fg=white}
\setbeamercolor{local structure}{fg=red}
\setbeamertemplate{itemize item}[ball]
\setbeamertemplate{enumerate item}[circle]
\setbeamercolor{bibliography entry author}{fg=black}
\setbeamercolor{bibliography item}{fg=black}
\setbeamercolor{alerted text}{fg=RoyalBlue}
\setbeamertemplate{frametitle continuation}{}
\setbeamertemplate{footline}[frame number]
\setbeamertemplate{navigation symbols}{} 
\setbeamersize{text margin left=15pt, text margin right=15pt}

% ---tcolorbox---
\usepackage{tcolorbox}
\tcbuselibrary{raster,skins,theorems}
\newtcolorbox{bluebox}[2][]{enhanced,
colframe=RoyalBlue!40!white,
colback=RoyalBlue!10!white,
coltitle=black,
drop fuzzy shadow, title={#2}
,#1}
\newtcolorbox{redbox}[2][]{enhanced,
colframe=DarkRed!40!white,
colback=DarkRed!10!white,
coltitle=black,
drop fuzzy shadow, title={#2}
,#1}

% ---Ignore the Warnings---
\usepackage{silence}
\WarningFilter{latexfont}{Some font shapes}
\WarningFilter{latexfont}{Font shape}
\ExplSyntaxOn
\msg_redirect_name:nnn{hooks}{generic-deprecated}{none}
\ExplSyntaxOff

% \usepackage{newtxmath}

% ---citation---
% \usepackage{usebib}
% \newbibfield{author} 
% \newbibfield{year} 
% \newbibfield{journal} 
% \newbibfield{doi} 
% \bibinput{ref}

% \makeatletter
% \newcommand*{\journal}{\begingroup\@makeother\#\@mylink}
% \newcommand*{\@mylink}[1]{\href{http://dx.doi.org/\usebibentry{#1}{doi}}{\usebibentry{#1}{journal}}\endgroup} 
% \makeatother

% \newcommand*{\citefone}[2]{
%   \begin{tikzpicture}[remember picture, overlay]
%     \node[anchor=north east, align=left] at ($(current page.north east)-(0,0.0)$){
%     {\tiny
%       \cite{#1}
%       #2,
%       \journal{#1}
%       (\usebibentry{#1}{year}).
%     }
%     };
%   \end{tikzpicture}
% }

% \newcommand*{\citeftwo}[4]{
%   \begin{tikzpicture}[remember picture, overlay]
%     \node[anchor=north east, align=left] at ($(current page.north east)-(0,0.0)$){
%     {\tiny
%       \cite{#1}
%       #2,
%       \journal{#1}
%       (\usebibentry{#1}{year}).
%     }
%     \\[-2.4ex]
%     {\tiny
%       \cite{#3}
%       #4,
%       \journal{#3}
%       (\usebibentry{#3}{year}).
%     }
%     };
%   \end{tikzpicture}
% }

% \newcommand*{\citefthree}[6]{
%   \begin{tikzpicture}[remember picture, overlay]
%     \node[anchor=north east, align=left] at ($(current page.north east)-(0,0.0)$){
%     {\tiny
%       \cite{#1}
%       #2,
%       \journal{#1}
%       (\usebibentry{#1}{year}).
%     }
%     \\[-2.4ex]
%     {\tiny
%       \cite{#3}
%       #4,
%       \journal{#3}
%       (\usebibentry{#3}{year}).
%     }
%     \\[-2.4ex]
%     {\tiny
%       \cite{#5}
%       #6,
%       \journal{#5}
%       (\usebibentry{#5}{year}).
%     }
%     };
%   \end{tikzpicture}
% }

% \newcommand*{\citefonev}[3]{
%   \begin{tikzpicture}[remember picture, overlay]
%     \node[anchor=north east, align=left, text width=#3cm] at ($(current page.north east)-(0,0.0)$){
%     {{\fontsize{5pt}{0pt}\selectfont
%       \cite{#1}
%       #2,
%       \journal{#1}
%       (\usebibentry{#1}{year}).\par}
%     }
%     };
%   \end{tikzpicture}
% }

% \newcommand*{\citeftwov}[5]{
%   \begin{tikzpicture}[remember picture, overlay]
%     \node[anchor=north east, align=left, text width=#5cm] at ($(current page.north east)-(0,0.0)$){
%     {{\fontsize{5pt}{0pt}\selectfont
%       \cite{#1}
%       #2,
%       \journal{#1}
%       (\usebibentry{#1}{year}).\par}

%       {\fontsize{5pt}{0pt}\selectfont
%       \cite{#3}
%       #4,
%       \journal{#3}
%       (\usebibentry{#3}{year}).\par}
%     }
%     };
%   \end{tikzpicture}
% }

% \newcommand*{\citefthreev}[7]{
%   \begin{tikzpicture}[remember picture, overlay]
%     \node[anchor=north east, align=left, text width=#7cm] at ($(current page.north east)-(0,0.0)$){
%     {{\fontsize{5pt}{0pt}\selectfont
%     \cite{#1}
%     #2,
%     \journal{#1}
%     (\usebibentry{#1}{year}).\par}

%     {\fontsize{5pt}{0pt}\selectfont
%     \cite{#3}
%     #4,
%     \journal{#3}
%     (\usebibentry{#3}{year}).\par}

%     {\fontsize{5pt}{0pt}\selectfont
%     \cite{#5}
%     #6,
%     \journal{#5}
%     (\usebibentry{#5}{year}).\par}
%     }
%     };
%   \end{tikzpicture}
% }


% ---Title---
\title{
  Spring School 2024
}
\author{
  Abe Lab. \ M1
  \texorpdfstring{\\}{}
  \texorpdfstring{\vspace*{3pt}}{}
  Itsuki Miyane
}
\date{Sunday, April 7th, 2024}


\begin{document}

\begin{frame}

  \setbeamertemplate{blocks}[rounded][shadow=true]
  \setbeamercolor{block body}{bg=blue!10!white, fg=black}

  \begin{block}{}
    \centering
    Spring School 2024 @Izukawana
    \\
    \Large
    Moduli stabilization

    on (for/at ?) supersymmetric magnetized D9-brane model
  \end{block}

  \begin{center}
    Abe Lab. \ M1 \\
    Itsuki Miyane

    \vspace*{5pt}

    Saturday, April 6th, 2024
  \end{center}

  \begin{center}
    *I will speak in Japanese though this slide is written in English.
  \end{center}
\end{frame}

\begin{frame}{Topics}
  \begin{itemize}
    \item
          Review the senior thesis
    \item
          Report progress and difficulties I met
  \end{itemize}
\end{frame}


\section{Introduction}

\begin{frame}
  \huge \secname
\end{frame}

\begin{frame}{Motivation}

  To solve the problems in the Standard Model, \textcolor{Green}{higher dimensional models} were proposed.

  \vspace{10pt}

  \pause

  In general, these theories contain extra fields related to
  \begin{itemize}
    \item
          the size and shape of the compactified extra dimension,
    \item
          hence the metric and the gravity
  \end{itemize}
  in its 4d effective field theory.

  \vspace{10pt}

  \pause

  \setbeamertemplate{blocks}[rounded][shadow=true]
  \setbeamercolor{block body}{bg=LimeGreen!10!white, fg=black}

  \begin{block}{}
    \centering
    \large
    Such a field is called \textcolor{DarkMagenta}{\textbf{moduli fields}}.
  \end{block}

\end{frame}


\begin{frame}{Moduli stabilization}


  $\cdots$

  These procedures are called \textcolor{Red}{moduli stabilization}.

\end{frame}


\begin{frame}{Purpose of my study}

  We will discuss the \textcolor{Red}{moduli stabilization} on \textcolor{DarkOrange}{magnetized torus model}.




\end{frame}


\section{Magnetized torus model}

\begin{frame}
  \huge \secname
\end{frame}

\begin{frame}{Torus compactification and Magnetic flux}




\end{frame}


\section{Determination of the overall factor}

\begin{frame}
  \huge \secname
\end{frame}

\begin{frame}{$F$-term potential}




\end{frame}





\section{Summary}


\begin{frame}{Summary}




\end{frame}


% --------------------------

% \newcounter{Appendix}
% \setcounter{Appendix}{\value{framenumber}}
% \setcounter{section}{0}
% \renewcommand{\thesubsection}{\Alph{subsection}}
% \makeatletter
%    \renewcommand{\theequation}{\thesubsection.\arabic{equation}}
%    \@addtoreset{equation}{section}

%    \renewcommand{\thefigure}{\thesubsection.\arabic{figure}}
%    \@addtoreset{figure}{section}

%    \renewcommand{\thetable}{\thesubsection.\arabic{table}}
%    \@addtoreset{table}{section}
% \makeatother

% \section{付録}

% \begin{frame}[plain]
%   \frametitle{\ }
%   \huge \secname
% \end{frame}

% \begin{frame}[plain]
%   \frametitle{\thesubsection\ \subsecname}








% \end{frame}

% --------------------------

\section{Reference}
\begin{frame}[plain,allowframebreaks]{Reference}
  
  \begin{thebibliography}{99}
    \beamertemplatetextbibitems

    \bibitem{cite}
      hogehoge

  \end{thebibliography}

\end{frame}

% \setcounter{framenumber}{\value{Appendix}}
\end{document}
