\documentclass[unicode,a4paper,11pt]{ltjsarticle}

\usepackage{luatexja-fontspec}
\setmainfont{TeX Gyre Termes}
% \setmainjfont[BoldFont = IPAGothic]{IPAMincho}
\setmathrm{Latin Modern Roman}
\setmainjfont{Noto Sans JP}

% ---Display \subsubsection at the Index
% \setcounter{tocdepth}{3}

% ---Setting about the geometry of the document----
% \usepackage{a4wide}
% \pagestyle{empty}

% ---Physics and Math Packages---
\usepackage{amssymb,amsfonts,amsthm,mathtools}
\usepackage{physics,braket,bm}

% ---underline---
\usepackage[normalem]{ulem}

% ---cancel---
\usepackage{cancel}

% --- surround the texts or equations
% \usepackage{fancybox,ascmac}

% ---settings of theorem environment---
% \usepackage{amsthm}
% \theoremstyle{definition}

% ---settings of proof environment---
% \renewcommand{\proofname}{\textbf{証明}}
% \renewcommand{\qedsymbol}{$\blacksquare$}

% ---Ignore the Warnings---
\usepackage{silence}
\WarningFilter{latexfont}{Some font shapes}
\WarningFilter{latexfont}{Font shape}
\WarningFilter{latexfont}{Size substitutions}
\ExplSyntaxOn
\msg_redirect_name:nnn{hooks}{generic-deprecated}{none}
\ExplSyntaxOff

% ---Insert the figure (If insert the `draft' at the option, the process becomes faster.)---
\usepackage{graphicx}
% \usepackage{subcaption}

% ----Add a link to a text---
\usepackage{url,hyperref}
\usepackage[dvipsnames,svgnames]{xcolor}
\hypersetup{colorlinks=true,citecolor=FireBrick,linkcolor=Navy,urlcolor=purple}
% ---refer `texdoc xcolor' at the command line---

% ---Tikz---
% \usepackage{tikz,pgf,pgfplots,circuitikz}
% \pgfplotsset{compat=1.15}
% \usetikzlibrary{intersections,arrows.meta,angles,calc,3d,decorations.pathmorphing}

% ---Add the section number to the equation, figure, and table number---
\makeatletter
   \renewcommand{\theequation}{\thesection.\arabic{equation}}
   \@addtoreset{equation}{section}
   
   \renewcommand{\thefigure}{\thesection.\arabic{figure}}
   \@addtoreset{figure}{section}
   
   \renewcommand{\thetable}{\thesection.\arabic{table}}
   \@addtoreset{table}{section}
\makeatother

% ---enumerate---
% \renewcommand{\labelenumi}{$\arabic{enumi}.$}
% \renewcommand{\labelenumii}{$(\arabic{enumii})$}

% ---Index---
% \usepackage{makeidx}
% \makeindex 

% ---Fonts---
% \renewcommand{\familydefault}{\sfdefault}

% ---Title---
\title{
    2024年度\ 春の学校\ しおり
}
\author{
  校長:宮根
}
\date{最終更新:\today}

\begin{document}

\maketitle
\tableofcontents

\hspace*{5pt}

* This document is the handbook about the spring seminar. The English version begins from page \pageref{eng_page}.

\clearpage

\section{スケジュール}

\begin{center}
  \begin{tabular}{cll}\hline
    日        & 時間                 & 予定                                      \\ \hline
    4/6\ (土) & 13:00                & セミナーハウス到着                        \\
              &                      & 荷物を部屋において、ゼミ室へ。            \\
              & 13:20                & 発表開始                                  \\
              & 16:00                & 発表終了                                  \\
              &                      & 入浴など。                                \\
              & 18:00\ $\sim$\ 19:00 & 夕食                                      \\
              & 19:00\ $\sim$\ 22:00 & 宴会                                      \\
              &                      & (片づけを含めた時間)                      \\
              & 23:00                & 就寝                                      \\ \hline
    4/7\ (日) & 7:00                 & 起床                                      \\
              & 7:30\ $\sim$\  8:30  & 朝食                                      \\
              & 9:00                 & チェックアウト\ \&\ 発表開始              \\
              & 12:10                & 発表終了                                  \\
              &                      & お茶会・写真撮影など\ $\rightarrow$\ 解散
  \end{tabular}
\end{center}

\subsection*{備考}
\begin{itemize}
  \item
        入浴ができるのは16:00\ $\sim$\ 22:00です。
  \item 
        食事は上記の時間内に済ませてください。
  \item 
        一応、チェックアウトは10:00までです。
  \item
        去年は(買い出し班が)少し早めに到着したら怒られました。
\end{itemize}


\clearpage

\section{発表タイムテーブル}

\begin{center}
  \begin{tabular}{clcl}\hline
    日        & 時間                 & 発表時間\ (分) & 発表者            \\ \hline
    4/6\ (土) & 13:20\ $\sim$\ 13:45 & 25             & 中里 弘道\ (教授) \\
              & 13:50\ $\sim$\ 14:00 & 10             & 落合 誠\ (助教)   \\
              & 14:05\ $\sim$\ 14:15 & 10             & 渡辺 あかね\ (D)  \\
              & 14:20\ $\sim$\ 14:30 & 10             & 徳永 尚文\ (D)    \\
              & 14:30\ $\sim$\ 14:50 &                & (20分\ 休憩)      \\
              & 14:50\ $\sim$\ 15:00 & 10             & 岩村 海飛\ (D)    \\
              & 15:05\ $\sim$\ 15:45 & 40             & 谷口 永希\ (M)    \\
              & 15:50\ $\sim$\ 16:00 & 10             & 芝山 駿介\ (M)    \\\hline
    4/7\ (日) & 9:00\ $\sim$\ 9:25   & 25             & 安倍 博之\ (教授) \\
              & 9:30\ $\sim$\ 10:10  & 40             & 小市 明勢\ (M)    \\
              & 10:15\ $\sim$\ 10:55 & 40             & 永井 駿平\ (M)    \\
              & 10:55\ $\sim$\ 11:10 &                & (15分\ 休憩)      \\
              & 11:10\ $\sim$\ 11:20 & 10             & 嶋田 直希\ (M)    \\
              & 11:25\ $\sim$\ 11:35 & 10             & 宮根 一樹\ (M)    \\
              & 11:40\ $\sim$\ 12:10 & 30             & Raiyan Haque\ (B) \\
  \end{tabular}
\end{center}

\subsection*{備考}
\begin{itemize}
  \item
        発表時間は\textbf{質疑応答込み}の時間です。タイムキーピングをお願いします。
\end{itemize}


\clearpage

\section{部屋割り}

\begin{center}
  \begin{tabular}{cl}\hline
    部屋番号 & メンバー                                      \\ \hline
    301      & 中里 弘道                                     \\
    302      & 安倍 博之                                     \\
    303      & 渡辺 あかね                                   \\
    202      & 岩村 海飛、 徳永 尚文、落合 誠、永井 駿平     \\
    203      & 小市 明勢、 谷口 永希、宮根 一樹、嶋田 直希   \\
    204      & 芝山 駿介、 佐久間 紀丞、下田 幹人、堀内 嵩真 \\
    205      & 花村 晃、吉田 聡一郎、Raiyan Haque
  \end{tabular}
\end{center}

\subsection*{備考}
\begin{itemize}
  \item
        従わなくて結構です。
  \item
        宿泊部屋での飲酒は禁止だそうです。
\end{itemize}


\section{電車利用の際の参考経路}


\section{その他・注意事項}





\clearpage

\setcounter{section}{0}

Here is the English version. Since I spent a lot of time making this version, I hope some English students to read this eagerly.

\label{eng_page}

\section{Schedule}





\section{Timetable}




\section{Room}




\clearpage

\section{リンク集 / Links}

\begin{itemize}
  \item
        \href{https://www.waseda.jp/inst/student/facility/seminar/facility/izukawana}{ホームページ / Webpage}
  \item
        \href{https://www.waseda.jp/inst/student/facility/seminar/flow/tips}{利用上の心得 / Usage tips}
\end{itemize}


\end{document}
