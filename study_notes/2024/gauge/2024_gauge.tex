\documentclass[unicode,a4paper,11pt]{ltjsarticle}

\usepackage{luatexja-fontspec}
\setmainfont{TeX Gyre Termes}
\setmainjfont[BoldFont = IPAGothic]{IPAMincho}
\setmathrm{Latin Modern Roman}
% \setmainjfont{Noto Sans JP}

% ---Display \subsubsection at the Index
% \setcounter{tocdepth}{3}

% ---Setting about the geometry of the document----
% \usepackage{a4wide}
% \pagestyle{empty}

% ---Physics and Math Packages---
\usepackage{amssymb,amsfonts,amsthm,mathtools}
\usepackage{physics,braket,bm}

% ---underline---
\usepackage[normalem]{ulem}

% ---cancel---
\usepackage{cancel}

% --- surround the texts or equations
% \usepackage{fancybox,ascmac}

% ---settings of theorem environment---
\theoremstyle{definition}
\newtheorem{dfn}{定義}[section]
\newtheorem{prop}{命題}[section]
\newtheorem{thm}{定理}[section]
\newtheorem{exm}{例}[section]
\newtheorem{rmk}{注}[section]
\newtheorem{exc}{演習}[section]

% ---settings of proof environment---
\renewcommand{\proofname}{\textbf{証明}}
\renewcommand{\qedsymbol}{$\blacksquare$}

% ---Ignore the Warnings---
\usepackage{silence}
\WarningFilter{latexfont}{Some font shapes}
\WarningFilter{latexfont}{Font shape}
\WarningFilter{latexfont}{Size substitutions}
\ExplSyntaxOn
\msg_redirect_name:nnn{hooks}{generic-deprecated}{none}
\ExplSyntaxOff

% ---Insert the figure (If insert the `draft' at the option, the process becomes faster.)---
\usepackage{graphicx}
% \usepackage{subcaption}

% ----Add a link to a text---
\usepackage{url,hyperref}
\usepackage[dvipsnames,svgnames]{xcolor}
\hypersetup{colorlinks=true,citecolor=FireBrick,linkcolor=Navy,urlcolor=purple}
% ---refer `texdoc xcolor' at the command line---

% ---Tikz---
% \usepackage{tikz,pgf,pgfplots,circuitikz}
% \pgfplotsset{compat=1.15}
% \usetikzlibrary{intersections,arrows.meta,angles,calc,3d,decorations.pathmorphing}

% ---Add the section number to the equation, figure, and table number---
\makeatletter
   \renewcommand{\theequation}{$\thesection.\arabic{equation}$}
   \@addtoreset{equation}{section}
   
   \renewcommand{\thefigure}{\thesection.\arabic{figure}}
   \@addtoreset{figure}{section}
   
   \renewcommand{\thetable}{\thesection.\arabic{table}}
   \@addtoreset{table}{section}
\makeatother

% ---enumerate---
\renewcommand{\labelenumi}{(\roman{enumi})}
% \renewcommand{\labelenumii}{$(\arabic{enumii})$}

% ---Index---
% \usepackage{makeidx}
% \makeindex 

% ---Fonts---
% \renewcommand{\familydefault}{\sfdefault}

% ---Title---
\title{
  ゲージ理論と幾何
}
\author{
  宮根\ 一樹
}
\date{最終更新日:\today}

\begin{document}

\maketitle
\tableofcontents

\vspace*{5pt}

\subsection*{To Doリスト}

\begin{itemize}
  \item 
  注\ref{rmk:vector_str_TpM}の内容の吟味・補足。
  \item 
  証明\ref{prf:prop_1_1}。
\end{itemize}

\clearpage
\section{多様体とその周辺}

\subsection{多様体}

まずは多様体の定義から。

\begin{dfn}[多様体]
  ハウスドルフ空間$M$に対して、$M$が開集合$U_{i}$によって
  \begin{equation}
    U
    =
    \bigcup_{i}U_{i}
  \end{equation}
  で表され、各$U_{i}$に対して、$U_{i}$から$n$次元ベクトル空間$\mathbb{R}^{n}$への全単射$\bm{x}_{U_{i}}$があって
  \begin{enumerate}
    \item
          像$\bm{x}_{U_{i}}(U)$は$\mathbb{R}^{n}$の開集合で、$\bm{x}_{U_{i}}$は$U_{i}$から$\bm{x}_{U_{i}}(U)$への同相写像。
    \item
          $U_{i}\cap U_{j}\neq\phi$ならば、写像
          \begin{equation}
            \bm{x}_{U_{j}}\circ\bm{x}_{U_{i}}^{-1}
            :
            \bm{x}_{U_{i}}(U_{i}\cap U_{j})\rightarrow\bm{x}_{U_{j}}(U_{i}\cap U_{j})
          \end{equation}
          が全単射で$C^{\infty}$かつ逆写像も同様。
  \end{enumerate}
  を満たすとき、
  \begin{itemize}
    \item
          組$\{(U_{i},\bm{x}_{U_{i}})\}$の全体は$M$に\textbf{$C^{\infty}$構造}を与え、
    \item
          $M$を\textbf{$C^{\infty}$多様体}という。
  \end{itemize}
\end{dfn}

\begin{exm}[直積多様体]
  $m$次元$C^{\infty}$多様体$M$、$n$次元$C^{\infty}$多様体$N$の$C^{\infty}$構造が$\{(U_{\alpha},\phi_{\alpha})\}_{\alpha\in A},\{(V_{\beta},\psi_{\beta})\}_{\beta\in B}$で定められているとき、
  \begin{equation}
    (\phi_{\alpha}\times\psi_{\beta})(x,y)
    =
    (\phi_{\alpha}(x),\psi_{\beta}(y))
  \end{equation}
  で定義しておけば、$\{(U_{\alpha}\times V_{\beta},\phi_{\alpha}\times\psi_{\beta})\}_{(\alpha,\beta)\in A\times B}$は$M\times N$上に$C^{\infty}$構造が定められて、$M\times N$は$C^{\infty}$多様体になる。これは直積多様体。
\end{exm}

\begin{exm}[$n$次元球面]
  $\mathbb{R}^{n+1}$に対して
  \begin{equation}
    S^{n}
    =
    \{
    (x^{1},\cdots,x^{n+1})\in\mathbb{R}^{n+1}
    |
    \sum (x^{i})^2=1
    \}
  \end{equation}
  において、各$i=1,\cdots,n+1$に対して
  \begin{equation}
    U_{i}^{\pm}
    \equiv
    \{
    (x^{1},\cdots,x^{i+1})\in S^{n}
    |
    x^{i}\gtrless 0
    \}
  \end{equation}
  とおいて\footnote{
    複号$\pm$と$\gtrless$の上下は一致しているものとする。
  }、$x_{i}^{\pm}:U_{i}^{\pm}\rightarrow\mathbb{R}^{n}$を
  \begin{equation}
    x_{i}^{\pm}(x^{1},\cdots,x^{n+1})
    =
    (x^{1},\cdots,\widehat{x^{i}},\cdots,x^{n+1})
    \in \mathbb{R}^{n}
  \end{equation}
  とする\footnote{
    ハット$\widehat{\ }$はその成分がないことを表す。
  }と、$x^{\pm}_{i}$の像は$\mathbb{R}^{n}$の開球(内部を含む)である。これらは全単射で、$\{(U_{i}^{\pm},x_{i}^{\pm})\}_{i=1,\cdots,n}$が$S^{n}$上に$C^{\infty}$構造を定めるため、$S^{n}$は多様体。
\end{exm}

\begin{exm}[射影空間]
  $\mathbb{K}=\mathbb{R}\ \mathrm{or}\ \mathbb{C}$の元
  \begin{equation}
    \bm{x}=(x^{0},x^{1},\cdots,x^{n}),
    \bm{y}=(y^{0},y^{1},\cdots,y^{n})
    \in
    \mathbb{K}^{n+1}
  \end{equation}
  に対して、ある$\alpha\in\mathbb{K}$があって$\bm{y}=\alpha\bm{x}$なら$\bm{y}\sim\bm{x}$だとする。するとこれは同値。$[\bm{x}]$は同値類で、それら全体の集合を$P^{n}(\mathbb{K})$とすれば、$P^{n}(\mathbb{K})$は射影空間。

  これが$C^{\infty}$多様体であることを見るために、$C^{\infty}$構造を構成する。まず、
  \begin{equation}
    \pi
    :
    \mathbb{K}^{n+1}\ \{0\}\in\bm{x}
    \rightarrow
    [\bm{x}]\ni P^{n}(\mathbb{K})
  \end{equation}
  で、$P^{n}(\mathbb{K})$の位相を$\mathbb{K}^{n+1}$の位相から定義する。したがって、$P^{n}(\mathbb{K})$の開集合が考えられて、それを$U_{i}=\{[(x^{0},x^{1},\cdots,x^{n})]\in P^{n}(\mathbb{K})|x^{i}\neq 0\}$として、写像$\bm{x}_{i}:U_{i}\rightarrow\mathbb{K}^{n}$を
  \begin{equation}
    \bm{x}_{i}([(x^{0},x^{1},\cdots,x^{n})])
    =
    \left(
    \frac{x^{0}}{x^{i}}
    ,\cdots,
    \widehat{
      \frac{x^{i}}{x^{i}}
    },\frac{x^{n+1}}{x^{i}}
    \right)
  \end{equation}
  とする。すると、$\{(U_{i},x^{i})\}_{i=1,\cdots,n+1}$は$C^{\infty}$構造を定める。

  最後に、この写像がwell-definedなことを確認する。$P^{n}(\mathbb{K})$の2つの元
  \begin{equation}
    [(x^{0},\cdots,x^{n})]
    ,
    [(y^{0},\cdots,y^{n})]
  \end{equation}
  が等しいとする。したがって、ある$\alpha\in\mathbb{K}$が存在して$y^{i}=\alpha x^{i}$なので、このことを用いれば
  \begin{equation}
    \bm{x}_{i}([(y^{0},\cdots,y^{n})])
    =
    \left(
    \frac{y^{0}}{y^{i}}
    ,\cdots,
    \widehat{
      \frac{y^{i}}{y^{i}}
    },\frac{^{n+1}}{y^{i}}
    \right)
    =
    \left(
    \frac{x^{0}}{x^{i}}
    ,\cdots,
    \widehat{
      \frac{x^{i}}{x^{i}}
    },\frac{x^{n+1}}{x^{i}}
    \right)
    =
    \bm{x}_{i}([(x^{0},\cdots,x^{n})])
  \end{equation}
  であり、well-defined。
\end{exm}

\begin{exm}
  $C^{\infty}$多様体の開集合$U$もまた$C^{\infty}$多様体。$C^{\infty}$構造は、元の多様体の構造と$U$の共通部分をとればよい。また、閉集合は多様体とは限らない。
\end{exm}

\begin{dfn}[座標近傍と局所座標系]
  $C^{\infty}$多様体$M$の開集合$W$、写像$\psi:W\rightarrow\mathbb{R}^{n}$が次の条件を満たすとき、$(W,\psi)$を\textbf{座標近傍}という:
  \begin{enumerate}
    \item
          $\psi:W\rightarrow\psi(W)$は同相写像。
    \item
          $C^{\infty}$構造を$\{(U,\bm{x}_{U})\}$とするとき、$U\cap W\neq\phi$なる全ての$U$に対して、写像
          \begin{equation}
            \bm{x}_{U}\circ\psi^{-1}
            :
            \psi(W\cap U)\rightarrow \bm{x}_{U}(W\cap U)
          \end{equation}
          とその逆写像は$\mathbb{R}^{n}$の意味で$C^{\infty}$写像。
  \end{enumerate}

  また、$p\in U$に対して、$\mathbb{R}^{n}$の点$\bm{x}_{U}(p)$を
  \begin{equation}
    \bm{x}_{U}(p)
    =
    (x^{1}(p),\cdots,x^{n}(p))
  \end{equation}
  と表すことがある。このとき、$x^{1},\cdots,x^{n}$を\textbf{局所座標系}という。
\end{dfn}

\begin{rmk}
  $C^{\infty}$構造$\{(U,\bm{x}_{U})\}$も定義から座標近傍。
\end{rmk}

\begin{dfn}[$C^{\infty}$写像と微分同相]
  $M,N$は$C^{\infty}$多様体、写像$f:M\rightarrow N$を考える。$f$が次の条件を満たすとき、\textbf{$C^{\infty}$写像}という:
  \begin{flushleft}
    $M,N$の任意の座標近傍$(U,\bm{x}_{U}),(V,\bm{y}_{V})$に対して、$V\cap f(U)\neq\phi$ならば、写像
    \begin{equation}
      \bm{y}_{V}\circ f\circ\bm{x}_{U}^{-1}
      :
      \bm{x}_{U}(U)\rightarrow\bm{y}_{V}(V)
    \end{equation}
    は、$\mathbb{R}^{m}$から$\mathbb{R}^{n}$への$C^{\infty}$写像。
  \end{flushleft}
  また、$f:M\rightarrow N$が全単射で上の意味で$C^{\infty}$写像であり、逆写像$f^{-1}$も同様ならば、$f$は$M$と$N$の間の\textbf{微分同相}という。
\end{dfn}

\begin{exm}
  3次元球面$S^{3}$と2次の特殊ユニタリー群
  \begin{equation}
    SU(2)
    =
    \{U\in GL(2,\mathbb{C})|UU^{\dag}=1,\det U=1\}
  \end{equation}
  は微分同相。
  \begin{proof}
    \label{prf:eg_s3_unitary}
    $SU(2)$は、定義から
    \begin{equation}
      SU(2)
      =
      \left\{
        \left.
        \begin{pmatrix}
          a & -\bar{b} \\
          b & \bar{a}
         \end{pmatrix}
        \right|
        a,b\in \mathbb{C}
        ,\ 
        |a|^2+|b|^2
        =
        1
      \right\}
    \end{equation}
    と書ける。2つの複素数は、その実部と虚部を具体的に書いてしまえば
    \begin{equation}
      (\Re a)^2
      +
      (\Im a)^2
      +
      (\Re b)^2
      +
      (\Im b)^2
      =
      1
    \end{equation}
    なので、$S^{3}$の上に乗っている。(本当は同型であることをもっとまじめに言わないとダメなのだろうが。)
  \end{proof}
\end{exm}

\begin{dfn}[曲線と接ベクトル、接ベクトル空間]
  $\mathbb{R}$の開区間$(a,b)$から$C^{\infty}$多様体$M$への$C^{\infty}$写像$c:(a,b)\rightarrow M$を$M$上の\textbf{曲線}という。以後は簡単のため、$t=0$で$c(0)=p\in M$とする。つまり、この曲線は点$p$を通る。

  $p$を通る2つの曲線$c_{1},c_{2}$に対して、その座標近傍$(U,\bm{x}_{U})$で$(\bm{x}_{U}\circ c_{1})^{\prime}(0)=(\bm{x}_{U}\circ c_{2})^{\prime}(0)$であるとき、$c_{1}\sim c_{2}$とする。この定義によって定まる曲線$c$の同値類を
  \begin{equation}
    c'(0)
    \mathrm{\ or\ }
    \left.\dv{}{t}c(t)\right|_{t=0}
  \end{equation}
  と書くことにし、これらを\textbf{接ベクトル}という。

  $p$を通る曲線の同値類全体を$T_{p}M$とする。すると、$T_{p}M$はベクトル空間の構造をもつので、この$T_{p}M$を$p$における$M$の\textbf{接ベクトル空間}という。
\end{dfn}

\begin{rmk}[$T_{p}M$のベクトル空間としての構造]
  \label{rmk:vector_str_TpM}
  $\mathbb{R}^{m}$の構造をうまく入れればよくて、
  \begin{equation}
    ac_{1}+bc_{2}
    \equiv
    \bm{x}^{-1}\circ(a\bm{x}\circ c_{1}+b\bm{x}\circ c_{2})
  \end{equation}
  でたぶんOK?
\end{rmk}

\begin{dfn}
  $p$の近傍で定義された$C^{\infty}$関数$f$に対して、接ベクトル$X_{p}\in T_{p}M$に沿った微分係数を
  \begin{equation}
    X_{p}(f)
    \equiv
    (f\circ c)^{\prime}(0)
  \end{equation}
  と定義する。ここで、$c(t)$は同値類$X_{p}$の代表元。
\end{dfn}

\begin{exm}
  $(U,x^{1},\cdots,x^{n})$を座標近傍として、$p\in U$に対して
  \begin{equation}
    (x_{0}^{1},\cdots,x_{0}^{n})
    \equiv
    (x^{1}(p),\cdots,x^{n}(p))
  \end{equation}
  とする。$p$を通る曲線
  \begin{equation}
    \bm{x}\circ c_{i}(t)=(x_{0}^{1},\cdots,x_{0}^{i}+t,\cdots,x_{0}^{n})
  \end{equation}
  に対して、接ベクトル$c_{i}^{\prime}(0)$を
  \begin{equation}
    \left( \pdv{}{x^{i}} \right)_{p}
  \end{equation}
  で表すと、
  \begin{equation}
    \left\{
    \left( \pdv{}{x^{1}} \right)_{p}
    ,\cdots,
    \left( \pdv{}{x^{n}} \right)_{p}
    \right\}
  \end{equation}
  は接空間$T_{p}M$の基底となる。
\end{exm}

\begin{dfn}[ベクトル場]
  まず、
  \begin{equation}
    TM=\bigcup_{p\in M}T_{p}M
  \end{equation}
  とする。対応$X:M\rightarrow TM$が次の条件を満たすとき、$X$を$M$上の\textbf{ベクトル場}という:
  \begin{enumerate}
    \item
          任意の$p\in M$に対して、$X(p)=X_{p}\in T_{p}M$。
    \item 
    任意の$f\in C^{\infty}(M)$について、写像$p\mapsto X_{p}(f)$は$C^{\infty}$写像。
  \end{enumerate}
  以後、$M$上のベクトル場全体を$\mathfrak{X}(M)$と書く。
\end{dfn}

\begin{rmk}[ベクトル場の局所座標表示]
  ベクトル場$X$を$p$の座標近傍$(U,x^{1},\cdots,x^{n})$に制限すると
  \begin{equation}
    X_{p}
    =
    \sum X^{i}(p)\left( \pdv{}{x^{i}} \right)_{p}
    ,\ 
    X^{i}\in C^{\infty}(U)
    \label{eqn:x}
  \end{equation}
  と表すことができる。上の定義(ii)は、$X^{i}$が$C^{\infty}$であることを言っている。もう1つの近傍を$(V,y^{1},\cdots,y^{n})$とすると、
  \begin{equation}
    X_{p}
    =
    \sum Y^{i}(p)\left( \pdv{}{y^{i}} \right)_{p}
    \label{eqn:y}
  \end{equation}
  と書けるが、$y^{i}(p)=y^{i}(x^{1}(p),\cdots,x^{n}(p))$と書けることから、\eqref{eqn:x}を書き直すと
  \begin{equation}
    X_{p}
    =
    \sum X^{i}(p)\left( \pdv{y^{j}}{x^{i}} \right)\left( \pdv{}{y^{j}} \right)_{p}
  \end{equation}
  となるので、\eqref{eqn:y}と比較して
  \begin{equation}
    Y^{i}
    =
    \sum X^{j}(p)\left( \pdv{y^{i}}{x^{j}} \right)
  \end{equation}
  である。
\end{rmk}

\begin{dfn}[交換子積]
  $X,Y\in\mathfrak{X}(M),\ f\in C^{\infty}(M)$に対して
  \begin{equation}
    [X,Y]_{p}(f)
    \equiv
    X_{p}(Y(f))
    -
    Y_{p}(X(f))
  \end{equation}
  とすれば、$[X,Y]\in\mathfrak{X}(M)$であることが分かる。これを\textbf{交換子積}という。
\end{dfn}

\begin{rmk}
  $[X,Y]\in\mathfrak{X}(M)$であることは
  \begin{align}
    [X,Y]_{p}(f)
    &=
    X_{p}(Y(f))
    -
    Y_{p}(X(f))
    \nonumber
    \\
    &=
    X_{p}
    \left( 
      \sum Y^{i}\pdv{f}{y^{i}}
     \right)
     -
     Y_{p}
     \left(  
      \sum X^{j}\pdv{f}{x^{j}}
     \right)
     \nonumber
     \\
     &=
     \sum\left[  
      X_{j}(p)
      \left(  
        \pdv{Y^{i}}{x^{j}}\pdv{f}{y^{i}}
        +
        Y^{i}(p)\pdv{f}{x^{j}}{y^{i}}
      \right)
     \right]
     -
     Y^{i}(p)
     \left(  
      \pdv{X^{j}}{y^{i}}\pdv{f}{x^{j}}
      +
      X^{j}(p)\pdv{f}{y^{i}}{x^{j}}
     \right)
     \nonumber
     \\
     &=
     \sum
     \left(  
      X^{j}\pdv{Y^{i}}{x^{j}}\pdv{}{y^{i}}
      -
      Y^{i}\pdv{X^{j}}{y^{i}}\pdv{}{x^{j}}
     \right)_{p}
     f
  \end{align}
  ということから分かる。交換子積じゃないと、2階微分が相殺しない。
\end{rmk}

\begin{prop}
  $a,b\in \mathbb{R},\ X,Y,Z\in \mathfrak{X}(M)$とするとき
  \begin{gather}
    [aX+bY,Z]
    =
    a[X,Z]
    +
    b[Y,Z]
    \\
    [X,Y]
    +
    [Y,X]
    =
    0
    \\
    [[X,Y],Z]
    +
    [[Y,Z],X]
    +
    [[Z,X],Y]
    =
    0
  \end{gather}
  が成立する。特に最後の等式は、ヤコビ恒等式といわれる。
\end{prop}

\begin{proof}
  \label{prf:prop_1_1}
  (更新中)
\end{proof}

\begin{dfn}
  ベクトル場$X$について、任意の$p\in M$に対して$p$を通る曲線$c_{p}:\mathbb{R}\rightarrow M$で
  \begin{equation}
    c_{p}(0)
    =
    p
    ,\ 
    c^{\prime}_{p}(t)
    =
    X_{c_{p}(t)}
    ,\ 
    t\in\mathbb{R}
  \end{equation}
  となるものが存在するとき、$X$は\textbf{完備である}という。

  また、この曲線$c_{p}$を$X$の\textbf{積分曲線}という。

  $t\in\mathbb{R}$に対して、写像
  \begin{equation}
    \varphi_{t}:M\ni p\mapsto c_{p}(t)\in M
  \end{equation}
  は$M$の微分同相写像を定め、
  \begin{equation}
    \varphi_{s}\circ\varphi_{t}(p)
    =
    \varphi_{s+t}(p)
    \label{eqn:one_parameter_grp}
  \end{equation}
  である\footnote{
    この関係を示すには、常微分方程式論の考えが必要である。詳しい証明はともかく、説明が欲しいなら\cite{Nakahara:2003}を参照のこと。
  }。ここで、$\{\varphi_{t}\in M|t\in \mathbb{R}\}$は$X$によって生成された$M$上の\textbf{1パラメター変換群}とよばれ、$\{\varphi_{t}(p)|t\in\mathbb{R}\}$を$p$を通る\textbf{軌道}という。
\end{dfn}



























% ----------------------------------------
% \clearpage

% \makeatletter
% \renewcommand{\appendix}{\par
%   \setcounter{section}{0}%
%   \setcounter{subsection}{0}%
%   \gdef\presectionname{\appendixname}%
%   \gdef\postsectionname{}%
%   \gdef\thesection{\presectionname\@Alph\c@section\postsectionname}%
%   \gdef\thesubsection{\@Alph\c@section.\@arabic\c@subsection}%
%   \renewcommand{\theequation}{\@Alph\c@section.\arabic{equation}}%
%   \renewcommand{\thefigure}{\@Alph\c@section.\arabic{figure}}%
%   \renewcommand{\thetable}{\@Alph\c@section.\arabic{table}}%
% }
% \makeatother
% \appendix

% \section{***}



% ----------------------------------------
\clearpage
\bibliography{ref}
\bibliographystyle{ytphys}

\nocite{Mogi:2001}
\nocite{Nakahara:2003}

\end{document}
