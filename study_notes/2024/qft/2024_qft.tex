\pdfoutput=1
\documentclass[a4paper,pdftex,11pt]{article}
% \usepackage[whole]{bxcjkjatype}
\usepackage[T1]{fontenc}
\usepackage{tgtermes}

% ---Display \subsubsection at the Index
\setcounter{tocdepth}{3}

% ---Setting about the geometry of the document----
\usepackage{a4wide}
% \pagestyle{empty}

% ---Physics and Math Packages---
\usepackage{amssymb,amsfonts,amsthm,mathtools}
\usepackage{physics,braket,bm,slashed}

% ---underline---
\usepackage[normalem]{ulem}

% ---cancel---
\usepackage{cancel}

% --- surround the texts or equations
% \usepackage{fancybox,ascmac}

% ---settings of theorem environment---
% \theoremstyle{definition}
% \newtheorem{dfn}{Definition}
% \newtheorem{prop}{Proposition}
% \newtheorem{thm}{Theorem}

% ---settings of proof environment---
% \renewcommand{\proofname}{\textbf{Proof}}
% \renewcommand{\qedsymbol}{$\blacksquare$}

% ---Ignore the Warnings---
\usepackage{silence}
\WarningFilter{latexfont}{Some font shapes,Font shape}
\ExplSyntaxOn
\msg_redirect_name:nnn{hooks}{generic-deprecated}{none}
\ExplSyntaxOff

% ---Insert the figure (If insert the `draft' at the option, the process becomes faster.)---
% \usepackage{graphicx}
% \usepackage{subcaption}

% ----Add a link to a text---
\usepackage{url,hyperref}
\usepackage[dvipsnames,svgnames]{xcolor}
\hypersetup{colorlinks=true,citecolor=FireBrick,linkcolor=Navy,urlcolor=purple}
% \usepackage[whole,autotilde]{bxcjkjatype}

% ---Tikz---
\usepackage{tikz,pgf,pgfplots,circuitikz}
\pgfplotsset{compat=1.15}
\usetikzlibrary{intersections, arrows.meta, angles, calc, 3d, decorations.pathmorphing}
\usepackage[compat=1.1.0]{tikz-feynhand}

% ---Add the section number to the equation, figure, and table number---
\makeatletter
   \renewcommand{\theequation}{\thesection.\arabic{equation}}
   \@addtoreset{equation}{section}
   
   \renewcommand{\thefigure}{\thesection.\arabic{figure}}
   \@addtoreset{figure}{section}
   
   \renewcommand{\thetable}{\thesection.\arabic{table}}
   \@addtoreset{table}{section}
\makeatother

% ---enumerate---
% \renewcommand{\labelenumi}{$\arabic{enumi}.$}
% \renewcommand{\labelenumii}{$(\arabic{enumii})$}

% ---Index---
% \usepackage{makeidx}
% \makeindex 

% ---Fonts---
% \renewcommand{\familydefault}{\sfdefault}

% ---footnotes---
\renewcommand{\thefootnote}{$\ast$\arabic{footnote}}

% ---Title---
\title{My Study Notes 2024 / QFT}
\author{Itsuki Miyane}
\date{Last modified:\ \today}

\begin{document}

\maketitle

\tableofcontents

\clearpage
\section{Introduction}
















\clearpage
\section{Functional Method}

















\clearpage
\section{Geometry of the Spacetime}

We will

























\clearpage
\section{Perturbation Theory Anomalies}

In specific circumstances, quantum corrections can destroy symmetries of the classical equations of motion.

\subsection{Intuitive Example: The Axial Current in Two Dimensions}








\clearpage
\appendix
\section{Some notes}

\subsection{Normalization of Maxwell Lagrangian}

The Maxwell lagrangian is the form
\begin{equation}
  \mathcal{L}
  =
  NF^{\mu\nu}F_{\mu\nu}
\end{equation}
where $F^{\mu\nu}=\partial^{\mu}A^{\nu}-\partial^{\nu}A^{\mu}$ is a field strength. We will determine the constant $N$ to lagrangian contains the term
\begin{equation}
  \mathcal{L}
  =
  \frac{1}{2}\dot{A}_{1}^2
  +
  \frac{1}{2}\dot{A}_{2}^2
  +
  \frac{1}{2}\dot{A}_{3}^2
  +
  \cdots
  \ .
\end{equation}
Now expanding the term $F^{\mu\nu}F_{\mu\nu}$ carefully, we obtain
\begin{align}
  F^{\mu\nu}F_{\mu\nu}
   & =
  (\partial^{\mu}A^{\nu}-\partial^{\nu}A^{\mu})
  (\partial_{\mu}A_{\nu}-\partial_{\nu}A_{\mu})
  \nonumber
  \\
   & =
  2((\partial^{\mu}A^{\nu})(\partial_{\mu}A_{\nu})-(\partial_{\mu}A^{\nu})(\partial_{\nu}A_{\mu}))
  \nonumber
  \\
   & =
  -2(\dot{A}_{1}^2+\dot{A}_{2}^2+\dot{A}_{3}^2)+\cdots
\end{align}
and thus, the constant $N$ satisfies the condition $-2N=1/2$ and $N=-1/4$. That's why we get Maxwell Lagrangian as\footnote{aaa}
\begin{equation}
  \mathcal{L}
  =
  -\frac{1}{4}F^{\mu\nu}F_{\mu\nu}
  \ .
\end{equation}














\end{document}
