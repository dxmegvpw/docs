\pdfoutput=1
\documentclass[a4paper,pdflatex,ja=standard]{bxjsarticle}

% ---Setting about the geometry of the document----
% \usepackage{a4wide}
% \pagestyle{empty}

% ---Physics and Math Packages---
\usepackage{amssymb,amsfonts,amsthm,mathtools}
\usepackage{physics,braket,bm}

% ---underline---
\usepackage{ulem}

% --- sorround the texts or equations
% \usepackage{fancybox,ascmac}

% ---settings of theorem environment---
% \usepackage{amsthm}
% \theoremstyle{definition}

% ---settings of proof environment---
\renewcommand{\proofname}{\uline{\textbf{証明}}}
\renewcommand{\qedsymbol}{$\blacksquare$}

% ---Ignore the Warnings---
\usepackage{silence}
\WarningFilter{latexfont}{Some font shapes,Font shape}

% ---Insert the figure (If insert the `draft' at the option, the process becomes faster)---
\usepackage{graphicx}
% \usepackage{subcaption}

% ----Add a link to a text---
\usepackage{url}
\usepackage{xcolor,hyperref}
\hypersetup{colorlinks=true,citecolor=orange,linkcolor=blue,urlcolor=magenta}
\usepackage{bxcjkjatype}

% ---Tikz---
\usepackage{tikz,pgf,pgfplots,circuitikz}
\pgfplotsset{compat=1.15}
\usetikzlibrary{intersections,arrows.meta,angles,calc,3d,decorations.pathmorphing}

% ---Add the section number to the equation, figure, and table number---
\makeatletter
   \renewcommand{\theequation}{\thesubsection.\arabic{equation}}
   \@addtoreset{equation}{subsection}
   
   \renewcommand{\thefigure}{\thesection.\arabic{figure}}
   \@addtoreset{figure}{section}
   
   \renewcommand{\thetable}{\thesection.\arabic{table}}
   \@addtoreset{table}{section}
\makeatother

% ---enumerate---
\renewcommand{\labelenumi}{\arabic{enumi}.}
\renewcommand{\labelenumii}{(\roman{enumii})}

% ---Index---
% \usepackage{makeidx}
% \makeindex 

% ---Fonts---
\renewcommand{\familydefault}{\sfdefault}

% ---Title---
\title{東京大学\ 平成29年\ 物理学専攻\ 院試\ 解答例}
\author{ミヤネ}
\date{最終更新:\today}

\newcommand{\prb}[2]{
  \phantomsection
  \addcontentsline{toc}{subsection}{問題 #1: #2}
  \subsection*{第#1問\phantom{#2}}
  \setcounter{subsection}{#1}
  \setcounter{equation}{0}
}

\begin{document}

\maketitle

\tableofcontents
\clearpage

\section{数学パート}

\prb{1}{微積分}
\begin{enumerate}
  \item 
  \begin{enumerate}
    \item 
    収束するのは$s>0$.
    \begin{proof}
      調べればよいのは,
      \begin{equation}
        \lim_{R\rightarrow\infty}
        \int_{0}^{R}
        te^{-st}
        \dd t
        \label{laplace_t}
      \end{equation}
      の収束です.ここで,$f(t)=te^{-st}$とおくと,
      \begin{equation}
        tf(t)
        =
        t^3e^{-st}
        \rightarrow
        0
        \quad
        (t\rightarrow\infty)
      \end{equation}
      なので,$M>0$に対して,ある$\Lambda>0$が存在して
      \begin{equation}
        t^2f(t)<M
        ,\ 
        t>\Lambda
      \end{equation}
      です.このような$M,\Lambda$について
      \begin{equation}
        \lim_{R\rightarrow\infty}
        \int_{\Lambda}^{R}
        f(t)
        \dd t
        \leq
        \lim_{R\rightarrow\infty}
        \int_{\Lambda}^{R}
        \frac{M}{t^2}
        \dd t
        <
        \infty
      \end{equation}
      なので,積分\eqref{laplace_t}は収束します.
    \end{proof}
    値は
    \begin{align}
      L[t]
      &=
      \int_0^\infty
      te^{-st}
      \dd t
      \nonumber
      \\
      &=
      \left[  
        -\frac{1}{s}te^{-st}
      \right]_0^\infty
      +
      \left[  
        -\frac{1}{s^2}e^{-st}
      \right]_0^\infty
      =
      \frac{1}{s^2}
    \end{align}
    です.

    \item 
    収束するのは$s\neq 0$です.
    \begin{proof}
      簡単で
      \begin{equation}
        e^{-st}\sin\omega t
        <
        e^{-st}
      \end{equation}
      を積分すればよいです.
    \end{proof}
    値ですが,2回部分積分してみると
    \begin{equation}
      L[\sin\omega t]
      =
      \frac{\omega}{s^2}
      -
      \frac{\omega^2}{s^2}L[\sin\omega t]
    \end{equation}
    となることがわかるので
    \begin{equation}
      L[\sin\omega t]
      =
      \frac{\omega}{s^2+\omega^2}
    \end{equation}
    です.
    
    \item 
    収束するのは$s>0$のときです.証明は$f(t)=t$のときとほぼ同じ.積分の値は,途中で$t=x^2$と変数変換をしてみると
    \begin{align}
      L[\sqrt{t}]
      &=
      \int_0^\infty
      \sqrt{t}e^{-st}
      \dd t
      \nonumber
      \\
      &=
      2\int_0^\infty
      x^2e^{-sx^2}
      \dd x
      \nonumber
      \\
      &=
      \frac{1}{2}\sqrt{\frac{\pi}{s^3}}
    \end{align}
    です\footnote{
      nz先生の講義を覚えてるならGaussian integral
      $$
        \int_0^\infty
        e^{-s x^2}
        \dd x
        =
        \frac{1}{2}\sqrt{\frac{\pi}{s}}
      $$
      を,$s$で微分して
      $$
        \int_0^\infty
        x^2e^{-sx^2}
        \dd x
        =
        \frac{1}{4}
        \sqrt{\frac{\pi}{s^3}}
      $$
      とすれば早いです.
    }.

  \end{enumerate}

  \item 
  \begin{enumerate}
    \item 
    $f_1*f_2$を変換すると
    \begin{equation}
      L[f_1*f_2]
      =
      \int_0^\infty
      \dd t
      \int_0^t
      \dd t^{\prime}\ 
      e^{-s(t-t^{\prime})}
      f_1(t-t^{\prime})
      \cdot
      e^{-st^{\prime}}
      f_2(t^{\prime})
    \end{equation}
    となります.この積分範囲は$0\leq t^{\prime}<t$なので,$t-t^{\prime}>0,t^{\prime}>0$の領域で積分するのと等価です.よって,
    \begin{equation}
      L[f_1*f_2]
      =
      \int_0^\infty
      \dd t_1
      \int_0^\infty
      \dd t_2\ 
      e^{-st_1}
      f_1(t_1)
      \cdot
      e^{-st_2}
      f_2(t_2)
      =
      F_1(s)
      F_2(s)
    \end{equation}
    となります.

    \item 
    両辺をLaplace変換すると
    \begin{equation}
      F(s)
      =
      \frac{1}{s^2+1}
      +
      \frac{1}{s^2+1}F(s)
    \end{equation}
    なので,
    \begin{equation}
      F(s)
      =
      \frac{1}{s^2}
    \end{equation}
    です.

    \item 
    $L[t]=1/s^2$だったので$f(t)=t$.
  \end{enumerate}
  
  \item 
  計算すればいいのは
  \begin{equation}
    f(t)
    =
    \frac{1}{2\pi i}
    \int_{\gamma-i\infty}^{\gamma+i\infty}
    \frac{s}{(s-1)^2}e^{st}
    \dd s
  \end{equation}
  です.極はすべて経路が囲む閉じた領域内にあって,積分の経路はそれら極を反時計回りにまわるので\footnote{
    ここらへんのお話は,そういったものとして許してください.
  },
  \begin{equation}
    \frac{1}{2\pi i}\int_{\gamma-i\infty}^{\gamma+i\infty}
    \frac{s}{(s-1)^2}e^{st}
    \dd s
    =
    \lim_{s\rightarrow1}
    \dv{}{s}\left[  
      se^{st}
    \right]
    =
    (t+1)e^t
  \end{equation}
  となります.

\end{enumerate}

\subsection*{補足}

\begin{itemize}
  \item 
  もともとLaplace変換は微分or積分方程式をある程度機械的に解けるのがメリットなので,2.(iii)をダイレクトに計算するのはナンセンスでしょう.ただ,もちろん,実際には計算できて,
  \begin{equation}
    f(t)
    =
    \frac{1}{2\pi i}
    \int_{-i\infty}^{+i\infty}
    \frac{1}{s^2}e^{st}
    \dd s
    =
    \lim_{s\rightarrow0}
    \left[ te^{st} \right]
    =
    t
  \end{equation}
  です.

  \item 
  3.も
  \begin{align}
    L[e^t](s)
    &=
    \int_0^\infty
    e^{-(s-1)t}
    \dd t
    \nonumber
    \\
    &=
    \frac{1}{s-1}
    \\
    L[te^t](s)
    &=
    \int_0^\infty
    te^{-(s-1)t}
    \dd t
    \nonumber
    \\
    &=
    \frac{1}{(s-1)^2}
  \end{align}
  を知っていれば,すぐわかりそうです.

\end{itemize}





\clearpage
\prb{2}{線形代数}
\begin{enumerate}
  \item 
  \begin{equation}
    J(\bm{a})
    =
    \begin{pmatrix}
      0 & -a_3 & a_2 \\
      a_3 & 0 & -a_1 \\
      -a_2 & a_1 & 0
    \end{pmatrix}
    .
  \end{equation}

  \item 
  固有方程式は
  \begin{equation}
    \lambda
    (\lambda^2-|\bm{a}|^2)
    =
    0
  \end{equation}
  なので,$\lambda=0,\pm|\bm{a}|$が固有値です.

  \item 
  $P(\bm{a})\bm{x}=-\bm{a}\times(\bm{a}\times\bm{x})$の関係があることを使います.
  \begin{enumerate}
    \item 
    $\bm{A}\times(\bm{B}\times\bm{C})=(\bm{A}\cdot\bm{C})\bm{B}-(\bm{A}\cdot\bm{B})\bm{C}$と$\bm{A}\times\bm{A}=0$がわかっていればOK.

    \item 
    計算すると
    \begin{equation}
      P(\bm{a})\bm{b}
      =
      -\bm{a}\times(\bm{a}\times\bm{b})
      =
      -(\bm{a}\cdot\bm{b})\bm{a}
      +
      (\bm{a}\cdot\bm{a})\bm{b}
      =
      \bm{b}
      .
    \end{equation}

    \item 
    任意のベクトル$\bm{x}$を$\bm{a}$とそれに垂直な2つの単位ベクトル$\bm{b}_1,\bm{b}_2$に分解して
    \begin{equation}
      \bm{x}
      =
      a\bm{a}
      +
      \sum_{i=1,2}
      b_i\bm{b}_i
    \end{equation}
    とすれば,前問の結果から
    \begin{equation}
      P(\bm{a})^2\bm{x}
      =
      \sum_{i=1,2}
      b_i\bm{b}_i
      =
      P(\bm{a})\bm{x}
    \end{equation}
    なので,$P(\bm{a})^2=P(\bm{a})$です.

  \end{enumerate}

  \item 
  展開して,偶数と奇数の項に分割して考えると
  \begin{align}
    \exp\left( \theta J(\bm{a}) \right)
    &=
    \sum_{n=0}^{\infty}
    \frac{\theta^n}{n!}J^n
    \nonumber
    \\
    &=
    1
    +
    \sum_{k=1}^{\infty}
    \frac{\theta^{2k}}{(2k)!}
    (-J)^{2k}
    +
    \sum_{k=0}^{\infty}    
    \frac{\theta^{2k+1}}{(2k+1)!}
    (-J)^{2k+1}
    \nonumber
    \\
    &=
    1
    +
    (\cos\theta-1)P(\bm{a})
    +
    \sin\theta J(\bm{a})P(\bm{a})
  \end{align}
  です.よって,係数比較すればOK.

  \item 
  それぞれ計算すれば
  \begin{equation}
    \left\{
      \begin{alignedat}{1}
        \exp(\theta J(\bm{a}))\bm{a}
        &=
        \bm{a}
        \nonumber
        \\
        \exp(\theta J(\bm{a}))\bm{b}
        &=
        \bm{b}\cos\theta+\bm{c}\sin\theta
        \nonumber
        \\
        \exp(\theta J(\bm{a}))\bm{c}
        &=
        -\bm{b}\sin\theta+\bm{c}\cos\theta
      \end{alignedat}
    \right.
  \end{equation}
  となります.これはちょうど,$\bm{a}$を軸にした角度$\theta$の回転になっています.

  \item 
  この変換は「\uline{座標軸}を軸$\bm{a}$に対して$-\theta$だけ回転させたのち,軸$\bm{b}$について\uline{ベクトル}を$\varphi$だけ回転させる操作」に対応しています.したがって,回転軸は
  \begin{equation}
    \bm{e}
    =
    \bm{b}\cos\theta
    +
    \bm{c}\sin\theta
  \end{equation}
  であり,角度は$\chi=\varphi$です.

\end{enumerate}

\subsection*{補足}
\begin{itemize}
  \item 
  $\bm{e}=(e_1,e_2,e_3)$を軸にする角度$\chi$の回転は
  \begin{align}
    &\hspace*{-10pt}
    \exp(\chi J(\bm{e}))
    \nonumber
    \\
    &=
    \begin{pmatrix}
      \cos\chi
      +
      e_x^2(1-\cos\chi)
      &
      e_xe_y(1-\cos\chi)
      -
      e_z\sin\chi
      &
      e_ze_x(1-\cos\chi)
      +
      e_y\sin\chi
      \\
      e_xe_y(1-\cos\chi)
      -
      e_z\sin\chi
      &
      \cos\chi
      +
      e_y^2(1-\cos\chi)
      &
      e_ye_z(1-\cos\chi)
      -
      e_x\sin\chi
      \\
      e_ze_x(1-\cos\chi)
      -
      e_y\sin\chi
      &
      e_ye_z(1-\cos\chi)
      +
      e_x\sin\chi
      &
      \cos\chi
      +
      e_z^2(1-\cos\chi)
    \end{pmatrix}
    \label{rodrigues}
  \end{align}
  と書けます.今回の回転行列は
  \begin{align}
    &\hspace*{-10pt}
    \exp(\theta J(\bm{a}))\exp(\varphi J(\bm{b}))\exp(-\theta J(\bm{a}))
    \nonumber
    \\
    &=
    \begin{pmatrix}
      \cos\varphi
      &
      -\sin\theta\sin\varphi
      &
      \cos\theta\sin\varphi
      \\
      \sin\theta\sin\varphi
      &
      \cos^2\theta-\sin\theta\cos\theta\cos\varphi
      &
      \sin\theta\cos\theta-\cos^2\theta\cos\varphi
      \\
      -\cos\theta\sin\varphi
      &
      \sin\theta\cos\theta-\sin\theta\cos\theta\cos\varphi
      &
      \sin^2\theta
      +
      \cos^2\theta\cos\varphi
    \end{pmatrix}
  \end{align}
  なので,確かに\eqref{rodrigues}で$\bm{e}=(0,\cos\theta,\sin\theta), \chi=\varphi$とおけば一致しています.

\end{itemize}



\clearpage
\section{物理パート}
\prb{1}{量子力学}
\begin{enumerate}
  \item 
  微分は
  \begin{equation}
    \pdv[2]{}{x_1}
    =
    \frac{1}{4}
    \pdv[2]{}{x}
    +
    \frac{1}{2}
    \pdv[2]{}{x}{y}
    +
    \dv[2]{}{y}
    ,\ 
    \pdv[2]{}{x_2}
    =
    \frac{1}{4}
    \pdv[2]{}{x}
    -
    \frac{1}{2}
    \pdv[2]{}{x}{y}
    +
    \dv[2]{}{y}
  \end{equation}
  となるので,
  \begin{equation}
    H
    =
    -\frac{\hbar^2}{2m}
    \left(  
      \frac{1}{2}\pdv[2]{}{x}
      +
      2\pdv[2]{}{y}
    \right)
    +
    \frac{m\omega^2}{2}
    \left(  
      2x^2+\frac{y^2}{2}
    \right)
    +
    \frac{m\omega^2\alpha}{4}
    y^2
  \end{equation}
  です.

  \item 
  $y$についてのポテンシャルは
  \begin{equation}
    V_{y}(y)
    =
    \frac{m\omega^2(\alpha+1)}{2}y^2
  \end{equation}
  なので,これが下に凸であるためには
  \begin{equation}
    \alpha\geq-1
  \end{equation}
  が必要です.よって,$\alpha_0=-1$.

  \item 
  $\psi(x,y)=X(x)Y(y)$と変数分離すれば,定常状態のSchr\"{o}dinger方程式は
  \begin{equation}
    \left\{
      \begin{alignedat}{1}
        -
        \frac{\hbar^2}{4m}
        \pdv[2]{X}{x}
        +
        m\omega^2 x^2 X(x)
        &=
        E_x X(x)
        \\
        -
        \frac{\hbar^2}{m}
        \pdv[2]{Y}{y}
        +
        \frac{m\omega^2(1+\alpha)}{4}y^2 Y(y)
        &=
        E_y Y(y)
      \end{alignedat}
    \right.
  \end{equation}
  となります.ただし,$E=E_x+E_y$です.基底状態の波動関数の形が分かっているので,
  \begin{equation}
    X(x)
    =
    \left( \frac{c_x}{\pi} \right)^{1/4}
    e^{-(c_x/2)x^2}
    ,\ 
    Y(y)
    =
    \left( \frac{c_y}{\pi} \right)^{1/4}
    e^{-(c_y/2)y^2}
  \end{equation}
  を代入していきます.$X$については
  \begin{equation}
    \left(  
      -\frac{\hbar^2 c^2}{4m}
      +
      m\omega^2
    \right)
    x^2
    =
    E_x
    -
    \frac{\hbar^2 c_x}{4m}
  \end{equation}
  となるので,この式が任意の$x$で成立するためには$c_x=2m\omega/\hbar$であることが必要なので
  \begin{equation}
    E_x
    =
    \frac{\hbar\omega}{2}
    ,\ 
    X(x)
    =
    \left( \frac{2m\omega}{\pi\hbar} \right)^{1/4}
    \exp\left[ -\frac{m\omega}{\hbar}x^2 \right]
  \end{equation}
  となります.$Y$についても同様で
  \begin{equation}
    E_y
    =
    \sqrt{1+\alpha}\frac{\hbar\omega}{2}
    ,\ 
    Y(y)
    =
    \left( \frac{m\omega\sqrt{1+\alpha}}{2\pi\hbar} \right)^{1/4}
    \exp\left[ -\frac{m\omega\sqrt{1+\alpha}}{4\hbar}y^2 \right]
  \end{equation}
  です.

  \item 
  $t=0$での波動関数は
  \begin{equation}
    \Psi
    (x,y;t=0)
    =
    \sqrt{\frac{m\omega}{\pi\hbar}}
    e^{-m\omega x^2/\hbar}
    e^{-m\omega y^2/4\hbar}
  \end{equation}
  です.したがって,
  \begin{equation}
    f(x,k)
    =
    \int_{-\infty}^{+\infty}
    \Psi
    (x,y;t=0)
    e^{-iky}
    \dd y
    =
    2\exp\left[  
      -\frac{m\omega}{\hbar}x^2
      -
      \frac{4\hbar k^2}{m\omega}
    \right]
  \end{equation}
  です.

  \item 
  時間発展を考えればよいので
  \begin{equation}
    \Psi(x,y;t)
    =
    e^{-iHt/\hbar}\Psi(x,y;t=0)
    =
    e^{-im\omega^2x^2t/\hbar}
    \int_{-\infty}^{\infty}
    \frac{\dd k}{2\pi}
    \exp\left[ \frac{i\hbar t}{4m}\pdv[2]{}{x} \right]f(x,k)
    \exp\left[ \frac{i\hbar t}{m}\pdv[2]{}{y} \right]e^{iky}
  \end{equation}
  であり,それぞれ$\exp$を展開して計算すると
  \begin{align}
    \exp\left[ \frac{i\hbar t}{4m}\pdv[2]{}{x} \right]f(x,k)
    &=
    2\exp\left[ -\frac{4\hbar k^2}{m\omega}-\frac{i\omega t}{2}\left( 1-\frac{2m\omega x^2}{\hbar} \right) \right]e^{-m\omega x^2/\hbar}
    ,
    \\  
    \exp\left[ \frac{i\hbar t}{m}\pdv[2]{}{y} \right]e^{iky}
    &=
    \exp\left[ -\frac{i\hbar k^2t}{m} \right]e^{iky}
  \end{align}
  となるので\footnote{
    $\exp$のなかにoperatorがあるときは,そこが固有値におきかわります.
  },
  \begin{align}
    &\hspace*{20pt}
    \int_{-\infty}^{\infty}
    \frac{\dd k}{2\pi}
    \exp\left[ \frac{i\hbar t}{4m}\pdv[2]{}{x} \right]f(x,k)
    \exp\left[ \frac{i\hbar t}{m}\pdv[2]{}{y} \right]e^{iky}
    \nonumber
    \\
    &=
    2\exp
    \left[  
      -\frac{i\omega t}{2}
      \left(  
        1
        -
        \frac{2m\omega}{\hbar}x^2
      \right)
      -
      \frac{m\omega}{\hbar}x^2
      -
      \frac{m\omega y^2}{4\hbar(4+i\omega t)}
    \right]
    \nonumber
    \\
    &\hspace*{3cm}
    \times
    \int_{-\infty}^{\infty}
    \frac{\dd k}{2\pi}
    \exp
    \left[  
      -
      \frac{\hbar(4+i\omega t)}{m\omega}
      \left(  
        k-\frac{im\omega y}{2\hbar(4+i\omega t)}
      \right)^2
    \right]
    \nonumber
    \\
    &=
    2
    \sqrt{\frac{\pi m\omega}{\hbar(1+i\omega t)}}
    \exp
    \left[  
      -\frac{i\omega t}{2}
      \left(  
        1
        -
        \frac{2m\omega}{\hbar}x^2
      \right)
      -
      \frac{m\omega}{\hbar}x^2
      -
      \frac{m\omega y^2}{4\hbar(4+i\omega t)}
    \right]
  \end{align}
  です.したがって,波動関数は
  \begin{equation}
    \Psi(x,y,t)
    =
    2
    \sqrt{\frac{\pi m\omega}{\hbar(1+i\omega t)}}
    \exp
    \left[  
      -\frac{i\omega t}{2}
      -
      \frac{m\omega}{\hbar}x^2
      -
      \frac{m\omega y^2}{4\hbar(4+i\omega t)}
    \right]
  \end{equation}
  となります.

\end{enumerate}



\clearpage
\prb{2}{統計力学}
\begin{enumerate}
  \item 
  分配関数は
  \begin{equation}
    Z
    =
    \sum_{\sigma=\pm}
    e^{-\beta\mathcal{H}_1}
    =
    2\cosh\left( \frac{\mu H}{k_BT} \right)
  \end{equation}
  なので
  \begin{equation}
    p_+
    =
    \dfrac{e^{\mu H/k_BT}}{2\cosh(\mu H/k_BT)}
    ,\ 
    p_-
    =
    \dfrac{e^{-\mu H/k_BT}}{2\cosh(\mu H/k_BT)}
  \end{equation}
  です.

  \item 
  \begin{equation}
    \ev*{\sigma}
    =
    (+1)\cdot\dfrac{e^{\mu H/k_BT}}{2\cosh(\mu H/k_BT)}
    +
    (-1)\cdot\dfrac{e^{-\mu H/k_BT}}{2\cosh(\mu H/k_BT)}
    =
    \tanh\left( \frac{\mu H}{k_BT} \right)
    .
  \end{equation}

  \item 
  格子A上のある粒子に着目すると,そのスピンの平均値は
  \begin{equation}
    \ev*{\sigma_A}
    =
    \tanh\left( \frac{\mu H_{\mathrm{eff,A}}}{k_B T} \right)
  \end{equation}
  と表すことができます.よって,
  \begin{equation}
    \ev*{\sigma_A}
    =
    \tanh\left(  
      -\frac{4J}{k_B T}\sigma_B
      +
      \frac{\mu H}{k_B T}
    \right)
  \end{equation}
  です.格子Bについても同様.したがって,
  \begin{equation}
    f(x)
    =
    \tanh\left(  
      -\frac{4J}{k_B T}x
      +
      \frac{\mu H}{k_B T}
    \right)
    .
  \end{equation}
  ただし,$z=4$としました.

  \item 
  $H=0$のとき,self-consistentな方程式は
  \begin{equation}
    x
    =
    \tanh\left( \frac{4J}{k_BT}x \right)
  \end{equation}
  です.右辺の$\tanh$の原点での傾きが1より小さければ非自明な解が存在するので,
  \begin{equation}    
    \left.
    \dv{}{x}
    \left[  
      \tanh\left( \frac{4J}{k_BT}x \right)
    \right]
    \right|_{x=0}
    =
    \frac{4J}{k_B T}\leq 1
  \end{equation}
  より,
  \begin{equation}
    T_N
    =
    \frac{4J}{k_B}
  \end{equation}
  です.

  \item 
  計算すれば,
  \begin{equation}
    \chi
    =
    \frac{\mu^2}{k_BT}
    \left(  
      \dfrac{1}{\cosh^2\left( \frac{4J}{k_BT}\ev*{\sigma_A} \right)}
      +
      \dfrac{1}{\cosh^2\left( \frac{4J}{k_BT}\ev*{\sigma_B} \right)}
    \right)
    .
  \end{equation}

  \item 
  平均場近似は
  \begin{equation}
    \mu H_{\mathrm{eff,A}}
    =
    -4J\ev*{\sigma_B}
    -4J'\ev*{\sigma_A}
  \end{equation}
  でよいでしょう.よって,格子A上の粒子のスピンの平均値は
  \begin{equation}
    \ev*{\sigma_A}
    =
    \tanh
    \left(  
      -\frac{4}{k_BT}
      \left( J\ev*{\sigma_B}+J'\ev*{\sigma_A} \right)
    \right)
  \end{equation}
  なので,self-consistent方程式は
  \begin{equation}
    x
    =
    \tanh\left( \frac{4(J-J')}{k_BT}x \right)
  \end{equation}
  となり,同様の計算で転移温度
  \begin{equation}
    T_c
    =
    \frac{J-J'}{k_B}
  \end{equation}
  を得ます.また,$J'\rightarrow J$で$T_c\rightarrow 0$となっています.

\end{enumerate}


\clearpage
\prb{3}{力学}
\begin{enumerate}
  \item 
  運動エネルギーを$K$,ポテンシャルを$U$とすれば,$L=K-U$なので,
  \begin{equation}
    L
    =
    \frac{1}{2}m_1^2(\dot{x}_1^2+\dot{y}_1^2)
    +
    \frac{1}{2}m_2^2(\dot{x}_2^2+\dot{y}_2^2)
    -
    mg(l_1-y_1)
    -
    mg(l_2-y_2)
  \end{equation}
  です.

  \item 
  座標は
  \begin{equation}
    x_1=l_1\sin\theta_1
    ,\ 
    y_1=l_1\cos\theta_1
    ,\ 
    x_2=l_1\sin\theta_1+l_2\sin\theta_2
    ,\ 
    y_2=l_1\cos\theta_1+l_2\cos\theta_2
  \end{equation}
  です.

  \item 
  近似すると
  \begin{align}
    &
    x_1=l_1\theta_1
    ,\ 
    y_1=l_1
    ,\ 
    x_2=l_1\theta_1+l_2\theta_2
    ,\ 
    y_2=l_1+l_2
    ,\ 
    \\
    &
    \dot{x}_1=l_1\dot{\theta_1}
    ,\ 
    \dot{y}_1=-l_1\dot{\theta}_1\theta_1
    ,\ 
    \dot{x}_2=l_1\dot{\theta}_1+l_2\dot{\theta}_2
    ,\ 
    \dot{y}_2=-l_1\dot{\theta}_1\theta_1-l_2\dot{\theta}_2\theta_2
  \end{align}
  なので,定数と$\theta$の3次以上の項を除けば
  \begin{align}
    L
    &=
    \frac{1}{2}m_1l_1^2\dot{\theta}_1^2
    +
    \frac{1}{2}m_2
    \left( 
      l_1\dot{\theta}_1^2 
      +
      l_2\dot{\theta}_2^2
    \right)
    -
    m_1gl_1\cdot\frac{1}{2}\theta_1^2
    -
    m_2g\left( l_2-l_1\left( 1-\frac{1}{2}\theta_1^2 \right)-l_2\left( 1-\frac{1}{2}l_2\theta_2^2 \right) \right)
    \nonumber
    \\
    &=
    \frac{1}{2}(m_1+m_2)l_1^2\dot{\theta}_1^2
    +
    m_2l_1l_2\dot{\theta}_1\dot{\theta}_2
    +
    \frac{1}{2}m_2l_2^2\dot{\theta}_2^2
    -
    \frac{1}{2}(m_1+m_2)gl_1\theta_1^2
    -
    \frac{1}{2}m_2gl_2\theta_2^2
  \end{align}
  です.

  \item 
  Euler-Lagrange
  \begin{equation}
    \dv{}{t}
    \pdv{L}{\dot{\theta}_i}
    -
    \pdv{L}{\theta_i}
    =
    0
  \end{equation}
  から,
  \begin{gather}
    (m_1+m_2)l_1^2\ddot{\theta}_1
    +
    m_2l_1l_2\ddot{\theta}_2
    +
    (m_1+m_2)gl_1\theta_1
    =
    0
    \\
    m_2l_2^2\ddot{\theta}_2
    +
    m_2l_1l_2\ddot{\theta}_1
    +
    m_2gl_2\theta_2
    =
    0
  \end{gather}
  です.

  \item 
  代入してみると
  \begin{equation}
    \left\{
      \begin{alignedat}{1}
        -
        a_1\omega^2
        (m_1+m_2)
        l_1^2
        -
        a_2
        \omega^2
        m_2l_1l_2
        +
        a_1(m_1+m_2)gl_1
        &=
        0
        \\
        -a_2\omega^2
        m_2l_2^2
        -
        a_1\omega^2
        m_2l_1l_2
        +
        a_2\omega^2m_2gl_2
        &=
        0
      \end{alignedat}
    \right.
  \end{equation}
  となるので,$a\coloneqq a_2/a_1$とおいて$a$を消去すれば
  \begin{equation}
    \omega^4
    -
    g
    \left(  
      \frac{1}{l_1}+\frac{1}{l_2}
    \right)
    \left(  
      1+\frac{m_2}{m_1}
    \right)
    \omega^2
    +
    \frac{g^2}{l_1l_2}
    \left(  
      1+\frac{m_2}{m_1}
    \right)
    =
    0
  \end{equation}
  なので,これを解けば
  \begin{equation}
    \omega
    =
    \sqrt{
      \frac{1}{2}
      \left\{  
        g
        \left(  
          \frac{1}{l_1}+\frac{1}{l_2}
        \right)
        \left(  
          1+\frac{m_2}{m_1}
        \right)
        \pm
        \sqrt{
          g^2
          \left(  
            \frac{1}{l_1}+\frac{1}{l_2}
          \right)^2
          \left( 1+\frac{m_2}{m_1} \right)^2
          -
          \frac{4g^2}{l_1l_2}
          \left( 1+\frac{m_2}{m_1} \right)
        }
      \right\}      
    }
  \end{equation}
  です.(複号任意)

  \item 
  $m_2/m_1\ll 1$なので,
  \begin{equation}
    \omega
    =
    \sqrt{
      \frac{g}{2l_1l_2}
      \left\{  
        l_1+l_2
        \pm
        |l_1-l_2|
      \right\}
    }
    \label{eigen_ang}
  \end{equation}
  となります.$|l_1-l_2|\gg l_1+l_2$のときは,
  \begin{equation}
    \omega
    =
    \sqrt{
      \frac{g}{2}
      \left|  
        \frac{1}{l_1}-\frac{1}{l_2}
      \right|
    }
  \end{equation}
  です.よって,
  \begin{equation}
    \frac{a_2}{a_1}
    =
    \frac{m_1+m_2}{m_2}
    \times
    \frac{g-\omega^2l_1}{\omega^2l_2}
    =
    \frac{m_1+m_2}{m_2}
    \times
    \frac{l_1(2l_2-|l_1-l_2|)}{l_2|l_1-l_2|}
    \label{ratio}
  \end{equation}
  となるので,例えば
  \begin{equation}
    a_1=m_2l_2|l_1-l_2|
    ,\ 
    a_2=(m_1+m_2)l_1(2l_2-|l_1-l_2|)
  \end{equation}
  です\footnote{
    \eqref{ratio}の比が保たれていれば,なんでもよいでしょう.
  }.

  \item 
  \eqref{eigen_ang}は常に成立しています.つまり
  \begin{equation}
    \omega^2
    =
    \frac{g(l\pm|l-2l_2|)}{2(l-l_2)l_2}
    =
    \frac{g}{l_2}
    \text{\ or\ }
    \frac{g}{l-l_2}
  \end{equation}
  です.第1式は自明なので,第2式を考えることにすると
  \begin{equation}
    \frac{g}{l\omega^2}
    =
    1-\frac{l_2}{l}
  \end{equation}
  です.図は省略.$l_2\gg l_1$のとき,質点1が止まって質点2が振動するのは,$\omega\rightarrow\infty$のとき,
  \begin{equation}
    \frac{a_2}{a_1}
    =
    \frac{1+m_2/m_1}{m_2/m_1}
    \times
    \frac{g/\omega^2-l_1}{l_2}
    \rightarrow
    \frac{1+\mathcal{O}(m_2/m_1)}{\mathcal{O}(m_2/m_1)}
  \end{equation}
  となるからです.

\end{enumerate}






\end{document}
