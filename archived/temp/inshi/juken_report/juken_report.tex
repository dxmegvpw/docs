\pdfoutput=1
\documentclass[a4paper,pdflatex,ja=standard]{bxjsarticle}

% ---Display \subsubsection at the Index
% \setcounter{tocdepth}{3}

% ---Setting about the geometry of the document----
% \usepackage{a4wide}
% \pagestyle{empty}

% ---Physics and Math Packages---
\usepackage{amssymb,amsfonts,amsthm,mathtools}
\usepackage{physics,braket,bm}

% ---underline---
\usepackage{ulem}

% ---cancel---
% \usepackage{cancel}

% --- surround the texts or equations
\usepackage{fancybox,ascmac}

% ---settings of theorem environment---
% \usepackage{amsthm}
% \theoremstyle{definition}

% ---settings of proof environment---
% \renewcommand{\proofname}{証明}
% \renewcommand{\qedsymbol}{\empty}

% ---Ignore the Warnings---
\usepackage{silence}
\WarningFilter{latexfont}{Some font shapes,Font shape}

% ---Insert the figure (If insert the `draft' at the option, the process becomes faster.)---
\usepackage{graphicx}
% \usepackage{subcaption}

% ----Add a link to a text---
\usepackage{url}
\usepackage{xcolor,hyperref}
\hypersetup{colorlinks=true,citecolor=orange,linkcolor=blue,urlcolor=magenta}
\usepackage[whole,autotilde]{bxcjkjatype}

% ---Tikz---
% \usepackage{tikz,pgf,pgfplots,circuitikz}
% \pgfplotsset{compat=1.15}
% \usetikzlibrary{intersections,arrows.meta,angles,calc,3d,decorations.pathmorphing}

% ---Add the section number to the equation, figure, and table number---
\makeatletter
   \renewcommand{\theequation}{\thesection.\arabic{equation}}
   \@addtoreset{equation}{section}
   
   \renewcommand{\thefigure}{\thesection.\arabic{figure}}
   \@addtoreset{figure}{section}
   
   \renewcommand{\thetable}{\thesection.\arabic{table}}
   \@addtoreset{table}{section}
\makeatother

% ---enumerate---
% \renewcommand{\labelenumi}{$(\arabic{enumi})$}
% \renewcommand{\labelenumii}{$(\arabic{enumii})$}

% ---Index---
% \usepackage{makeidx}
% \makeindex 

% ---Fonts---
\renewcommand{\familydefault}{\sfdefault}

% ---Title---
\title{受験記}
\author{ミヤネ}
\date{最終更新:\today}

\begin{document}

\maketitle

*この文書を共有するのはいいですが,親しい友人程度にとどめてください(理由は脚注\ref{caution}).

\tableofcontents
\clearpage

\setcounter{section}{-1}

\section{はじめに}

これは,令和6年度入学の早稲田大学と東京大学の大学院試験の受験の様子を記録したメモをTeXで書き起こしたものになります.大体のものは試験の当日 or 翌日に記述するようにしています.そのため,勢いに任せて書いているのでかなり読みづらいかもしれませんが,許してください\footnote{
  その分,臨場感があるかと思います(笑)
}.特に面接の内容はあまり出回らないと思うので,参考にしていただければと思います\footnote{
  特に,東京大学は面接の内容を漏らしてはいけないらしいと聞いていたので,残してみたかったです.
  \label{caution}
}.


\section{受験記}

\subsection{早稲田大学}

AB研究室を受験しました.特に,筆記試験については特に言うことはないと思います.順調に受けました\footnote{
  計算用紙がないのがちょっと大変だったかもしれません.
}.

面接は,みんな言っていると思いますが,ちょっと怖かったです.面接官はAB先生とNZ先生,あと原子核のTK先生でした.基本的に進行はTK先生でした.AB先生やNZ先生は,その場で質問したり,茶々を入れる感じでした.以下では面接がどんなだったかについて,覚えているかぎりのことを書いていこうと思います.みなさんの参考になればと思います\footnote{
  ちなみに,面接の当日に書いてますし,あまりあとから編集をしないようにします.
}.
    
試験の内容ですが,まず前半は(というよりもほとんどこれでしたが)試験の解きなおしでした.「試験の感触はどうでしたか?」と言われたので,正直に
\begin{itemize}
  \item 
  線形代数は多分OK.

  \item 
  力学は角運動量のほうの運動方程式の係数を少し間違えた(Rをかけてなかった)ので,解き方自体はあっていると思うが,(5)以降の答えが少し間違っていると思う.

  \item 
  電磁気は(1)$\sim$(4)まではOK.それ以降は問題の意味がわからなかった.が,解きなおしたら多分分かった.(後で述べますが,実は(4)以外は全部間違ってました.)
  
  \item 
  量子力学は最後の最後でちょっとミスった.けどあとはOK.
\end{itemize}
と伝えました.(正直に書いているので,私の学力の参考にしてください)

上の内容を伝えたので,力学とか電磁気を解かされると思いましたが,実際に解かされたのは量子力学だったので,もしかしたら解かされる問題はある程度決まっているのかもしれません\footnote{
  NZ研の同僚もそうらしかったです.
}.

そんなこんなで,量子力学の最後の設問(8)を先生の前で解きました.ちゃんと解きなおしはしてあったので,ここらへんはスムーズでした.記憶にあるやりとりをリストアップしておくと
\begin{itemize}
  \item 
  「『最後ミスった』って言っていましたが,どこだったんですか?」TK先生より
  
  \item 
  「どうやら今回はハイゼンベルグ方程式から解いたみたいですが,(5)を使っても解けるはずですよね」AB先生より.確かにその通りです.
  
  \item 
  「ちなみに,今回は$\hat{O}$が消滅演算子の役割$(\hat{O}\ket{0}=0)$を課していましたが,もし,$O^{\dag}$のほうを真空にかけて消えるとしたらどうなりますか?試しに聞いてみるだけですが.」NZ先生より.これについてはその場で考えようとしたのですが,そうしたところ「知ってると思ったので聞いただけです.知らないならいいですよ.」と言われました.ちょっとショックでした.
\end{itemize}    
といった感じです.研究室のゼミと同じ感じだったので,それを想像しておけばよいかと思います.(ただし,計算ミスをしても,先生はアシストしてくれないので,ミスの箇所を探しているときの沈黙が怖かったです.その点は研究室のゼミとは違いました.)

量子力学の話が終わって,今度は力学と電磁気の話に移りました.まず,電磁気について出来をもう一度きかれて,上と同じように答えたところ,TK先生が「いや~,君の認識と点数が一致してないんだけどな~.(1)とかはどういう感じに解いたんですか?口頭でいいので.」と聞かれました.私は正直に「電荷を帯びているとして,単位長さの円筒領域でガウスの法則を使って電場をだして,積分すれば電位がもとまるので,あとは電荷と電位の比を計算しただけです.」と答えたところ「ふ~ん,そうですか.」といった反応だったので,「え,こうやるんじゃないんですか?」と思わず聞いてしまいました.それに対して「いや,私たちの聞きたいことはもう聞けたので,これ以上は言えません.」とあっさり返されたのですが,結局,自分の答えがあってるのかあってないのかがわからなくて怖かったです\footnote{試験なのでしかたないですが.ちなみに,面接が終わってから「単位長さ当たりの円筒」を考えていたのがまずかったことに気がつきました.それを考えるのは無限の長さの円筒の場合です.だから,たしかに私の答案は(4)以外はダメだったんだと思います.
}.

力学については,何回かやりとりがあったのですが,すぐ終わったので詳しく覚えてないです.たしか,NZ先生に「ここらへんの話,授業でやんなかったんですか?」と聞かれて「力学Bでやりました」と答えたところ,「ここらへんは得意なんじゃないんですか?」と聞かれたような気がします.脈絡がなさ過ぎてあまり覚えてません.たしか「正直,定義は覚えていましたが具体的な計算は慣れてません」と答えてところ,「典型的な問題だから,これぐらいはできてくれないとね~」みたいなことを言われて,また背筋が凍ったような記憶はあります\footnote{
  後になって知ったのですが,どうやら先生たちの手元には私の成績表があったようなので,そこに書いてある力学Bの私の成績を見ていったのかもしれません.
}.

だいたいこんなやり取りをした後,着席をして志望理由や院生生活の抱負について質問されました.この後半のやり取りについては,特に言うことはないかと思います.質問は
\begin{itemize}
  \item 
  志望理由(TK先生)
  
  \item 
  「やってみたい研究はあるんですか?」(NZ先生)

  \item 
  「研究と勉強はまったく違うということが分かってきていると思うのですが,大学院で研究することについての抱負をお願いします.」(AB先生)
\end{itemize}
といった普通の内容でしたし,私自身も当たり障りのない返答をしました.後半については,NZ研を受けた友達もおんなじ感じだったらしいので,なんか決まっているんでしょう.

早稲田の面接で総じて思ったことは「先生たちがめちゃくちゃ不安をあおってくる」ということでしょうか.「これぐらいできなきゃね」みたいな感じの雰囲気になるのは院試の定番なんだと思いますし,実際そんな感じに思われているのかもしれませんが,それでも受験する側の学生にとっては心臓に悪かったです.もう,こればかりは気にしないようにするしかないんじゃないかと思います.
\begin{flushright}
  (2023 7/16)  
\end{flushright}

そういえば,受かってました.私よりもしごかれていた同僚も受かっていたので,AB研やNZ研はかなり受かりやすいかもしれません\footnote{
  ちなみに,後から研究室の先輩に聞いたのですが,(数年前の)外部からの受験者は容赦なく落としていたそうです.
}.

私の結果は,まあそれでよかったわけですが,情報系のSWD研を受けていた友人が何故か落ちていました.前々から「僕,先生に嫌われてるかもしれない」と聞いていたので,たぶんそれなんじゃないかなと思っています\footnote{
  あくまで私の感想ですが,その人は普通に物理はできていたと思います.
}.4,5月くらいに外部受験をすると伝えたところ,先生の態度が変わったとのことらしいです.研究室を選ぶときはそういった情報とかもちゃんと集めておくと良いかもしれません.
\begin{flushright}
  (2023 7/31)
\end{flushright}

\subsection*{追記}

\begin{itemize}
  \item 
  宇宙系のI研も3人中1人だけが合格したそうです(受かったのは私の友達でした.おめでとう!).こちらに関しては,落ちた人達は普通に実力不足だと思いますが.I先生が言うには「うちの研究室かなりムズイで~」らしいです.

  \item
  AB・NZ研の合宿のときに「今年,AB先生は問題を担当したりしました?」と訊いたところ「もちろん言えませんが,まあ,物理の問題作る人なんて限られますよね(笑)」と返ってきました.これ,絶対何か関わっていますね.量子力学とかですかね(笑)
\end{itemize}


\subsection{東京大学}

筆記試験は順調に終わりました.出来を書いておくと
\begin{itemize}
  \item 
  量子力学

  一応,すべての設問に解答しました.内容が全然知らないものだったので,答えが合ってるかは確証を持てませんが,「***となることを示せ」のタイプの問題で,ちゃんとその結果が出てきていたのでそんなに大きくは転んでいないはず.最後の説明の設問はそれっぽいことを書いたと思うのですが,まあ,それが間違っていることと,途中の議論で少し減点されると思って\uline{80/100}としておきましょう.

  \item
  統計力学

  前半の設問1から設問5は知ってた内容だったので答えもあってると思いますが,逆に後半は分かんなかったです.完全に盲点でした\footnote{
    ちゃんと勉強しておけばよかった...
  }.得点の傾斜も考えて\uline{40/100}としておきます.

  \item 
  電磁気

  これは一応全部解答しました.最後の説明問題は自信ないですが,その分を考慮して\uline{80/100}としておきましょう.ただ,電磁気もあまり見たことないような計算処理だったので,答えには確証がありません.

  \item
  数学

  微積分の最後の設問は答えを書いたけど自信なし\footnote{
    正確に言うと,自分で仮定を入れちゃいました.それが合ってたらいいのですが.
  }.行列の問題は最後がわかりませんでした.これも\uline{80/100}でいいんじゃないでしょうか.
\end{itemize}
といった感じです.ある程度低く見積もっても\uline{280/400}で7割はいきそうです.もちろん,もっとできた受験生もいると思いますし,素粒子理論なのでボーダーが読めないのはありますが,ひとまず自分の目標の点数は超えることができたはずです.(これで無理だったら,素粒子分野ってやっぱり手強いなと思います.)

ただ,数学と統計力学はもう少し戦えました.特に統計力学の吸着の内容は完全に見落としていました.それがちょっと心残りです.できた人がいたらお見事だと思います.

ひとまず,後は結果を待つだけです.8/31まで生きた心地がしません...(ところで,問題用紙も回収されちゃったので,解きなおしもできません.このことを知らなかったので焦ってます.)
\begin{flushright}
  (2023 8/23)  
\end{flushright}




\section{その他}

大学院受験に役立ちそうなことや,(もしあったら)書いておきたいことなどを残しておこうと思います.

\subsection{研究室見学}




\subsection{内部生の有志の解答例}

研究室見学の際に貰えたものがあるので,もし欲しい方がいたら私のメールアドレス\footnote{
  \href{mailto:itsuki.miyane@gmail.com}{itsuki.miyane@gmail.com}
}
に,氏名・所属している学部学科・趣旨を本文に書いて送ってください\footnote{
  私の知り合いの方であっても,メールを送っていただければと思います.
}.件名等はあってもなくても大丈夫ですが,迷惑メールだとみなされてブロックされない程度にはフォーマルにお願いします.私の苗字は「宮根」です.返信でそのファイルを送付しようと思います.

\uline{また「取り扱いには気をつけて」と言われて渡されたので,受け取った方も変なことをしないでください.}     
      







\end{document}
