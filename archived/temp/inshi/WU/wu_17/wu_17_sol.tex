\pdfoutput=1
\documentclass[a4paper,pdflatex,ja=standard]{bxjsarticle}

% ---Setting about the geometry of the document----
% \usepackage{a4wide}
% \pagestyle{empty}

% ---Physics and Math Packages---
\usepackage{amssymb,amsfonts,amsthm,mathtools}
\usepackage{physics,braket,bm}

% ---underline---
\usepackage{ulem}

% --- sorround the texts or equations
% \usepackage{fancybox,ascmac}

% ---settings of theorem environment---
% \usepackage{amsthm}
% \theoremstyle{definition}

% ---settings of proof environment---
% \renewcommand{\proofname}{\textbf{証明}}
% \renewcommand{\qedsymbol}{$\blacksquare$}

% ---Ignore the Warnings---
\usepackage{silence}
\WarningFilter{latexfont}{Some font shapes,Font shape}

% ---Insert the figure (If insert the `draft' at the option, the process becomes faster)---
\usepackage{graphicx}
% \usepackage{subcaption}

% ----Add a link to a text---
\usepackage{url}
\usepackage{xcolor,hyperref}
\hypersetup{colorlinks=true,citecolor=orange,linkcolor=blue,urlcolor=magenta}
\usepackage{bxcjkjatype}

% ---Tikz---
\usepackage{tikz,pgf,pgfplots,circuitikz}
\pgfplotsset{compat=1.15}
\usetikzlibrary{intersections,arrows.meta,angles,calc,3d,decorations.pathmorphing}

% ---Add the section number to the equation, figure, and table number---
\makeatletter
   \renewcommand{\theequation}{\thesection.\arabic{equation}}
   \@addtoreset{equation}{section}
   
   \renewcommand{\thefigure}{\thesection.\arabic{figure}}
   \@addtoreset{figure}{section}
   
   \renewcommand{\thetable}{\thesection.\arabic{table}}
   \@addtoreset{table}{section}
\makeatother

% ---enumerate---
\renewcommand{\labelenumi}{$(\arabic{enumi})$}
% \renewcommand{\labelenumii}{$(\arabic{enumii})$}

% ---Index---
% \usepackage{makeidx}
% \makeindex 

% ---Fonts---
\renewcommand{\familydefault}{\sfdefault}

% ---Title---
\title{早稲田大学\ 2017年\ 物理学専攻\ 院試\ 解答例}
\author{ミヤネ}
\date{最終更新:\today}

\newcommand{\prb}[2]{
  \clearpage
  \phantomsection
  \addcontentsline{toc}{section}{問題 #1: #2}
  \section*{問題番号\fbox{#1}\ (#2)}
  \setcounter{section}{#1}
  \setcounter{equation}{0}
}

\begin{document}

\maketitle

\tableofcontents
\clearpage

\prb{1}{線形代数}

\begin{enumerate}
  \item 
  固有行列をとけば,
  \begin{equation}
    \lambda_\pm
    =
    \pm i
    .
  \end{equation}

  \item 
  対応する規格化された固有ベクトルは
  \begin{equation}
    \bm{u}_+
    =
    \frac{1}{\sqrt{2}}
    \begin{pmatrix}
      1 \\
      -i
    \end{pmatrix}
    ,\ 
    \bm{u}_-
    =
    \frac{1}{\sqrt{2}}
    \begin{pmatrix}
      1 \\
      i
    \end{pmatrix}
  \end{equation}
  です.したがって,
  \begin{equation}
    U
    =
    \left( \bm{u}_+\ \bm{u}_- \right)
    =
    \frac{1}{\sqrt{2}}
    \begin{pmatrix}
      1 & 1 \\
      -i & i
    \end{pmatrix}
  \end{equation}
  とすれば,対角化できます.

  \item 
  計算すれば
  \begin{equation}
    A^2
    =
    -
    \begin{pmatrix}
      1 & 0 \\
      0 & 1
    \end{pmatrix}
    =
    -1
  \end{equation}
  です.ついでに,もうちょっと計算しておくと
  \begin{equation}
    A^3
    =
    -A
    ,\ 
    A^{4}
    =
    1
    ,\ 
    A^5
    =
    A
    ,\ \cdots
  \end{equation}
  となっています.

  \item 
  展開して4で割った余りで分類します.すると
  \begin{align}
    e^{\phi A}
    &=
    \sum_{n=0}^{\infty}
    \frac{1}{(4n)!}\phi^{4n}
    +
    \sum_{n=0}^{\infty}
    \frac{1}{(4n+1)!}\phi^{4n+1}A
    -
    \sum_{n=0}^{\infty}
    \frac{1}{(4n+2)!}\phi^{4n+2}
    -
    \sum_{n=0}^{\infty}
    \frac{1}{(4n+3)!}\phi^{4n+3}A
    \nonumber
    \\
    &=
    \sum_{n=0}^{\infty}
    \frac{(-1)^n}{(2n)!}\phi^{2n}
    +
    A
    \sum_{n=0}^{\infty}
    \frac{(-1)^n}{(2n+1)!}\phi^{2n+1}
    \nonumber
    \\
    &=
    \begin{pmatrix}
      \cos\phi & -\sin\phi \\
      \sin\phi & \cos\phi
    \end{pmatrix}
  \end{align}
  です.よって,$A$は$\text{SO}(2)$のgenerator.

  \item 
  次の行列
  \begin{equation}
    S
    =
    \begin{pmatrix}
      0 & 1 & 0 \\
      -1 & 0 & 0 \\
      0 & 0 & 0
    \end{pmatrix}
    ,\ 
    T
    =
    \begin{pmatrix}
      1/2 & 0 & 0 \\
      0 & 1/2 & 0 \\
      0 & 0 & 1
    \end{pmatrix}
  \end{equation}
  を用いれば
  \begin{equation}
    M
    =
    \Omega S
    +
    \gamma T
  \end{equation}
  と分解できます.ここで,$S$と$T$が可換なので,
  \begin{equation}
    e^{-Mt}
    =
    e^{-\Omega t S}e^{-\phi t T}
  \end{equation}
  が成立しています.したがって,それぞれの因子を計算すればよいです.$S$と$T$については
  \begin{equation}
    \left\{
      \begin{alignedat}{1}
        &
        S^2
        =
        -
        \begin{pmatrix}
          1 & 0 \\
          0 & 0  
        \end{pmatrix}
        ,\ 
        S^3
        =
        -S
        ,\ 
        S^4
        =
        \begin{pmatrix}
          1 & 0 \\
          0 & 0  
        \end{pmatrix}
        ,\ 
        S^5
        =
        S
        ,\ \cdots
        \\
        &
        T^n
        =
        \begin{pmatrix}
          1/2^n & 0 & 0 \\
          0 & 1/2^n & 0 \\
          0 & 0 & 1
        \end{pmatrix}
      \end{alignedat}
    \right.
  \end{equation}
  が成立しているので,
  \begin{equation}
    \left\{
      \begin{alignedat}{1}
        &
        e^{-\Omega t S}
        =
        \begin{pmatrix}
          \cos\Omega t & \sin\Omega t & 0 \\
          -\sin\Omega t & \cos\Omega t & 0 \\
          0 & 0 & 1
        \end{pmatrix}
        \\
        &
        e^{-\phi tT}
        =
        \begin{pmatrix}
          e^{-\gamma t/2} & 0 & 0 \\
          0 & e^{-\gamma t/2} & 0 \\
          0 & 0 & e^{-\gamma t}
        \end{pmatrix}
      \end{alignedat}
    \right.
  \end{equation}
  です\footnote{
    $e^{-\Omega tS}$の$(3,3)$成分が1であることに注意.よくあるミスなので.
  }.よって,
  \begin{equation}
    e^{-Mt}
    =
    \begin{pmatrix}
      \cos\Omega t & \sin\Omega t & 0 \\
      -\sin\Omega t & \cos\Omega t & 0 \\
      0 & 0 & 1
    \end{pmatrix}
    \begin{pmatrix}
      e^{-\gamma t/2} & 0 & 0 \\
      0 & e^{-\gamma t/2} & 0 \\
      0 & 0 & e^{-\gamma t}
    \end{pmatrix}
    =
    e^{-\gamma t/2}
    \begin{pmatrix}
      \cos\Omega t & \sin\Omega t & 0 \\
      -\sin\Omega t & \cos\Omega t & 0 \\
      0 & 0 & e^{-\gamma t/2}
    \end{pmatrix}
  \end{equation}
  となります.

  \item 
  一般解は,$\bm{c}=(c_1,c_2,c_3)$を定ベクトルとして
  \begin{equation}
    \bm{v}(t)
    =
    e^{-Mt}\bm{c}
    -
    M^{-1}\bm{b}
  \end{equation}
  です.ただし,$M^{-1}$は
  \begin{equation}
    M^{-1}
    =
    \begin{pmatrix}
      \dfrac{\gamma}{2}\cdot\dfrac{1}{\gamma^2/4+\Omega^2}
      &
      \dfrac{\Omega}{\gamma^2/4+\Omega^2}
      &
      0
      \\
      -\dfrac{\Omega}{\gamma^2/4+\Omega^2}
      &
      \dfrac{\gamma}{2}\cdot\dfrac{1}{\gamma^2/4+\Omega^2}
      &
      0
      \\
      0 & 0 & \dfrac{1}{\gamma}
    \end{pmatrix}
  \end{equation}
  です.したがって,初期条件は
  \begin{equation}
    \begin{pmatrix}
      a_1 \\
      a_2 \\
      a_3
    \end{pmatrix}
    =
    \begin{pmatrix}
      c_1 \\
      c_2 \\
      c_3
    \end{pmatrix}
    -
    \begin{pmatrix}
      \dfrac{\gamma}{2}\cdot\dfrac{1}{\gamma^2/4+\Omega^2}
      &
      \dfrac{\Omega}{\gamma^2/4+\Omega^2}
      &
      0
      \\
      -\dfrac{\Omega}{\gamma^2/4+\Omega^2}
      &
      \dfrac{\gamma}{2}\cdot\dfrac{1}{\gamma^2/4+\Omega^2}
      &
      0
      \\
      0 & 0 & \dfrac{1}{\gamma}
    \end{pmatrix}
    \begin{pmatrix}
      0 \\
      0 \\
      \Gamma
    \end{pmatrix}
  \end{equation}
  なので,これを解くと
  \begin{equation}
    \begin{pmatrix}
      c_1 \\
      c_2 \\
      c_3
    \end{pmatrix}
    =
    \begin{pmatrix}
      a_1 \\
      a_2 \\
      a_3 + \gamma/\Gamma
    \end{pmatrix}
    =
    \bm{v}(0)
    +
    \frac{1}{\gamma}\bm{b}
  \end{equation}
  です.よって,
  \begin{equation}
    \bm{v}(t)
    =
    e^{-Mt}\bm{v}(0)
    +
    \frac{1}{\gamma}(e^{-Mt}-1)\bm{b}
  \end{equation}
  がもとめる解です.

\end{enumerate}

\section*{補足}

\begin{itemize}
  \item 
  $A$と$B$が可換なら$e^{A+B}=e^Ae^B$が成立します.証明は普通の数のときと同様で
  \begin{align}
    e^{A+B}
    &=
    \sum_{n=0}^{\infty}
    \frac{1}{n!}(A+B)^n
    \nonumber
    \\
    &=
    \sum_{n=0}^{\infty}
    \sum_{k=0}^{n}
    \frac{1}{(n-k)!k!}A^{n-k}B^k
    \nonumber
    \\
    &=
    \left(  
      \sum_{n=0}^{\infty}
      \frac{1}{n!}A^n
    \right)
    \left(  
      \sum_{n=0}^{\infty}
      \frac{1}{n!}B^n
    \right)
    =
    e^Ae^B
  \end{align}
  です.
  

\end{itemize}










\clearpage

\prb{2}{物理数学}

\begin{enumerate}
  \item 
  $f(x)$のフーリエ級数展開は
  \begin{equation}
    f(x)
    =
    \frac{a_0}{2}
    +
    \sum_{n=1}^{\infty}
    (
      a_n \cos nx
      +
      b_n \sin nx
    )
  \end{equation}
  だとします.このとき,
  \begin{align}
    a_0
    &=
    \frac{1}{\pi}
    \int_{-\pi}^{\pi}
    |x|\dd x
    \nonumber
    \\
    &=
    \frac{2}{\pi}
    \cdot
    \frac{\pi^2}{2}
    =
    \pi
    \\
    a_n
    &=
    \frac{2}{\pi}
    \int_0^{\pi}
    x\cos nx
    \dd x
    \nonumber
    \\
    &=
    \left\{
      \begin{alignedat}{1}
        0
        &\quad
        (n:\text{even})
        \\
        -
        \frac{4}{\pi n^2}
        &\quad
        (n:\text{odd})
      \end{alignedat}
    \right.
    \\
    b_n
    &=
    \frac{1}{\pi}
    \int_{-\pi}^{\pi}
    |x|\sin nx
    \dd x
    =
    0
  \end{align}
  なので,
  \begin{equation}
    |x|
    =
    \frac{\pi}{2}
    -
    \sum_{m=0}^{\infty}
    \frac{4}{(2m+1)^2\pi}
    \cos (2m+1)x
  \end{equation}
  です.

  \item 
  (1)で$x=0$とすれば
  \begin{equation}
    \sum_{n=0}^{\infty}
    \frac{1}{(2n+1)^2}
    =
    \frac{\pi^2}{8}
  \end{equation}
  です.

  \item 
  (積分が可能か否か云々はここでは議論しないことにします.値がちゃんと出てるなら,別によいでしょう.)

  $\varepsilon>0$をとれば
  \begin{align}
    c_n
    &=
    \lim_{\varepsilon\rightarrow+0}
    \int_{\varepsilon}^{1}
    x^n\log x
    \dd x
    \nonumber
    \\
    &=
    \lim_{\varepsilon\rightarrow+0}
    \left[  
      -
      \frac{1}{n+1}\varepsilon^{n+1}\log\varepsilon
      -
      \frac{1-\varepsilon^{n+1}}{(n+1)^2}
    \right]
    =
    -
    \frac{1}{(n+1)^2}
  \end{align}
  となります.ただし,
  \begin{equation}
    \lim_{\varepsilon\rightarrow+0}\varepsilon^{n+1}\log\varepsilon
    =
    0
  \end{equation}
  を用いました.

  \item 
  $t=1-x$で変数変換をして,$\log(1-t)$を展開すれば
  \begin{equation}
    A
    =
    -\sum_{n=1}^{\infty}\frac{1}{n}
    \int_{0}^{1}t^{n-1}\dd t
    =
    -\sum_{n=1}^{\infty}\frac{1}{n^2}
  \end{equation}
  となります.(2)の結果から値を決めるためには,
  \begin{equation}
    T
    \coloneqq
    \sum_{n=1}^{\infty}
    \frac{1}{n^2}
  \end{equation}
  としたとき,$T$を偶数と奇数の和に分割して
  \begin{equation}
    T
    =
    \sum_{n=1}^{\infty}
    \frac{1}{(2n)^2}
    +
    \sum_{n=0}^{\infty}
    \frac{1}{(2n+1)^2}
    =
    \frac{1}{4}
    \sum_{n=1}^{\infty}
    \frac{1}{n^2}
    +
    S
  \end{equation}
  となることを用いましょう.したがって,
  \begin{equation}
    T
    =
    \sum_{n=1}^{\infty}
    \frac{1}{n^2}
    =
    \frac{4}{3}S
    =
    \frac{\pi^2}{6}
  \end{equation}
  なので,
  \begin{equation}
    A
    =
    -
    \frac{\pi^2}{6}
  \end{equation}
  です.

  \item 
  連立方程式です.つまり
  \begin{equation}
    \left\{
      \begin{alignedat}{1}
        A-B
        &=
        \frac{A}{2}
        \\
        A+B
        &=
        2C
      \end{alignedat}
    \right.
  \end{equation}
  を解けば\footnote{
    $A-B$についての方程式は$t=x^2$と変数変換すればよいです.
  }
  \begin{equation}
    B
    =
    -
    \frac{\pi^2}{12}
    ,\ 
    C
    =
    -\frac{\pi^2}{8}
    .
  \end{equation}

\end{enumerate}








\clearpage

\prb{3}{力学}

\begin{enumerate}
  \item 
  運動方程式は
  \begin{equation}
    M\dv[2]{v}{t}
    =
    -
    Mg
    -
    \alpha v(t)
  \end{equation}
  なので,これを解くと
  \begin{equation}
    v(t)
    =
    C_1
    e^{-\alpha t/M}
    -
    \frac{M}{\alpha}g
  \end{equation}
  です.初期条件を満たすように定数$C_1$を決めると
  \begin{equation*}
    v(t)
    =
    \frac{M}{\alpha}g
    \left(  
      1-e^{-\alpha t/M}
    \right)
  \end{equation*}
  となります.また,これを積分して初期条件を満たすように定数を決めれば
  \begin{equation}
    z(t)
    =
    \frac{M}{\alpha}g
    \left\{  
      t
      +
      \frac{M}{\alpha}
      \left(  
        e^{-\alpha t/M}
        -
        1
      \right)
    \right\}
    .
  \end{equation}
  図は\ref{ans_vt},\ref{ans_zt}です.

  \begin{figure}[ht]
    \begin{minipage}[ht]{0.49\columnwidth}
      \centering
      \begin{tikzpicture}
        \draw[thin,-Stealth] (-0.3,0)--(4,0)node[above]{$t$};
        \draw[thin,-Stealth] (0,-0.3)--(0,4)node[left]{$v(t)$};
        \draw[very thick,samples=100,domain=0.0:4.0]plot(\x,{3.3*(1-exp(-2*\x))});
        \draw[dotted](-0.3,3.3)--(4,3.3);
      \end{tikzpicture}
      \caption{$v(t)$のグラフ}
      \label{ans_vt}
    \end{minipage}
    \begin{minipage}[ht]{0.49\columnwidth}
      \centering
      \begin{tikzpicture}
        \draw[thin,-Stealth] (-0.3,0)--(4,0)node[above]{$t$};
        \draw[thin,-Stealth] (0,-0.3)--(0,4)node[left]{$z(t)$};
        \draw[very thick,samples=100,domain=0.0:3.0]plot(\x,{\x+1-exp(-5*\x)});
        \draw[dotted](-0.3,0.8)--(-0.3+3.3,0.8+3.3);
      \end{tikzpicture}
      \caption{$z(t)$のグラフ}
      \label{ans_zt}
    \end{minipage}
  \end{figure}

  \item 
  E.O.M.は
  \begin{equation}
    M\dv{v}{t}
    =
    -
    Mg
    -
    \beta v^2(t)
  \end{equation}
  なので,これを変数分離すると
  \begin{equation}
    \frac{1}{v^2+Mg/\beta}\dv{v}{t}
    =
    -
    \frac{\beta}{M}
  \end{equation}
  です.これを積分すると
  \begin{equation}
    \sqrt{
      \frac{\beta}{Mg}
    }
    \tan^{-1}\left(  
      \sqrt{\frac{\beta}{Mg}}v
    \right)
    =
    -
    \frac{\beta}{M}t
    +
    C_2
  \end{equation}
  となります.初期条件を満たすのは$C_2=0$なので
  \begin{equation}
    v(t)
    =
    \sqrt{\frac{Mg}{\beta}}
    \tan\left(  
      \sqrt{\frac{\beta g}{M}}t
    \right)
  \end{equation}
  が解です\footnote{
    \href{https://uxhpu.net/physics/inertial_resistance/}{これ}によると,すこし答えが違っています.そもそも最初の運動方程式が違うようです.
  }.図は省略.

  \item 
  説明は少し面倒ですが,感覚的にはこんな感じでしょう.つまり,運動量の変化が力積なので,流体粒子1つあたり,だいたい$mV$の力がかかります.物体が速度$V$で動くとき,物体に衝突する粒子の個数は$V$に比例します.したがって,$V$個の粒子が$mV$だけ力積をかけるので,かかる力は$V^2$に比例する,といった感じです\footnote{
    少し調べてみましたが,ちゃんとやると\href{https://www.kem3.com/esrp/lecture/Mech/Viscosity/viscosityo.html}{こんな感じ}らしいです.
  }.

  \item 
  略.

  \item 
  略.

\end{enumerate}











\clearpage
\prb{4}{電磁気学}

\begin{enumerate}
  \item 
  2つめと3つめは曲面$S$上で積分,4つめは空間$V$上で積分をします.するとそれぞれ
  \begin{equation}
    \oint_{\partial S}\bm{H}\cdot\dd \bm{l}
    =
    \int_{S}
    \left(  
      \bm{j}
      +
      \pdv{\bm{D}}{t}
    \right)
    \cdot
    \dd \bm{S}
    ,\ 
    \oint_{\partial S}\bm{E}\cdot\dd \bm{l}
    =
    -
    \pdv{}{t}
    \int_{S}
    \bm{B}
    \cdot
    \dd \bm{S}
    ,\ 
    \oint_{\partial V}
    \bm{B}\cdot\dd^2 \bm{S}
    =
    0
  \end{equation}
  となります\footnote{
    ストークスの定理は$\text{rot}$をの面積分を線積分に,ガウスの定理は$\text{div}$の体積積分を面積分にします.
  }.

  \item 
  原点に電荷がある場合は,そもそも一般形がわかって
  \begin{equation}
    \bm{E}(\bm{r})
    =
    \frac{q}{4\pi \varepsilon_0 r^2}\cdot\frac{\bm{r}}{r}
    .
  \end{equation}

  \item 
  ガウスの法則の積分形より
  \begin{equation}
    \int
    \bm{E}(\bm{r})
    \cdot
    \bm{n}
    \dd S
    =
    \frac{A}{\varepsilon_0}
    \int
    e^{-2r/a}
    \dd V
  \end{equation}
  です.対称性より$\bm{E}(\bm{r})\cdot\bm{n}=E(r)$で,右辺は
  \begin{equation}
    \int
    e^{-2R/a}
    \dd V
    =
    4\pi^2
    \int_0^r
    R^2e^{-2R/a}
    \dd R
    =
    -
    \frac{ar^2}{2}e^{-2r/a}
    -
    \frac{a^2}{2}re^{-2r/a}
    -
    \frac{a^3}{4}e^{-2r/a}
    +
    \frac{a^3}{4}
  \end{equation}
  となるので,
  \begin{equation}
    \bm{E}(r)
    =
    \frac{\pi A}{\varepsilon_0r^2}
    \left[  
      -
    \frac{ar^2}{2}e^{-2r/a}
    -
    \frac{a^2}{2}re^{-2r/a}
    -
    \frac{a^3}{4}e^{-2r/a}
    +
    \frac{a^3}{4}
    \right]
    \frac{\bm{r}}{r}
  \end{equation}
  となります.ただし,$r$を$R$と書き換えてました.

  \item 
  電荷が2つあるとき,それぞれの電荷の寄与を考えて足せばよいです\footnote{
    補足を参照.trivialなことなので,あまり真面目に示す恩恵はないですが.
  }.だけど,計算が面倒なので略.

  \item 
  原点からの極座標が$(r,\theta,\varphi)$の位置にある小片がもつ電荷は$\rho(r)\times r^2\dd \Omega$で,これの時間変化が電流.したがって,
  \begin{equation}
    \dd j
    =
    \dv{Q}{t}
    =
    \rho(r)
    r^2\sin\theta\dd \theta
    \dv{\varphi}{t}
    =
    A\omega r^2 e^{-2r/a}\sin\theta\dd \theta
  \end{equation}
  です.この小片が原点に作る磁場は
  \begin{equation}
    \dd \bm{H}
    =
    \frac{1}{4\pi}\int\frac{\dd \bm{j}\times \bm{r}}{r^3}
  \end{equation}
  ですが,これらの$z$軸方向以外の成分は積分で打ち消します\footnote{
    感覚からも明らかですが,\eqref{infinitesimal}のほかの成分を考えると計算で示すことができます.ほかの成分は$\sin\varphi\dd\varphi$や$\cos\varphi\dd\varphi$が残るので,これを積分すると消えます.
  }.したがって,
  \begin{equation}
    \dd H_z
    =
    \frac{1}{4\pi}\int\frac{(\dd \bm{j}\times \bm{r})_z}{r^3}
  \end{equation}
  をすべての空間で計算してやればよいです.ここで,
  \begin{align}
    (\dd \bm{j}\times \bm{r})_z
    &=
    \left(  
      \begin{pmatrix}
        -|\bm{j}|r\sin\theta\sin\varphi\dd \varphi\\
        |\bm{j}|r\sin\theta\cos\varphi\dd \varphi\\
        0
      \end{pmatrix}
      \times
      \begin{pmatrix}
        -r\sin\theta\cos\varphi \\
        -r\sin\theta\sin\varphi \\
        -r\cos\theta
      \end{pmatrix}
    \right)_z
    \nonumber
    \\
    &=
    2|\bm{j}|r^2\sin^2\theta\dd\varphi
    \nonumber
    \\
    &=
    2A\omega r^3 e^{-2r/a}\sin^3\dd \theta\dd \varphi
    \label{infinitesimal}
  \end{align}
  なので\footnote{
    図を描いてみてください.
  },
  \begin{align}
    H_z
    &=
    \frac{A\omega}{2\pi}
    \int_{0}^{2\pi}\dd \varphi
    \int_{0}^{\pi}\sin^3\theta\dd \theta
    \int_{0}^{\infty}e^{-2r/a}\dd r
    \nonumber
    \\
    &=
    \frac{2}{3}A\omega a
  \end{align}
  がもとめる磁場です.

\end{enumerate}


\subsection*{補足}

\begin{itemize}
  \item 
  解答で「電荷が2つあるとき,それぞれの電荷の寄与を考えて足せばよい」といいましたが,たぶんこんな感じだと思います.電荷密度が2つの項からなるとき,それらを
  \begin{equation}
    \rho(\bm{r})
    =
    \rho_1(\bm{r})
    +
    \rho_2(\bm{r})
  \end{equation}
  とおきます.これらの電荷がつくる電場を$\bm{E}_1,\bm{E}_2$を
  \begin{equation}
    \int_S
    \bm{E}_1
    \cdot
    \bm{n}
    \dd S
    =
    \frac{1}{\varepsilon_0}
    \int\rho_1(\bm{r^{\prime}})\dd^3\bm{r}^{\prime}
    ,\ 
    \int_S
    \bm{E}_2
    \cdot
    \bm{n}
    \dd S
    =
    \frac{1}{\varepsilon_0}
    \int\rho_2(\bm{r^{\prime}})\dd^3\bm{r}^{\prime}
  \end{equation}
  と定義すれば,ガウスの法則から
  \begin{equation}
    \bm{E}
    =
    \bm{E}_1
    +
    \bm{E}_2
  \end{equation}
  となります.これを積分するので,結局,ポテンシャルも電荷$\rho_1,\rho_2$がつくるものに分解することができます.

  \item 
  電荷$\rho(\bm{r})$がつくる電位$\phi(\bm{r})$は
  \begin{equation}
    \phi(\bm{r})
    =
    \frac{1}{4\pi\varepsilon_0}
    \int
    \frac{\rho(\bm{r}^{\prime})}{|\bm{r}-\bm{r}^{\prime}|}
    \dd \bm{r}^{\prime}
  \end{equation}
  です.これのgradientをとると
  \begin{equation}
    \bm{E}(\bm{r})
    =
    -
    \bm{\nabla}\phi(\bm{r})
    =
    \frac{1}{4\pi\varepsilon_0}
    \int
    \frac{\bm{r}-\bm{r}^{\prime}}{|\bm{r}-\bm{r}^{\prime}|^3}\rho(\bm{r}^{\prime})
    \dd \bm{r}^{\prime}
  \end{equation}
  となり,これのdivergenceをとると
  \begin{equation}
    \bm{\nabla}
    \cdot
    \bm{E}(\bm{r})
    =
    -
    \bm{\nabla}^2
    \phi(\bm{r})
    =
    -
    \frac{1}{4\pi\varepsilon_0}
    \int
    \bm{\nabla}^2
    \left(  
      \frac{1}{|\bm{r}-\bm{r}^{\prime}|}
    \right)\rho(\bm{r}^{\prime})
    \dd \bm{r}^{\prime}
    =
    \frac{\rho(\bm{r})}{\varepsilon_0}
  \end{equation}
  となり,ちゃんとMaxwell方程式とconsistentです.

\end{itemize}










\clearpage
\prb{5}{量子力学}

\begin{enumerate}
  \item 
  計算すれば$k^2=2m\omega^2$.

  \item 
  代入すればOK.

  \item 
  $k=\sqrt{2mE/\hbar^2}$とおけば,一般解は
  \begin{equation}
    \varphi(x)
    =
    Ae^{+ikx}
    +
    Be^{-ikx}
    ,\ 
    \varphi^{\prime}(x)
    =
    ik\left(  
      Ae^{+ikx}
      -
      Be^{-ikx}
    \right)
  \end{equation}
  と書けます.これに境界条件を代入すれば
  \begin{equation}
    A
    \sin\left(  
      \frac{kL}{2}
    \right)
    =
    B
    \sin\left(  
      \frac{kL}{2}
    \right)
    =
    0
  \end{equation}
  であればよいことがわかり,
  \begin{equation}
    A=B=\frac{1}{\sqrt{2L}}
    ,\ 
    \frac{kL}{2}
    =
    2n\pi
    \quad
    \left( n\in\mathbb{Z} \right)
  \end{equation}
  です.よって,波動関数とエネルギー固有値は
  \begin{equation}
    \varphi_n(x)
    =
    \sqrt{\frac{2}{L}}\cos k_n x
    ,\ 
    E_n
    =
    \frac{8\pi^2\hbar^2}{mL^2}n^2
  \end{equation}
  です.

  \item 
  境界条件は$e^{ikL}=1$から
  \begin{equation}
    A=B=\frac{1}{\sqrt{2L}}
    ,\ 
    kL
    =
    2n\pi
    \quad
    \left( n\in\mathbb{Z} \right)
  \end{equation}
  なので
  \begin{equation}
    \varphi_n(x)
    =
    \sqrt{\frac{2}{L}}\cos k_n x
    ,\ 
    E_n
    =
    \frac{2\pi^2\hbar^2}{mL^2}n^2
  \end{equation}
  です.

  \item 
  第$n$励起状態の$k$次のエネルギーは,もし,縮退がなければ
  \begin{equation}
    E_n^{(k)}
    =
    \int_{-L/2}^{L/2}
    \dd x\ 
    {\varphi^{*}_{n}}^{(0)}(x)
    V(x)
    {\varphi_n}^{(k-1)}(x)
    \label{fract}
  \end{equation}
  です.基底で$k=1$なら,ポテンシャルは$L/2$周期になっているので
  \begin{equation}
    E_0^{(1)}
    =
    \int_{-L/2}^{L/2}
    \dd x\ 
    {\varphi^{*}_{0}}(x)
    V(x)
    {\varphi_0}(x)
    =
    \frac{V_0}{\sqrt{2L}}
    \int_{-L/2}^{L/2}
    \cos\left( \frac{4\pi x}{L} \right)
    \dd x\     
    =
    0
  \end{equation}
  です.$k=2$のときは,
  \begin{equation}
    \varphi_0^{(1)}(x)
    =
    \sum_{m\neq 0}
    \varphi_m^{(0)}(x)
    \int\dd x^{\prime}\ 
    \frac{{\varphi_m^*}^{(0)}(x^{\prime})V(x^{\prime})\varphi_0^{(0)}(x^{\prime})}{E_0^{(0)}-E_m^{(0)}}
  \end{equation}
  であり,積分は
  \begin{equation}
    \frac{4V_0}{L}
    \int_0^{L/2}\dd x^{\prime}\ 
    \cos k_mx^{\prime}\cos \left( \frac{4\pi}{L}x^{\prime} \right)
    =
    V_0\delta_{m2}
  \end{equation}
  なので,
  \begin{equation}
    \varphi_0^{(1)}(x)
    =
    -
    \frac{V_0}{E_2^{(0)}}\varphi_2^{(0)}(x)
    =
    -
    \sqrt{\frac{2}{L}}\frac{V_0}{E_2^{(0)}}\cos\left( \frac{4\pi}{L}x \right)
  \end{equation}
  となります.したがって,
  \begin{equation}
    E_0^{(2)}
    =
    -
    \frac{mV_0}{\pi^2\hbar^2}\sqrt{\frac{L^3}{32}}
  \end{equation}
  です.

  \item 
  計算は簡単で,
  \begin{equation}
    E_1^{(1)}
    =
    \frac{2V_0}{L}
    \int_0^{L/2}
    \left(  
      \cos\left( \frac{4\pi}{L}z \right)
      +
      \cos^2\left( \frac{4\pi}{L}z \right)
    \right)
    \dd x
    =
    \frac{1}{2}V_0
  \end{equation}
  です\footnote{
    $\cos(4\pi x/L)$は周期が$L/2$なので積分しても0で,$\cos^2(4\pi x/L)$は周期が$L$なのでそっちの寄与を考えれば早いです.
  }.固有状態は
  \begin{align}
    \int_{-L/2}^{L/2}
    {\varphi_m^{(0)}}^{*}
    V
    \varphi_1^{(0)}
    \dd x
    &=
    \frac{4V_0}{L}
    \int_{0}^{L/2}
    \cos\left( \frac{2\pi}{L}x \right)
    \cos\left( \frac{4\pi}{L}x \right)
    \cos\left( \frac{2m\pi}{L}x \right)
    \dd x
    \nonumber
    \\
    &=
    \left\{
      \begin{alignedat}{1}
        \frac{1}{2}V_0
        &
        \quad
        (m=1,3)
        \\
        0
        &
        \quad
        (\text{otherwise})
      \end{alignedat}
    \right.
  \end{align}
  より\footnote{
    周期性を気にすれば,残るのが$m=1,3$のときなのがわかります.例えば,周期$L/3,L$の三角関数の積の周期は$3L/2,L/2$の線形和になります.そして,それに周期$L/m$のものがかかれば,周期
    $$
      \frac{L}{m+3}
      ,\ 
      \frac{L}{m-3}
      ,\ 
      \frac{L}{m+1}
      ,\ 
      \frac{L}{m-1}
    $$
    の関数に分離されます.これらは全部$\cos$で,$0$から$L/2$で積分したときに残るのは$m=1,3$のときのみです.気になったら周期$L/N$の$\cos$を$0$から$L/2$で積分して,$N$が1より大きいと全部消えることをチェックするとよいでしょう.
    \label{note}
  },
  \begin{equation}
    \varphi_1^{(1)}(x)
    =
    \sum_{m\neq 1}
    \frac{\ev*{\varphi_m^{(0)}|V|\varphi_1^{(0)}}}{E_1^{(0)}-E_m^{(0)}}
    \varphi_m^{(0)}
    =
    -\sqrt{\frac{2}{L}}
    \frac{V_0 mL^2}{32\pi^2\hbar^2}
    \cos\left( \frac{6\pi}{L}x \right)
  \end{equation}
  となります.

  \item 
  波動関数へのポテンシャルの寄与は,$k_\lambda=2\pi a$の項を追加するような形になると考えられます.したがって,(1)の答えに$k_\lambda^2$のような項が加わった形が極限でしょう.つまり
  \begin{equation}
    k^2+4\pi^2 a^2
    =
    2m\omega^2
  \end{equation}
  です.

\end{enumerate}


\subsection*{補足}

\begin{itemize}
  \item 
  (3),(4)の規格化ですが
  \begin{equation}
    \varphi(x)
    =
    A\left(  
      e^{+ikx}
      +
      e^{-ikx}
    \right)
    =
    2A\cos kx
  \end{equation}
  とすれば,
  \begin{equation}
    \int_{-L/2}^{L/2}
    \varphi^{*}(x)
    \varphi(x)
    \dd x
    =
    2L|A|^2=1
  \end{equation}
  なので,
  \begin{equation}
    A
    =
    \frac{1}{\sqrt{2L}}
  \end{equation}
  です.

  \item 
  ちょっと,ここから書くことは自信がないですが,設問(7)をもう少しちゃんと考えてみましょう.ポテンシャルが
  \begin{equation}
    V(x)
    =
    V_0
    \cos\left( \frac{2N\pi}{L}x \right)
  \end{equation}
  のときは,
  \begin{equation}
    \ev*{\varphi_m^{(0)}|V|\varphi_k^{(0)}}
    =
    \frac{4V_0}{L}
    \int_{0}^{L/2}
    \cos\left( \frac{2k\pi}{L}x \right)
    \cos\left( \frac{2N\pi}{L}x \right)
    \cos\left( \frac{2m\pi}{L}x \right)
    \dd x
  \end{equation}
  なので,$N,m$を固定したとき,生き残るのは$k=|N\pm m|$なので\footnote{
    注釈\ref{note}を参考にしてください.
  },摂動で生き残るのは$k=N+m$です.よって,このポテンシャルの摂動は
  \begin{equation}
    \varphi_m^{(1)}(x)
    =
    \frac{\ev*{\varphi_m^{(0)}|V|\varphi_{N+m}^{(0)}}}{E_m-E_{N+m}}\varphi_{N+m}(x)
  \end{equation}
  といった形で波動関数に現れます.よって,第$m$励起状態の波動関数の2階微分の固有値は
  \begin{equation}
    \pdv[2]{\varphi_m}{x}
    =
    \pdv[2]{\varphi_m^{(0)}}{x}
    +
    \pdv[2]{\varphi_m^{(1)}}{x}    
    =
    k_m^2\varphi_m^{(0)}
    +
    k_{m+N}^2\varphi_m^{(1)}
  \end{equation}
  となります.したがって,分散関係も
  \begin{equation}
    k_m^2+k_{m+N}^2
    =
    2m\omega^2
  \end{equation}
  のようになると考えられますが,
  \begin{equation}
    k_{m+N}^2
    =
    \frac{2(m+N)\pi}{L}
    \rightarrow
    2\pi a
  \end{equation}
  です.したがって,$k$と$\omega$の関係は
  \begin{equation}
    k^2+4\pi^2 a^2
    =
    2m\omega^2
  \end{equation}
  になると考えられます.

\end{itemize}

\clearpage

\prb{6}{統計力学}

\begin{enumerate}
  
  \item 

  場合の数は
  \begin{equation}
    W
    =
    \frac{M!}{N!(M-N)!}
  \end{equation}
  なので,
  \begin{equation}
    S
    =
    k_B\log W
    =
    k_B
    \log\frac{M!}{N!(M-N)!}
  \end{equation}
  です.

  \item 

  ある格子に着目して,考えてみましょう.そこに粒子が入る確率は$N/M$で,隣接する格子に粒子が入る確率も$N/M$.隣接格子は$z=4$なので,格子ひとつあたりのエネルギー期待値は$-4\varepsilon (N/M)^2$です.これを$M$個の格子で考えるとき,各相互作用については重複を取り除いて
  \begin{equation}
    \bar{E}
    =
    -2\varepsilon M\left( \frac{N}{M} \right)^2
  \end{equation}
  です.

  \item 
  
  $F=\bar{E}-TS$より
  \begin{equation}
    Z[\beta,E]
    =
    W(\bar{E})e^{-\beta \bar{E}}
  \end{equation}
  なので
  \begin{equation}
    Z[\beta,\bar{E}]
    =
    \exp\left[ -2\varepsilon M\left( \frac{N}{M} \right)^2 \right]\log\frac{M!}{N!(M-N)!}
  \end{equation}
  です.

  \item 
  
  \begin{align}
    F
    &=
    -2\varepsilon M\left( \frac{N}{M} \right)^2
    -
    \frac{1}{\beta}\log \frac{M!}{N!(M-N)!}
    \nonumber
    \\
    &\sim
    -2\varepsilon M\left( \frac{N}{M} \right)^2
    -\frac{1}{\beta}\left[ M\log M
    -
    N\log N
    -
    (M-N)\log (M-N) \right]   
    .
  \end{align}

  \item 

  \begin{equation}
    p
    =
    -\pdv{F}{V}
    =    
    -
    \frac{2\varepsilon}{a}\left( \frac{N}{M} \right)^2
    -
    \frac{1}{a\beta}
    \log\left( 1-\frac{N}{M} \right)
    .
    \label{pressure}
  \end{equation}
  ただし,$V=aM$として,$M$の微分を計算しました.

  \item 
  もう一度,$M$で微分すると思えば
  \begin{equation}
    \pdv{p}{V}
    =
    \frac{4\varepsilon}{a^2M}\left( \frac{N}{M} \right)^2
    -
    \frac{1}{a^2\beta}
    \cdot
    \frac{1}{1-N/M}
    \cdot
    \frac{N}{M^2}
  \end{equation}
  です.

  \item 

  $\pdv*{p}{V}=0$となるような$M(>0)$が存在すれば,相転移が起こることがわかります.そのような$M$を決定する方程式は
  \begin{equation}
    \frac{4\varepsilon}{a^2M}\left( \frac{N}{M} \right)^2
    -
    \frac{1}{a^2\beta}
    \cdot
    \frac{1}{1-N/M}
    \cdot
    \frac{N}{M^2}
    =
    0
  \end{equation}
  です.これを整理すれば
  \begin{equation}
    M^2
    -
    \frac{4\varepsilon N}{k_BT}M
    +
    \frac{4\varepsilon N^2}{k_BT}
    =
    0
  \end{equation}
  であり,これが実数解をもつためには
  \begin{equation}
    T
    \leq
    \frac{\varepsilon}{k_B}
    \label{ans67}
  \end{equation}
  が必要です.2次関数
  \begin{equation}
    f(M)
    \coloneqq
    M^2
    -
    \frac{4\varepsilon N}{k_BT}M
    +
    \frac{4\varepsilon N^2}{k_BT}
  \end{equation}
  を考えれば,$M>0$の解をもつことがわかる\footnote{
    軸が$y$軸よりも右側にあって,$f(0)>0$なので.
  }ので,もとめるのは\eqref{ans67}です.

\end{enumerate}

\end{document}
