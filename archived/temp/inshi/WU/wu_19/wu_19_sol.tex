\pdfoutput=1
\documentclass[a4paper,pdflatex,ja=standard]{bxjsarticle}

% ---Setting about the geometry of the document----
% \usepackage{a4wide}
% \pagestyle{empty}

% ---Physics and Math Packages---
\usepackage{amssymb,amsfonts,amsthm,mathtools}
\usepackage{physics,braket,bm}

% ---underline---
\usepackage{ulem}

% --- sorround the texts or equations
% \usepackage{fancybox,ascmac}

% ---settings of theorem environment---
% \usepackage{amsthm}
% \theoremstyle{definition}

% ---settings of proof environment---
% \renewcommand{\proofname}{\textbf{証明}}
% \renewcommand{\qedsymbol}{$\blacksquare$}

% ---Ignore the Warnings---
\usepackage{silence}
\WarningFilter{latexfont}{Some font shapes,Font shape}

% ---Insert the figure (If insert the `draft' at the option, the process becomes faster)---
% \usepackage{graphicx}
% \usepackage{subcaption}

% ----Add a link to a text---
\usepackage{url}
\usepackage{xcolor,hyperref}
\hypersetup{colorlinks=true,citecolor=orange,linkcolor=blue,urlcolor=magenta}
\usepackage{bxcjkjatype}

% ---Tikz---
\usepackage{tikz,pgf,pgfplots,circuitikz}
\pgfplotsset{compat=1.15}
\usetikzlibrary{intersections,arrows.meta,angles,calc,3d,decorations.pathmorphing}

% ---Add the section number to the equation, figure, and table number---
\makeatletter
   \renewcommand{\theequation}{\thesection.\arabic{equation}}
   \@addtoreset{equation}{section}
   
   \renewcommand{\thefigure}{\thesection.\arabic{figure}}
   \@addtoreset{figure}{section}
   
   \renewcommand{\thetable}{\thesection.\arabic{table}}
   \@addtoreset{table}{section}
\makeatother

% ---enumerate---
\renewcommand{\labelenumi}{$(\arabic{enumi})$}
% \renewcommand{\labelenumii}{$(\arabic{enumii})$}

% ---Index---
%\usepackage{makeidx}
%\makeindex 

% ---Fonts---
\renewcommand{\familydefault}{\sfdefault}

% ---Title---
\title{早稲田大学\ 2019年\ 物理学専攻\ 院試\ 解答例}
\author{ミヤネ}
\date{最終更新:\today}

\newcommand{\prb}[2]{
  \clearpage
  \phantomsection
  \addcontentsline{toc}{section}{問題 #1: #2}
  \section*{問題番号\fbox{#1}\ (#2)}
  \setcounter{section}{#1}
  \setcounter{equation}{0}
}

\begin{document}

\maketitle

\tableofcontents
\clearpage

\prb{3}{力学}

\begin{enumerate}
  
  \item 

  \begin{equation}
    l_0
    \coloneqq
    y_0
    +
    \frac{Mg}{k}
    .
  \end{equation}

  \item 

  \begin{equation}
    \left\{
      \begin{alignedat}{1}
        \text{車輪}\ 
        &:\ 
        m
        \dv[2]{y_1}{t}
        =
        N-mg+k(y_2-y_1-l_0)
        \\       
        \text{物体}\ 
        &:\ 
        M
        \dv[2]{y_2}{t}
        =
        -Mg-k(y_2-y_1-l_0)
        \\ 
      \end{alignedat}
    \right.
    .
  \end{equation}

  \item 

  \begin{equation}
    \dv[2]{y}{t}
    =
    -g
    -
    \frac{k}{M}(y-l_0)
    -
    \frac{\pi^2 a}{l^2}\cos\frac{\pi}{l}x
    .
  \end{equation}

  \item 

  $0<t<2\pi/\omega$でのE.O.M.は
  \begin{equation}
    \dv[2]{y}{t}
    =
    -\omega_0^2 y
    -
    a\omega^2\cos\omega t
  \end{equation}
  です.$2\pi/\omega$以降は強制振動の項が落ちるだけです.したがって,一般解は
  \begin{equation}
    y(t)
    =
    \left\{
      \begin{alignedat}{1}
        y_0+A\cos\omega_0 t+B\sin\omega_0 t+\frac{a\omega^2}{\omega^2-\omega_0^2}\cos\omega t
        \ &\ (0<t<2\pi/\omega)
        \\
        y_0+A\cos\omega_0 t+B\sin\omega_0 t
        \ &\ (2\pi/\omega<t)
      \end{alignedat}
    \right.
  \end{equation}
  です.

  \item 

  初期条件$y(0)=y_0,\dot{y}(0)=0$を満たす$0<t<2\pi/\omega$での解は
  \begin{equation}
    y(t)
    =
    y_0
    +
    \frac{a\omega^2}{\omega^2-\omega_0^2}
    (\cos\omega t-\cos\omega_0 t)
  \end{equation}
  であり,
  \begin{equation}
    \lim_{\omega\rightarrow\omega_0}
    \frac{\cos\omega t-\cos\omega_0 t}{\omega-\omega_0}
    =
    -t\sin\omega_0 t
  \end{equation}
  であることに注意すれば
  \begin{equation}
    y(t)
    =
    \left\{
      \begin{alignedat}{1}
        y_0
        -
        \frac{a\omega_0}{2}t\sin\omega_0 t
        \ &\ (0<t<2\pi/\omega)
        \\
        y_0
        \ &\ (2\pi/\omega<t)
      \end{alignedat}
    \right.
  \end{equation}
  となります\footnote{
    $t=2\pi/\omega\eqqcolon T$では$y(T)=y_0,\dot{y}(T)=0$なので,この条件で接続すると$t>T$では定常解$y(t)\equiv y_0$となります.
  }.

  \item 

  略.

  \item 

  \begin{equation}
    \dv[2]{y}{t}
    =
    -\omega_0^2(y-y_0)
    -
    2\beta \dv{y}{t}
    -
    a\omega^2\cos\omega t
    .
  \end{equation}

  \item 

  $t\rightarrow\infty$では
  \begin{equation}
    \dv[2]{y}{t}
    =
    -
    2\beta \dv{y}{t}
    -
    a\omega^2\cos\omega t
  \end{equation}
  となっているので,一般解は
  \begin{equation}
    y(t)
    =
    A+Be^{-2\beta t}
    -
    \frac{2a\beta\omega}{\omega^2+4\beta^2}\sin\omega t
    +
    \frac{a\omega^2}{\omega^2+4\beta^2}\cos\omega t 
  \end{equation}
  です\footnote{
    やり方はいろいろあるかと思いますが,$v(t)\coloneqq \dot{y}(t)$とおいて考えてみるのがよいかと思います.
  }.

  \item 

  \begin{equation}
    y(t)
    =
    A+Be^{-2\beta t}
    -
    \frac{2a\beta\omega_0}{\omega_0^2+4\beta^2}\sin\omega_0 t
    +
    \frac{a\omega_0^2}{\omega_0^2+4\beta^2}\cos\omega_0 t 
    .
  \end{equation}
  グラフは,$t\gg 1$なので(次元あってないのでこの書き方はよくないですが),指数関数の項を落としてただの三角関数.

\end{enumerate}

\clearpage

\prb{4}{電磁気学}

\begin{enumerate}
  
  \item 

  $z=0$では,$x$方向の成分が同じで
  \begin{equation}
    E_0
    -
    F_xe^{ik_x^{\prime}x}
    =
    T_xe^{ik_x^{\prime\prime}x}
  \end{equation}
  なので\footnote{
    $e^{-i\omega t}$は共通なので落としてます.
  },任意の$x\in\mathbb{R}$でこの等式が成立するためには$k_x^{\prime}=k_x^{\prime\prime}=0$です.また,その場合,$E_x-F_x=T_x$.

  \item 

  $\bm{\nabla}\cdot\bm{E}^{\prime}$は
  \begin{equation}
    \bm{\nabla}\cdot\bm{E}^{\prime}
    =
    k_z^{\prime}F_z e^{ik_z^{\prime}-i\omega t}
    =
    0
  \end{equation}
  なので,$F_z=0$です.$T_z=0$も同様.$\bm{\nabla}\cdot\bm{E}^{\prime\prime}=0$を計算するだけです.

  \item 

  $\bm{\nabla}\times\bm{E}+\pdv*{\bm{B}}{t}=0$より,$\bm{E}=\bm{E_0}e^{i(\bm{k}\cdot\bm{r}-\omega t)}$といった形にかけるなら
  \begin{equation}
    \bm{B}
    =
    \frac{1}{\omega}
    \bm{k}\times\bm{E}
  \end{equation}
  です.したがって,入射光については,$k\rightarrow-k$として
  \begin{equation}
    B_0
    =
    -
    \frac{k}{\omega}E_0
  \end{equation}
  となります.反射光と屈折光も同様に考えれば
  \begin{equation}
    B_0^{\prime}
    =
    -
    \frac{k_z^{\prime}}{\omega}F_x
    ,\ 
    B_0^{\prime\prime}
    =
    -
    \frac{k_z^{\prime\prime}}{\omega}T_x
  \end{equation}
  です.

  \item 

  磁場の界面方向について
  \begin{equation}
    \frac{1}{\mu_0}B_0
    +
    \frac{1}{\mu_0}B_0^{\prime}
    =
    \frac{1}{\mu_0}B_0^{\prime\prime}
  \end{equation}
  なので
  \begin{equation}
    kE_0
    +
    k_z^{\prime}F_0^{\prime}
    =
    k_z^{\prime\prime}T_0^{\prime\prime}
  \end{equation}
  です.

  \item 

  $\bm{\nabla}\times\bm{E}+\pdv*{\bm{B}}{t}=0$の回転をとって,ベクトル解析の公式を用いれば
  \begin{equation}
    \left(  
      \bm{\nabla}^2
      -
      \varepsilon\mu_0\pdv[2]{}{t}
    \right)
    \bm{E}
    =
    0
    \label{wave_eq}
  \end{equation}
  です.

  \item 

  \eqref{wave_eq}にそれぞれの解を入れましょう.

  \item 
  
  \begin{equation}
    \left|\frac{F_x}{E_0}\right|^2
    =
    \left( \frac{1-n}{1+n} \right)^2
    .
  \end{equation}

  \item 

  反射率は
  \begin{equation}
    \left|\frac{F_x}{E_0}\right|^2
    =
    \left( \frac{1-i\sqrt{|\varepsilon|/\varepsilon_0}}{1+i\sqrt{|\varepsilon|/\varepsilon_0}} \right)^2
  \end{equation}
  です.また,$E_z^{\prime\prime}=T_x e^{|k_z^{\prime\prime}|z}e^{-i\omega t}$となるので,$z\rightarrow-\infty$で屈折光は$E_z^{\prime\prime}\rightarrow 0$となります.

\end{enumerate}

\clearpage

\prb{5}{量子力学}

\begin{enumerate}
  
  \item 

  $\{a,a^{\dagger}\}=\ketbra*{0}{0}+\ketbra*{1}{1}=1.$

  \item 

  $n=\ketbra*{1}{1}$なので,$H_1=E_0(2n-\{a,a^{\dagger}\}).$

  \item 

  $\mel{\psi}{n}{\psi}=\beta.$

  \item 

  $x^2=1/4$なので,$\mel{\psi}{x^2}{\psi}=(|\alpha|^2+|\beta|^2)/4.$

  \item 

  固有値は$\pm E_0$で,固有状態は$\ket{\pm}=\ket{0}\pm\ket{1}$です.

  \item 

  $H_2=E_0^2$に気をつければ
  \begin{align}
    U_2(t)
    &=
    \sum_{n=0}^{\infty}
    \frac{1}{(2n)!}\left( \frac{iE_0t}{\hbar} \right)^{2n}
    -i
    \sum_{n=0}^{\infty}
    \frac{1}{(2n+1)!}\left( \frac{iE_0t}{\hbar} \right)^{2n+1}
    \begin{pmatrix}
      0 & 1 \\
      1 & 0
    \end{pmatrix}
    \nonumber
    \\
    &=
    \begin{pmatrix}
      \cos(E_0 t/\hbar) & -i\sin(E_0 t/\hbar)\\
      -i\sin(E_0 t/\hbar) & \cos(E_0 t/\hbar)
    \end{pmatrix}
  \end{align}
  です.

  \item 

  \begin{equation}
    \ket{\phi(t)}
    =
    \begin{pmatrix}
      \cos(E_0 t/\hbar) & -i\sin(E_0 t/\hbar)\\
      -i\sin(E_0 t/\hbar) & \cos(E_0 t/\hbar)
    \end{pmatrix}
    \begin{pmatrix}
      1 \\
      0
    \end{pmatrix}
    =
    \begin{pmatrix}
      \cos(E_0 t/\hbar) \\
      -i\sin(E_0 t/\hbar)
    \end{pmatrix}
    .
  \end{equation}

  \item 

  $E_0^{(1)}=\mel{0^{(0)}}{H_2}{0^{(0)}}=0.$

  \item 

  \begin{equation}
    E_0^{(2)}
    =
    \frac{|\mel{0^{(1)}}{H_2}{0^{(0)}}|^2}{E_0-E_1}
    =
    -\frac{1}{2}E_0
    .
  \end{equation}

  \item 

  固有方程式は
  \begin{equation}
    \begin{vmatrix}
      \lambda+1 & -\gamma \\
      -\gamma & \lambda-1
    \end{vmatrix}
    =
    0    
  \end{equation}
  なので,$E_0=-\sqrt{1+\gamma^2}E_0$が基底状態のエネルギーの厳密解です.$\gamma\ll 1$とすれば
  \begin{equation}
    E_0
    \sim
    -E_0
    -\frac{\gamma^2}{2}E_0
    =
    E_0^{(0)}
    +
    \gamma^2 E_0^{(2)}
  \end{equation}
  なので,よく近似ができています.

\end{enumerate}

\clearpage

\prb{6}{統計力学}

\begin{enumerate}
  
  \item 
  粒子$i$のエネルギー量子数を$n_i$とすると,
  \begin{equation}
    \varepsilon_i
    =
    \hbar\omega\left( n_i+\frac{1}{2} \right)
  \end{equation}
  であり,$M=\sum n_i$です.したがって,$E=\hbar\omega(M+N/2)$より
  \begin{equation}
    M
    =
    \frac{E}{\hbar\omega}-\frac{N}{2}
  \end{equation}
  です.

  \item 

  \begin{equation}
    W
    =
    \frac{M!}{N!(M-N)!}
    .
  \end{equation}

  \item 

  $M/N=E/N\hbar\omega-1/2$を代入すれば
  \begin{equation}
    S
    =
    k_B N
    \left[  
      \frac{2E}{N\hbar\omega}
      \log\frac{2E+N\hbar\omega}{2E-N\hbar\omega}
      +
      \frac{1}{2}
      \log\frac{4E^2-N^2\hbar^2\omega^2}{4N^2\hbar^2\omega^2}
    \right]
  \end{equation}
  です.

  \item 

  \begin{equation}
    \dv{S}{E}
    =
    \frac{k_B}{\hbar\omega}
    \log\frac{2E+N\hbar\omega}{2E-N\hbar\omega}
    .
  \end{equation}

  \item 

  $\ev*{E}$が満たす式は
  \begin{equation}
    \frac{k_B}{\hbar\omega}
    \log\frac{2\ev*{E}+N\hbar\omega}{2\ev*{E}-N\hbar\omega}
    =
    \frac{1}{T}
  \end{equation}
  なので,
  \begin{equation}
    \ev*{E}
    =
    \frac{N\hbar\omega}{2}\coth(\hbar\omega/2k_B T)
  \end{equation}
  です.

  \item 

  \begin{equation}
    z
    =
    \sum_{n=0}^{\infty}e^{-\beta\hbar\omega(n+1/2)}
    =
    \frac{e^{-\beta\hbar\omega/2}}{1-e^{-\beta\hbar\omega}}
    .
  \end{equation}

  \item 

  \begin{equation}
    \ev*{E}
    =
    \frac{N\hbar\omega}{2}
    \coth(\beta\hbar\omega/2)
    .
  \end{equation}

\end{enumerate}

\end{document}
