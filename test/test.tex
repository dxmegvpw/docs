\documentclass[ja=standard,xelatex]{bxjsarticle}
\usepackage{amsmath}%数式を書くためのパッケージ
%ちなみに「%」の右側はコメントになります
\title{テスト}
\subtitle{test}
\author{}%今回は略
\date{}%今回は略

\begin{document}
\maketitle
\section{第一部}
\subsection{第一章}
ここに本文を書いていく。

段落と段落の間(改行)は1行あける必要があるので注意。字下げも勝手にやってくれる。こんな感じにね。ちゃんと字下げされてるでしょ?便利。

\$で囲めば数式が書ける。$\hat{H}\psi_1 = E\psi_1$

別立てで数式を書くこともできる
\begin{align}
    \hat{H}\psi &= E\psi
\end{align}

まあこんな感じでこれまた勝手にいい感じのレイアウトにしてくれる。便利。
\end{document}
